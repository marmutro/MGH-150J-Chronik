\chapter{1975-1999}

\begin{history}

    \subsubsection*{Kant. Musikfest in Sempach., 22. Juni 1975}

    Wir spielen in der Festhalle das Aufgabestück \enquote{Nostalgische
        Ouverture}, wir bekommen von der Jury schöne 153 von 180 Punkten. In der
    Kirche spielen wir das Selbstwahlstück \enquote{Ouverture Symphonique}.
    Hier erhalten wir 160 Punkte. Mit der Totalpunktzahl von 313 erreichen
    wir den 5. Rang in der 2. Stärkeklasse.

    Von der Marschmusikjury erhalten wir am Nachmittag leider nur 29 Punkte.


    \subsubsection*{Musiktag in Hergiswil b.W. 22. Mai 1977}

    Mit Privatautos begeben wir uns an den Musiktag in Hergiswil b.W. Experte
    Alois Gschwind Dornach begutachtete unseren Vortrag von \enquote{Simfonietta
        Pastorale}. Am Nachmittag ziehen wir mit dem Marsch \enquote{Kameraden} am
    aufmerksamen Publikum vorbei.


    \subsubsection*{Musiktag Römerswil, 20. Mai 1978}

    Mit dem PW begeben wir uns nach Römerswil. Wir treten mit dem Stück
    \enquote{Royal Processional} von John J. Morrissey an. Als Experte amtet
    Eduard Muri, Zürich. Er wertet unseren Vortrag als sehr gut, stellt
    lediglich hie und da einige Wischerchen fest, gibt uns aber die Empfehlung
    diesen Vortrag nicht an einem Fest zu spielen. Als Marschmusikexperte amten
    Fridolin Bünter für das Musikalische und Josef Keist kontrolliert die
    Aufstellung. Er stellt einige Fehler fest, zum Beispiel Melden,
    Instrumentenhaltung und Abmarschieren. Im Expertenbericht merkt er an, es
    wäre schade, dass ein solch flottes Korps nicht etwas mehr Schneid und Rasse
    zeigt.


    \subsubsection*{Eidg. Musikfest Lausanne,  20. und 21. Juni 1981}

    Nach sehr harter Probearbeit -- in 100 Tagen war der Verein 50-mal beisammen
    -- fahren wir mit einem Car der Auto AG Rothenburg nach Lausanne.

    Wir spielen im Palais de Beaulieu Saal vor leider nur etwa 30 Personen.
    Zuerst spielen wir unser Selbstwahlstück \enquote{Synfonietta Pastorale} und
    danach das Aufgabenstück \enquote{London Scherzo}. Nach einem Gesamtfoto zur
    Erinnerung stellen wir uns am Nachmittag auf zum \enquote{Marsch zur Feier
        des Tages}. Am Sonntag nehmen wir Teil am farbenprächtigen Einzug von 150
    Musikvereinen ins Fussballstadion. Leider war die erreichte Punktzahl von
    Total 103.5 Punkten nicht das was wir erwartet hatten, dafür erhielten wir
    für die Marschmusik 46 Punkte.


    \subsubsection*{Solothurner Kantonales Musikfest in Balsthal, 24. Juni 1984}

    Zuerst spielen wir das Aufgabstück 3. Klasse \enquote{Capriccio für
        Blasmusik} in der Kirche. Danach das Selbstwahlstück \enquote{London
        River} in der Turnhalle. Wir erreichen Total 338 Punkte. Auf dem
    Computer sind wir auf den 1. Platz vorgerückt. Am Nachmittag erreichen
    wir bei der Marschmusik 88 Punkte, d.h. 16. von 56 Vereinen.

    Wegen eines heftigen Gewitters findet der Festakt im Festzelt statt, wo wir
    wegen Platzmangel in der Mitte zusammenstehen. Als unser 1. Rang verkündet
    wird fliegen die Hüte unters Zeltdach.

    Auf dem Hildisrieder Schulhausplatz erwartete uns viel Volk und die
    Dorfvereine, um uns zu gratulieren.


    \subsubsection*{Unterwaldner Musikfest Sarnen, 13. und 14. Juni 1987}

    Wir pielen zuerst den Marsch \enquote{Diavolezza}. Am Nachmittag spielen wir
    in der Turnhalle die Ouverture \enquote{A Suite for Switzerland} von Roy
    Newsome. Da die Resultate erst am Sonntag bei der Rangverkündigung bekannt
    gegeben werden, beginnt nun das grosse Rätselraten um gewonnene oder evtl.
    verlorene Punkte. Nach dem Bankett stieg dann in der Linden in Sarnen ein
    Fest, so dass die Wirtschaft den Musikanten und die Musikanten dem Wirt
    sicher noch einige Zeit in Erinnerung bleiben wird.

    Auch am Sonntag traf sich eine stattliche Zahl Hildisrieder Musikanten zur
    Rangverkündigung in Sarnen ein. Der 20. Platz von 28 Teilnehmern war eher
    eine magere Ausbeute. Besser ging es in der Marschmusik: Mit
    90$\sfrac{1}{2}$ Punkten belegen wir den 5. Rang.


    \subsubsection*{Musiktag in Menznau, 10. und 11. Juni 1989}

    Wir treffen am Samstagabend um 19 Uhr ein. Die Vorprobe haben wir im Saal
    des Gasthof Krone. In der vollen Rickenhalle spielen wir um 20.47 Uhr das
    Stück \enquote{March Prelude} von Edward Gregson. Im anschliessenden
    Gespräch mit dem Experten fand dieser dann durchwegs lobende Worte. Zur
    Marschmusik mussten wir am Sonntag um 14.30 antreten. Wir spielen den Marsch
    \enquote{King Size} von Fred L. Frank.

    Um 16 Uhr ist im Festzelt die Veteranenehrung. Drei unserer Kameraden
    befinden sich auch im Kreis der geehrten. Zum ersten Mal in der Geschichte
    der MGH kann ein Vereinsmitglied für 60 Jahre aktives musizieren geehrt
    werden. Kaspar Troxler bekommt zu diesem Anlass die Medaille des
    internationalen Musikbundes CISM. Am Abend feiern wir die Veteranen im
    Löwen.


    \subsubsection*{Kantonales Musikfest Schüpfheim, 16. und 17. Juni 1990}

    Bereits um 7.40 müssen wir in Schüpfheim sein und mit der Vorprobe beginnen.
    Bei einem lockeren Einblasen versuchen wir die Nervosität in den Zügeln zu
    halten. Um 8.45 gilt es zum ersten Mal ernst. Im Saal des Hotels Adler
    tragen wir das Aufgabenstück \enquote{Dies acterna} von Pascal Faver vor.
    Die Jury bewertet unsere Leistung mit 159.5 Punkten. Diese Punktzahl
    erweckte zuerst ein bisschen Missmut, war aber in der Endabrechnung gar
    nicht so schlecht. Um 9.30 stellten wir uns der Jury in der Pfarrkirche mit
    dem Selbstwahlstück \enquote{A Holiday Suite} von Eric Ball. Uns gelang die
    Vorstellung super, die Experten bedankten sich mit 168 Punkten. Dies gab
    zusammen 327.5 Punkte, was den ausgezeichneten 5. Rang bedeutete.

    Nächster Auftritt war die Marschmusikkonkurrenz am Nachmittag. Der Start zum
    Marsch \enquote{Hessen} gelang nicht ganz nach Wunsch, mit 41 Punkten
    landeten wir im Mittelfeld.

    Danach kam der gemütliche Teil des Festes. Schüpfheim zeigte sich natürlich
    auch da von der besten Seite. Am Sonntagabend wurden wir, wieder
    heimgekehrt, von den Dorfvereinen empfangen.


    \subsubsection*{Unterwaldner Musikfest in Hergiswil, 13. Juni 1992}

    Gut vorbereitet stellen wir uns um 14.25 in der Aula des Schulhauses mit dem
    Selbstwahlstück \enquote{Oregon} von Jacob de Haan der Jury. Diese bewertet
    unseren Vortrag mit 152 Punkten. Um 15.00 spielen wir in der Turnhalle das
    Aufgabenstück \enquote{Der Torero und die Zigeunerin} vom Luzerner
    Komponisten Otto Haas. Mit unserem Vortrag mochten wir der Jury sehr gut zu
    gefallen. Sie bedankte sich mit 169 Punkten. Mit einem Total von 321 Punkten
    reicht dies zum 4. Platz von 12 Konkurrenten in der 2. Stärkeklasse.

    Um 17.15 marschierten wir mit dem Marsch \enquote{Front and Center} von Pat
    Lee an den Kampfrichtern vorbei. Wir mochten nicht ganz zu überzeugen, die
    erreichten 83 Punkte reichten nur zum 17. Rang.

    Nach einem feinen Nachtessen geniessen wir  noch manche frohe Stunde im
    Lopperdorf.


    \subsubsection*{Jubiläums-Musiktag in Rothenburg, 20. und 21. Juni 1992}

    In Rothenburg findet der Jubiläums-Musiktag statt. Um 18.45 spielen wir in
    der Chärnshalle das Stück \enquote{Oregon}. Wir mochten unseren Experten
    Ives Illi mit dem Vortrag zu überzeugen, was er mit einem guten
    Expertenbericht honorierte.

    Am Sonntag Mittag eröffnen wir die Marschmusikvorträge mit dem Marsch
    \enquote{Bärner Musikante} von Walter Joseph.


    \subsubsection*{Kant. Musiktag in Ermensee, 4. Juni 1994}

    Mit dem Marsch \enquote{Gret little Army} bestreiten wir die
    Marschmusikkonkurrenz. Die Marschdisziplin wird als sehr gut, die
    musikalische Darbietung eher als mässig beurteilt. Am Abend stellen wir uns
    dann in der voll besetzten Mehrzweckhalle mit der \enquote{Romantischen
        Ouverture} von Stephan Jäggi der Jury. Mit voller Konzentration und grossem
    Einsatz geben wir diese Komposition zum Besten. Dies wurde mit grossem
    Applaus und einem guten Jurybericht belohnt.

    Natürlich widmen wir uns jetzt auch mit gleichem Einsatz dem gemütlichen
    Teil dieses Festes, was bei einigen bis in die Morgenstunden anhielt.


    \subsubsection*{Kant. Musikfest Reiden, 17. Juni 1995}

    Gut vorbereitet nehmen wir an diesem musikalischen Wettstreit in der 2.
    Klasse Brass Band teil. Wir stellen uns mit dem Selbstwahlstück
    \enquote{Triptych for Brass Band} von Philip Sparke und dem Aufgabenstück
    \enquote{Un Soupcon de Paganini} der Jury. Mit 85.2 im Aufgabe- und 89.0
    Punkten im Selbstwahlstück (Total 174.2 Punkte) belegen wir den 7. Rang von
    14 teilnehmenden Vereinen. Mit diesem Mittelfeldplatz sind wir zufrieden,
    kalkulierten jedoch bei unserem anspruchsvollen Selbstwahlstück mit etwas
    mehr Punkten.

    Bei gutem Festwetter bestreiten wir anschliessend den Marschmusikwettbewerb.
    Dieser Auftritt lässt bezüglich Marschdisziplin sowie der musikalischen
    Leistung zu wünschen übrig, was uns auch die Jury bestätigt. (49. Rang, 85.4
    Punkte)


    \subsubsection*{Eidgenössisches Musikfest Interlaken, 15. und 16. Juni 1996}

    Im Löwen spielen wir uns auf den musikalischen Wettstreit ein. Vor der
    Abfahrt geben wir unseren Firmlingen noch ein Ständchen.

    Anschliessen fahren wir via Lungern (Mittagessen) nach Interlaken. Das
    sommerliche Wetter verspricht ein Superfest. Angespannt tragen wir im
    vollgestopften Aareparksaal das Aufgabestück \enquote{Offside} von Christian
    Henting der Jury vor. Mit 138 erreichten Punkte sind wir nur mässig
    zufrieden.

    Nun verschieben wir in die Kirche von Unterseen. Gut vorbereitet spielen wir
    da unser Selbstwahlstück \enquote{A Saddleworth Festival Ouverture} von Goff
    Richards. Nach diesem Vortrag haben wir ein gutes Gefühl. Hoffentlich
    bestätigen dies auch die Experten. Und siehe da, mit 149 Punkten im
    Selbstwahlstück, Total 287 sind wir ganz vorne in der Rangliste. Im 9. Rang
    von 41 klassierten Vereinen sind wir mehr als zufrieden.

    Nun wollen wir auch in der Marschmusik einen guten Eindruck hinterlassen.
    Diesmal hat uns das Gefühl getäuscht. Mit 107 Punkten und einem 12. Rang von
    41 Vereinen werden unsere Erwartungen weit übertroffen.

    Nun werden diese erzielten Punkte bis in die Morgenstunden gefeiert.
    Gutgelaunt kehren wir am Sonntag via Brünig wieder nach Hildisrieden zurück.
    Mit einem Grossaufmarsch werden wir von der ganzen Bevölkerung empfangen.


    \subsubsection*{Musiktag in Hergiswil bei Willisau., 13. Juni 1998}

    Um 17.12 beginnen wir mit dem Marsch \enquote{Fidelity} von Henk Hogestein.
    Nach einem kurzen Imbiss tragen wir in der vollen Mehrzweckhalle unser
    Konzertstück \enquote{Four Little Maids} von John Carr vor.

    Von Thomas Rüedi bekommen wir einen sehr guten Expertenbericht. Einige
    Verbesserungsmöglichkeiten sieht er in der Intonation und der Dynamik.

\end{history}