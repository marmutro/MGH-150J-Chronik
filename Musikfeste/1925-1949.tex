\chapter{1925-1949}

\begin{history}

    \subsubsection*{1926 11. Kantonal Musikfest in Hochdorf 29./30. Mai}

    Direktion: Alois Troxler.\\
    Dass diesem Feste vermehrte Proben vorangingen, ist selbstverständlich. Mit
    74 Proben auf dem Buckel, konnten sie doch gut gerüstet und mit vollster
    Zuversicht antreten.

    Leider erlitt der Dirigent Mitte April einen Unfall mit dem Motorrad und
    verletzte den linken Oberarm, was ihn am Dirigieren hinderte.

    Vom Verein zweimal eingeladen, erschienen Herr Ritzmann, Direktor der
    Musikgesellschaft Emmen und Herr Jaggi, Direktor der Feldmusik Hochdorf an
    den Proben und gaben durch fachliche Kritik weise Ratschläge und gute Lehren
    für den bevorstehenden Wettkampf.

    Die Hildisrieder hatten sich in der 1. Klasse angemeldet, und mussten in
    dieser Kategorie nur ein Selbstwahlstück spielen. Mit dem Verein
    konkurrierten noch sieben Verbandsund vier Gastvereine. Es war die maximale
    Punktzahl von 50 Punkten erreichbar.

    Ouverture zu \enquote{Till Eulenspiegel} von Corradi\\
    Ein Expertenbericht liegt leider nicht vor. Aus dem Protokoll entnehmen wir:
    Von den 50 erreichbaren Punkten holten wir deren 45, was uns den dritten
    Rang einbrachte. Musikgesellschaft Nebikon als erster erreichte 45 Punkte.

    An der Veteranenehrung in der Festhütte konnte zwei wackeren Musikanten
    Kaspar Troxler, Moos, seit 1894 Mitglied und Kaspar Suter, Dorf, seit 1896
    Mitglied die silberne Veteranen-Medaille ausgehändigt werden.


    \subsubsection*{1933 Kantonal Musiktag in Rickenbach, 14. Mai}

    Direktion: Alois Estermann\\
    Nach sieben Jahren Unterbruch fanden wir es für angezeigt, wieder einmal ein
    solches Fest zu besuchen. Mit grosser Verspätung waren wir mit dem
    Mostereiauto, Tenue zivil, in Rickenbach eingetroffen. Zur vorgesehenen
    Probe reichte es nicht mehr. Als wir zum Wertkampfspiel bereit waren,
    stellte der Dirigent mit Schrecken fest, dass der Tenorhornist noch fehlte.
    Just beim Eintritt hatte er entdeckt, dass ein Ventil streikte. Mit einem
    Musikfreund aus Hellbühl wurde schnell das Horn ausgetauscht und nun
    Glückauf zur Fahrt ins Glück.

    Lustspiel Onverture \enquote{Die Fahrt ins Glück} von Friedemann

    Der Musikexperte, Herr Rosenberg aus Thun schrieb folgenden Bericht:\\
    Die Musikgesellschaft Hildisrieden hat mit dem Vortrag dieser Komposition
    eine fleissige und gut präparierte Vorarbeit gezeigt. Die Ouverture wurde
    von der energischen und umsichtigen Direktion im Vortrag gut zu Gehör
    gebracht. Viel Mühe war auf die dynamischen Schattierungen und Kontraste
    verwendet worden. Diese Arbeit hat sich gelohnt — sie ist mit gut zu
    bezeichnen. Als Ratschläge: Auf die harmonische Reinheit ist immer wieder
    Wert zu legen. Tonleiter und Akkordübungen verbunden mit dynamischen
    Kontrasten erzielen mit der Zeit überraschende Resultate. Das zeitweise noch
    hörbare Nachdrücken des Tones muss abgestellt werden.


    \subsubsection*{1934 13. Kantonal Musikfest in Wolhusen Direktion: Alois Estermann}


    Ouverture zum \enquote{Liederspiel Mirella} von C. Friedemann 1. Rang, 47
    Punkte

    Diesen Expertenbericht möchten wir im Wortlaut wiedergeben: Hier stellte
    sich dem Kampfgericht ein Musikkors vor, an welchem man seine aufrichtige
    Freude haben konnte. Der Dirigent hatte seine Leute vollständig in der Hand
    und die Bläser folgten seinem Stabe mit einem Anpassungsvermögen, dass der
    Zuhörer nie das Gefühl bekam, es seien Schwierigkeiten zu überwinden. Das
    nennt man musizieren, was auch bei der Rangierung seinen Lohn fand.

    Die Wahl des Wettstückes war für den Verein eine glückliche, da die
    Ouverture allen Ausübenden wirklich gut lag. Mit einem schneidig
    aufgefassten Presto beginnend, sah man sofort, dass der Dirigent wusste, was
    er wollte und auch was er von seinen Musikanten verlangen konnte. Die
    Sechszehntelnote im 1. und 2. Takt, welche sich später des öfters
    wiederholten, wurde zu kurz geblasen, was den Anschein erweckte, sie würde
    verschluckt. Es ist ja recht, wenn solche Stellen schön kurz geblasen
    werden, aber immerhin müssen derartige Noten noch wahrzunehmen sein. Im 5.
    Takt vor dem Allegro wurden die betonten Achtelsnoten etwas zu kurz
    genommen. Die ersten zwei Takte im Allegro stimmten nicht genau. Sehr schön
    und duftig wurde nun das Allegretto durchgeführt. Im Pocolento kam der erste
    Staccatoakkord nicht präzis zusammen. Tenorpartie war gut und insgesamt
    schöne Beachtung der dynamischen Zeichen. Die Schlussakkorde wurden
    plötzlich etwas zu schnell. Durchwegs war die Stimmung sehr gut abgetönt.
    Alle vorgenannten Mängel waren jedoch nur so geringfügiger Natur, dass das
    Kampfgericht mit Freuden diese schöne Punktzahl erteilte, Punktierung:

    Harmonische Reinheit 9, Rhythmik 10, Dynamik 9, Auffassung 9, Gesamteindruck
    10, Total 47 Punkte.



    \subsubsection*{1938 14. Kantonal Musikfest in Willisau}

    Direktion: A. Estermann 3. Klasse, 8. Rang, 83 Punkte

    Unser Verein, erstmals in der dritten Klasse konkurrierend, hat neben dem
    Selbstwahlstück noch ein Aufgabenstück zu spielen. Dieses Konzertstück, das
    eigens. Für dieses Fest komponiert wurde, mussten alle Vereine der dritten
    Klasse spielen.

    Der Dorfkönig von Steinbeck, Selbstwahlstück

    Auszug vom Expertenbericht: Beim Vortrag hätten wir uns ein genaueres
    Studium gewünscht. Dabei wollen wir absolut nicht verkennen, dass wohl
    fleissig studiert worden ist, aber die unbedingt notwendige Präzision,
    wodurch eine einwandfreie Wiedergabe gewährleistet wird, fehlte. Man fasse
    diese Ermahnung nicht falsch auf und befleisse sich nach folgenden Weisungen
    die künftigen Proben zu führen und die Spielfreudigkeit wird sich wesentlich
    heben, denn viel guter Wille ist in der wackeren Schar vorhanden.

    Ritter Blaubart von C. Friedemann, Aufgabenstück

    Für Hildisrieden müssen wir feststellen, dass bei intensivem Studium eine
    noch bessere Leistung hätte geboten werden können. Auch hier ist man in den
    Fehler verfallen, kein reines Forte und Fortissimo zu spielen. Die
    Einleitung verlor durch die wenig beachteten dynamischen Zeichen ihre ganze
    Wirkung.

    Frohe Heimkehr von Schild, Marsch-Wettbewerb Marschmusik-Punktierung:

    Auffassung der Aufgabe 10, musikalische Ausführung 9, militärische
    Ausführung 8, Gesamteindruck 9, Total 36 Punkte

    \subsubsection*{1946 15. Luzerner Kantonal Musikfest Ebikon 25./26. Mai}

    Direktion: Josef Elmiger 3. Klasse, 1. Rang, 89 Punkte

    Ouverture zur komischen Oper: \enquote{Die Nürnberger Puppe} von Adam

    Auszug vom Expertenbericht von Heinrich Steinbeck: Es ist empfehlenswert,
    wenn das Anfangstempo etwas ruhiger genommen wird. Die Ausführung war
    korrekt, nur tonlich zu heftig. Die Wiedergabe des ziemlich schweren Werkes
    war, mit Ausnahme der Tonkultur, die nicht immer so gut war wie alle anderen
    Faktoren, von packender Wirkung. Zu loben ist auch die geschmackvolle
    Phrasierung und hauptsächlich noch die frische und lebendige Auffassung des
    tüchtigen Dirigenten. Der Gesamteindruck war trotz der erwähnten kleinen
    Unzulänglichkeiten sehr gut und die schöne Punktzahl lässt erkennen, dass
    das Kampfgericht an dem Vortrag der Musikgesellschaft Hildisrieden seine
    Freude hatte.

    Festspielreigen von L. Kempter, Aufgabestück

    Auszug vom Expertenbericht: Von Georges Acby, Freiburg. Einzig die unreinen
    Eingangsakkorde belasteten den Faktor harmonische Reinheit, sonst war die
    Stimmung sehr gut. Das Zusammenspiel litt im 4. Takt an ungenügender
    Präzision. Wenn das dim. vor der Reprise gut gelang, so fehlte der Coda die
    begeisternde, abschlusskrönende Energie. Die Interpretation war im grossen
    und ganzen gut. Sie hätte mit etwas mehr Schmiss und Leichtflüssigkeit an
    Effekt gewaltig gewonnen.

    Marschmusik: St. Triphon von A. Ney

    Bericht von Walter Spieler: Auffassung, Auswahl, Plazierung waren sehr gut.
    Die musikalische Ausführung war gut. Die Stimmung war mit Ausnahme der
    Trompeten gut. Die militärische Ausführung war sehr gut. Jedoch entspricht
    die genommene Schrittlänge 83 cm, nicht den Bestimmungen. Haltung, Richtung,
    Deckung und Marschordnung waren gut. Tempo 120. Mit kürzerem Schritt und
    ruhigerem Tempo wird der Verein bei einem späteren Fest ein vorzügliches
    Resultat erreichen.

    \subsubsection*{1948 Luzerner Kantonal Musiktag in Rain, 9. Mai}

    Direktion Libero Bazzani

    Ouverture \enquote{Exelsior} von A. Ney, Selbstwahlstück

    Marschmusik: Sebastianmarsch

    Leider gibt das Protokoll über diesen Anlass sehr wenig Auskunft. 31
    Musikvereine gaben dem Festort Rain die Ehre. Der Nähe halber begaben wir
    uns zu Fuss zum Festort. Der Expertenbericht gibt für die Marschmusik das
    Prädikat: Vorzüglich, für die Ouverture: Im allgemeinen gut.


    \subsubsection*{1948 Eidg. Musikfest in St. Gallen, 10. Juli}

    Direktion: Libero Bazzani

    3. Klasse, Blechmusik, 2. Rang, Silberlorbeer

    Der Beschluss zum Besuche des Eidg. Musikfestes in St. Gallen wurde an der
    Generalversammlung vom 27. Februar 1948 einstimmig gefasst. Dirigent und
    Musikanten waren sich bewusst, mit dem positiven Entschluss ein grosses
    Opfer aufgeladen zu haben. Der erstmalige Besuch eines eidgenössischen
    Festes, an dem über zweihundert Vereine aus allen Gauen des Landes
    teilnahmen, reizte uns. Alle waren bereit, ihr Bestes zu geben.

    Le Lac Maudit, dramatische Ouverture von H. Statz. instrumentiert von Otto
    Zurmühle, Selbstwahlstück

    Jahrhundertklänge von St. Jäggi, Marschmusik

    Mit diesen beiden Vorträgen stellten wir uns den Kampfrichtern. Nur aus
    achtundzwanzig Mann bestehend, waren wir einer der kleinsten Vereine. Auf
    der riesigen Bühne in der Festhalle, eng zusammensitzend, sahen wir aus wie
    Zwerge. Wir hielten uns tapfer und Dir. Bazzani führte seine Getreuen
    schneidig über alle Klippen des sehr anspruchsvollen Werkes. Der grosse
    Applaus zeigte, dass sehr gut musiziert wurde. Zu einem Vorzüglich reichte
    es nicht ganz wie der einte oder andere gehofft hatte. Doch dürfen wir mit
    diesem Erfolg zufrieden sein.

    Marschmusik: Militärspielleiter Bazzani zeigte da sein ganzes Können:
    Resultat vorzüglich.

\end{history}
