\section*{1874-1925}

\begin{history}

    \subsubsection*{1881 1. Kantonales Musikfest in Münster}

    Die musikalischen Anforderungen von damals und heute sind natürlich nicht
    miteinander vergleichbar. Dennoch verdienen unsere Musikanten, die es wagten
    mit nur acht Mann ein Kantonales Musikfest zu besuchen, für ihren Mut die
    vollste Anerkennung.

    Als Selbstwahlstick wurde die \enquote{Lustspiel-Ouverture} von Keller-Bela,
    arrangiert vom Musikdirektor Lampart aus Luzern, gewählt. Über 15 Vereine
    erschienen zum Feste, das am 17. Juli durchgeführt wurde. Von den
    konkurrierenden Vereinen sind noch bekannt: Stadtmusik Luzern, Feldmusik
    Sarnen, Stadtmusik Sursee, Stadtmusik Willisau, Harmonie Sempach und die
    Musikgesellschaften Ettiswil, Rickenbach, Kriens, Weggis, Hellbühl, Schongau
    und Kleinwangen.

    Unserer Musikantengruppe war ein sehr schöner Erfolg beschieden, das beweist
    uns heute noch eine Erinnerungstafel. Sie lautet: Musikgesellschaft
    Hildisrieden 1. Preis, 2. Klasse, Lorbeerkranz für gute Leistung. Das gute
    Abschneiden freute die ganze Dorfschaft, gab den Musikanten aber Mut und
    Vertrauen und spornte sie zu weiteren Taten an.

    Die Stadtmusik Luzern, von unserem Erfolg selbst erfreut, machte bei der
    Heimkehr Halt auf dem Dorfplatz und gab ein Ständchen zum Besten.

    \subsubsection*{1924 11. Kantonal Musiktag in Sempach}

    Direktion: Alois Troxler\\
    Das Festwetter war den Musikanten nicht hold und machte bis gegen Abend ein
    trübes Gesicht. Dessen ungeachtet zogen sie früh morgens, um das Banner
    geschart, zum friedlichen Wettkampf. Was dem einten oder andern an Courage
    noch mangelte, das ersetzte der von zarter Hand gestiftete Ehrentrunk beim
    Empfang im Hotel Kreuz.

    Am Fest beteiligten sich 12 Verbands- und neun Nichtverbandsvereine.
    Hildisrieden war noch Nichtverbandsmitglied. Nach Eich und Alberswil wurden
    sie zum Wettkampf aufgerufen.

    Das Selbstwahlstück: Ouverture \enquote{Milanese} von Baumann.\\
    Die in Heftform zugestellte Kritik des Kampfgerichtes, vertreten durch die
    Herren H. Genhart, Musikdirektor, Langenthal, und G. Sauter,
    Obermusikmeister in Würzburg, lautete wie folgt: Hildisrieden
    Musikgesellschaft, Dir, Alois Troxler, 4. Rang.

    Das Beste von allem war der Rhythmus. Auf diesem Gebiet konnten verschiedene
    recht ansprechende Resultate beobachtet werden, weniger glücklich dagegen
    fand sich die Gesellschaft hinsichtlich Stimmung und Dynamik ab. Hier gibt
    es noch viel zu tun und die Mühe würde sich lohnen, dem mit solch
    leistungsfähige, scheinbar gut veranlagtem Material, lässt sich bei
    fachkundiger, sachlicher Leitung schon etwas erreichen. Der Gesamteindruck
    der offenbar tüchtigen Gesellschaft war gut und berechtigt zu schönen
    Hoffnungen, vorausgesetzt, dass der Dirigent den besagten Fehlern Herr wird.


\end{history}
