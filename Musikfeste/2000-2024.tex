\chapter{2000-2024}

\begin{history}

    \subsubsection*{Kant. Musikfest Kriens, 27. Mai 2000}

    Bei strömendem Regen fahren wir am Morgen mit der Auto AG Rothenburg nach
    Kriens. Nach dem Einspielen tragen wir in der Kirche das Aufgabestück
    \enquote{Focus} von Leon Vargas und in der Mehrzweckhalle das
    Selbstwahlstück \enquote{North-West Passage} von Roy Newsome vor.

    In der Rangverkündigung wurden wir mit einem schlechten Ergebnis
    konfrontiert. Mit 155 Punkten belegen wir den 13. Rang von 14 Vereinen in
    derselben Klasse. In der Marschmusik konnten wir einen Mittelfeldplatz
    erreichen.


    \subsubsection*{Eidg. Musikfest Friboug, 16. und 17. Juni 2001}

    Bei regnerischem Wetter, aber in guter Stimmung reisen wir am Samstagmorgen
    mit einem Car nach Fribourg. Nach 5 Minuten Strammstehen in strömenden Regen
    konnten wir unser Marschmusikvortrag absolvieren. Das lange Warten hat sich
    gelohnt: Mit 103 Punkten sind wir in der oberen Ranglistenhälfte gelandet.

    Anschliessend bereiten wir uns auf die Konzertstücke vor. Mit dem
    Aufgabestück \enquote{Alpine Variations} von Bertrand Moren und dem
    Selbstwahlstück \enquote{Prelude and Celebration} von James Curnow treten
    wir gut vorbereitet vor die Jury. Die beiden Darbietungen gelingen uns recht
    gut. Mit je 150 Punkten stehen wir am Schluss auf dem guten 19. Rang von 46
    teilnehmenden Vereinen.

    Am Abend erkunden wir die Altstadt von Fribourg. Der herzliche Empfang von
    der Hildisrieder Bevölkerung vom Sonntagabend ist ein schöner Abschluss
    eines gelungenen Musikfestes.


    \subsubsection*{Musiktag in Grosswangen, 24. Mai 2003}

    Bei hervorragendem Wetter reisten wir mit zwei Bussen vom „Froschaugen-
    Winu“ nach Grosswangen. Nach dem Einspielen warteten wir gut vorbereitet auf
    den Vortrag von unserem Selbstwahlstück \enquote{Schattdorf Impressionen}.

    Nach dem nicht allzu schlechten Jurybericht von Philippe Bach stand nur noch
    die Marschmusik bei 25 Grad vor der Türe: Mit dem bekannten
    \enquote{Bundesrat Gnägi Marsch} erreichten wir mit 48.3 Punkten den
    zufrieden stellenden 42. Rang.

    Nach dem ersten Bier, offeriert von der Begleitperson, bedankten wir uns mit
    der Zweitaufführung vom Marsch.


    \subsubsection*{Kantonaler Musiktag in Büron, 22./23. Mai 2004}

    Früh morgens starteten wir mit einem Kleinbus nach Büron, um uns mit anderen
    Vereinen zu messen. Vor fast leerem Saal spielten wir um 9 Uhr mit der
    Unterhaltungsnummer \enquote{Hollywood} auf. Nach dem Expertengespräch mit
    Thomas Rüedi stand uns ein langer Tag vor der Tür.

    Als wir uns dann um 17.00 Uhr zum Showblock formierten und mit dem Marsch
    \enquote{Ravanello} 48.1 Punkte erreichten konnte das Fest beginnen.

    Als uns die Chauffeuse um ca. 4 Uhr in der Früh abholte, konnte sie Dank
    energischem Eingreifen doch noch eine Fahnenentführung verhindern.


    \subsubsection*{Kant. Musikfest in Nottwil, 28./29. Mai 2005}

    Unter der Leitung von Kobi Banz erreichten wir in der 2. Stärkeklasse Brass
    Band den 13. Rang mit 82.0 beim Selbstwahlstück \enquote{Dancing in the
        Park} und 85.3 im Teststück \enquote{Newstaed}. In der Marschmusik reihten
    wir uns am Anfang des Mitteldrittels auf den 14. Rang ein.


    \subsubsection*{Eidg. Musikfest in Luzern, 17./18. Juni 2006}

    In der Bruchmattturnhalle eröffneten wir als erste in der 2. Klasse Brass
    Band den Konzertmorgen mit dem Pflichtstück \enquote{Fanfare and Funk} und
    erzielten 258 Punkte. Anschliessend spielten wir unser Selbstwahlstück
    \enquote{Dimensions} von Peter Graham, für das wir 245 Punkte erhielten, was
    uns insgesamt auf den 14. Platz mit 503 Punkten brachte.

    Unser Auftritt in der Marschmusik war besonders erfolgreich; mit 253 Punkten
    erreichten wir den hervorragenden 4. Platz mit dem Marsch \enquote{Triumph}
    von Hans Heusser.


    \subsubsection*{Musiktag Gettnau, 9./10. Juni 2007}

    Gut vorbereitet machten wir uns mit dem Car auf den Weg nach Gettnau, der
    Heimatgemeinde unseres Dirigenten. Alle waren unter Druck, schliesslich
    wollten wir einen guten Eindruck hinterlassen, was uns auch gelang, mit dem
    schwierigen Selbstwahlstuck \enquote{Resurgam} von Eric Ball. Unser Juror
    Thomas Wyss gab einen guten Bericht ab, was uns motivierte, auf der
    Marschmusikstrecke noch einmal unser Bestes zugeben.

    Als dann die beiden A-Noten (technischer Eindruck) verteilt waren, wussten
    wir, dass wir das ganze mit einer guten B-Note (Interpretation und
    Choreographie) noch wettmachen können. Denn mit unserer guten Kameradschaft
    war das für uns kein Problem eine 6.0 zu erreichen...


    \subsubsection*{Musiktag Eschholzmatt, 7. Juni 2008}

    Wir spielten das fetzige Konzertstück \enquote{Shine as the Light} von Peter
    Graham. Der Experte war durchaus begeistert. Nur zum spielen mit Dämpfer
    meinte er \enquote{die Leute, die spiele mit Daampfff, müssen haben ein
        wenig meeehr exposition!}

    Mit dem Marsch \enquote{Furchtlos und Treu!} von Julius Fucik erhielten wir
    49 Punkte.


    \subsubsection*{Oberwalliser Musikfest Susten-Leuk, 13. Juni 2009}

    Die Darbietung der Marschmusik begann wie für uns üblich mit Tambour-Beginn,
    Spiel vom Marsch \enquote{Fanfare en Fête} von Jean-Pierre Fleury und
    anschliessendem Spiel-Halt. Doch an diesem Tag lief es anders als erwartet.

    Die Walliser Zuschauer, die sich entlang der Strecke versammelt hatten,
    standen noch immer da, Hunderte Meter weiter, links und rechts der Parade.
    Sie erwarteten offensichtlich mehr von uns, und so, getrieben von ihrer
    ungeduldigen Anteilnahme, formierten wir uns erneut und wiederholten
    denselben Marsch noch drei Mal.

    Nachdem die Marschmusik schliesslich zum Abschluss gekommen war, widmeten
    wir uns den Konzertstücken. Für das Selbstwahlstück   \enquote{Music for a
        Festival} von Philip Sparke bekamen wir von der Jury 271 Punkte, für das
    Aufgabenstück \enquote{A Scots Miscelany} von Alan Fernie 258 Punkte.

    Obwohl wir als ausserkantonaler Verein nicht offiziell rangiert wurden,
    hätte unsere Leistung für einen stolzen 5. Rang gereicht. Diesen Erfolg
    feierten wir ausgelassen mit einem selbstmitgebrachten Pokal.


    \subsubsection*{Kant. Musikfest Willisau, 12./13. Juni 2010}

    Die MGH eröffnete den Marschmusikwettbewerb mit dem Marsch \enquote{Fanfares
        en Fête}. Wir erzielten 45.5 von 60 möglichen Punkten und landeten auf
    dem 27. Platz unter 36 teilnehmenden Vereinen.

    Nach einer kurzen Erholung begann um 15:40 Uhr das Einspielen, gefolgt von
    den Vorträgen. Auf der warmen Bühne präsentierte die MGH zunächst das
    Pflichtstück \enquote{Symphonic Contrasts} von Etienne Crausaz und
    anschliessend das Selbstwahlstück \enquote{The Prizewinners} von Philip
    Sparke. Die Musikanten und ihr Dirigent Michael Rösch waren mit ihrer
    Darbietung zufrieden und hofften auf eine gute Bewertung.

    Bei der Mitternachtsverkündung erfuhr die MGH, dass sie auf dem 20. Platz
    unter 26 Mitbewerbern gelandet war. Anschliessend wurde der Jurybericht
    genauestens studiert, um den Grund für diesen enttäuschenden Rang zu
    verstehen.

    \subsubsection*{Eidg. Musikfest St. Gallen, 18./19. Juni 2011}

    Aufgrund einiger technischer Anmeldeprobleme bei den Organisatoren musste
    die MGH ihr Können am Sonntagnachmittag um 16 Uhr als letzter Verein zum
    Besten geben.

    Beim Selbstwahlstück \enquote{Chorale and Toccata} von Stephen Bulla
    erhielten wir 83 Punkte. Die Juroren vom Aufgabenstück \enquote{Odin, King
        of Asgard} von Bertrand Moren gaben uns 78.6 Punkte. Mit diesen Punkten
    landeten wird auf dem 13. Rang von 17 in der 2. Klasse Brass Band.

    Beim Marschvortrag \enquote{The Drum Major} von J.J. Taylor bewerteten uns
    die Juroren mit 82 Punkten.


    \subsubsection*{Musiktag Aesch, 9. Juni 2012}

    Am frühen Morgen besammelten sich die Musikanten/innen und Ehrendamen für
    die Fahrt nach Aesch. Nach einem Zmorgen mit Zopf und Kaffee ging es weiter
    in das Instrumentendepot, wo wir unsere Bagage deponierten. Anschliessend
    gab es einen kurzen Marsch von 10min ins Einspiellokal.

    Der Auftritt war aus unserer Sicht zufriedenstellend. Bei einer Stelle war
    jedoch aus irgendeinem Grund kein Cornet mehr zu hören. Unser Experte meinte
    jedoch, wir haben "bon travail" geleistet und sollten noch ein bisschen
    "plus piano" spielen. Von unserem Patzer erwähnte er nichts.

    Danach stand bereits die Marschmusik mit dem Marsch \enquote{Gruss an Bern}
    von Carl Friedemann auf dem Programm. Wunderschön gestartet, guter Sound,
    konzentriertes Marschieren bis -- ja bis wir am Schluss in die
    Höudi-Fan-Kurve gelangten, in welcher fleissig applaudiert wurde. Dadurch
    verstand Philipps "Spiel-Halt" nur noch knapp die erste Reihe. Je weiter
    nach hinten, umso holpriger und verschobener war der Halt.

    Trotzdem reichte es mit 49.4 Punkten auf den 9. Platz von 16 und es wurde
    wie immer bis tief in die Nacht gefeiert, was man gewissen Uniformen noch
    immer entnehmen kann. Irgendein Hobby-Clown sprühte zu später Zeit weissen
    Bauschaum umher, von welchem unsere ich-geh-erst-nach-Hause-wenns-hell-ist
    Musikanten leider noch getroffen wurden.


    \subsubsection*{Musiktag Wauwil, 31. Mai 2014}

    Die Musikanten und Musikantinnen der Musikgesellschaft Hildisrieden konnten
    schon am Samstagmorgen um 10.15 Uhr ihr Vortragsstück  \enquote{Kingdom of
        Dragons} von Philip Harper dem Publikum sowie dem Experten Armin Bachmann
    vortragen.

    Kurz darauf konnten die schöneren Musikanten weiter brillieren, wobei die
    anderen bei der Fotoaufnahme in die hinteren Reihen stehen mussten. Schon
    bald war dann auch Mittag und allesamt gesellten sich am Bankett an den
    Tisch. Auch dort konnten sich mit einem zügigen Ablauf die Organisatoren und
    Helfer auszeichnen.

    Um 14.15 Uhr stellte sich die MGH zur Marschmusik auf mit dem Marsch
    \enquote{Bundesrat Gnägi} von Albert Benz, den wir auswendig spielten. Die
    zufriedenstellende Darbietung wurde mit 48.9 Punkten bewertet.


    \subsubsection*{Kant. Musikfest Sempach, 6. Juni 2015}

    Am Samstagnachmittag 14.45 standen die Musikantin und Musikanten unter der
    brütenden Sonne auf der Marschmusikstrecke bereit für die Parademusik mit
    dem Marsch \enquote{Geb. Füs. Bataillon 48} von Hans Flury. Mit der
    kräftigen Unterstützung der mitgereisten Hildisrieder Fans gelang nach
    mehreren Jahren die Zielmarke von 50 Punkten. Diese wurde sogar um 0,6
    Punkte übertroffen, weshalb die MGH schliesslich mit dem 5. Rang belohnt
    wurde.

    Bei den Vorträgen vom Selbstwahlstück \enquote{A Malvern Suite} von Philip
    Sparke und dem Aufgabenstück \enquote{Strawabar} von Manuel Renggli konnte
    ebenfalls auf unsere Fans und Musikfreunde gezählt werden, weshalb die Ränge
    gut gefüllt und die Stimmung fantastisch war.  Diese übertrug sich auf die
    Musikantin und Musikanten, welche eine gute Darbietung boten und somit auf
    dem 10. Platz klassiert wurden.


    \subsubsection*{Eidg. Musikfest Montreux, 11. Juni 2016}

    Der anhaltende Regen verzögerte den Beginn der Marschmusikvorträge bis nach
    15 Uhr. Um 16:03 Uhr zeigte die Musikgesellschaft Hildisrieden trotz
    schwieriger Witterungsbedingungen eine beeindruckende Leistung und erhielt
    für ihren Marsch \enquote{Geb. Füs. Bataillon 48} von Hans Flury
    ausgezeichnete 90 Punkte, was den 2. Platz in ihrer Ranglistengruppe
    bedeutete.

    Am Abend bekam die MGH für den Vortrag vom Selbstwahlstück \enquote{Turris
        Fortissima} von Steven Ponsford. 92.6 Punkte. Die Aufführung vom
    Aufgabenstück \enquote{Caverns} von Fabian Künzli bewertete die Jury mit
    88.3 Punkten. Mit dem 10. Platz sicherten wir uns einen Platz in der
    oberen Hälfte der Rangliste.


    \subsubsection*{Musiktag Schüpheim, 27. Mai 2017}

    Auf der Parademusikstrecke am Samstagnachmittag um 15.30 Uhr gelang uns mit
    dem Marsch \enquote{Viva Arogno} von Josef Walter der am höchsten bewertete
    Auftritt in der Geschichte der Musikgesellschaft Hildisrieden. Mit
    sensationellen 53.1 Punkten wurde der Sieg mit lediglich 0.2 Punkten
    verfehlt.

    Auch beim Konzertvortrag mit dem Konzertstück \enquote{Lord of all} von Dean
    Jones hatte der Juror Thomas Rüedi neben einigen Verbesserungsmöglichkeiten
    viel Lob zu verteilen.


    \subsubsection*{Musiktag Eschenbach, 2. Juni 2018}

    Bei heissen Temperaturen spielten wir das Konzertstück  \enquote{Trittico}
    von James Curnow in der grossen Dreifach-Turnhalle. Der Experte gab uns eine
    gute Bewertung, obwohl immer etwas verbessert werden kann.

    Bei der Parademusik wurde es dann richtig warm an den Füssen. Die MGH bekam
    für den Marsch \enquote{Viva Arogno} von Josef Walter von der Jury 49.6
    Punkte, was den 6. Schlussrang ergab. Einige mokierten aber, die extra
    angagierte Tambourengruppe sei nicht schön marschiert.


    \subsubsection*{Innerschweizer Musikfest, Hergiswil, 16. Juni 2019}

    Wir spielten als erster Verein am Sonntagmorgen, konnten dafür aber schon
    auf der Bühne einspielen. Für das Selbstwahlstück \enquote{Cristo Redentor}
    von Steven Ponsford erhielten wir 93 Punkte, für das gleich darauf folgende
    Aufgabenstück \enquote{Oneiric Tales} von Eddy Debons sogar 94 Punkte.

    Bald zeigte sich, dass wir es auf den 1. Platz der 4 teilnehmenden 2. Klasse
    Brassbands geschafft hatten. Es freute uns besonders, dass wir sogar die BB
    Root geschlagen hatten.

    Die Marschzeit war erst am späten Nachmittag geplant. Durch weitere
    Verzögerungen spielten wir schlussendlich erst nach 17.30 den Marsch
    \enquote{Menzberg} von Mario Bürki. Wir bekamen von der Jury dafür 87.3
    Punkte.

    \subsubsection*{Kant. Musikfest Emmen, 19. Juni 2022}

    Zuerst mussten wir zur Marschmusik antreten. Da es an diesem Wochenende sehr
    heiss war, warteten wir im Schatten unter der Autobahnbrücke.

    Bei über 35° spielten wir den Marsch \enquote{Menzberg} von Mario Bürki. Wir
    bekamen von der Jury 84.8 Punkte. Der dadurch resultierende 1. Rang - den
    wir leider mit Egolzwil teilen mussten - wurde am Abend frenetisch gefeiert.

    Zuvor galt es aber noch, das Selbstwahlstück \enquote{Shine as the Light}
    von Peter Graham und das Aufgabenstück Aufgabenstück \enquote{Cascades} von
    Sami Lörtscher den Juroren zu präsentieren. In der extrem heissen Halle
    erreichten beide Vorträge jeweils 90 Punkte, was uns auf den 5. Platz in der
    Rangliste brachte.

    \subsubsection*{Marschpreis.LU, 3. Sept. 2022}

    Wir spielten den Marsch \enquote{Margam Abbay} von T.J. Powell.

    Unser Transportvehikel war ein altes Postauto. Wir starteten den Marschpreis
    in der Mehrzweckhalle Rain und erhielten für unsere Performance 96 Punkte.
    Damit kamen wir in Rain auf den 2. Rang über alle Bands, knapp hinter der
    Kirchenmusik Flühli.

    Der nächste Auftritt war auf dem Vorplatz der Halle in Neuenkirch. Hier
    bekamen wir vom Juror 92 Punkte. Das reichte für den 4. Schlussrang über
    alle Bands. Dafür holten wir den Spezialpreis für die beste
    Marschbegleitung.

    Danach fuhren wir zum Knutwiler Bad, hier hatten die Organisatoren eine
    Bühne in einem gedeckten, aber offenen Festzelt gebaut. Auch hier war der
    Experte begeistert und gab uns 97 Punkte. Damit kamen wir auf den 1. Rang,
    vor der Kirchenmusik Flühli und holten zusätzlich den Spezialpreis für das
    bestgespielte Bassteil.

    Der letzte Standort für uns war die Lindenhalle in Gunzwil. Hier bekamen wir
    abermals tolle 94.5 Punkte. Das reichte hier aber nur für den 6. Rang.

    Mit der Gesamtpunktzahl 377.5 erreichten wir auf der Gesamtrangliste den 2.
    Rang, knapp hinter der Kirchenmusik Flühli. In der Kategorie 2./3. Brassband
    konnten wir den Sieg für uns entscheiden.


    \subsubsection*{Kant. Musiktag Wolhusen, 19. Mai 2024}

    Direkt nach der Fasnacht begannen die Proben vom Konzertstück \enquote{The
        Raid} von Oliver Waespi. In diesem anspruchsvollen Stück wurde eine Sage
    von einem Raub und das Sörenberger Lied verarbeitet.

    In der musikalisch \enquote{trockenen} 3-fach Turnhalle Berghof spielten wir
    am Pfingstsonntag kurz vor Mittag das Konzertstück. Der Experte Philip Bach
    war begeistert, auch die zahlreich anwesenden Zuhörer bestätigten den tollen
    Auftritt.

    Nach dem Gesamtfoto und einer kurzen Mittagspause bereiteten wir uns auf die
    Parademusik vor. Der kurze Regenguss hörte zum Glück auf bevor wir uns zum
    Richten aufstellten. Obwohl wir wetterbedingt nur 3 Proben machen konnten,
    bekamen wir für den Marsch \enquote{Rumisberg} von Walter Joseph von den
    Juroren 89.5 Punkte. Dies reichte überraschenderweise für den 3. Platz in
    der 2. Klass Brass Band.

\end{history}