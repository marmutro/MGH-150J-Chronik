\section*{2000-2024}

\begin{history}

    \subsubsection*{Kant. Musikfest Kriens, 27. Mai 2000}

    Bei strömendem Regen fahren wir am Morgen mit der Auto AG Rothenburg nach
    Kriens. Nach dem Einspielen tragen wir in der Kirche das Aufgabestück
    \enquote{Focus} von Leon Vargas und in der Mehrzweckhalle das
    Selbstwahlstück \enquote{North-West Passage} von Roy Newsome vor.

    In der Rangverkündigung wurden wir mit einem schlechten Ergebnis
    konfrontiert. Mit 155 Punkten belegen wir den 13. Rang von 14 Vereinen in
    derselben Klasse. In der Marschmusik konnten wir einen Mittelfeldplatz
    erreichen.


    \subsubsection*{Eidg. Musikfest Friboug, 16. und 17. Juni 2001}

    Bei regnerischem Wetter, aber in guter Stimmung reisen wir am Samstagmorgen
    mit einem Car nach Fribourg. Nach 5 Minuten Strammstehen in strömenden Regen
    konnten wir unser Marschmusikvortrag absolvieren. Das lange Warten hat sich
    gelohnt: Mit 103 Punkten sind wir in der oberen Ranglistenhälfte gelandet.

    Anschliessend bereiten wir uns auf die Konzertstücke vor. Mit dem
    Aufgabestück \enquote{Alpine Variations} von Bertrand Moren und dem
    Selbstwahlstück \enquote{Prelude and Celebration} von James Curnow treten
    wir gut vorbereitet vor die Jury. Die beiden Darbietungen gelingen uns recht
    gut. Mit je 150 Punkten stehen wir am Schluss auf dem guten 19. Rang von 46
    teilnehmenden Vereinen.

    Am Abend erkunden wir die Altstadt von Fribourg. Der herzliche Empfang von
    der Hildisrieder Bevölkerung vom Sonntagabend ist ein schöner Abschluss
    eines gelungenen Musikfestes.


    \subsubsection*{Musiktag in Grosswangen, 24. Mai 2003}

    Bei hervorragendem Wetter reisten wir mit zwei Bussen vom „Froschaugen-
    Winu“ nach Grosswangen. Nach dem Einspielen warteten wir gut vorbereitet auf
    den Vortrag von unserem Selbstwahlstück \enquote{Schattdorf Impressionen}.

    Nach dem nicht allzu schlechten Jurybericht von Philippe Bach stand nur noch
    die Marschmusik bei 25 Grad vor der Türe: Mit dem bekannten
    \enquote{Bundesrat Gnägi Marsch} erreichten wir mit 48.3 Punkten den
    zufrieden stellenden 42. Rang.

    Nach dem ersten Bier, offeriert von der Begleitperson, bedankten wir uns mit
    der Zweitaufführung vom Marsch.


    \subsubsection*{Kantonaler Musiktag in Büron, 22./23. Mai 2004}

    Früh morgens starteten wir mit einem Kleinbus nach Büron, um uns mit anderen
    Vereinen zu messen. Vor fast leerem Saal spielten wir um 9 Uhr mit der
    Unterhaltungsnummer \enquote{Hollywood} auf. Nach dem Expertengespräch mit
    Thomas Rüedi stand uns ein langer Tag vor der Tür.

    Als wir uns dann um 17.00 Uhr zum Showblock formierten und mit der
    Marschmusik 48.1 Punkte erreichten konnte das Fest beginnen.

    Als uns die Chauffeuse um ca. 4 Uhr in der Früh abholte, konnte sie Dank
    energischem Eingreifen doch noch eine Fahnenentführung verhindern.


    \subsubsection*{Kant. Musikfest in Nottwil, 28./29. Mai 2005}

    Unter der Leitung von Kobi Banz erreichten wir in der 2. Stärkeklasse Brass
    Band den 13. Rang mit 82.0 beim Selbstwahlstück \enquote{Dancing in the
        Park} und 85.3 im Teststück \enquote{Newstaed}. In der Marschmusik reihten
    wir uns am Anfang des Mitteldrittels auf den 14. Rang ein.


    \subsubsection*{Eidg. Musikfest in Luzern, 17./18. Juni 2006}

    Mit leerem Magen trafen wir uns um 05.00 Uhr im Chrüz zum spendierten
    Morgenessen vom Chrüzwirt. Dann ging es weiter mit dem Car in die
    Leuchtenstadt. Angespannt spielten wir in der Bruchmattturnhalle ein. Wir
    hatten die Ehre den Konzertmorgen in der 2. Klasse Brass Band mit dem
    Aufgabestück \enquote{Fanfare and Funk}zu eröffnen. Die Begleitung mit der
    Bassgitarre war ungewohnt, passte aber gut zu dem funkigen Stück. 258 Punkte
    haben wir heraus gespielt, dann ging es weiter zum Selbstwahlstück. 245
    Punkte konnte nun geerntet werden, welches uns den 14. Schlussrang mit 503
    Punkten ergab.

    Bei der Marschmusik erreichten wir 253 Punkte, welches uns den 4. Rang
    bescherte.

    \subsubsection*{Musiktag Gettnau, 9./10. Juni 2007}

    Gut vorbereitet machten wir uns mit dem Car auf den Weg nach Gettnau, der
    Heimatgemeinde unseres Dirigenten. Alle waren unter Druck, schliesslich
    wollten wir einen guten Eindruck hinterlassen, was uns auch gelang, mit dem
    schwierigen Selbstwahlstuck \enquote{Resurgam}. Unser Juror Thomas Wyss gab
    einen guten Bericht ab, was uns motivierte, auf der Marschmusikstrecke noch
    einmal unser Bestes zugeben.

    Als dann die beiden A-Noten (technischer Eindruck) verteilt waren, wussten
    wir, dass wir das ganze mit einer guten B-Note (Interpretation und
    Choreographie) noch wettmachen können. Denn mit unserer guten Kameradschaft
    war das für uns kein Problem eine 6.0 zu erreichen...


    \subsubsection*{Musiktag Eschholzmatt, Frühling 2008}

    TODO überarbeiten

    Mann, mann, mann, wieder einmal lag es an den Musikanten viel Alkohol zu
    vernichten. Vorher gaben wir allerdings noch einmal unser aller Bestes. Und
    das reichte zumindest um unseren Juror zu überzeugen. Einzig die Leute, die
    spiele mit Daampfff, müssen haben ein wenig meeehr exposition! Naja... zum
    gemütlichen Abschluss fanden sich die Musikanten dann vor einem Bier in den
    diversen kleinen Bars auf dem Festgelände.


    \subsubsection*{Oberwalliser Musikfest Susten-Leuk, Frühling 2009}

    TODO Erinnerungen sammeln


    \subsubsection*{Kant. Musikfest Willisau, 12./13. Juni 2010}

    Gut vorbereiten und mit viel Zuversicht reisten wir mit dem Car nach
    Willisau, das das der letzte Auftritt unter der Leitung von Michael Rösch
    war, wussten zu diesem Zeitpunkt nur die beiden Präsidenten.

    Ohne unter Druck der Titelverteidigung zu stehen, konnten wir das geübte der
    Fachjury vortragen. Sichtlich zufrieden warteten wir auf die Rangierung und
    würden enttäuscht. In der hinteren Hälfte der Rangliste war die MGH zu
    finden aber zum Glück noch vor Beromünster.

    Am Sonntag schafften es doch noch diverse Musikantinnen, an der
    Veteranenehrung unseren Kant. Veteranen Armin Schmid und Beat Koller
    teilzunehmen.


    \subsubsection*{Eidg. Musikfest St. Gallen, 18./19. Juni 2011}

    Aufgrund einiger technischer Probleme bei den Organisatoren, musste die MGH
    ihr Können am Sonntag als letzter Verein zum Besten geben. Beim Konzert
    kassierten wir 161.67 Punkte von 200 ab und bei der Marschmusik waren es 82
    von 100. Da das Fest bereits um 20.00 Uhr zu Ende war, musste im Car noch
    weitergefeiert werden.

    Fazit: Minus ein Bildschirm und Suche eines neuen Carunternehmens.


    \subsubsection*{Musiktag Aesch, 9. Juni 2012}

    Am frühen Morgen besammelten sich die Musikanten/innen und Ehrendamen für
    die Fahrt nach Aesch. Nach einem Zmorgen mit Zopf und Kaffee ging es weiter
    in das Instrumentendepot, wo wir unsere Bagage deponierten. Anschliessend
    gab es einen kurzen Marsch von 10min ins Einspiellokal.

    Der Auftritt war aus unserer Sicht zufriedenstellend. Bei einer Stelle war
    jedoch aus irgendeinem Grund kein Cornet mehr zu hören. Unser Jurist meinte
    jedoch, wir haben "bon travail" geleistet und sollten noch ein bisschen
    "plus piano" spielen. Von unserem Patzer erwähnte er nichts.

    Danach stand bereits die Marschmusik auf dem Programm. Wunderschön
    gestartet, guter Sound, konzentriertes Marschieren bis -- ja bis wir am
    Schluss in die Höudi-Fan-Kurve gelangten, in welcher fleissig applaudiert
    wurde. Dadurch verstand Philipps "Spiel-Halt" nur noch knapp die erste
    Reihe. Je weiter nach hinten, umso holpriger und verschobener war der Halt.

    Trotzdem reichte es mit 49.4 Punkten auf den 9. Platz von 16 und es wurde
    wie immer bis tief in die Nacht gefeiert, was man gewissen Uniformen noch
    immer entnehmen kann. Irgendein Hobby-Clown sprühte zu später Zeit weissen
    Bauschaum umher, von welchem unsere ich-geh-erst-nach-Hause-wenns-hell-ist
    Musikanten leider noch getroffen wurden.


    \subsubsection*{Musiktag Wauwil, 31. Mai 2014}

    Die Organisatoren des Musiktages in Wauwil konnten anders als ihre
    letztjährigen  Vorgänger aus Hildisrieden von ihrem Wetterglück profitieren.
    Sie stellten ein gelungenes Fest auf die Beine wo alle ihre Erwartungen
    erfüllen konnten. Durch die Kompaktheit des Festgeländes konnten die
    Verschiebungszeiten kurz gehalten werden.

    Die Musikanten und Musikantinnen der Musikgesellschaft Hildisrieden konnten
    schon am Samstagmorgen um 10.15 Uhr ihr Vortragsstück dem Publikum sowie dem
    Juror vortragen. Kurz darauf konnten die schöneren Musikanten weiter
    brillieren, wobei die anderen bei der Fotoaufnahme in die hinteren Reihen
    stehen mussten. Schon bald war dann auch Mittag und allesamt gesellten sich
    am Bankett an den Tisch. Auch dort konnten sich mit einem zügigen Ablauf die
    Organisatoren und Helfer auszeichnen.

    Um 14.15 Uhr spielte die MGH zur Marschmusik auf. Die zufriedenstellende
    Darbietung konnte jedoch nicht alle überzeugen, da doch plötzlich von einer
    gewissen Jurorin nicht genannte Körperteile zum \enquote{Verhängnis} wurden.


    \subsubsection*{Kant. Musikfest Sempach, 6. Juni 2015}

    TODO Erinnerungen sammeln.


    \subsubsection*{Eidg. Musikfest Montreux, 11. Juni 2016}

    TODO Erinnerungen sammeln.


    \subsubsection*{Musiktag Schüpheim, 27. Mai 2017}

    TODO Erinnerungen sammeln.


    \subsubsection*{Musiktag Eschenbach, 2. Juni 2018}

    TODO Erinnerungen sammeln.



    \subsubsection*{Innerschweizer Musikfest, Hergiswil, 16. Juni 2019}

    TODO Erinnerungen sammeln.


    \subsubsection*{Kant. Musikfest Emmen, 19. Juni 2022}

    TODO Erinnerungen sammeln.


    \subsubsection*{Marschpreis.LU, 3. Sept. 2022}

    TODO Erinnerungen sammeln.


\end{history}