\chapter{1950-1974}

\begin{history}

    \subsubsection*{1955 17. Kantonal Musikfest in Malters}

    Direktion: Roman Höhener 3. Klasse, 5. Rang, 89 Punkte

    Frohgemut und in bester Stimmung zogen wir nach Malters. Statt eines
    Empfanges durch schöne Ehrendamen begoss man uns mit einer recht kalten
    Regendusche. Es drohte noch zu schneien. So suchten wir auf kürzestem Wege
    das Probelokal, um uns einzuspielen, Selbstwahlstück: \enquote{Antigone},
    Ouverture von F. Roussaeu

    Aufgabestück: \enquote{Die Piraten von Penzance}, Ouverture von Sullivan

    Marschmusik: Spielbeginn von Crescenzi

    Kampfrichter:

    Stephan Jaeggi, Musikdirektor, Bern

    Goxtlieb Zimmerli, Musikdirektor, Gerlafingen

    Emil Dörle, Musikdirektor, Freiburg i.Br. Wertkampfbericht:

    Selbstwahlstück: Interpretation, Rhythmik und Dynamik konnten in dieser
    Aufführung mit sehr gut taxiert werden, während für die harmonische Reinheit
    und Tonkultur das Prädikat \enquote{gut} eingesetzt werden konnte. Die
    Tonkultur steht der technischen Ausführung nach. Hier muss in diesem Korps,
    welches im Allgemeinen über respektable Techniker verfügt, die systematische
    Ausbildung einsetzen.

    Aufgabestück: In der Ausführung des Aufgabestückes machte das Korps
    durchwegs einen sehr guten Eindruck. Die Dynamik konnte hier sogar mit der
    Note 10 belohnt werden. Harmonisch und dynamisch war alles sehr sauber
    differenziert, so dass die Aufführung sehr gut imponierte. Im Bezug der
    Interpretation muss jedoch dem begabten Dirigenten klargemacht werden, dass
    er sein Temperament nicht zügellos gehen lassen darf. Diese Ouvertüre von
    Sullivan verlangt im Ganzen einen Schuss trockenen englischen Humors,
    welcher nicht im hastigen Tempi erreicht werden kann. Marschmusik: Der recht
    günstige Gesamteindruck könnte bei grösseren dynamischen Kontrasten und
    sorgfältiger musikalischer Wiedergabe erhöht werden.


    \subsubsection*{1957 Eidg. Musikfest in Zürich}

    Direktion: Albert Sommerhalder

    3. Klasse, 1. Rang, Goldlorbeer

    Die Musikgesellschaft Hildisrieden, die unter der tüchtigen Leitung Albert
    Sommerhalders qualitativ sehr gut vorbereitet wurde, erlebte am Eidg.
    Musikfest den absouten Höhepunkt in der bisherigen Vereinsgeschichte. Ihre
    Leistungen wurden im Selbstwahl- sowie im Aufgabestück und bei der
    Marschmusik mit dem Prädikat \enquote{Vorzüglich} beurteilt. Dem Erfolg
    entsprechend fand im Dorfe ein begeisterter Empfang auf dem Dorfplatz statt,

    Kampfbericht von Otto Zurmühle Luzern

    \enquote{Cäsar und Cleopatra}, Ouvertüre von Boedijn

    Das Korps besitzt eine grosse dynamische Spannweite, die um so mehr zur
    Geltung kommt, als das Piano in der Regel gepflegt ausgeführt wird. Im
    Weiteren ist das Bestreben nach rhythmischer Prägnanz und Geschlossenheit
    unverkennbar, was dem Vortrag den Charakter des Sicheren, Selbstbewussten
    verleiht. Zwischen A und B erfreute die gepflegte, ausgeglichene Tongebung.
    Auch das bei C beginnende Allegro moderato fand eine überzeugende
    Ausdeutung, bis im Takte 80 in den Überleitungsvierteln der tiefen Stimmen
    Unsicherheiten auftraten.

    \enquote{Hebes}, Ouvertüre von J. Godard

    Auch die Wiedergabe des Aufgabenstückes bedeutete eine respektable Leistung.
    Kleine Intonationstrübungen störten bei Ziffer 5, wo die Oktavführung der
    Melodie nicht ganz rein war, in den Takten 99 und 100, wo sich die
    Tenorhörner nicht ganz fanden, im Takte II, wo die Melodieführung bei den
    Sopraninstrumenten lag, zu wünschen übrig liess.

    Marschmusik-Bericht von Fridolin Bünter: Musketiermarsch vor. Lüthold,

    Die Musikgesellschaf: Hildisrieden ist mit einer vorbildlichen Arbeit
    aufgetreten. Sie war musikalisch prima vorbereitet und hatte eine
    erfreuliche Marschordnung.

    \subsubsection*{    1962 Kant. Musiktag in Hitzkirch 20. Mai}

    Direktion: Franz Limacher

    Es waren nur ein Selbstwahlstück und ein Marsch vorzutragen. Nach
    sorgfältiger Vorbereitung mit vielen Proben hofften wir auf ein gutes
    Abschneiden. Auszug aus dem Expertenbericht von Musikdirektor Mener. Dieser
    mittelstarke und ausgeglichen besetzte Verein in reiner Blechbesetzung
    erfreute mit einer sehr frischen, den Charakter der Komposition gut
    erfassten Wiedergabe der dankbaren Ouvertüre pastorale von Paul Huber. Eine
    in allen Registern disziplinierte Ausführung der rhythmischen und
    bewegungsmässigen Belange, gute klangliche Ausgeglichenheit und schöne
    dynamische Gestaltung zeugten vom musikalischen Gestaltungswillen ihres
    Dirigenten, der dem Werk zu einer erfreulich geschlossenen Darstellung
    verhalf. Auch die harmonische Reinheit war im Gesamten gesehen recht
    erfreulich gepflegt und zeigte nur vorübergehende Schwankungen, z.B. in den
    etwas getrübten Begleitakkorden des Andante ab T. 10, in den noch zu wenig
    ausgeglichenen harmonischen Folgen der "Takte 96/97, im Unisono T, 108/109,
    oder durch einige falsche Intonationen der tiefen Stimmen in T. 145.

    Im Übrigen freute man sich aber ebenso an vielen Einzelheiten, die eine
    solide bläserische Qualität bezeugen. So der klangschöne und reine
    Unisonoeingang, das sehr lebendig und nur ganz im ersten Moment etwas zu
    hart im Forte interpretierte Allegro, das sehr wirkungsvoll gestaltet war,
    oder das sehr schön empfundene und mit Wärme ausgedrückte Seitenthema ab C,
    D, od. H.

    Marsch: Der österreich. Soldat von Adam Prohaszka Bericht von Walter
    Spieler:

    Die Aufstellung erfolgte zweckmässig, der Marsch wurde sehr günstig gewählt,
    Die Wiedergabe sollte mit etwas mehr Temperament erfolgen. Bei einer
    schwungvollen Wiedergabe und einer richtigen Dynamik liesse sich die
    Darbietung viel vorteilhafter gestalten. Der Gesamteindruck war trotz
    einiger Bemerkungen sehr gut.

    \subsubsection*{1965 7. Schwyzer Kantonal Musikfest in Brunnen}

    Direktion: Franz Limacher

    3. Klasse, 1. Rang, 99 Punkte

    Für die Musikgesellschaft Hildisrieden war es das erste ausserkantonale
    Musikfest. Der Besuch hat sich gelohnt, erreichten wir doch die höchste
    Punktzahl aller konkurrierender Vereine. Aus dem Expertenbericht.

    1. Selbstwahlstück: \enquote{Ouvertüre pastorale} von Paul Huber

    Der Verein überraschte in jeder Hinsicht mit einem ausgezeichneten Vortrag
    der sehr guten Ouvertüre von Paul Huber. Man freute sich an einer makellosen
    Stimmung und einer fein ausgewogenen Dynamik: Auch der Faktor Rhythmus muss
    trotz gelegentlich zaghaften Begleitfiguren als hervorragend bezeichnet
    werden. Aus den durchwegs ausgeglichenen Klangbildern sowohl in weichen
    Pianopartien wie in der strahlenden ff-Unisono sind ganz besonders noch die
    glänzenden Bässe und die ausgezeichneten Trompetensolisten zu erwähnen. Da
    das ganze Werk sehr klar dirigiert und musikalisch interpretiert wurde, sind
    nur zwei Kleinigkeiten erwähnenswert: nicht ganz präziser Einsatz des
    Schlagzeuges in den Takten 48-49, etwas zaghafte, doch schön klingende
    Begleitfiguren im Andate von 110-114, sonst wurde in jeder Beziehung
    vorbildlich musiziert und der Verein ist zu seiner Leistung herzlich zu
    beglückwünschen.

    2. Aufgabenstück: Notline von F. Boisson

    Das Aufgabestück war für diesen sehr guten Verein natürlich eine leichte
    Sache und wurde ebenfalls einwandfrei durchgearbeitet und gut interpretiert.
    Nach gut klingender Unisons-Einleitung und fein empfundenen \enquote{dim}
    bei Takt 3-4 wurde sehr schön gesungen, die Viertelbegleitung dürfte noch
    etwas besser abgesetzt werden, weil in der Melodie der 2, Schlag immer
    angebunden ist, damit der rhythmische Ablauf noch klarer wird. Für die
    übrigen Faktoren gilt das im Wettstück erwähnte Lob: In allen Teilen eine
    vorbildliche Arbeit,

    Wir hoffen gerne, der wohlverdiente Goldlorbeer möge Ansporn sein für
    weitere Taten.


    \subsubsection*{Luzerner Kantonal Musitag in Littau, 16. Juni 1968}

    Direktion: Franz Limacher

    \enquote{Rhapsody on Negro Spiritual} von E. Boll

    Ein Expertenbericht legt leider nicht vor. Aus dem Protokoll ist nur zu
    entnehmen, dass der sehr geschätzte Direktor Franz Limacher, sowie jedes
    einzelne Vereinsmitglied, zum guten Gelingen beigetragen haben. Die gute
    Kritik in der Berichterstattung zu unserem Vortrag kann als ganzer Erfolg
    bewertet werden.

    Marschmusik: Sebastiansmarsch

    In militärischer Haltung stellen wir uns der Marschmusikexpertise. Schneidig
    und stramm ziehen wir am aufmerksamen Publikum vorbei. Leider beinträchtigt
    ein starker Regen die Marschmusikdemonstration. Glanz und Effekt der
    Instrumente und Uniformen leiden darunter.

    \subsubsection*{Bezirksmusikfest in Aidlingen, 18./19. Juli 1970}

    Ein Musikfest ausserhalb unseres Landes zu besuchen, war schon lange der
    Wunsch vieler Musikanten. Es gibt Gelegenheit, das Nützliche mit dem
    Angenehmen zu verbinden, aber auch andere Sitten und Bräuche kennen zu
    lernen. Uns mit deutschen Musikvereinen messen, um so den musikalischen
    Stand vergleichen zu können, war für uns sehr interessant. Wir waren
    bestrebt, unsere Heimat würdig zu vertreten und gingen gut vorbereitet an
    das Fest.

    Die 47 Personen umfassende Reisegesellschaft wurde in Aidlingen herzlich
    empfangen. Der Musikverein Eintracht begrüsste uns mit schneidigen Märschen
    und Bürgermeister Häge richtete ein herzliches Willkomm an die
    Schweizergäste.

    Den gemütlichen Abend im Festzelt eröffnete die Musikgesellschaft
    Hildisrieden. Unsere Darbietungen ernteten grossen Beifall. Anschliessend
    spielten die Aidlinger Musikanten zum Tanze auf und gaben humoristische
    Einlagen

    Sonntag, den 19. Juli, Wettspielkonzert Wettkampfstück: Intrada festiva, von
    S. Jäggi Stundenchor: Konzert für Blasorchester v. H. Haase Klasse:
    Oberstufe

    Note: sehr gut

    Auszeichnung: 1. Rang, 114 Punkte

    Sonntagnachmittag 13.00 Uhr, grosser Festzug der

    28 beteiligten Musikvereine zum Rathausplatz. Massenchor aller am Festzug
    beteiligten Kapellen, anschliessend Nachmittagskonzert im Festzelt. Zum
    Abschied überreichte uns Bürgermeister Häge zu Handen unseres
    Gemeindeammanns eine Medaille, als Zeichen freundschaftlicher Verbundenheit.
    Helmut Schlorz schenkte uns ein Aquarell von Aidlingen, das heute unser
    Probelokal ziert. Musikpräsident Alois Gassmann dankte für die schönen
    Geschenke und für die überaus herzliche Aufnahme. Er gab der Hoffnung
    Ausdruck, dass die freundschaftlichen Bande erhalten bleiben und das nächste
    Zusammentreffen in der Schweiz stattfinden werde. Ein denkwürdiges Fest, an
    das wir uns noch lange und gerne erinnern werden.


    \subsubsection*{Kantonalmusiktag in Reiden, 16. Juni 1974}

    Mit Privatauto begeben wir uns nach Reiden. Am Vormittag treten wir in der
    Pfarrkirche zur Aufführung an. Am Nachmittag bestreiten wir bei heissem
    Wetter die Marschmusikkonkurrenz.


\end{history}