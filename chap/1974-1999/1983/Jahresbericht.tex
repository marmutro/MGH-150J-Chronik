\begin{multicols}{2}

    % \subsection{Jahresbericht}

    \begin{itemize}

        \item[]2. Jan.\\
        Nach einem Glas Wein im Pavillon beim Schulhaus gehts mit Marschmusik
        zum Löwen, wo am Zunftbot der neue Zunftmeister Albert Wyss-Käppeli
        Baumeister inthronisiert wird. Wir spielen ihm 2 Märsche.

        \item[]8. und 15. Jan.\\
        Wir halten unser Winterkonzert mit einigen Neuerungen: Erhöhte Bühne für
        Schlagzeug und Bässe, einheitliche Kartons für die Notenständer und
        einem Podium für den Dirigenten.

        \item[]22. Jan.\\
        Konzert im Shopping Center Emmen.

        \item[]6. Feb.\\
        Wir verschönern das Fest der Installation von Pfarrer Paulo Brenni.

        \item[]9. Feb.\\
        Schulbesuch von Zunftmeister Albert Wyss, den wir musikalisch umrahmen.

        \item[]10. April\\
        Bei angenehmem Frühlingswetter helfen wir im üblichen Rahmen den weissen
        Sonntag zu verschönern.

        \item[]19. April\\
        Wir beschliessen den Kauf von 2 Euphonium Besson mit 4 Ventilen.

        \item[]8. Mai\\
        Muttertagsständchen nach der Maiandacht.

        \item[]14. Mai\\
        Konzert in der Kirche, wir spielen Werke von Händel, Dvorak und Hayden.
        Die Besucherzahl war leider eher mager.

        \item[]2. Juni\\
        Wegen schlechtem Wetter wird der Fronleichnamgottesdienst in der Kirche
        durchgeführt.

        \item[]14. Juni\\
        Ständchen im Alterheim Beromünster.

        \item[]28. Juni - 15. Juli\\
        Quartierständchen im Sandgütsch, Lenzenweid, Birkenweg,
        Sempacherstrasse, Hochdorferstrasse und Waldmatte.

        \item[]1. Aug.\\
        Nach langer Hitzeperiode kommt am 1. August der ersehnte Regen. Die
        Feier auf dem Schulhausplatz wird trotzdem durchgeführt.

        \item[]24. und 25. Sept.\\
        Am Herbstfest hat es leider weniger Eintritte als letztes Jahr.

        \item[]5. Nov.\\
        Wir unterhalten die Delegierten des Regionalverbandes der
        Raiffeisenkasse Ob-Nidwalden und Luzern im Löwen während dem
        Mittagessen.

        \item[]29. Nov.\\
        Wir wählen Fahnengotte Lisbeth Lang und Fahnengötti Fritz Amrein.

    \end{itemize}

\end{multicols}
