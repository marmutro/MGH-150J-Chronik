\begin{multicols}{2}

    % \subsection{Jahresbericht}

    \begin{itemize}

        \item[]2. Jan.\\
        Nach dem Ständchen auf dem Löwenplatz begleiten wir den abtretenden
        Zunftmeister Martin Estermann durchs Dorf bis Zunftbot im Rest. Kreuz.
        Mit grossem Applaus wird Josef Zwinggi-Amstutz als neuer Zunftmeister
        inthronisiert.

        \item[]7. und 14. Jan.\\
        Jahreskonzert

        \item[]12. Jan.\\
        Traditionelles Nachtessen des Vereins im Löwen. Einige meinten, sie
        haben zu wenig Fleisch und gaben sich mit dem Knochen oder mit dem
        Knochenmark zufrieden was schlussendlich auch gesund ist.

        \item[]26. Jan.\\
        Zunftmeisterpaar Zwinggi-Amstutz macht seinen Schulbesuch. Wir
        verschönern die Veranstaltung mit unseren Klängen. Nach der
        Orangenschlacht sind wir zu einem Kaffee in den Löwen eingeladen.

        \item[]2. April\\
        Wir begleiten die Kommunionskinder bei der Weiss-Sonntag Prozession zur
        Kirche. Nach dem Platzkonzert offeriert uns Kreuzwirt Hugo Fleischli ein
        Glas Wein.

        \item[]14. Mai\\
        Pfingsten und Muttertag. Am heutigen Ehrentag der Mutter warten wir
        wieder mit einem Platzkonzert auf dem Dorfplatz auf.

        \item[]18. Mai\\
        Wir lockern die Orientierungsversammlung über die bevorstehende
        Abstimmung über die Uni Luzern mit unseren Darbietungen auf.

        \item[]20. Mai\\
        Musiktag Römerswil. Mit dem PW begeben wir uns nach Römerswil. Wir
        treten mit dem Stück "`Royal Processional"' von John J. Morrissey an.
        Als Experte amtet Eduard Muri, Zürich. Er wertet unseren Vortrag als
        sehr gut, stellt lediglich hie und da einige Wischerchen fest, gibt uns
        aber die Empfehlung diesen Vortrag nicht an einem Fest zu spielen. Als
        Marschmusikexperte amten Fridolin Bünter für das Musikalische und Josef
        Keist kontrolliert die Aufstellung. Er stellt einige Fehler fest, zum
        Beispiel Melden, Instrumentenhaltung und Abmarschieren. Im
        Expertenbericht merkt er an, es wäre schade, dass ein solch flottes
        Korps nicht etwas mehr Schneid und Rasse zeigt.

        \item[]30. Mai\\
        Da Direktor Franz Limacher in Römerswil zum kant. Veteran ernannt wurde,
        wollen wir ihn auch noch etwas feiern. Wir begeben uns nach der Probe
        zum gemeinsamen Nachtessen ins Rest. Schlacht.

        \item[]2. Juli\\
        Um der Einweihung des neu erstellten Tennisplatzes den festlichen Rahmen
        zu geben werden wir zu einem Ständchen aufgeboten.

        \item[]15. Juli\\
        Nach einigen Verschiebungen wegen Wetter und Fussball WM können wir bei
        eher kühlem Sommerwetter da bald zur Tradition gewordene Ständchen
        abhalten.

        Neben den Runden Bier und Kaffee vom Wirt Josef Wey und Hans Meier
        wollte auch der Vertreter der Harmonie Sempach "`Miggu"' Birrer eine
        Runde bezahlen, aber am Schluss scheint er es doch nicht mehr zu wissen
        und das bereitete unserem Kassier einige Sorgen.

        \item[]23. Juli\\
        Bei einem strahlenden Sonntag können wir das Waldfest abhalten. Wir
        beginnen wie üblich um 11 Uhr mit Braten vom Spiess. Auch den ganzen
        Nachmittag können wir uns an einer sehr hohen Besucherzahl erfreuen, das
        uns einen Reinerlös von 3184 Fr. ergibt.

        \item[]25. Juli\\
        Ständchen zum 40-jährigen Priesterjubiläum von unserem Pfarrer Josef
        Jost.

        \item[]1. Aug.\\
        Um der Nationalen Aktion den 1. August musikalisch zu umrahmen werden
        wir aufs SChlachtfeld aufgeboten. Nationalrat Valentin Oehen dankt für
        die Zusage und meind es sei für ihn nicht selbstverständlich, seien sie
        doch ein eher kauziger Verein.

        Am Abend nehmen wir wie üblich an der weltlichen Feier auf dem
        Schulhausplatz teil.

        \item[]2. und 3. Sept.\\
        Da am Donnerstag noch Schnee bis 1000m und das Wetter als regnerisch
        angekündigt wurde wollten wir den Ausflug zuerst absagen. Aber am
        Samstagmittag scheint die Sonne hinter den Wolken hervor, als sich die
        Reiselustigen auf dem Schulhausplatz versammelten. Wir fahren mit dem
        bis zum letzten Platz gefüllten Car der Firma Galliker Ballwil nach
        Flumserberge Tannenboden. Nach dem Zimmer- und Massenlagerbezug ist
        Gelegenheit zum Gottesdienstbesuch. Am Abend wird uns im Sporthotel
        Tannenboden ein reichhaltiges Nachtessen serviert. Anschliessend
        geniessen wir den restlichen Abend im Sääli mit Unterhaltung der
        Ronspatzen und an der Bar. Es wird gesungen und Duett geblasen. Ein
        Flumser, der neben an sass und hörte wie unser beste Schütze Kaspar
        Troxler geehrt wurde, machte er ein Spiel mit Kaspar, mit den
        Pfannendeckel. Kaspar muss versuchen, den Schädel des Flumsers zu
        treffen, was ihm aber leider nie gelang.

    \end{itemize}

\end{multicols}

