\begin{history}

    % \subsection{Jahresbericht} 1980

    \begin{itemize}

        \item[]2. Jan\\
        Wir besammeln uns im Schulhaus zum Zunftbot. Mit Marschmusik geht es zum
        Löwen. Wir gratulieren dem neu gewählten Zunftmeister Alfred Bachmann
        Mitteltannen mit 2 Märschen.

        \item[]5. und 12. Jan.\\
        An diesen beiden Samstagen halten wir unser Winterkonzert.

        \item[]13. Feb.\\
        Wir begleiten Zunftmeister Alfred Bachmann zum Schulbesuch mit
        Marschmusik durchs Dorf.

        \item[]1. und 2. April\\
        Wir nehmen nach harter Probearbeit 14 Stücke auf Tonband auf im
        Bureschopf des Rest. Schlacht. Die Aufnahme macht Herr Braun aus Luzern.

        \item[]13. April\\
        Bei strahlendem Frühlingswetter dürfen wir die Kommunionkinder mit einem
        Prozessionsmarsch in die Kirche begleiten. Nach dem Ständchen geht es
        mit einem Marsch bis zum Rest. Kreuz.

        \item[]11. Mai\\
        Muttertagständli nach der Maiandacht.

        \item[]5. Juni\\
        Mit 2 Liedern helfen wir, den Fronleichnamsgottesdienst zu verschönern.

        \item[]11. Juni\\
        Wir müssen Abschied nehmen von unserem Musikkameraden Silvester Troxler,
        der leider schon mit 56 Jahren von uns gegangen ist.

        \item[]15. Juni\\
        Der Hildisrieder Peter Widmer wurde heute kant. Schwingerkönig in
        Flühli. Wir empfangen in mit einem Festzug.

        \item[]6. Juli\\
        Zur 100 Jahr Feier der Feldmusik Neuenkirch spielen wir den Marsch
        \enquote{Schwyzer Soldaten} und im Festzelt \enquote{Our Gallant Infantry} und
        \enquote{Blinkende Sterne}.

        \item[]12. Juli\\
        Schlusshock für alle Helferinnen und Helfer der Chilbi im Schopf bei
        Walter Rüttimann.

        \item[]20. Juli\\
        Wir führen das Waldfest durch, obwohl das Wetter recht zweifelhaft ist.
        Aber Petrus zwang uns, das Fest um 16 Uhr abzubrechen.

        \item[]27. Juli\\
        Heute meint es das Wetter gut mit dem Waldfest und wir erwirtschaften
        3888 Fr.

        \item[]23. und 24. Aug.\\
        Wir fahren mit einem neuen Car von Estermann Beromünster zur Schwägalp
        beim Säntis. Nach dem Zimmerbezug besuchen wir den Abendgottesdienst in
        der Bergkapelle. Ein Ständchen vom Männerchor aus Herisau und unsere
        Ronspatzen erfreute alle Ausflügler. Am Sonntagmorgen ist der Säntis
        nebelfrei, sodass die meisten mit der Schwebebahn hochfahren und die
        Aussicht geniessen. Danach geht es weiter auf den Kronberg, die einen
        wandern, die anderen nehmen den Car und die Seilbahn. Nach einem kurzen
        Aufenthalt in Appenzell geht es zurück nach Hause.

        \item[]7. Sept.\\
        Eine Fahnendelegation nimmt an der 100 Jahr Feier der Harmonie Sempach
        teil.

        \item[]5. Dez.\\
        Als Abschluss nach 20 Jahren als Direktor möchte Franz Limacher das
        eidgenössische Musikfest 1981 in Lausanne besuchen. Wir stimmen zu
        $\sfrac{2}{3}$ zu.

        \item[]13. Dez.\\
        Konzert im Shopping Center Emmen.


    \end{itemize}

\end{history}
