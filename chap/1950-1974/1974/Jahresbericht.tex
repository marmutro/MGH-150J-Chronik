\subsection*{1974}

\begin{history}


    \begin{itemize}
        \item 3. Jan.\\
              Bei der Probe macht der Präsident den Vorschlag des Vorstandes, beim
              Winterkonzert dem Verein aus finanziellen Gründen kein Nachtessen mehr
              zu bezahlen, dafür beim Konzert in Emmen ein Bier zu offerieren. Ohne
              Diskussion wird der Vorschlag angenommen.

        \item 5. und 12. Jan.\\
              An diesen Abenden bringen wir das Jubiläumskonzert zu Gehör. Wir können
              uns an einem vollbesetzten Saal erfreuen. Eintritte 600 Personen. Seit
              mehreren Jahren konnten wir nicht mehr so viele Zuhörer begrüssen.

        \item 10. Jan.\\
              Auch dieses Jahr pilgern wir nach Emmen um das Winterkonzert auch der
              Emmer Bevölkerung zu Gehör zu bringen. Der Reinerlös soll dieses Jahr an
              das seraphische Liebeswerk weitergeleitet werden. Aber es wird sich
              bloss um ein Trinkgeld handeln, denn die Zuhörerzahl beschränkt sich auf
              ca. 60 Stück. Nachher inkludieren wir im Restaurant Sternen noch ein
              Bier.

        \item 18. Jan.\\
              Zunftbot im Rest. Kreuz. Um $\sfrac{1}{4}$ vor 8 Uhr besammeln wir uns
              beim letztjährigen Zunftmeister Josef Estermann Bauernhof um noch unter
              seiner Macht den letzten Tropfen zu geniessen. Aber es soll noch nicht
              der letzte sein an diesem Abend, denn es regnet während des Einzuges ins
              Kreuz in Strömen.

              Der zum letzten mal amtierende Zunftpräsident Albin Estermann gibt die
              Nomination von Alois Gassmann Sandgütsch als neuer Zunftmeister bekannt.
              Mit grossem Applaus wird er in sein Amt eingesetzt.

        \item 1. Feb.\\
              100. Generalversammlung im Rest. Kreuz

        \item 2. Feb.\\
              Die Firma Schuler aus Rothenturm hat unsere Uniform im Rohbau soweit
              fertig, dass sie heute jedem Musikanten angepasst werden kann.

        \item 15. Feb.\\
              Der Zunftmeister Alois Gassmann besucht die Schüler. Um 15 Uhr besammeln
              wir uns beim neuen Schulhaus. Dann gehts mit Marschmusik durchs Dorf zum
              alten Schulhausplatz. Dort bringen wir einige Stücke zu Gehör und der
              Zunftmeister lädt uns zu einem Imbiss im Kreuz ein.

        \item 21. Feb.\\
              Da es schmutziger Donnerstag ist führen wir am Abend den traditionellen
              Musikhock im Kreuz durch. Es geht bald so richtig fastnächtlich zu. Es
              erscheinen gegen 20 Maskierte, was für den kleinen Raum doch schon viel
              ist. Bei einer zügigen Tanzmusik, die Hugo Fleischlin organisiert hat,
              kommt Jung und Alt in den Genuss des Beinschwingens.

        \item 5. März\\
              Herr Schuler jun. aus Rothenturm orientiert uns über sein
              Krawattensortiment. Mit grossem Mehr wird beschlossen, eine gemusterte
              Krawatte zur neuen Uniform zu kaufen.

        \item 11. März\\
              OK Präsident Julius Bieri lädt den Verein ein zur Verteilung der
              Adressen für die Bettelaktion. Jedem anwesenden Musikanten werden einige
              Posten zugeteilt, sodass dem Kassier von heute an die Säcke bald gefüllt
              werden sollten.

        \item 20. März\\
              Begrüssung der Neuzuzüger. Der Gemeinderat lädt alle 132 Neuzugezogenen
              in den Löwen ein. Auch unser Verein begrüsst sie auf musikalische Weise.

        \item 2. April\\
              Firma Schuler aus Rothenturm liefert die neuen Uniformen an den Verein
              aus. Jeder Musikant kleidet sich ein. An 6 Hosen muss eine kleine
              Abänderung vorgenommen werden. Herr Schuler muntert uns auf zur neuen
              Uniform Sorge tu tragen, wenigstens die ersten paar Jahre.

        \item 4. 5. 7. Mai \textbf{100-Jahr Feier der Musikgesellschaft
                  Hildisrieden}\\
              Das Jubiläumsfest beginnt mit einem festlichen Eröffnungskonzert am
              Samstagabend, an dem die alte, stahlblaue Uniform mit rassigen Klängen
              verabschiedet wird. Der befreundete Musikverein Aidlingen aus
              Deutschland gibt ein Galakonzert in der zum Bersten gefüllten Festhalle
              und bringt auch  das Bierzelt mit Stimmungsmusik auf jubilierende
              Hochstimmung.

              Mit der Totenehrung auf dem Friedhof beginnt das sonntägliche Programm,
              und der Marsch \enquote{alte Kameraden} erklang in den Sonntagmorgen.
              Der feierliche Gottesdienst gibt dem Fest einen würdigen Anfang.

              Nachher trifft man sich zum gemütlichen Frühschoppenkonzert, vorgetragen
              von Musikverein Aidlingen. Die Vorträge der Stadtmusik Laufen umrahmen
              das Bankett. Der farbenfrohe Einzug der geladenen Musikvereine aus der
              Umgebung sowie die alte und neue Uniform bildeten den Höhepunkt des
              Festes. Die Veteranen wurden durch den Vereinspräsidenten Fritz Disler
              mit Nelken und weissem Wein geehrt aus Dankbarkeit für die vielen
              Verdienste dies die dem Verein hergaben.

              Das Fest dehnt sich am Sonntagabend weiter aus. Als Stargast konzertiert
              die Musikgesellschaft Engelberg, welche zum grossartigen Ambiente im
              Festzelt beiträgt und viel Applaus ernteten.

              Am Schlussabend der Geburtstagsparty zeigt sich das musikalische Können
              der Musikgesellschaft Hildisrieden unter der Leitung von Franz Limacher.
              Der unermüdliche Aschi Lehmann versteht es vorzüglich, die Feststimmung
              noch einmal so richtig zu heben.

        \item 10. Juni\\
              Einstimmig wird beschlossen, dass jeder Musikant sein Hemd zur Uniform
              selber bezahlt.

        \item 16. Juni\\
              Kantonalmusiktag in Reiden. Mit Privatauto begeben wir uns nach Reiden.
              Am Vormittag treten wir in der Pfarrkirche zur Aufführung an. Am
              Nachmittag bestreiten wir bei heissem Wetter die Marschmusikkonkurrenz.

        \item 3. Juli\\
              Heute Abend sind all OK Mitglieder, deren Frauen und all Musikanten und
              alle die an der Hundertjahrfeier mitgewirkt haben zu einem Schlusshock
              in die Obstlagerhalle bei Hermann Wolf eingeladen.

        \item 6. Juli\\
              Wir nehmen Teil an der Schlachtfeier. Wir besammeln uns in Sempach beim
              alten Schulhaus. Punkt 9 Uhr geht es los mit Marschmusik durchs
              Städtchen Richtung Schlacht. Nach der Schlachtfeier machen wir noch eine
              Radioaufnahme, dann geht es wieder nach Sempach in die Festhalle zum
              Mittagessen.

        \item 28. Juli\\
              Bei schönstem, sommerlichen Wetter können wir das verschobene Waldfest
              durchführen. Wir beginnen wieder wie letztes Jahr nach 11 Uhr mit Suppe
              und Spatz. Die Nachfrage nach diesen von Metzger Walter Odermatt gut
              vorbereiteten Vögeln ist besser als letztes Jahr. Die Festwirtschaft
              sowie am späten Nachmittag auch die Kaffeebude läuft zur vollen
              Zufriedenheit.

        \item 1. Aug.\\
              Wir nehmen an der 1. August Feier teil. Sie wird wieder von der Zunft
              organisiert. Nach der kirchlichen Feier bewegt sich ein Fackelzug
              Richtung Breite. Nach dem Schweizerpsalm offeriert der Gemeinderat auf
              der Festwiese das \enquote{Nationalgetränk} Bier und Most.

        \item 17.-18. Aug.\\
              Musikausflug auf die grosse Scheidegg. Per Postautobus begeben wir uns
              nach Luzern und mit der Bahn nach Meiringen. Von dort bringt uns ein Bus
              bis zur Schwarzwaldalp. Nach dem Zobig beginnt der ca. 2 Std. Marsch auf
              die grosse Scheidegg, die wir noch vor dem Regen erreichen. Nach dem
              Nachtessen spielen die Ronspatzen. Gut gelaunt wird gesungen, geblasen
              und Kaffee getrunken. Am Sonntagmorgen marschieren wir bis nach
              Grindelwald First. Nach dem Mittagessen geht es mit der Sesselbahn nach
              Grindelwald und dann mit Bus und Bahn nach Luzern, wo wir die Abendmesse
              besuchen.

        \item 17. Sept.\\
              OK Präsident Julius Bieri und OK Kassier Hermann Wolf erläutern im Löwen
              die Abrechnung der Hundertjahrfeier. Durch Sammlung gingen in die Kasse
              75785 Fr. Dieser Betrag stammt von 456 Gönnern. Es wurden 46 Uniformen
              bezahlt, diese kosten 31267 Fr. Als Enderlös vom Fest bleiben 51476Fr.

        \item 3. Okt.\\
              Wir beschliessen nach der Probe den Kauf von 4 neuen Posaunen. Der
              Eintritt am Konzert 1975 wird auf 6 Fr. festgesetzt.

    \end{itemize}

\end{history}
