\begin{multicols}{2}

    \subsection{1950-1974}

    «Immer vorwärts», so heisst die Parole unserer Musikgesellschaft.
    Stillstand heisst Rückgang. 1955 besuchte
    man das Kantonale Musikfest in Malters und 1957
    das Eidgenössische in Zürich. Ebenso vertrat der Verein
    an den Musiktagen von Hitzkirch und Littau, sowie
    am Schwyzer Kantonal-Musikfest in Brunnen unsere
    Gemeinde. Dass Musik keine politischen Grenzen kennt,
    zeigt uns die Einladung nach Aidlingen (Stuttgart) zum
    dortigen Bezirksmusikfest

    Eine besondere Anerkennung möchten wir unseren
    Dirigenten widmen. Ihr Bestreben ist, den musikalischen
    Stand des Vereins zu heben und die Leistungen
    zu verbessern. Mit grossen Fähigkeiten und eisernem
    Willen gelang es ihnen, die Musikgesellschaft Hildisrieden
    musikalisch weiter emporzuführen. In ihren Bestrebungen
    fanden sie stets ungeteilte und geschlossene Unterstützung
    durch Vorstand und Musikanten. Wenn es trotzdem hin und wieder
    kleine Trübungen gab, so
    waren sie sicher nicht allzu ernst gemeint, denn das
    Wohl und die Ehre des Vereins standen immer im
    Vordergrund, Ihre Ratschläge, ihre grosse Aufopferung
    und Treue zur Musikgesellschaft verdient aufrichtigen
    Dank.

    Der schönste Brauch im Vereinsleben ist die Pflege des
    geselligen und kameradschaftlichen Lebens durch
    gegenseitigen Besuch. Da werden die freundschaftlichen
    Bande unter den Musikvereinen enger geknüpft. Die
    Besuche der Stadtmusik Laufen im Jahre 1952, der
    Harmonie Neuenkirch 1961, sowie jener der Aidlinger
    Blaskapelle 1972 freuten uns sehr. So werden alte,
    liebe Freundschaftsbande aufgefrischt und neue angebahnt.
    Möge dieser schöne Brauch immer erhalten
    bleiben.

    Die seit 1953 alljährlich im Hildisriederwald stattfindenden
    Waldfeste stärken die Vereinskasse (denn
    dieser Anlass wird nicht des Festes wegen organisiert).
    Dieser Anlass verursacht den Musikanten viel Mühe
    und Arbeit, ist aber bei der Bevölkerung sehr beliebt,
    wie der gute Besuch immer beweist. Die heissen
    Sommertage locken die Musikfreunde in den kühlen
    Wald, um bei Musik und Tanz einige gemütliche
    Stunden zu geniessen.

    Der 3. Juli 1954 war ein ganz besonderer Tag für die
    Musikgesellschaft, Unter dem Motto «Zoge am Boge»
    konzertierten wir in einer volkstümlichen Sendung im
    Radiostudio Basel, nebst dem Jodelklub Pilatus, Luzern,
    und dem Handharmonika-Duett Lustenberger, Emmenbrücke.
    Der schöne Erfolg gab dem Verein den nötigen
    Impuls für die Weiterreise nach Laufen zur dortigen
    Stadtmusik, um an der Jubiläumsfeier teilzunehmen.
    Die Feierlichkeiten eröffneten die Musikgesellschaft
    Konkordia Allschwil, der Männerchor Laufen und
    schliesslich die Hildisrieder Musikanten. Die musikalischen Darbietungen,
    sowie die humoristischen Einlagen von Mathias Jutz als Conferencier ernteten
    rauschenden Beifall. Im September gleichen Jahres
    durften wir die Grenzbesetzungssoldaten der Mitr Kp. 4
    mit einem Konzert erfreuen

    Trotz Schneegestöber am Auffahrtsmorgen vom
    19. Mai 1955, wurde dann der Abend zu einer sehr
    eindrucksvollen Feier. Der Empfang zu Ehren des neu
    gewählten Regierungsrates Dr. Werner Bühlmann
    war ein einmaliges Erlebnis

    Die Musikgesellschaft ist 1955 auf 31 Mitglieder angewachsen, und der Vorstand setzte sich wie folgt
    zusammen:

    Präsident: Hans Troxler, Ausserbuchen
    Aktuar: Kaspar Troxler, Moos
    Kassier: Josef Rüttimann, Dorf
    Mat.-Verwalter: Otto Estermann, Traselingen
    Beisitzer: Hans Suppiger, Dorf

    1956 wirkte unser Verein an der Luzerner-Fasnacht
    mit. Die Musikanten waren als Zahnärzte verkleidet.
    Der zahlenmässig erstarkte Verein stand 1957 vor der
    Wahl einer neuen Uniform. Mit tatkräftiger Unterstützung seitens der vielen
    Gönner konnte die Neuuniformierung beschlossen werden.

    Mit dem neuen Kleid zogen unsere Musikanten
    erstmals an das eidgenössische Musikfest in
    Zürich. Neben vielen Proben und Auftritten gehört
    hin und wieder eine verdiente, gemütliche Reise ins
    Programm, um die Kameradschaft zu beleben. Wir
    erinnern an die Ausflüge an den Rheinfall 1950, ins
    Bündnerland 1952, ins Tessin 1956, auf die Rigi 1958,
    ins Urnerland 1960, ins Lötschental 1962, auf die
    Schwägalp 1964, in das Engadin 1967, die Drei-See-Rundfahrt 1969 und die Reise auf die Rieder- und
    Bettmeralp 1972.

    Da viele Musikanten zu den besten Schützen unserer
    Gemeinde zählen, ist es für die Musikgesellschaft stets
    eine Freude, diese bei der Heimkehr von grossen
    Schützenfesten feierlich zu empfangen. Am Freundschaftstreffen 1961 mit der Harmonie Neuenkirch
    in Neuenkirch übergab Präsident Silvester Troxler
    seinen Musikfreunden die Ehrenurkunde.

    Am 24. Februar 1962 konzertierten unsere Musikanten
    zur Eröffnung des neuerbauten Gasthof zum Roten
    Löwen und ebenso im November zum Antrinket im
    renovierten Restaurant Kreuz.

    Der 29. November 1962 war ein denkwürdiger Tag für
    unser Dorf, Regierungsrat Franz Xaver Leu übergab
    die neue Dorfstrasse dem Verkehr. Dem Dorfausbau
    mussten neun Häuser weichen. Der "`Vierwaldstätterhof"', als`grösstes Verkehrshindernis, war lange Zeit
    ein dankbares Fasnacht-Sujet, das allerdings schon
    Jahre vorher vom unvergesslichen Mathis Jutz mehrmals
    entrümpelt wurde.

    Die Generalversammlung vom 12. März 1966 beschloss
    die Neuanschaffung von Instrumenten. Man kaufte
    7 Flügelhörner & Fr. 580.—, 2 Euphonium zu Franken
    1350.—, 1 Es-Bass & Fr. 2200,— der Marke Wilson
    von Kurath, Flums, sowie 5 Courtois-Trompeten A
    Fr. 530.—.

    Im Sommer 1970 erlebten die Hildisrieder einen Pfarr-
    auftritt, Dekan Josef Jost von Beromünster, vorher
    Pfarrer in Hochdorf wurde Pfarrherr von Hildisrieden.
    An der Delegiertenversammlung des Kantonalen
    Musikverbandes 1972 in Kriens wurde der Musiktag
    Hildisrieden übertragen.

    Kann man sich ein Fest denken ohne Musik? Nach
    meiner Auffassung sicher nicht. Wo immer die Klänge
    einer Musik ertönen, herrscht froher Sinn und Gemütlichkeit.
    Wohin auch die Musikgesellschaft Hildisrieden gerufen wird, ist sie bereit, mitzumachen.
    Mit besonderer Freude folgt sie jeweils den Einladungen
    von Nachbarsektionen zu einer Banner- oder Uniformweihe.

    Die Stadtmusik Laufen, mit der sie freundschaftliche
    Beziehungen pflegt, rief sie zu einem Gala-Konzert.
    Ein Kurplatzkonzert in Luzern zu geben und mit
    ihren Darbietungen Gäste aus aller Welt zu erfreuen,
    zählt zu ihren dankbaren Ereignissen. Die Hildisrieder-Fasnacht beleben und der Götschizunft eine
    Freude zu bereiten beweist, dass sie auch Verständnis
    für echtes Brauchtum hat. Auch andere Feste verschönern, gehört zu ihrer Aufgabe. Mit den Vereinen
    von Hildisrieden begeht sie nach vollendetem Tagewerk
    den 1. August.

    So oft der hochwürdige Bischof zur Firmung unsere
    Gemeinde besuchte, oder ein Pfarrer oder Primiziant
    Einzug hielt, fehlte auch das zu ihren Ehren gegebene
    Ständchen auf dem Dorfplatz nicht.

    Ein schöner Brauch, den der Verein pflegt, ist, den
    neuvermählten Kameraden ein Ständchen zu bringen.
    Die gleiche Aufmerksamkeit wird auch den verehrten
    Jubilaren anlässlich des Geburtstages oder der Goldenen
    Hochzeit zuteil. Oft schon stand die Musikgesellschaft
    am Grabe eines lieben Aktiv- oder Ehrenmitgliedes.
    Wehmutsvoll senkte sich jeweils das Vereinsbanner
    unter den Klängen des letzten Musikgrusses in die kühle
    Gruft. Auch die grösseren und kleineren Ausmärsche
    in die nähere Umgebung, die speziell unseren Gönnern
    und Freunden gewidmet sind, und die sich zur Ausbildung in der Marschmusik eignen, seien hier erwähnt.
    Wir schön, dass die Musikgesellschaft nach dem Motto
    handelt: «In Freud und Leid, zum Spiel bereit.»


\end{multicols}
