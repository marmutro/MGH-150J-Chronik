\subsection{2008}

\begin{history}


    \begin{itemize}

        \item 2. Jan. Zunftbot\\
              Nach dem Unterhaltungsprogramm zur Wahl des neuen närrischen Oberhaupts
              stand es dann fest. Hans Luterbach wurde zum neuen Meister der
              Götschizunft erkoren. Und zur Feier des Tages spielten wir dem
              Zunftmeister ein nettes Ständchen.

        \item 12./13.01. Jahreskonzert\\

              Mit dem Motto Swingin' Groove spielten wir auf der Inpulsbühne zum
              diesjährigen Jahreskonzert auf. Die Musikkommission hatte wieder einmal
              ein anspruchsvolles aber auch interessantes Programm zusammengestellt.
              Eines der Höhepunkte war sicherlich die musikalische Vorlesung von
              \enquote{Forza del Destino}. Ein weiteres Highlight wie jedes war, war
              die gute alte Musigbaar... Und nach einigen Stunden, oder vielleicht
              auch nach ein wenig mehr gingen auch die letzten zu Bett.

        \item Skitag\\
              Nach dem Jahreskonzert mussten wir uns sputen, das winterliche Weiss
              noch nutzen zu können. Also machten wir uns mit grossem Bus ganz schön
              früh auf richtung ... Bei richtig guten Verhältnissen und schönem Wetter
              genossen wir ein par Schwünge auf dem Schnee. Und nach dem Mittag
              mussten wir ach schon bald wieder ans Après-Ski denken. Nach einem
              gelungenen Ausgang, schlummerten wir dann wieder nach Hause.

        \item Jubilarenkonzerte\\
              Und wie jedes Jahr begeisterten auch dieses Jahr wieder rund ...
              Senioren und Seniorinnen mit einem Ständchen zu deren Geburtstagen.

        \item Weisser Sonntag / Muttertag / Firmung\\
              Neben den Jubilarenständchen gehören die Ständchen am Weissen Sonntag,
              am Muttertag und an der Firmung zu den wichtigen Kulturarbeiten der MGH.
              An den jeweiligen Anlässen verpassten wir die Szenen mit unserer
              musikalischen Unterstützung in einen passenden Rahmen.

        \item Musiktag Eschholzmatt\\
              Mann, mann, mann, wieder einmal lag es an den Musikanten viel Alkohol zu
              vernichten. Vorher gaben wir allerdings noch einmal unser aller Bestes.
              Und das reichte zumindest um unsern Juror zu überzeugen. Einzig die
              Leute, die spiele mit Daampfff, müssen haben ein wenig meeehr
              exposition! Naja... zum gemütlichen Abschluss fanden sich die Musikanten
              dann vor einem Bier in den diversen kleinen Bars auf dem Festgelende.

        \item Abschlusshöck\\
              Der Abschlusshöck begann diesmal gleich von Beginn an kulinarisch. Mit
              Bachmann-Tonis berüchtigtem Speck und Käse und Brot und... Wein. Danach
              gings auf einen Fussmarsch. Und zwar richtig Schützenhaus. Dort
              vergnügten wir uns mit feinem Fleisch vom Grill und feinen hausgemachten
              Salaten. Und zum Finale: Marinas sagenhaftes Dessertbuffet. Ein Traum
              für jeden Süssigkeitenliebhaber. Aber leider: „Treumli send jo geng so
              schnäu verbiil.." Für Unterhaltung sorgte ein Fussballmatch der
              Europameisterschaften in der Schweiz und in Österreich



        \item Herbstkonzert\\
              Eine Neukreation der MGH! Und, wie sich zeigen sollte,
              erfolgsverdächtig. Unter offenem Himmel und bei bestem Wetter boten wir
              beste Unterhaltungsmusik. Das begeisterte Publikum konnte sich nach dem
              Konzert auch noch verköstigen. Die Musikanten servierten Köstlichkeiten
              vom Grill mit Salaten und Dessertbuffet.

        \item Lotto\\
              Und dann, kurz vor Jahresende hiess es für Ürsu wieder einmal: Ran an
              die Organisation des Lottos... Unser Sorgenkind war zwar leicht besser
              als letztes Jahr doch das kann es nicht sein. Vieleicht ist ein neues
              Datum die Lösung?


    \end{itemize}

\end{history}
