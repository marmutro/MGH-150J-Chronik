\subsection*{2023}
\begin{history}


      \begin{itemize}

            \item 2. Januar 2023 - Probetag sowie Zunftbot\\
                  Wie alle Jahre hatte die MGH am 02. Januar den Probetag für
                  das Jahreskonzert durchgeführt. Anschliessend fand der
                  legendäre Zunfbot im Roten Löwen statt. Auch dieses Jahr
                  begleitete die MGH die Götschizunft vom Schulhaus zum Roten
                  Löwen um der neue Zunfmeister zu erkören. De Oli Rüttimann ist
                  mit grossem Applaus zum Zunftmeister 2023 gewählt worden.
                  Unter dem Motto e guet, e Schnauz mer gfaut führt Oli die
                  Fasnacht in Hildisrieden an. Ab dem 02. Januar ist unser Franz
                  nicht mehr alleine der Schnauzträger. Gefährlich für Franz,
                  ein Traum für Hildegard. Später haben wir noch bei Oli
                  gefeiert und einige Schnauzbierli genossen.

            \item 13. und 14. Januar 2023 Jahreskonzerte\\
                  Am Wochenende vom 13. / 14. Januar 2023 haben die
                  Jahreskonzerte unter dem Motto \enquote{Himmlische Momente}
                  stattgefunden. An beiden Abenden konnten wir die Halle mit
                  Besucher/innen wiederum füllen und boten somit sehr
                  unterhaltsame Jahreskonzerte.

            \item  Geburt von Alina Disler\\
                  Am 3. März 2023 hat Alina, die Tochter von Raphaela und Bitu,
                  das Licht der Welt erblickt. Eine Delegation der MGH hat den
                  weiten Weg ins Entlebuch nach Schüpfheim gemacht, um mit einem
                  schönen Schild zu gratulieren. An dieser Stelle nochmals ein
                  herzliches Dankeschön von meiner Seite.

            \item Winteranlass\\
                  Am 4. März konnten wir unseren Winteranlass durchführen.
                  Organisiert wurde dieser vom Euphonium/Bariton-Register. Ein
                  Fussmarsch in Richtung Möister, dazwischen ein Aperitif und
                  dann ein feines Fondue in der Sennerei führten zu einem
                  gelungenen, gemütlichen Anlass. Besten Dank für die
                  Organisation, und schon jetzt sind wir sehr gespannt auf den
                  nächsten Winter-, Verzeihung, Frühlingsanlass der Solo
                  Cornets.

            \item  Weisser Sonntag\\
                  Wie jedes Jahr haben wir die Erstkommunikanten mit einer
                  Prozession durch das Dorf in die Kirche begleitet. Nach der
                  Kirche haben wir ein kurzes Ständchen zum Besten gegeben und
                  so den Anlass musikalisch umrahmt. Ein herrliches Bild von
                  unserem Öli Stöpsu, das zeigt: Wir sind erstens bereit und
                  zweitens völlig sicher.

            \item 850 Jahre Hildisrieden\\
                  Vom 28. bis 30. April konnte Hildisrieden seinen 850.
                  Geburtstag feiern. Auch eine Delegation der MGH war vertreten
                  und hat das Fest musikalisch mitgestaltet.

            \item  Chilbi\\
                  Am Sonntag, den 30. April, konnten wir bei trockenem Wetter
                  und vielen Besuchern unsere Chilbi feiern. Die MGH war wieder
                  mit einem Stand vor Ort, und die Besucher hatten ihren Spass
                  am Blasrohr-Schiessen. Auch unser Festkassierer Jönu hatte
                  seinen Spass, aber darüber erzählt er sicher selbst später
                  gerne.

            \item Veteranenehrung\\
                  Am Freitag, den 2. Juni, fand die Veteranenehrung in Ruswil
                  statt. Auch hier wurden wir bei herrlichem Wetter nach Ruswil
                  gebracht und feierten mit unseren Veteranen Beat, Maxx und
                  Toni mit. Ein Highlight war natürlich unser Toni, den wir für
                  60 Jahre Musizieren und somit als CISM Veteran feiern durften.
                  In der Zwischenzeit haben unsere Backrow-Kollegen bei Toni
                  zuhause einen Baum aufgestellt und später ein gut sichtbares
                  Plakat dazugestellt. Bis morgen werden Toni zu Hause noch
                  einige MGH-Mitglieder/Veteranen empfangen haben und dafür
                  gesorgt haben, dass niemand dursten musste.

            \item Fronleichnam\\
                  Am 8. Juni haben wir wie jedes Jahr eine Prozession durch das
                  Dorf bis zur Schule durchgeführt, gefolgt von einem Ständchen,
                  das wir zu unserem Besten gegeben haben. Das Risotto und den
                  Nachmittag konnten wir dieses Jahr in vollen Zügen geniessen,
                  da wir im Anschluss keine Hauptprobe hatten.

            \item  Abschlussgrillen\\
                  Am 24. Juni hatten wir unser Abschlussgrillen. Diesmal etwas
                  kleiner, aber dennoch in sehr gemütlichem Rahmen.

            \item  Neue Uniformierung in Rain\\
                  Am 26. August konnten unsere Nachbarn, das Blasorchester Rain,
                  ihre Neueuniformierung feiern. Das Wetter war nicht so gut,
                  daher fand der Anlass im Zelt statt. Wir konnten uns von
                  unserer besten Seite zeigen und zwei sehr gut gespielte
                  Märsche präsentieren.

            \item 2. und 3. September MGH Vereinsreise\\
                  Diese legendären Reise wurde organisiert von Silvan und Max.

                  Am Morgen fuhren wir los mit dem Car und stärkten uns bei der
                  Raststätte Kempthal noch für den weiteren Tag, dann fuhren wir
                  weiter nach Pfullendorf.

                  Nach dem Einchecken nahmen wir am Brassfestival Musikprobe
                  teil, bei dem wir grossartige Brass-Musik geniessen und das
                  eine oder andere Bierchen nehmen konnten.

                  Am nächsten Morgen fuhren wir mit etwas kleinen Augen mit dem
                  Schiff nach Konstanz. Spätestens da war es von Vorteil, dass
                  der Magen bei allen wieder in Ordnung war. Ein paar Stunden
                  konnten wir dann in Konstanz verweilen, bevor wir uns bereits
                  wieder auf den Rückweg machten.

                  Nochmals ein herzliches Dankeschön an die zwei Organisatoren
                  Silvan und Max. Es war ein absolut hammermässig gut
                  organisierter Ausflug. Gefeiert haben die zwei das bereits am
                  4. Januar ausgiebig im Löie und in der LTH Bar unten. Details
                  kenne ich nicht, aber am besten geht ihr direkt zu den
                  Organisatoren und fragt nach.

            \item  Lotto\\
                  Am 20. und 21. Oktober konnten wir im Löie unser Lotto wie
                  gewohnt durchführen. Zweimal konnten wir den Löie wieder mit
                  vielen eifrigen Lotto-Besuchern füllen, was uns natürlich sehr
                  gefreut hat. Der Vize Beni sowie die zwei Einkäufer Kili und
                  Silvan haben dabei am meisten Arbeit eingesetzt und das Lotto
                  wieder sehr gut organisiert.

            \item  Marktleben\\
                  Beim Marktleben am 28. Oktober hat eine MGH Delegation
                  mitgewirkt und alle Marktbesucher musikalisch unterhalten.

            \item  Helfereinsatz in Rain\\
                  Zwischendurch haben wir uns von unserer besten Seite gezeigt
                  beim Helfereinsatz beim Konzert bei unseren Rainer Nachbarn.
                  Jeder dort hat seine Stärken gezeigt.

            \item Jubiläumskonzert\\
                  Am 3. Dezember fand unser Jubiläumskonzert statt. Nach dem
                  Ständchen gab es den gewohnten Aperitif, und die MGH hatte
                  etwas Zeit, mit unseren Jubilaren zu verbringen.

            \item Geburt von Mael\\
                  Am 4. Dezember konnten Kerstin und Oli den Mael zum ersten Mal
                  in die Arme nehmen. Die MGH gratulierte mit einem Bäumchen bei
                  Kerstin und Oli und hiess den Mael herzlich willkommen.
      \end{itemize}

\end{history}
