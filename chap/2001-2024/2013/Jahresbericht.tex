\subsection*{2013}
\begin{history}


    \begin{itemize}

        \item 2. Januar, Zunftbot\\
              Wie jedes Jahr eröffneten wir unser Vereinsjahr mit dem traditionellen
              Zunftbot. Walter Aecherli heisst der Regent der Fasnacht 2013. Mit einem
              Ständli gratulierten wir dem neuen Zunftmeisterpaar. Unter dem Motto
              \enquote{Gmüeser-Party} wurde anschliessend in der Stube des
              Zunftmeisters gemütlich und ruhiger als auch schon auf eine glatte
              Fasnacht angestossen

        \item 6. Januar, Probesonntag\\
              Bevor wir den Probesonntag starteten, drehten wir am frühen Morgen
              unsere Kreiselparade - ein Video als Werbung für den kommenden Musiktag
              2013. An den vielen farblosen Gesichtern und den teils nicht ganz
              geraden Linien beim Marschieren war anzunehmen, dass die
              Lüötertüter-Mitglieder von uns am Vorabend bereits einen Auftritt
              hatten. Unser Bäseriser Musikant kämpfte sich ebenfalls voller Mühe aus
              dem Bett. Er kam zwar zu spät aber immerhin doch noch zur
              Paradenaufnahme - dummerweise ohne seinen MGH-Hut, wodurch ihm das
              mitmarschieren nicht gestattet wurde - blöd. Anschliessend an den
              Videodreh fand der Probesonntag im Inpuls statt.

        \item 12. / 13. Januar, Jahreskonzerte\\
              Dos Jahreskonzert 2013 stand ganz im Zeichen des kantonalen Musiktages
              vom kommenden Juni in Hildisrieden. Unter dem Motto Nachtexpress stellte
              Marina eine unterhaltsame Ansage zusammen. Diverse Musikanten wurden als
              eine prominente Person verkleidet und verfilmt, während Marina im Studio
              die Liederwünsche dieser entgegennahm. Auch der OK Präsident Jakob
              Estermann hatte eine Rolle im Video. Das Jahreskonzert stoss aufgrund
              der amüsanten Ansage und der abwechslungsreichen Stückwahl auf extrem
              grosses Lob und viele begeisterte Reaktionen. Vor dem Konzert am
              Samstag, fand wie jedes Jahr die legendäre Bahnhofguuggete in Luzern
              statt, weiche natürlich die Fasnächtler unter uns sehr reizte, jedoch
              aufgrund des Konzertes nicht besucht wurde. Nicht, dass man am Konzert
              noch einen dummen Kopf hat - oder wie war das, 'ja es sitzen ja noch
              zwei links von mir, die es eh besser können’, Pitsch?

        \item 3. März, Jubilaren-Apérokonzert\\
              Auf musikalischem Weg gratulierten wir mit einem Ständli den
              eingeladenen Jubilaren. Unser Pfarrer Josef Hauser wurde 80ig Jahre alt
              und war daher ebenfalls eingeladen. Er kam etwas verspätet zu uns, da er
              in der Kirche schon einen Geburi-Apéro hatte. Am Ende des Ständlis
              verdankte er die schöne Musik und meinte, er habe absichtlich schneller
              gepredig, jedoch: \enquote{hed de ander eifach nömm welle höre Orgele}.

        \item 15. März, Generalversammlung\\
              Die GV fand im Restaurant roter Löwen in Hildisrieden statt. Es verlief
              alles reibungslos und kurzweilig, sodass schon früh das köstliche
              Abendessen genossen wurde.

        \item 14. April, Weisser Sonntag\\
              Nach dem Einzug mit den Kindern durch das Dorf, stärkten wir uns bei
              einem Gipfeli im Löwen. Anschliessend umrahmten wir den Apéro im Inpuls.

        \item 27. / 28. April, Chilbi\\
              Die Chilbi war mal wieder nicht das Gelbe vom Ei. Der Samstag lief
              Überhaupt nicht gut. Am Sonntag war es etwas besser. Das Ergebnis war
              deutlich negativ. An unserem MGH-Stand haben wir bereits Werbung für den
              Musiktag 2013 gemacht.

        \item 11. Mai, Muttertag und Gönneranlass Musiktag 2013\\
              Zum ersten Mal hat die MGH am Muttertag nicht ein Ständli nach der
              Kirche durchgeführt, sondern am Samstagabend im roten Löwen ein Konzert
              zu ehren aller Müttern veranstaltet. Die Ansage der Stücke wurde auf
              verschiedene Musikanten/Innen aufgeteilt. Zuvor fand ebenfalls im Löwen
              der Gönneranlass für den Musiktag statt, welcher durch die Ronspatzen
              musikalisch umrahmt wurde.

        \item 19. Mai, Firmung\\
              Den frisch Gefirmten gratulierten wir mit einem abwechslungsreichen
              Ständli.

        \item 25. Mai, Luzerner Kantonales Jugendmusikfest\\
              Das Jugendmusikfest war ein voller Erfolg und ein einmaliges Erlebnis.
              Das riesige Zelt auf dem Rasenplatz bebte während der ganzen
              Rangverkündigung - unglaublich diese Stimmung.


        \item 29. Mai Country Fäscht\\
              Das Country Fäscht zog viele Countryfans an, welche den ganzen Abend
              tanzten, wodurch diese etwas das Trinken vergassen. Aber dafür waren ja
              wir Musikanten anwesend und viele andere Gäste. So wurde am Mittwoch vor
              Fronleichnam bis tief in die Nacht im grossen Zelt und dem Partyliner
              gefeiert.

        \item 1. / 2. Juni Luzerner Kantonaler Musiktag 2013\\
              Die Durchführung des Musiktages war ebenfalls ein voller Erfolg. Der
              Ablauf mit den Vereinen klappte und das Treiben in den verschiedenen
              Beizli geführt durch Hildisrieder Vereine war einmalig. Leider konnte
              die Marschmusik aufgrund des schlechten Wetters nur am Sonntag
              durchgeführt werden. Wir MGH-ler sind jedoch total stolz auf diesen
              gelungenen Anlass.


        \item 14. Juni, Abschlusshöck\\
              Dieses Jahr war das Posaunen-Register für den Abschlusshöck zuständig.
              Sie luden die MCH ins Urnerland ein. Direkt am See genossen wir ein
              feines Nachtessen. Während dem begann es leider zu regnen. Die
              Schnelleren konnten sich noch unter dem Gartenhäuschen platzieren und
              die anderen suchten mit dem Tisch etwas Schutz unter den Bäumen. Davon
              liessen wir uns natürlich nicht unterkriegen und genossen etwas später
              noch ein grosszügiges Desserbuffet. Maxs Führung durch die Ziegel-Bar
              war ebenfalls sehr interessant.


        \item 21. Juni, 40 Jahre Musikschule Hildisrieden\\
              Die Musikschule Hildisrieden feierte ihr 40-Jähriges Jubiläum, zu
              welchem die MGH mit einem Ständli gratulierte. Anschliessend wurde im
              Inpuls noch ein Glässchen Wein getrunken, bevor man gemütlich nach Hause
              ging - falsch. Unser Dario lud noch zu sich in den Sonnerain ein, um ein
              Glässchen Whisky oder Wein zu geniessen. Dieser Einladung folgten einige
              MGH-ler sowie der Ensemble Dirigent. Es wurde auf Fleischlis Balkon
              getratscht, getrunken und wie Bürstenbinder geraucht. Als unsere
              Aushilfe auf dem zweiten Cornet Frau R.T. aus dem E. in H. mal für
              kleine Musikantinnen musste, verwechselte sie leider die Türen und stand
              blöder Weise im Zimmer des schlafenden Bruders, welcher etwas imitiert
              war über den nächtlichen Besuch der Gemeindeprösidentin. Als ob nichts
              wäre, wurde anschliessend auf dem Balkon noch bis tief in die Nacht
              gefachsimpelt.


        \item 15. September, Buss- und Bettag\\
              Der diesjährige Buss- und Betttag fand in der Kirche in Römerswil statt.
              Es war etwas eng, was jedoch das anschliessende Glas Most mit ein paar
              Möckli Sprinz und Brot wieder vergessen liess.


        \item 19./20. Oktober, Lotto\\
              Das Lotto lief gut und wir sind mit dem Ergebnis zufrieden. Der Freitag
              stand ganz im Zeichen einer Frau Miriam Bolzern aus Hochdorf. Zuerst
              riefen wir sie gefühlte 100 mal aus, bis sie endlich ihr Auto
              umparkierte und anschliessend gewann sie zufälliger Weise noch das
              Millionelos, welches sie bei den Speaker abholen musste und jeder sah,
              wer die Person mit den Parkschwierigkeiten war. Zufälle gibt's.

        \item 1. Dezember, Jubilaren-Apérokonzert\\
              Und nochmals durften wir unseren Jubilaren musikalisch zu ihrem
              Geburtstag gratulieren. Einzig unser fromme Chosé kam etwas zu spät ans
              Ständli, da er die Predigt in der Kirche noch bis zum Schluss geniessen
              wollte.



    \end{itemize}

\end{history}
