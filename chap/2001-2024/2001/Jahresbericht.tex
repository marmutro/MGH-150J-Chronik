\subsection{Jahresbericht 2001}
\begin{history}

    \begin{itemize}

        \item[]2. Jan.\\
        Wie immer eröffnen wir das Vereinsjahr mit dem traditionellen Zunftbot.
        Wir begleiten die Zeremonie durchs Dorf und gratulieren dem 64.
        Zunftmeister Franz Helfenstein mit einem Ständchen.

        \item[]13. und 14. Jan.\\
        Zum ersten Mal haben wir das Jahreskonzert unter ein Motto gestellt. Zum
        Thema \enquote{Jagd} hat die Musikkommission ein passendes Programm
        zusammengestellt. Die Ansage macht das Duo Klaus und Thomas Estermann.

        \item[]8. April\\
        Geburtstagsständchen für die Jubilaren im Inpuls.

        \item[]22. April\\
        Der weisse Sonntag ist ganz seinem Namen treu. Der Einzug wird wegen
        heftigem Schneefall abgesagt.

        \item[]5. und 6. Mai, Dorfchilbi\\
        Bei nasskaltem Wetter finden am Samstag nur wenige Festbesucher den Weg
        an die Hildisriederchilbi. Trotz der vielen Besucher am Sonntag müssen
        wir als durchführender Verein rote Zahlen schreiben.

        \item[]13. Mai, Muttertag\\
        Bei sommerlichem Wetter halten wir vor der Kirche ein Ständchen.

        \item[]7. Juni,  Gemeinschaftskonzert im Inpuls. \\
        Als Vorbereitung auf das Eidg. Musikfest halten wir zusammen mit der
        Feldmusik Hellbühl und Feldmusik Neuenkirch ein Vorbereitungskonzert.
        Viele Konzertbesucher geniessen das interessante und unterhaltsame
        Konzert.

        \item[]16. und 17. Juni\\
        Bei regnerischem Wetter, aber in guter Stimmung reisen wir am
        Samstagmorgen mit einem Car nach Fribourg. Nach 5 Minuten Strammstehen
        in strömenden Regen konnten wir unser Marschmusikvortrag absolvieren.
        Das lange Warten hat sich gelohnt: Mit 103 Punkten sind wir in der
        oberen Ranglistenhälfte gelandet. Anschliessend bereiten wir uns auf die
        Konzertstücke vor. Mit dem Aufgabestück \enquote{Alpine Variations} von
        Bertrand Moren und dem Selbstwahlstück \enquote{Prelude and Celebration}
        von James Curnow treten wir gut vorbereitet vor die Jury. Die bedien
        Darbietungen gelingen uns recht gut. Mit je 150 Punkten stehen wir am
        Schluss auf dem guten 19. Rang von 46 teilnehmenden Vereinen. Am Abend
        erkunden wir die Altstadt von Fribourg. Der herzliche Empfang von der
        Hildisrieder Bevölkerung vom Sonntagabend ist ein schöner Abschluss
        eines gelungenen Musikfestes.

        \item[]30. Juni\\
        Bei herrlichem Sommerwetter spielen wir das traditionelle Ständchen für
        die neugefirmten Kinder.

        \item[]14. Aug.\\
        Der Abschlusshoch der Chilbe dürfen wir bei Armin Schmid im Länzehof
        durchführen. Trotz des schönen Sommerwetters folgen nur wenige der
        Einladung.

        \item[]16. Sept., Eidg. Bettag\\
        Dank Schlechtwetter wir die Messe in die Kirche Rain verschoben. Wir
        bestreiten die Messebegleitung in reduzierter Besetzung.

        \item[]2. und 3. Nov., Lotto\\
        Der gutorganisierte Lottomatch lockt viele Besucher in den Löwensaal.

        \item[]17. und 18. Nov., Theater Trachtengruppe.\\
        An diesem Anlass bewähren wir uns  als komptetentes Wirteteam.

        \item[]25. Nov.\\
        Geburtstagsständchen für die Jubilaren im Inpuls.


    \end{itemize}

\end{history}
