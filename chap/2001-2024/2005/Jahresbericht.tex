\subsection*{2005}

\begin{history}


    \begin{itemize}

        \item 2. Januar, Zunftbot\\
              Wie jedes Jahr eröffnen wir unser Vereinsjahr mit dem traditionellen
              Zunftbot. Bruno Stadelmann heisst der neue Regent der Fasnacht 2005. Mit
              einem kleinen Ständli gratulierten wir dem neuen Zunftmeister.

        \item 10/11. Januar, Jahreskonzerte\\
              Mit dem Motto Fiesta hat die Musikkommission ein interessantes Programm
              ausgearbeitet und einstudiert. Mit kleinen Showeinlagen wurde der zweite
              unterhaltsame Teil versüsst. Für 30 Jahre aktives musizieren wurden
              Peter Käppeli und Pirmin Troxler gechrt. Nach vielen Stunden in der Bar
              wurde bald der Morgen eingeläutet.

        \item 3. April, Weisser Sonntag\\
              Mit dem Prozessionsmarsch begleiteten wir die zehn Kinder bei
              strahlender Sonne durch das Dorf. Nach der Messe spielten wir der
              Dorfgemeinde verschiedene Stücke aus dem „Marschbüechli“.

        \item 23. und 24. April, Frühlingsparty, Chilbi\\
              Dank gutem Einsatz vom OK und den
              Mitgliedern konnte unser Finanzchef ein sehr gutes Resultat präsentieren.
              Im Jahre 2008 liegt es wieder an uns den jetzigen Reingewinn zu
              übertreffen.

        \item 8. Mai, Muttertagsständli\\
              Bei viel Sonnenschein und einem unterhaltsamen Muttertagskonzert
              gratulierten wir den Hildisrieder Müttern.

        \item 28./29. Mai, Luzerner Kantonales Musikfest in Nottwil\\
              Unter der Leitung von Kobi Banz erreichten wir in der 2. Stärkeklasse
              Brass Band den 13. Rang mit 82.0 beim Selbstwahlstück \enquote{Dancing in the
                  Park} und 85.3 im Teststück \enquote{Newstaed}. In der Marschmusik reihten wir
              uns am Anfang des Mitteldrittels auf den 14. Rang ein. Man kann also
              sagen: Wir sind mit einem blauen Auge davongekommen. Als um 15.30 Uhr
              für uns der offizielle Teil beendet war, konnten wir uns ganz dem
              „Biertrinken“ widmen.

              Sonntag, der Tag der Veteranen. Zu unserer Freude duften wir zwei
              Veteranen stellen, die für 30 Jahre aktives musizieren geehrt wurden.
              Peter Käppeli und Pirmin Troxler sind jetzt stolze Besitzer vom
              blauweissen Kranz.

        \item 5. Juni, Jubilarenkonzert\\
              Mit einem kleinen Unterhaltungskonzert mit anschliessendem Apero
              gratulierten wir den Jubilaren.

        \item 20. Juni, Interne Besprechung über Ziele und Massnahmen\\
              Der Workshop sollte dazu dienen, um herauszufinden, was für uns das
              Richtige ist, wie bekommen wir wieder Spass am Musizieren und zu guter
              letzt - wie ist das realisierbar? Es waren zwar nicht alle der gleichen
              Meinung, aber es kristallisierte sich mit der Zeit immer mehr heraus,
              dass uns ein Direktionswechsel bevorsteht.

        \item 23. Juni, Kündigung Kobi Banz\\
              Drei Tage danach flatterte die Kündigung von Kobi in die gute Stube
              unseres Präsidenten. 15.01.06 sollte der letzte Auftritt mit Kobi sein.
              Nach einer Aussprache mit ihm konnten wir uns einigen, dass wir bestrebt
              sind, das Konzert 06 mit einem neuen Dirigenten durchzuführen. Als
              offiziellen letzten Auftritt wurde das 125-Jahr-Jubiläum der
              Musikgesellschaft Harmonie Sempach als Abschiedskonzert ernannt.

        \item 26. Juni, Amtseinsetzung Werner Bucher\\
              Unter dem Hildisrieder Inpuls-Vordach umrahmten wir den Apero mit
              Unterhaltungsmusik. Als Dank offerierte uns die Kirchgemeinde \enquote{Hörndli
                  met Ghacketem}.

        \item 9. September, Derniere Kobi Banz\\
              Nach der letzten Probe mit Kobi wurden wir von den Veteranen Peter
              Käppeli und Pirmin Troxler ins Rest. Kreuz eingeladen. Für „Speis und Trank“
              danken wir den beiden Veteranen recht herzlich.

        \item 10. September, 125 Jahre Harmonie Sempach\\
              Zum letzten Mal gaben wir unter der Leitung von Kobi Banz in der
              Festhalle Sempach anlässlich dem 125 Jahr Jubiläum der Musikgesellschaft Harmonie Sempach
              unser Bestes. Beim anschliessenden Umtrunk konnte dieser Abend als
              gelungen abgehakt werden.

        \item 17. September, Firmung\\
              Den frisch gefirmten Firmlingen gratulierten wir mit einem kleinen
              Unterhaltungskonzert im Foyer.

        \item 28. und 29. Oktober, Super Lotto\\
              Das gut organisierte Lotto ging zu unseren Gunsten flott über die Bühne.
              Einmal mehr hatten wir die angefressenen LottospielerInnen unter
              Kontrolle. Besten Dank Urs, für die gelungene Premiere.

        \item 4. Dezember, 25. Jubilarenkonzert\\
              Mit einem kleinen Unterhaltungskonzert unter der neuen Leitung von
              Michael Rösch und mit einem anschliessendem Apéro gratulierten wir den
              Jubilaren.



    \end{itemize}

\end{history}
