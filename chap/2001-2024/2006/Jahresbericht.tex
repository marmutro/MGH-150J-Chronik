\subsection{Jahresbericht 2006}

\begin{history}


    \begin{itemize}

        \item[]2. Januar, Zunftbot\\
        Wie jedes Jahr eröffnen wir unser Vereinsjahr mit dem traditionellen
        Zunftbot. Peter Käppeli heisst der neue Regent der Fasnacht 2006. Mit
        einem kleinen Ständli gratulierten wir dem neuen Zunftmeister.

        \item[]14./15. Januar, Jahreskonzerte\\
        Mit dem Motto \enquote{Rhythm 'n’ Brass} hat die Musikkommission ein
        interessantes Programm ausgearbeitet und einstudiert. Mit kleinen
        Showeinlagen vom Percussionregister wurde der zweite unterhaltsame Teil
        versüsst. Nach vielen Stunden in der Bar wurde bald der Morgen
        eingeläutet.

        \item[]2. April, Jubilaren-Aperokonzert\\
        Am wunderbaren Sonntagmorgen gratulierten wir mit einem kleinen
        Geburtstagskonzert den Hildisrieder jubilierenden. Beim anschliessenden
        Apero klang der morgen langsam aus.

        \item[]8. April, 1. Skitag\\
        Ist es nicht wunderschön, an einem sonnigen Samstagmorgen aufzustehen
        und die Ski's zu schultern? Der bis auf den letzten Platz gefüllte Bus
        konnte pünktlich Richtung Andermatt abreisen. Bei guter Witterung und
        sensationellen Schneeverhältnissen genossen alle beteiligten den tollen
        Wintertag.

        Dann, nach dem Fondueplausch ging es dem Alkohol an den Kragen. Guter
        Boden im Magen war nach dem Nachtessen bestimmt bei jedem vorhanden.

        Mann geht davon aus, dass der gut ausgebuchte Bus am morgen so gegen die
        Eins sämtliche MGH-ler sicher nachhause brachte. Ein unvergesslicher Tag
        geht zu Ende.

        \item[]14. Mai, Muttertagsständli\\
        Bei viel Sonnenschein und einem unterhaltsamen Muttertagskonzert
        gratulierten wir den Hildisrieder Müttern.

        \item[]2. Juni, Vorbereitungskonzert in Hitzkirch\\
        Mit den Vereinen MG Oberkirch und Harmoniemusik Hitzkirchtal bereiten
        wir uns in Hitzkirch gemeinsam auf das bevorstehende eidg. Musikfest
        vor.

        \item[]17./18. Juni, Eidg. Musikfest in Luzern\\
        Einmal mehr früh morgens aus der Bettruhe, Wettbewerb ist angesagt. Mit
        leerem Magen trafen wir uns um 05.00 Uhr im Chrüz zum spendierten
        Morgenessen vom Chrüzwirt. Dann ging es weiter mit dem Car in die
        Leuchtenstadt. Angespannt spielten wir in der Bruchmattturnhalle ein.
        Wir hatten die Ehre den Konzertmorgen in der 2. Klasse Brass Band mit
        dem Aufgabestück zu eröffnen. 258 Punkte haben wir heraus gespielt, dann
        ging es weiter zum Selbstwahlstück. 245 Punkte konnte nun geerntet
        werden, welches uns den 14. Schlussrang mit 503 Punkten ergab.

        Bei der Marschmusik erreichten wir 253 Punkte, welches uns den 4. Rang
        bescherte.

        Endlich war der Pflichtteil hinter uns und wir konnten uns gekonnt dem
        Gesellschaftlichen widmen.

        \item[]2.13. September, Vorstands- und Musikkommissionsausflug\\
        Mit Mumis Escorte ging es los Richtung Visp. Nach dem Hotelbezug in
        Brig. Nach dem Mittagessen wurden wir mit Weinglas und kulturellen
        Informationen eingedeckt. Dann ging es los in den Hängen der Heida.
        Diese Rebberge unter Visperterminen sind bekanntlich das höchstgelegene
        Weingut in Europa. Nach der siebengängigen Wanderung wurde es lustiger
        und lustiger. Am Abend verwandelte sich das Bergdörfli Oberstalden in
        eine Partyhölle. Jeder der konnte baute seinen Keller in eine Bar um.
        Das letzte Postauto brachte denn auch noch uns sicher nach Visp. Gut
        Gelaunt fand zum Schluss noch jeder seine Ruhe im Hotelbett.

        \item[]17. September. 100 Jahre Harmonie Rain\\
        Mit einer kleinen Marschmusik von der Kirche zur Turnhalle, mit einem
        kleinen Unterhaltungskonzert gratulierten wir der Harmonie Rain zum
        100jährigen bestehen.

        \item[]27. und 28. Oktober, Super Lotto\\
        Das gut organisierte Lotto ging zu unseren Gunsten flott über die Bühne.
        Einmal mehr hatten wir die angefressenen Lottospielerinnen unter
        Kontrolle. Besten Dank dem OK.

        \item[]3 Dezember, Jubilarenkonzert\\
        Mit einem kleinen Unterhaltungskonzert unter der neuen Leitung von
        Michael Rösch und mit einem anschliessendem Apero gratulierten wir den
        Jubilaren.

    \end{itemize}

\end{history}
