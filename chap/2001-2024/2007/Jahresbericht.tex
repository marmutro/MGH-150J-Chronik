\subsection*{2007}

\begin{history}


    \begin{itemize}

        \item 2. Januar, Zunftbot\\
              Wie jedes Jahr eröffnen wir unser Vereinsjahr mit dem traditionellen
              Zunftbot. Walter Rüttimann heisst der neue Regent der Fasnacht 2007. Mit
              einem kleinen Ständli gratulierten wir dem neuen Zunftmeister.

        \item 14./15. Januar, Jahreskonzerte\\
              Mit dem Motto \enquote{Irish Folks} hat die Musikkommission ein
              interessantes Programm ausgearbeitet und einstudiert.

        \item 24. Februar, 2. Skitag im Hasliberg\\
              Der bis auf den letzten Platz gefüllte Bus konnte pünktlich Richtung
              Hasliberg abreisen. Alle waren schneehungrig. Bei nicht ganz optimaler
              Witterung holten wir das Beste heraus. Als dann endlich abends auch die
              letzten den weg ins Tal gefunden haben, ging es weiter nach Sarnen um
              den Abend so richtig zu geniessen. Der Apero im Citypub, das Fondue im
              Rest. Jordan und das Bier im Mexican. Bevor die ersten an der Bar
              einschliefen, brachte uns der Chauffeur sicher Nachhause.

        \item 1. April, Jubilaren-Aperokonzert\\
              Nein, das ist kein Scherz, wir gratulieren an diesem Sonntagmorgen
              tatsächlich den Jubilaren aus der Gemeinde.

        \item 9. - 14. April, Lager Blechbläser-Jugendensemble in Isental\\
              Zusammen mit den Vereinen MG Isenthal und MG Römerswil organisierte
              Alexander Troxler für den Nachwuchs ein Jugendmusiklager. Dass das Lager
              erfolgreich war, wissen wir sehr zu schätzen, denn konnten wir viele
              neue Gesichter verpflichten. Isenthal bietete gute Lokalitäten für
              Registerproben und das abschliessende Konzert. Um die Teilnehmer im
              Gruppenunterricht zu fördern, mussten etliche Registerlehrer eingeflogen
              werden.

        \item 15. April, Weisser Sonntag\\
              Mit einer Prozession begleiteten wir die Kinder zur Kirche. Nach einer
              ausgiebigen Kaffeepause im Löwen gratulierten wir den Erstkommunikanten
              mit ein paar brassigen Klängen.

        \item 28./29. April, Chilbi\\
              Der Protokollführer verweilte zu diesem Zeitpunkt in Amerika, jedoch
              wird das mein Nachfolger noch berichtigen.

        \item 6. Mai, Firmung\\
              Den frisch Gefirmten gratulierten wir mit einem Ständli.

        \item 13. Mai, Muttertag\\
              Bei viel Sonnenschein und einem unterhaltsamen Muttertagskonzert
              gratulierten wir den Hildisrieder Müttern.

        \item 7. Juni, Frohnleichnam\\
              Nach der Prozession unterhielten wir die Bevölkerung mit einem kleinen
              Platzkonzert.

        \item 9. / 10 Juni, Kantonaler Musiktag Gettnau\\
              Gut vorbereitet machten wir uns mit dem Car wohlverstanden auf den Weg
              nach Gettnau, die Heimatgemeinde unseres Dirigenten. Alle waren unter
              Druck, schliesslich wollten wir einen guten Eindruck hinterlassen, was
              uns auch gelang, mit dem schwierigen Selbstwahlstuck \enquote{Resurgam}.
              Unser Juror Thomas Wyss gab einen guten Bericht ab, was uns motivierte,
              auf der Marschmusikstrecke noch einmal unser Bestes zugeben. Als dann
              die beiden A-Noten (technischer Eindruck)verteilt waren, wussten wir,
              dass wir das ganze mit einer guten B-Note (Interpretation und
              Choreographie) noch wettmachen können. Denn mit unserer guten
              Kameradschaft war das für uns kein Problem eine 6.0 zu erreichen...

        \item 15. Juni, Abschlusshöck\\
              Nach dem erfolgreichen Musiktag in Gettnau trafen sich die
              MGH-Mitglieder am Freitag zum gemeinsamen Anstossen auf den Abschluss
              des Frühlings 2007. Im Baumhaus Hildisrieden wurden der vergangene
              Musiktag, die dazugehörigen Fotos und neusten Gerüchte nochmals in
              gemütlicher Runde kommentiert und diskutiert. Zum Schluss einen
              herzlichen Dank den Organisatoren aus dem Bass-Register und den
              Dessert-Spendern Franz und Hildegard Dörig.

        \item 17. - 19. August, Musigreise\\
              Nach dreijähriger Unterbrechung traf sich am Freitag die MGH zu einer
              weiteren legendären Musigreise. Das Reiseprogramm versprach dabei eine
              Tour de Russl... ähm Romandie. Die Instrumente wurden dabei allerdings
              (schon fast traditionell) vergessen.

              In Wasen im Emmental wurde um 10:00 der erste Kaffee-Halt gemacht. Beim
              Hornussen zeigte sich schon bald mal, welche Talente sich wirklich
              hinter den Musikanten verstecken.

              Gegen Mittag machte die Reisegesellschaft dann in Niede\emph{rösch}
              halt, um sich für einen strengen Nachmittag zu stärken. Von unserem
              Dirigenten fehlte trotz Dorfnamen jede Spur.

              Danach nahm man Fahrt Richtung Ligerz (BE), dem \enquote{Dorf am
                  Röschtigraben}. Nach einem intensiven, viertelstündigen Aufstieg
              wurden die Musikanten beim Weinbau "Festiguet" von Rolf Teutsch
              empfangen. Die anschliessende Weinbaubesichtigung rückte dabei etwas
              in den Hintergrund, dafür gestaltete sich die Weindegustation umso
              länger. Gemäss Prophezeiungen des Weinbauers sollten die Musikanten
              nach dieser rund zweistündigen Degustation nicht nur die wunderbare
              Aussicht auf den Bielersee geniessen dürfen, sondern sogar das Dach
              Afrikas, den Kilimandscharo, entdecken. Nicht zum letzten Mal auf
              dieser dreitägigen Reise, wie sich herausstellte.

              Nach der Fahrt nach Murten und einem feinen Nachtessen durfte die Zeit
              jeder selber nutzen, mehr oder weniger lang Murten auszukundschaften.

              Früh am Samstagmorgen (10.15 Uhr) fuhr die Reisetruppe dann weiter mit
              Reiseziel Le Pont am Lac de Joux. Nach einer Mittagspause zeigte sich
              die MGH mit einer zweistündigen Wanderung zu den Grotten von Vallorbe
              (VD) von ihrer sportlichen Seite. Selbstverständlich durfte auch die
              Besichtigung der Grotten nicht fehlen. Bei etwas kühleren Temperaturen
              von rund 10 Grad zeigte der Führer die Attraktionen dieser Grotte, die
              seit 1974 öffentlich zugänglich ist.

              Der Abend wurde dann in der "olympischen Stadt" Lausanne verbracht. Beim
              gemeinsamen Nachtessen im "Ticino de Lausanne" und dem Besuch der
              ortsansässigen Brasserie wurde der Abend um die Ohren geschlagen.
              Anscheinend hat die Wanderung aber bei einigen Musikanten zu wenig
              gewirkt. Gerüchteweise soll die Nacht plötzlich sehr kurz und sogar noch
              Berge namens Kilimandscharo nach Lausanne verschoben worden sein Die
              eingetragene Nachtruhe um 22:00 wurde von der Reiseleitung
              höchstpersönlich um mehrere Stunden hinausgezögert, schliesslich ist man
              nur einmal in Lausanne.

              Sonntags wollte die Reisegruppe die sicherlich wunderbare und
              einzigartige Aussicht von der Turmspitze der Kathedrale in Lausanne
              geniessen. Doch des einen Leid, war des anderen Freud: Die Turmspitze
              kann an Tagen wie diesen nur ab 14:00 bezwungen werden. Einigen fiel
              wohl ein Stein vom Herzen. So reiste die MGH unverrichteter Dinge [ein
                      paar Gebete ausgenommen] weiter Richtung Montreux. Ein Teil der Gruppe
              erkundete Montreux, während dem die anderen im Aquaparc das Vergnügen
              suchten. Im späteren Nachmittag wurden dann alle wieder
              "zusammengesammelt". Auf dem Heimweg versuchten viele, die Musigreise
              bei einem Nickerchen nochmals zu Gemüte zu führen. Kurz vor 19:00 trafen
              die Musikanten unversehrt in Hildisrieden ein.

              An dieser Stelle gebührt Mätthu Rub und Dani Fleischli ein grosses
              Dankeschön für die perfekt organisierte \enquote{Tour de Romandi}.

        \item 4. August, Pavilionkonzert\\
              Einmal anders vergnügten wir uns im roten Löwen, mangels guter
              Witterung. War auch, gut, konnten wir uns wieder einmal dem
              Kameradschaftlichen widmen.

        \item 16. September, Berghof Gedenkfeier Gundolingen\\
              Bei windigen Verhältnissen begleiteten wir mit dem Kirchenchor den
              Berghofgottesdienst. Nach dem anschliessenden Unterhaltungskonzert
              durften wir uns noch Stärken \enquote{met Chäs, Brot ond Moscht, frösch
                  ab Press}.

        \item 26./27. Oktober, Lotto\\
              Jetzt müssen wir über die Bücher, oder ist die Druckfirma schuld am
              schlechten Lottoerfolg? Schlussendlich konnte unser Kassier Dani trotz
              des schlechten Samstags einen Erfolg verzeichnen.


    \end{itemize}

\end{history}
