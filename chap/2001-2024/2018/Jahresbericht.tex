\subsection*{2018}
\begin{history}

      Dieser Jahresbericht leider nicht vollständig.

      \begin{itemize}

            \item 2. Januar Zunftbot\\
                  Wie es in Hildisrieden Tradition ist, wird am 2. Jahrestag der
                  neue Zunftmeister erkoren.  Und auch diesmal begleitete die MGH
                  zuerst die Zunftdelegation, nach einem kleinen Apero, mit dem
                  Marsch Viva Arogno durchs Dorf bis zum Leuen. Anschliessend wurde
                  der Zunftbot im grossen Saal abgehalten. Der Zunftmeister Cornel
                  Estermann wurde feierlich zum neuen Amt erkoren. Die
                  Musikgesellschaft gratulierte wie gewohnt auf musikalische Weise.
                  Anschliessend verschoben sich die Feierlichkeiten nach
                  Traselingen.

            \item 20. / 21. Januar Jahreskonzerte\\
                  „Von Bildern inspiriert“ lautete unser Motto an den diesjährigen
                  Jahreskonzerten. Einerseits wurden die Musikanten von
                  verschiedensten Bildern und Klängen aus aller Welt inspiriert und
                  andererseits inspirierten die Musikanten den Künstler Martin Geel
                  mit der Musik. So konnte er die Klänge und Emotionen während dem
                  Konzert in seine Werke einfliessen lassen. Einige dieser Bilder
                  werden tatsächlich in Zukunft Wände in Hildisrieder Wohnungen
                  zieren. Höchst erfreulich waren auch die Besucherzahlen. Die
                  Darbietungen, welche die MGH auf der Bühne zeigen kann, scheint
                  den Konzertbesuchern zu gefallen. Am Samstag war die Inpulshalle
                  randvoll gefüllt und auch am Sonntag fanden zahlreiche
                  Musikfreunde den Weg nach Hildisrieden. Traditionell wurden die
                  Konzerte in der Bar feierlich zu Ende gebracht. Ein Musikant
                  konnte das kurz zuvor erlebte scheinbar nicht ganz fassen und
                  musste sich die Geschichte vor dem Inpuls noch einmal durch den
                  Kopf gehen lassen.

            \item  18. März Jubilarenkonzert

            \item 20. März Einzelprobe\\
                  Am letzten Dienstag wollte ein tüchtiger Musikant die wöchentliche
                  Gesamtprobe besuchen. Vor Ort fand er eine leere Bühne vor, denn
                  laut Jahresprogramm wurde in dieser Woche die Gesamtprobe vom
                  Dienstag auf den Donnerstag verschoben. Unwissentlich packte er
                  sein Instrument aus und startete mit der Probe, trotz der vielen
                  Absenzen, pünktlich um 20.00 Uhr. Der Probebetrieb lief 30 Minuten
                  und fand erst ein abruptes Ende, als das Register des 2. Und 3.
                  Cornets erschien um ihre geplante Spezialprobe abzuhalten.

      \end{itemize}

\end{history}
