\subsection*{2018}
\begin{history}

      \begin{itemize}

            \item 2. Januar Zunftbot\\
                  Wie es in Hildisrieden Tradition ist, wird am 2. Jahrestag der
                  neue Zunftmeister erkoren.  Und auch diesmal begleitete die
                  MGH zuerst die Zunftdelegation, nach einem kleinen Apero, mit
                  dem Marsch Viva Arogno durchs Dorf bis zum Leuen.
                  Anschliessend wurde der Zunftbot im grossen Saal abgehalten.
                  Der Zunftmeister Cornel Estermann wurde feierlich zum neuen
                  Amt erkoren. Die Musikgesellschaft gratulierte wie gewohnt auf
                  musikalische Weise. Anschliessend verschoben sich die
                  Feierlichkeiten nach Traselingen.

            \item 20. / 21. Januar Jahreskonzerte\\
                  „Von Bildern inspiriert“ lautete unser Motto an den
                  diesjährigen Jahreskonzerten. Einerseits wurden die Musikanten
                  von verschiedensten Bildern und Klängen aus aller Welt
                  inspiriert und andererseits inspirierten die Musikanten den
                  Künstler Martin Geel mit der Musik. So konnte er die Klänge
                  und Emotionen während dem Konzert in seine Werke einfliessen
                  lassen. Einige dieser Bilder werden tatsächlich in Zukunft
                  Wände in Hildisrieder Wohnungen zieren. Höchst erfreulich
                  waren auch die Besucherzahlen. Die Darbietungen, welche die
                  MGH auf der Bühne zeigen kann, scheint den Konzertbesuchern zu
                  gefallen. Am Samstag war die Inpulshalle randvoll gefüllt und
                  auch am Sonntag fanden zahlreiche Musikfreunde den Weg nach
                  Hildisrieden. Traditionell wurden die Konzerte in der Bar
                  feierlich zu Ende gebracht. Ein Musikant konnte das kurz zuvor
                  erlebte scheinbar nicht ganz fassen und musste sich die
                  Geschichte vor dem Inpuls noch einmal durch den Kopf gehen
                  lassen.

            \item 18. März Jubilarenkonzert

            \item 20. März Einzelprobe\\
                  Am letzten Dienstag wollte ein tüchtiger Musikant die
                  wöchentliche Gesamtprobe besuchen. Vor Ort fand er eine leere
                  Bühne vor, denn laut Jahresprogramm wurde in dieser Woche die
                  Gesamtprobe vom Dienstag auf den Donnerstag verschoben.
                  Unwissentlich packte er sein Instrument aus und startete mit
                  der Probe, trotz der vielen Absenzen, pünktlich um 20.00 Uhr.
                  Der Probebetrieb lief 30 Minuten und fand erst ein abruptes
                  Ende, als das Register des 2. Und 3. Cornets erschien um ihre
                  geplante Spezialprobe abzuhalten.

            \item 1. April  Erstkommunion\\
                  Wie jedes Jahr begleitete die MGH die Erstkommunikanten/innen
                  durch das Dorf vom Schulhaus in die Kirche mit einem
                  Prozessionsmarsch. Anschliessend der Kirchenmesse gaben wir
                  ein Ständli zu unserem Besten und genossen im Anschluss den
                  Apèro.

            \item 8. April LJBB Konzert in Hildisrieden\\
                  Bereits zum 2. male nach 2012 fand das Konzert der Luzerner
                  Jugend Brass Band in Hildisrieden statt. Speziell Stolz sind
                  wir darauf, dass sechs junge Hildisrieder Musikanten im Lager
                  Mitwirken uns somit ein Heimspiel geniessen dürfen.

            \item 28. April Brass in Concert in Rickenbach\\
                  Am 28. April genossen wir ein Konzert am Brass in Concert in
                  Rickenbach. Als Gast präsentierten wir ein gelungenes,
                  unterhaltsames Konzert unter der Leitung von Peter Stadelmann.

            \item 29. April Chilbi\\
                  Am Sonntag, 29. April 2018 haben wir bei gutem Wetter mit
                  vielen Besuchern die Chilbi in Hildisrieden feiern. Wenn es
                  auch finanziell nicht ein Gewinn-Geschäft ist, die MGH ist vor
                  Ort und Präsentiert sich gut. 02. Juni 2018 Kantonal Musiktag
                  in Eschenbach

            \item 2. Juni Musiktag Eschenbach\\
                  Gut vorbereitet machten wir uns auf den Weg Richtung
                  Eschenbach. Bei heissen Temperaturen gaben wir das Beste. Beim
                  Wettvortrag gab uns die Jury gute Bewertung. Obwohl immer
                  etwas verbessert werden kann. Bei der Parademusik wurde es
                  dann richtig warm an den Füssen. Die MGH konnte mit 49.6
                  Punkten den 6. Rang erspielen. Anschliessend genossen wir
                  endlich das langersehnte Bier danach und verbrachen einen
                  tollen Sommerabend in Eschenbach.

            \item Abschlusshöck\\
                  Nach der strengen Probezeit fand der MGH Abschlusshöck statt.
                  Mit einer Schnitzeljagt durch das Dorf kamen wir am Zielort in
                  der Schüür in Hildisrieden an. Wir genossen gemeinsam einen
                  schönen Abend mit Pizza und Bier. Herzlichen Dank für die
                  Organisation am EsHorn/Flügelhorn Register.

            \item  15. September Ehrenmitgliederkonzert\\
                  Am 15. September hat die MGH ein unterhaltsames Konzert
                  präsentiert vor unseren Ehrenmitgliedern/innen. Anschliessend
                  genossen wir den Nachmittag bei Kaffee und Kuchen.

            \item 19. Und 20. Oktober Lotto\\
                  Am 19. Und 20. Oktober haben wir im Roten Löwen unser Lotto
                  durchgeführt. Viele Besucher/innen spielten um tolle und
                  hochwertige Preise an diesen zwei Tagen.

      \end{itemize}

\end{history}
