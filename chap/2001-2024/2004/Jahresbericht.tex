\subsection{2004}

\begin{history}


    \begin{itemize}

        \item 2. Januar, Zunftbot\\
              Wie jedes Jahr eröffnen wir unser Vereinsjahr mit dem traditionellen
              Zunftbot. Auf Sepp der 8. folgt unser Aktivehrenmitglied Walter Troxler.
              Wir gratulierten dem Greenkeeper musikalisch zur Wahl als Zunftmeister
              2004.

        \item 10/11. Januar, Jahreskonzerte\\
              Unter dem Motto \enquote{Hits in Brass} haben wir ein
              abwechslungsreiches Konzertprogramm einstudiert.

        \item 13. März, Beerdigung Walter Troxler.\\
              Leider ist unser Zunftmeister und Aktivehrenmitglied von uns gegangen.
              Mit einer Fahnendelegation begleiteten wir ihn zu seiner Ruhestätte.

        \item 14. März, Gönnerbrunch Kesselpauken\\
              Als dank für die finanzielle Unterstützung zur Anschaffung der Timpani
              luden wir die Gönner zu einem Brunch ein. Für die musikalische
              Unterhaltung und ein Fass Bier war auch gesorgt.

        \item 4. April, Jubilarenkonzert\\
              Mit einem Geburtstagständli im Foyer gratulierten wir den Jubilaren zu
              ihren runden Geburtstagen.

        \item 25. April, Chilbi\\
              Bevor wir das Vergnügen haben, die Chilbi selber zu organisieren
              \enquote{besserten} wir mit dem Bogenschiessen unsere Vereinskasse auf.

        \item 1. Mai, Kirchweihfest\\
              Mit ein paar Unterhaltungsstücken umrahmten wir den Apéro
              der Kirchengemeinde im Foyer Inpuls.

        \item 9. Mai, Muttertagsständli\\
              Da an diesem Sonntag zu viele Musikanten abwesend waren, mussten wir aus
              eigener Regie das Ständli absagen.

        \item 22./23. Mai, Kantonaler Musiktag in Büron\\
              Früh morgens starteten wir mit einem Kleinbus nach Büron, um uns mit
              anderen Vereinen zu messen. Vor fast leerem Saal spielten wir um 9 Uhr
              mit der Unterhaltungsnummer Hollywood auf. Nach dem Expertengespräch mit
              Thomas Rüedi stand uns ein langer Tag vor der Tür. Als wir uns dann um
              17.00 Uhr zum Showblock formierten und mit der Marschmusik 48.1 Punkte
              erreichten konnte das Fest beginnen. Während sich die einen mit Bier
              trinken, die anderen mit konzertanten Musikvorträgen vergnügten, wurde
              es langsam morgen. Als uns Froschauge-Winus Chauffeuse um ca. 4 Uhr in
              der Früh abholte, konnte sie Dank energischem Eingreifen doch noch eine
              Fahnenentführung verhindern.

        \item 9./11. Juni\\
              Konzert mit den Kirchenchören Hildisrieden und Neudorf

        \item 12. Juni, Firmung. \\
              Den frisch gefirmten Firmlinge gratulierten wir mit einem
              Ständchen.

        \item 20. Juni, Neuuniformierung BBH Neuenkirch.\\
              Mit einem kleinen Unterhaltungskonzert gratulierten wir der BBH zur
              neuen Verkleidung.

        \item 4.5. September, Musigreise\\
              Mit Sack und Pack fuhren wir am Samstagmorgen ab Richtung Jura. Nach der
              Mittagsrast in ging es dann weiter nach Bassecourt. Nach kurzer
              Wartezeit war dann unsere Privatbahn mit Jahrgang 1935 startklar.
              Plötzlich wurde die Situation ganz heiss, denn wir wurden von drei
              Cowboys zum Halten gezwungen. Hildegard und Urs mussten wir opfern, um
              weiterzufahren. Nachdem der Tuba-Franz genügend Lösegeld
              zusammengekratzt hatte, um seine Frau und Urs freizukaufen fuhren wir
              mit der alten Lady weiter. Angekommen in Saignelegier checkten wir im
              Hotel ein. Nach dem Nachtessen was dann an der Zeit das Dorf unsicher zu
              machen. Bei Bier und Gesang wurde die Nacht immer kürzer.

              Am nächsten Morgen ging dann die Carfahrt weiter zum Fluss Le Doubs. Um
              die Doubsfälle zu erreichen stand uns ein kleiner Spaziergang bevor.
              Nach der Schifffahrt wurde das geschichtliche Wissen der meisten
              aufgebessert. Nach dem Mittagessen fuhren wir via Biel nach Hause.

        \item 19. September, Berghof-Gottesdienst Gundolinge.\\
              Mit dem Schweizerpsalm und ein paar
              Unterhaltungstücken unterhielten wir die  Berghofgemeinde bei Käse und
              Most.

        \item 30. und 31. Oktober, Super Lotto.\\
              Das gut organisierte Lotto ging zu
              unseren Gunsten flott über die Bühne. Einmal mehr hatten wir die
              angefressenen LottospielerInnen unter Kontrolle.

        \item 14. November, 6. Seetaler Jugendmusiktreffen.\\
              Dass es viele Zuhörer gibt, wussten wir, aber dass die Halle platzen
              könnte hätten wir nicht gedacht.

        \item 3. Dezember, Chlaushöck\\
              Wenn der Probebesuch besser gewesen wäre, hätte es für den
              Rest weniger gegeben.

        \item 5. Dezember, Jubilarenständli\\
              Mit einem kleinen Sonntagmorgenkonzert gratulierten wir der zweiten
              Hälfte der Jubilaren unserer Gemeinde.

        \item 9. Dezember, Weihnachtsfenstereröffnung Frauenbund\\
              Mit weihnachtlichen Klängen
              umrahmten wir die Eröffnung vom Weihnachtsfenster in der Waldmatt.
              Anschliessend wurden wir mit Kaffee, Glühwein und Kuchen verwöhnt.

    \end{itemize}

\end{history}
