\subsection*{2002}

\begin{history}

    \begin{itemize}

        \item 2. Jan., Zunftbot.\\
              Als Zunftmeister wird Herbert Rüttimann gewählt.

        \item 12. und 13. Jan.\\
              Dieses Jahr hat die Musikkommission ein unterhaltsames Konzert unter dem
              Motto \enquote{Acqua Musicale} zusammengestellt.

        \item 7. April, Weisser Sonntag. \\
              Wir führen den Einzug durchs Dorf an. Nach dem
              Gottesdienst geben wir ein Ständchen zum Besten.

        \item 20. April, Lublaska Konzert.\\
              Beim Konzert der Luzerner Blaskapelle im Zentrum
              Inpuls engagieren wir uns als Festwirt. Unter der bewährten Leitung von
              Armin Schmid bewirten wir die bies auf den letzten Platz gefüllte
              Inpulshalle.

        \item 27. und 28. April\\
              An der Dorfchilbi betreiben wir unseren Chilbistand.

        \item 12. Mai\\
              Anlässlich des Muttertages halten wir in der Kirche ein kleines Konzert.
              Anschliessend geben wir vor der Kirche ein Ständchen.

        \item 30. Mai\\
              Wir umrahmen den Fronleichnam-Feldgottesdienst auf dem Schulhausplatz
              musikalisch. Anschliessend führen wir die Prozession zur Kirche an.

        \item 9. Juni\\
              Geburtstagsständchen für die Jubilaren im Inpuls.

        \item 15. Juni\\
              Nach langem Warten kommen die Firmlinge doch noch aus der Kirche. Wir
              halten für sie ein Ständchen.

        \item 28.-30. Juni, Reise ins Südtirol\\
              Bei Regen führt fahren wir mit einem Car von
              Estermann Reisen Richtung Arlberg, Pfunds, Reschenpass nach Rentsch bei
              Bozen. Zwischen Zimmerbezug und Nachtessen blieb noch Zeit für einen
              Sprung in den Pool, die jüngeren versuchten es mitsamt Gartenstuhl.
              Am Samstagvormittag besuchen wir ein Weingut in
              Kaltern. Am Nachmittag vergnügen wir uns mit Wandern, Spazieren und
              Baden. Am Abend halten wir ein Kurkonzert in Eppan, das zum Teil vom
              Wind stark verweht wurde. Am Sonntag fahren wir via St. Maria, wo wir
              noch ein Frühschoppenkonzert halten, nach Hildisrieden.

        \item 25. und 26. Okt.\\
              Wir haben viele Besucher am Lotto im Löwensaal.

        \item 16. und 17. Nov.\\
              Service am Kirchenchorkonzert.

        \item 6. Dez., Chlaushock\\
              Nach der Gesamtprobe mit unseren ehemaligen Musikkameraden
              waren alle zum gemeinsamen Nüssli- und Mandarindli-Essen im roten Löwen
              eingeladen.


    \end{itemize}

\end{history}
