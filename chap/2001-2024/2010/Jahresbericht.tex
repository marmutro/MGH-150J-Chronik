\subsection*{2010}
\begin{history}


    \begin{itemize}

        \item 2. Januar, Zunftbot\\
              Wie jedes Jahr eröffnen wir unser Vereinsjahr mit dem
              traditionellen Zunftbot. Thomas Estermann, Buurehof heisst der
              neue Regent der Fasnacht 2010. Mit einem kleinen Ständli
              gratulierten wir dem neuen Zunftmeister.

        \item 15.16. Januar, Jahreskonzerte\\
              Mit dem Motto \enquote{Filmmusik} hat die Musikkommission ein
              interessantes Programm ausgearbeitet und einstudiert. Mit
              passenden Filmsequenzen von der entsprechenden Hollywoodproduktion
              wurde unsere Musik noch optisch untermalt. Auch in der Bar wurde
              anschliessend noch aus dem Vollen geschöpft.

        \item 6. März, Delegiertenversammlung LKBV und Skitag\\
              In die Vereinsgeschichte ist die Delegiertenversammlung 2010 in
              Nottwil eingegangen. Die Delegierten der Luzerner Musiksektionen
              haben uns zur Durchführung des Musiktages 2013 einstimmig
              zugestimmt. Der OK-Präsident Jakob Estermann kann nun definitiv
              mit dem Zusammenstellen des OK's beginnen. Eine wahre
              Herausforderung haben wir uns da aufgeladen.

        \item 14. März, Jubilaren-Apérokonzert\\
              Wie jedes Jahr fand im Foyer Inpuls vor zahlreichen Jubilarinnen
              unser Gratulationsständli statt

        \item 21. März, Veteranentagung LKBV\\
              Die gut organisierte Tagung fand ohne nennenswerte Unsicherheiten
              in der Inpulshalle statt. Ein besonderer Dank gilt dem OK unter
              der Leitung von Beat Koller, welcher dem Verein nach der
              Schlussabrechnung noch einen Reingewinn der Vereinskasse
              überweisen konnte.

        \item 11. April, Weisser Sonntag\\
              Mit einer Prozession begleiteten wir die Kinder zur Kirche. Nach
              einer ausgiebigen Kaffeepause im Löwen gratulierten wir den
              Erstkommunikanten mit ein paar brassigen Klängen.

        \item 23./24./25. April, Chilbi\\
              Endlich haben wir es geschafft, eine Chilbi durchzuführen, nach
              welcher einen gleichmässigen Gewinn den Vereinen ausbezahlt werden
              konnte. 5'000.- Franken ist eine schöne Summe, wenn man denkt,
              dass die letzten Jahre von einem Defizit überschattet waren. Nun
              müssen wir daran arbeiten, am Konzept festzuhalten und das Optimum
              herauszuholen.

        \item 13. Mai, Muttertag\\
              Bei viel Sonnenschein und einem unterhaltsamen Muttertagskonzert
              gratulierten wir den Hildisrieder Müttern.

        \item 30. Mai, Firmung\\
              Den frisch Gefirmten gratulierten wir mit einem Ständli.

        \item 3. Juni, Fronleichnahm\\
              Der schlechten Witterung entsprechend fand keine Prozession statt.
              Stattdessen unterhielten wir die Dorfbevölkerung mit einem Ständli
              unter dem Inpulsdach beim Apero. Anschliessend waren alle zum
              Mittagessen bei Risotto von der Kirchgemeinde eingeladen.

        \item 6. Juni, Vorbereitungskonzert\\
              In Oberkirch fand das Vorbereitungskonzert für das Kant. Musikfest
              in Wilisau statt. Im gut gefüllten Saal fand der erste Ernstfall
              statt.

        \item 12./13. Juni, Kant. Musikfest Willisau\\
              Die Musikgesellschaft Hildisrieden nahm am vergangenen Samstag, 12. Juni
              2010 am kantonalen Musikfest in Willisau teil. Sie erreichte  in der
              Kategorie 2. Klasse Brass Band leider nur den 20. Platz, genoss aber das
              Fest trotzdem in vollen Zügen.

              Am Samstag  war es endlich soweit. Das lang ersehnte und intensiv
              vorbereitete Musikfest stand vor der Tür und die Musikgesellschaft
              Hildisrieden reiste mit ihren zwei charmanten Ehrendamen los. In
              Willisau angekommen, wurde die MGH von der Begleitperson begrüsst und
              zum Fototermin gebracht, wo sich alle von ihrer Schokoladenseite zeigten
              - oder dies zumindest versuchten.

              Mit genügend Kohlenhydraten auf zur Marschmusik

              Gemäss Anweisung vom Dirigenten Michael Rösch genoss die MGH im
              Restaurant Schlossfeld eine ausgiebige Portion Spaghetti. Das Wort
              ausgiebig schienen die einen sehr ernst zu nehmen, wodurch sie bei der
              Marschmusik mit einigen Fortbewegungsproblemen zu kämpfen hatten. Die
              Musikgesellschaft Hildisrieden hatte die Ehre den Marschmusikwettbewerb
              mit dem Marsch «Fanfares en Fête» eröffnen zu dürfen. Das etwas zu
              schnelle Schritttempo bekamen die Musikanten/in mit den etwas kürzeren
              Beinen ziemlich zu spüren. Schlussendlich  erreichten sie 45.5 von 60
              Punkten und somit den 27. Rang von 36 Mitmarschierenden Vereinen.

              «La Grande Finale»

              Der Weg von der Marschmusikstrecke zum Vortragslokal war ziemlich
              «happig», da dieser ca. 300 Treppenstufen beinhaltete. Nachdem dieser
              Weg bewältigt und der Mineralstand beinahe ausgetrunken war, traf man
              sich um 15.40 Uhr zum Einspielen. Eine Stunde danach ging es los. Auf
              einer ziemlich warmen Bühne präsentierte die Musikgesellschaft
              Hildisrieden der Jury und dem anwesenden Publikum als erstes das
              Pflichtstück «Symphonic Contrasts» von Etienne Crausaz und anschliessend
              ihr Selbstwahlstück «The Prizewinners» von Philip Sparke. Die Musikanten
              und der Dirigent waren mit der Leistung zufrieden und hofften auf ein
              gutes Resultat.

              Leider nichts mit «Prizewinners»

              Anschliessend widmete sich die MGH einem weiterem Können – dem
              «Fäschte». So genossen die Musikanten einige gesellige Stunden, die
              einen beim gemütlichen Beisammensein, andere beim verfolgen des
              WM-Matches und die Jungmannschaft am unermüdlichen herumhüpfen vor der
              Bühne zu irischer Musik.

              Um Mitternacht startete die Rangverkündigung, wo die MGH auf dem
              überraschenden 20. Rang von 26. Mitwerbern aufgelistet war. Nun warten
              die Hildisrieder gespannt auf den Jury-Bericht, um nachvollziehen zu
              können, wies es «nur»  für den 20. Rang gereicht hat.

              Trotz diesem Resultat wurde natürlich bis tief in die Nacht gefeiert und
              in Gedanken die Rangliste ganz einfach umgedreht, um die MGH auf dem 7.
              Rang zu sehen und sich somit trotzdem als kleine «Prizewinner-lis» zu
              fühlen…

              Die Musikgesellschaft Hildisrieden bedankt sich bei allen Freunden,
              Familienmitgliedern und Fans, die sich am vergangenen Samstag Zeit
              genommen haben, sie an der Marschmusikstrecke, am Vortrag und beim
              Festen zu unterstützen.

              Gratulation an unsere treuen Seelen

              Ein weiteres Ereignis des Kantonalen Musikfestes in Willisau war die
              Veteranenehrung. Seit 30 Jahren musizieren Armin Schmid und Beat Koller
              aktiv in der MG Hildisrieden mit und wurden am Sonntag für diese
              langjährige Treue geehrt und zu Veteranen gekrönt. Die Musikanten und
              Musikantinnen gratulieren ihren Kantonalen Veteranen von ganzem Herzen
              und wünschen den beiden Herren beim Ausüben Ihres Hobbys weiterhin alles
              Gute.


        \item 19. Juni, Hochzeit Martin und Nadine Aregger\\
              Bei nicht ganz trockener Witterung gratulieren wir unserem
              Bassisten Martin Aregger alias Gegge und seiner Ehefrau Nadine zur
              Vermählung. Wir durften den Apéro im Schulhaus Beromünster
              musikalisch untermalen. Für den finanziellen Zustupf in die
              Vereinskasse danken wir dir herzlich.

        \item 25. Juni, Abschlusshock\\
              Der ad hok organisierte Abschlussevent, organisiert vom
              Perkussionsregister fand in der Schüür Hildisrieden statt. Bei
              Bier, Grill und Dessert vom Haldehof haben wir das vergangene
              Musikfest revue passiert und über lustige Erlebnisse lachen
              können. Den Perkussionisten herzlichen Dank.

        \item 3. Juli, Schriftl. Rücktritt von Michael Rösch\\
              Bedauerlicherweise entsprach die Art und Weise seines Abganges
              nicht unseren Vorstellungen. Das bevorstehende Kirchenkonzert
              konnten wir nun zum Abschiedskonzert umbenennen.

        \item 19. September, Bettag \& Abschiedskonzert\\
              Bei guter Witterung begleiteten wir mit dem Kirchenchor Rain den
              Berghofgottesdienst in Gundolingen. Nach dem anschliessenden
              Unterhaltungskonzert durften wir uns noch Stärken \enquote{met
                  Chäs, Brot ond Moscht, frösch ab Press}.

              Am Abend luden wir in die Kirche zum Abschlusskonzert vom Michael
              Rösch. Zur Abwechslung konnten wir den Jazzchor Not2sale aus
              Kriens dazu ziehen, um das Konzert ein bisschen kreativer zu
              gestalten. Dank den Inserenten und Konzertstückspendern konnten
              wir auch finanziell mit einem Erfolg abschliessen und sogar dem
              Chor einen symbolischen Beitrag überweisen. Beim anschliessenden
              Nachtessen im Löwen konnte aus der Türkollekte die erste
              Getränkerunde übernommen werden. Somit schliesst sich der Kreis
              mit der MGH unter der Leitung von Michael Rösch.

        \item 22./23. Oktober, Lotto\\
              Aus dem Misserfolg vom vergangenen Jahr haben wir Lehren gezogen
              und die Flyer etwas früher verschickt. Was hatte das für Folgen:
              Der Freitag war ein Lottomatch zum Aufwärmen, denn am Samstag
              platzt der Löwen aus allen Nähten. Der letzte Tisch wurde aus dem
              Keller geholt, die Besucher waren nicht mehr zu bremsen, bis wir
              die Notbremse zogen und die letzten Besucher wieder nachhause
              schickten. Hoffentlich kommen sie das nächste Jahr wieder.

        \item 30. November, Dirigentenwahl\\
              Nach fünf erfolgreichen Stellproben mit den Dirigenten Simon
              Gertschen, Jan Wyss, Philipp Gisler, Pascal Maillard und Richard
              Eicher konnten wir Philipp Gisler zum neuen Dirigenten der MGH
              wählen. Mit viel Zuversicht erwarteten wir den 14. Dezember 2010,
              bis Philipp den Verein endgültig übernahm. Bis zu diesem Zeitpunkt
              leitete Max das Korps und bereitete uns mit viele Engagements vor.
              Max, dir noch einmal herzlichen Dank.

        \item 5. Dezember, Jubilaren-Apérokonzert\\
              Unter der Leitung vom Max gratulierten wir den Jubilaren der
              zweiten Jahreshälfte.

    \end{itemize}

\end{history}
