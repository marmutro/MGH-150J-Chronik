\subsection*{2010}
\begin{history}


    \begin{itemize}

        \item 2. Januar, Zunftbot\\
              Wie jedes Jahr eröffnen wir unser Vereinsjahr mit dem traditionellen
              Zunftbot. Thomas Estermann, Buurehof heisst der neue Regent der Fasnacht
              2010. Mit einem kleinen Ständli gratulierten wir dem neuen Zunftmeister.

        \item 15.16. Januar, Jahreskonzerte\\
              Mit dem Motto \enquote{Filmmusik} hat die Musikkommission ein
              interessantes Programm ausgearbeitet und einstudiert. Mit passenden
              Filmsequenzen von der entsprechenden Hollywoodproduktion wurde unsere
              Musik noch optisch untermalt. Auch in der Bar wurde anschliessend noch
              aus dem Vollen geschöpft.

        \item 6. März, Delegiertenversammlung LKBV und Skitag\\
              In die Vereinsgeschichte ist die Delegiertenversammlung 2010 in Nottwil
              eingegangen. Die Delegierten der Luzerner Musiksektionen haben uns zur
              Durchführung des Musiktages 2013 einstimmig zugestimmt. Der OK-Präsident
              Jakob Estermann kann nun definitiv mit dem Zusammenstellen des OK's
              beginnen. Eine wahre Herausforderung haben wir uns da aufgeladen.

        \item 14. März, Jubilaren-Apérokonzert\\
              Wie jedes Jahr fand im Foyer Inpuls vor zahlreichen Jubilarinnen unser
              Gratulationsständli statt

        \item 21. März, Veteranentagung LKBV\\
              Die gut organisierte Tagung fand ohne nennenswerte Unsicherheiten in der
              Inpulshalle statt. Ein besonderer Dank gilt dem OK unter der Leitung von
              Beat Koller, welcher dem Verein nach der Schlussabrechnung noch einen
              Reingewinn der Vereinskasse überweisen konnte.

        \item 11. April, Weisser Sonntag\\
              Mit einer Prozession begleiteten wir die Kinder zur Kirche. Nach einer
              ausgiebigen Kaffeepause im Löwen gratulierten wir den Erstkommunikanten mit
              ein paar brassigen Klängen.

        \item 23./24./25. April, Chilbi\\
              Endlich haben wir es geschafft, eine Chilbi durchzuführen, nach welcher
              einen gleichmässigen Gewinn den Vereinen ausbezahlt werden konnte.
              5'000.- Franken ist eine schöne Summe, wenn man denkt, dass die letzten
              Jahre von einem Defizit überschattet waren. Nun müssen wir daran
              arbeiten, am Konzept festzuhalten und das Optimum herauszuholen.

        \item 13. Mai, Muttertag\\
              Bei viel Sonnenschein und einem unterhaltsamen Muttertagskonzert
              gratulierten wir den Hildisrieder Müttern.

        \item 30. Mai, Firmung\\
              Den frisch Gefirmten gratulierten wir mit einem Ständli.

        \item 3. Juni, Fronleichnahm\\
              Der schlechten Witterung entsprechend fand keine Prozession statt.
              Stattdessen unterhielten wir die Dorfbevölkerung mit einem Ständli
              unter dem Inpulsdach beim Apero. Anschliessend waren alle zum
              Mittagessen bei Risotto von der Kirchgemeinde eingeladen.

        \item 6. Juni, Vorbereitungskonzert\\
              In Oberkirch fand das Vorbereitungskonzert für das Kant. Musikfest in
              Wilisau statt. Im gut gefüllten Saal fand der erste Ernstfall statt.

        \item 12./13. Juni, Kant. Musikfest Willisau\\
              Gut vorbereiten und mit viel Zuversicht reisten wir mit dem Car nach
              Willisau, das das der letzte Auftritt unter der Leitung von Michael
              Rösch war, wussten zu diesem Zeitpunkt nur die beiden Präsidenten.

              Ohne unter Druck der Titelverteidigung zu stehen, konnten wir das geübte
              der Fachjury vortragen. Sichtlich zufrieden warteten wir auf die
              Rangierung und würden enttäuscht. In der hinteren Hälfte der Rangliste
              war die MGH zu finden aber zum Glück noch vor Beromünster.

              Am Sonntag schafften es doch noch diverse Musikantinnen, an der
              Veteranenehrung unseren Kant. Veteranen Armin Schmid und Beat Koller
              teilzunehmen.

        \item 19. Juni, Hochzeit Martin und Nadine Aregger\\
              Bei nicht ganz trockener Witterung gratulieren wir unserem Bassisten
              Martin Aregger alias Gegge und seiner Ehefrau Nadine zur Vermählung. Wir
              durften den Apéro im Schulhaus Beromünster musikalisch untermalen. Für
              den finanziellen Zustupf in die Vereinskasse danken wir dir herzlich.

        \item 25. Juni, Abschlusshock\\
              Der ad hok organisierte Abschlussevent, organisiert vom
              Perkussionsregister fand in der Schüür Hildisrieden statt. Bei Bier,
              Grill und Dessert vom Haldehof haben wir das vergangene Musikfest revue
              passiert und über lustige Erlebnisse lachen können. Den Perkussionisten
              herzlichen Dank.

        \item 3. Juli, Schriftl. Rücktritt von Michael Rösch\\
              Bedauerlicherweise entsprach die Art und Weise
              seines Abganges nicht unseren Vorstellungen. Das bevorstehende
              Kirchenkonzert konnten wir nun zum Abschiedskonzert umbenennen.

        \item 19. September, Bettag \& Abschiedskonzert\\
              Bei guter Witterung begleiteten wir mit dem Kirchenchor Rain den
              Berghofgottesdienst in Gundolingen. Nach dem anschliessenden
              Unterhaltungskonzert durften wir uns noch Stärken \enquote{met Chäs, Brot ond
                  Moscht, frösch ab Press}.

              Am Abend luden wir in die Kirche zum Abschlusskonzert vom Michael Rösch.
              Zur Abwechslung konnten wir den Jazzchor Not2sale aus Kriens dazu
              ziehen, um das Konzert ein bisschen kreativer zu gestalten. Dank den
              Inserenten und Konzertstückspendern konnten wir auch finanziell mit einem
              Erfolg abschliessen und sogar dem Chor einen symbolischen Beitrag
              überweisen. Beim anschliessenden Nachtessen im Löwen konnte aus der
              Türkollekte die erste Getränkerunde übernommen werden. Somit schliesst
              sich der Kreis mit der MGH unter der Leitung von Michael Rösch.

        \item 22./23. Oktober, Lotto\\
              Aus dem Misserfolg vom vergangenen Jahr haben wir Lehren gezogen und die
              Flyer etwas früher verschickt. Was hatte das für Folgen: Der Freitag war
              ein Lottomatch zum Aufwärmen, denn am Samstag platzt der Löwen aus allen
              Nähten. Der letzte Tisch wurde aus dem Keller geholt, die Besucher waren
              nicht mehr zu bremsen, bis wir die Notbremse zogen und die letzten
              Besucher wieder nachhause schickten. Hoffentlich kommen sie das nächste
              Jahr wieder.

        \item 30. November, Dirigentenwahl\\
              Nach fünf erfolgreichen Stellproben mit den Dirigenten Simon Gertschen,
              Jan Wyss, Philipp Gisler, Pascal Maillard und Richard Eicher konnten wir
              Philipp Gisler zum neuen Dirigenten der MGH wählen. Mit viel Zuversicht
              erwarteten wir den 14. Dezember 2010, bis Philipp den Verein endgültig
              übernahm. Bis zu diesem Zeitpunkt leitete Max das Korps und bereitete
              uns mit viele Engagements vor. Max, dir noch einmal herzlichen Dank.

        \item 5. Dezember, Jubilaren-Apérokonzert\\
              Unter der Leitung vom Max gratulierten wir den Jubilaren der zweiten
              Jahreshälfte.

    \end{itemize}

\end{history}
