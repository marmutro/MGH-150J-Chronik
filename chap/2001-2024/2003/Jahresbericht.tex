\subsection*{2003}

\begin{history}

    \begin{itemize}

        \item 2. Jan., Zunftbot\\
              Wie jedes Jahr eröffnen wir unser Vereinsjahr mit dem traditionellen
              Zunftbot Auf Herbert der 1. folgt Sepp der 8. Wir gratulieren dem Bauer
              vom Bäseris musikalisch zur Wahl als Zunftmeister 2003.

        \item 11. und 12. Jan., Jahreskonzerte\\
              Unter dem speziellen Motto \enquote{Musig för ond met Gescht} haben wir
              ein abwechslungsreiches Konzert unter der Leitung von Kobi Banz
              einstudiert.

        \item 2. März, Fasnachtsumzug Hildisrieden\\
              Bei trübem Wetter nahmen wir auch dieses Jahr unter dem Motto „s’Roma
              brennt“ am Umzug teil. Unter der initiativen Leitung von Toni Bachmann
              ist eine schöne, originelle Wagennummer entstanden.

              Toni besten Dank für deinen Einsatz.

        \item 9. März, Geburtstagsständli Pfarrer Hauser.\\
              Anlässlich des 70. Geburtstag von Pfarrer Josef Hauser überraschten wir
              ihn mit einem Ständchen im Löwen. Anschliessend lud uns die
              Kirchengemeinde zu einem Apéro ein.

        \item 6. April, Jubilarenständli\\
              Im Foyer Inpuls

        \item 25. April, Weisser Sonntag \& Chilbi\\
              Bei guter Witterung begleiteten wir die Weiss-Sontags Kinder beim Einzug
              durch das Dorf.

              Anschliessend an den Festgottesdienst gratulierten wie den Kindern mit
              einem Ständli.

              An der Chilbi versuchten wir ein weiteres Mal unsere Vereinskasse mit
              Bogenschiessen aufzubessern.

        \item 11. Mai, Muttertag\\
              Bei gutem Wetter gratulierten wir den Müttern mit einigen konzertanten
              Stücken zum Muttertag. Dadurch hatte Max doch noch die Möglichkeit, sein
              Solo der Dorfbevölkerung vorzutragen.

        \item 24. Mai, Luz. Kant. Musiktag in Grosswangen\\
              Bei hervorragendem Wetter reisten wir mit zwei Bussen vom „Froschaugen-
              Winu“ nach Grosswangen. Nach dem Einspielen warteten wir gut vorbereitet
              auf den Vortrag von unserem Selbstwahlstück \enquote{Schattdorf
                  Impressionen}. Nach dem nicht allzu schlechten Jurybericht von Philippe
              Bach stand nur noch die Marschmusik bei 25 Grad vor der Türe: Mit dem
              bekannten \enquote{Bundesrat Gnägi Marsch} erreichten wir mit 48.3
              Punkten den zufrieden stellenden 42. Rang. Nach dem ersten Bier,
              offeriert von der Begleitperson, bedankten wir uns mit der
              Zweitaufführung vom Marsch. Mit viel Bier, Kafi Luz und einer Hochbar
              beendeten wir den Musiktag mit einem beträchtlichen Höch und einer
              amüsanten Busfahrt zurück nach Hildisrieden.

        \item 14. Juni,  Firmung\\
              Den frisch gefirmten Firmlinge gratulierten wir mit einem Ständchen.

        \item 19. Juni, Fronleichnam\\
              Bei strahlendem Sonnenschein begleitet eine Kleinformation die Messe auf
              dem Schulhausplatz. Anschliessend begleiteten wir die Dorfbevölkerung
              mit einer Prozession zum Kirchenplatz für den Schlusssegen. Zum Schluss
              spielten wir noch ein Ständchen.

        \item 20. Juni\\
              Saisonschluss mit anschliessendem Bräteln

              Nach einer kurzen Probe mit eingefügtem Theorieteil fuhren wir mit den
              PW's in den Traselingerwald. Auf sehr guter Glut grillierte unser
              Fähnrich die SEG-Poulets und die Schläuche zu unserem Wohlergehen. Auf
              Wein und Bier gab es für wenige nur noch Herrgöttli bis um Morgen um 4.

        \item 6. September\\
              Begleitung der Abendmesse. Mit ein paar einfühlsamen und rhythmisch
              schwierigen Stücken begleiteten wir die Samstagabendmesse.

        \item 22. September\\
              100 Jahr Ernst Küng. Als endlich alle da waren, konnten wir unserem
              Jubilar aufspielen. Es wird schliesslich nicht jeder 100 Jahre alt. Nach
              dem offiziellen Teil wurde uns ein reichhaltiger Apero serviert. Vielen
              Dank den Organisatoren Strassengenossenschaft Waldmatt und der Gemeinde
              Hildisrieden

        \item 24./25. Oktober, Lotto\\
              Wir sind es uns langsam gewöhnt, dass unser Kassier nach dem Lotto eine
              erfolgreiche Abrechnung präsentieren kann.

        \item 21./22. November Service Theater Trachtengruppe\\
              Für den reibungslosen Service der Theaterbesucher waren wir heuer
              verantwortlich.

        \item 30. November, Jubilarenständchen\\
              Wir gratulierten mit einem Ständli im Foyer den
              Jubilaren.

        \item 5. Dezember, Chlaushock\\
              Nach strenger Probe war das gemütliche Beisammen sein mit
              Nüssli, Schoggi und Manderindli genau das richtige.

        \item 21. Dezember\\
              In zwei Gruppen aufgeteilt besuchten wir verschiedene Quartiere und
              brachten musikalische Weihnachtsstimmung mit.

              Als wir zum Schluss noch den Dorfkern beglückten war der offerierte
              Glühwein vom Präsi noch genau das richtige, was uns passieren konnte.

    \end{itemize}

\end{history}
