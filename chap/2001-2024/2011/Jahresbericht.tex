\subsection{2011}
\begin{history}


    \begin{itemize}

        \item 2. Januar, Zunftbot\\
              Wie jedes Jahr eröffneten wir unser Vereinsjahr mit dem traditionellen
              Zunftbot. Robert Emmenegger hiess der neue Regent der Fasnacht 2011. Mit
              einem kleinen Ständli gratulierten wir dem neuen Zunftmeisterpaar.

        \item 15./16. Januar, Jahreskonzerte\\
              Mit dem Motto \enquote{Amerika} hat die Musikkommission ein unterhaltsames
              Programm ausgearbeitet. Ein bekanntes Slück nach dem anderen sorgte für
              gute Stimmung in der Inpuls-Halle. Das romantische Duett \enquote{Träne} von
              Florian Ast alias Stephan Schneider und Francine Jordi alias Dario
              Fleischli sorgte da und dort für feuchte Augen. Anschliessend
              diskutierten die Gäste und Musikanten in der Bar bis tief in die Nacht
              über das Gesehene und Gehörte und rissen sich um ein Interview mit
              unserer Francine.

        \item 20. März, Jubilaren-Aperokonzert\\
              Wie jedes Jahr hatten die etwas jüngeren Musikanten auch diesmal mit dem
              frühen Aufstehen zu kämpfen, um im Foyer Inpuls zahlreichen Jubilarinnen
              und Jubilaren auf musikalische Art zum hohen Geburtstag zu gratulieren.

        \item 25. März, Generalversammlung\\
              Die 137. Generalversammlung fand im Restaurant Chrüz statt. Gemeinsam
              wurde auf das Jahr 2010 zurückgeblickt. Der OK-Chef des Musiktages 2013
              Jakob Estermann war ebenfalls anwesend und berichtete über den aktuellen
              Stand der Vorbereitungen.

        \item 29./30. April/1. Mai, Chilbi\\
              Nach einer erfolgreichen Chilbi im 2010 folgte bereits der nächste
              Tiefflug im Jahre 2011. Leider konnte wieder einmal mehr kein Gewinn für
              die Vereine verteilt werden. Trotzdem sorgte eine Kleinformation der MGH
              für gute Stimmung und amüsante Unterhaltung - Dank sei dem Weisswein.

        \item 8. Mai, Weisser Sonntag \& Muttertag\\
              Mit einer Prozession begleiteten wir die Kinder zur Kirche. Nach einer
              gemütlichen Kaffeepause im Löwen gratulieren wir den Erstkommunikanten
              mit einigen brassigen Klängen.

        \item 29. Mai, Firmung\\
              Den frisch Gefirmten gratulierten wir mit einem abwechslungsreichen
              Ständli.

        \item 3. Juni, Fronleichnam\\
              Bereits stand das nächste Ständli auf dem Programm. Einige Musikanten
              spielten bereits während dem Gottesdienst, nachdem wir von der Waage in
              die Kirche marschiert sind. Anschliessend begleitete die MGH den Apero
              im Inpuls, wo es darauf noch ein von der Kirchengemeinde offeriertes
              Mittagessen gab.

        \item 10. Juni, Vorbereitungskonzert\\
              Im Zentrum Inpuls führte die MGH mit der Musikgesellschaft Römerswil und
              der Harmonie Rain ein Vorbereitungskonzert für das eidgenössische
              Musikfest in St. Gallen durch. Die Festwirtschaft mit den berühmten
              Spiessli sowie die Ronspatzen sorgten für gute Unterhaltung.
              Anschliessend stand die Bar für ein gemütliches Ausklingen des Abends
              zur Verfügung.

        \item 18./19. Juni, Eidg. Musikfest St. Gallen\\
              Aufgrund einiger technischer Probleme bei den Organisatoren, musste die
              MGH ihr Können am Sonntag als letzter Verein zum Besten geben. Beim
              Konzert kassierten wir 161.67 Punkte von 200 ab und bei der Marschmusik
              waren es 82 von 100. Da das Fest bereits um 20.00 Uhr zu Ende war,
              musste im Car noch weitergefeiert werden. Fazit: Minus ein Bildschirm
              und Suche eines neuen Carunternehmens.

        \item 24. Juni, Abschlusshöck\\
              Der Abschlussevent, organisiert vom Bariton- und Euphoniumregister, fand
              im Partyraum in Ohmelingen statt. Bei köstlichen Grilladen wurde über
              die MGH und viel anderes gelacht und diskutiert. Der Höhepunkt war das
              gigantisch, schöne Feuerwerk in der Nacht. Diese Register müssen Geld
              haben!? Besten Dank bereits heute für das Hauptsponsoring am Musiktag
              2013.

        \item 17. September, Hochzeit Petra und Chregu Emi \& Festakt
              Bogenhüsli\\
              Nach dem Spalierstehen im Gormund bei schönstem Sonnenschein
              gratulierten wir unserem Präsi und seiner Petra mit einem Ständli zur
              Hochzeit. Anschliessend stärkte sich die MGH beim grosszügigen Apero.
              Darauf zogen die Musikanten weiter ins Bogenhüsli, wo der neue
              Sportplatz eingeweiht wurde. Ebenfalls durch ein Ständli und
              anschliessend durch die Ronspatzen sorgte man für musikalische
              Unterhaltung.

        \item 23. September, Gadenkonzert\\
              Zum ersten Mal fand das Herbstkonzert in einer Scheune statt. Mit
              urchigen Klängen und muhenden Kühen im Hintergrund durften wir im
              Haldenhof ein einmal anderes Auftrittsfeeling erleben. Dank unseren
              Liebsten, welche mit uns serviert und grilliert haben, blicken wir auf
              einen erfolgreichen Abend zurück.

        \item 21./22. Oktober, Lotto\\
              Leider hat dieses Jahr etwas mit dem Inserat in der Barnipost nicht
              geklappt. Eventuell ist dies ein Grund, wieso wir wesentlich weniger
              Besucher hatten. Trotzdem verkauften wir fleissig Karten und verteilten
              die schön hergerichteten Preise an die Gewinner. Vor allem unsere
              Ehrendame Sylvia Schurtenberger wird sich freuen, da Ihr Freund die
              Hauptpreis-Ferien gewonnen hat.

        \item 4. Dezember, Jubilaren-Apérokonzert\\
              Wiederum an einem frühen Sonntagmorgen gratulierte die MGH den Jubilaren
              der zweiten Jahreshälfte mit einer Rose und einigen musikalischen
              Klängen zu ihren Geburtstagen. Natürlich sahen auch an diesem
              Adventssonntag nicht alle gleich frisch aus der Wäsche. Was man nicht
              alles für die gute MGH auf sich nimmt....


    \end{itemize}

\end{history}
