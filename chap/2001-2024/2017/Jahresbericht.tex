\subsection*{2017}
\begin{history}

      \begin{itemize}

            \item 14./15. Januar Jahreskonzerte\\
                  Vor gut gefüllten Rängen konnte die MGH ihre Jahreskonzerte unter
                  dem Motto „Nordische Momente“ aufführen. Die musikalischen
                  Leistungen, welche die Musikanten zusammen mit dem Dirigenten
                  erbrachten, hat das Publikum beeindruckt. Viele Gratulationen
                  konnten entgegengenommen werden. Die Musikanten und Musikantin
                  wussten dies zu schätzen. Natürlich wurde auch dieses Jahr die Bar
                  nicht umsonst aufgerichtet, so konnten die Erfolge und teilweise
                  gut getarnten Misserfolge mit einem spätnächtlichen Bier oder
                  Schnaps gefeiert werden. Jedoch mahnte zuvor der Präsident zur
                  Vorsicht, da am Sonntagmittag ein Fotoshooting geplant war. Den
                  meisten ist dies relativ gut gelungen.

            \item 12. Februar Einweihung\\
                  Pastoralraum Inmitten der Fasnachtsferien konnte die MGH ein
                  ehrenvolles Engagement erbringen. Zur Einweihung des
                  Pastoralraumes in Sempach, umrahmte die Musikgesellschaft
                  Hildisrieden die Feier musikalisch. Nach einem zeitlich leicht
                  überzogenen Gottesdienst, konnten die Musikanten und Musikantin,
                  die Festgemeinde mit einem Marsch zur Festhalle führen. Einigen
                  hat jedoch das Warten vor der Kirche in der kalten Bise zugesezt.
                  So soll dem Stabführer der rechte Arm eingefroren sein. Eindeutig
                  zu sehen war dies an den Bewegungen zu den Kommandos oder eben
                  nicht. Unten angekommen und sauber abgeschlossen, wurde auf der
                  Bühne der Festhalle das restliche Konzert aufgeführt. Das Publikum
                  wusste den Auftritt zu schätzen und zeigte dies mit reichlich
                  Applaus.

            \item 4. März Musikantenskirennen Sörenberg\\
                  Am ersten Märzwochenende fand das alljährliche Musikantenskirennen
                  im Sörenberg statt. Unter den zahlreichen Entlebucher Musikanten,
                  waren auch einige Hildisrieder unterwegs. Der Vormittag wurde ganz
                  der Rennvorbereitung gewidmet. Da die Witterungsverhältnisse nicht
                  den Wünschen entsprach, wurde viel Wert auf die Mentale
                  vorbereitung im Restaurant Schwand gelegt. Mit Kaffee Luz und
                  einem grossen mocken Fleisch konnte der Rennhang kurz nach Mittag
                  von allen Teilnehmer bewältigt werden. Deren acht waren
                  Hildisrieder. Unten im Ziel angekommen gas wiederum Kafischnaps
                  und perfekte Stimmung für ein Skifest. Dieses wurde anschliessend
                  im Gade neben dem Parkplatz zusammen mit der Rangverkündigung
                  ghörig gefeiert. Und unter dem Motto „die schnellsten sind nicht
                  immer die Besten“ haben die Hildisrieder auf ihren Triumph
                  angestossen.

            \item 7. März Tonis 70er\\
                  An der ersten Gesamtprobe nach den Fasnchtsferien, überraschte uns
                  Toni nach der Probe mit einem prächtigen Fleischplättli. Er
                  scheute keinen Aufwand und servierte uns ein perfektes Menue. Auch
                  das Bier währenddessen und an den Kaffe danach hat er besorgt. Der
                  Vorstand war sich sofort einig. Es sollten und dürfen noch mehr
                  Musikanten bis 70 im Verein bleiben.

            \item 12. März Jubilarenkonzert\\
                  Das mittlerweile halbjährliche gewohnt gut besuchte
                  Jubilarenkonzert, bot auch im März 2017 ein abwechslungsreiches
                  und unterhaltsames Programm. Die Jubilaren waren sichtlich erfreut
                  und verdankten dies mit Applaus. In den Reihen der Musikanten
                  blieben jedoch zwei Stühle unbesetzt. Der eine kündigte dies
                  bereits sonntagmorgens um 5 Uhr früh auf dem Heimweg vom Ausgang
                  seinen Solo Cornet- Kollegen mit, dass es knapp werden könnte. Der
                  andere leere Platz war im Nachbarsregister zu finden. Dort schien
                  die Lust nach Schnee und einem wilden Pistenritt grösser zu sein,
                  als ein gemütliches Konzertchen mit den Senioren und
                  anschliessendem Apero.

            \item 23. April Weisser Sonntag\\
                  Wie es die Tradition will, begleiteten wir die Erstkommunikanten
                  mit dem Prozessionsmarsch durchs Dorf zur Kirche. Während des
                  Gottesdienstes konnten sich dann die Musikanin und Musikanten im
                  Leue stärken, um anschliessend vor der Kirche ein Ständli zum
                  Besten zu geben. Da Peter kurzfristig ein Terminkonflikt mit
                  Triengen bemerkte, durfte Max das Dirigentenamt ausüben. Die Gäste
                  und Kirchgänger waren aber sichtlich erfreut an unserem Auftritt.


            \item 30. April Chilbi\\
                  Um unseren Beitrag an der Hildisrieder-Chilbi beizutragen,
                  stellten wir auch dieses Jahr einen Chilbistand. Die Betreuung des
                  Standes übernahmen die jüngsten Musikanten. Ebenfalls an
                  Musikanten gingen die Podest Plätze. So konnte sich Mäni Estermann
                  den Sieg gutschreiben lassen. Der zweite Platz ging ins Haus des
                  Vize-Präsidenten. Es wird davon ausgegangen, dass alles mit
                  rechten Dingen von statten ging.

            \item  6. Mai Hochzeit Petra und Mätthu\\
                  Erfreulicherweise wurden wir bereits im Winter zum Hochzeitsapero
                  von Petra und Mätthu eingeladen. So versammelten wir uns dann auch
                  am Samstag, den 6 Mai vor der Kapelle Mariazell in Sursee und
                  standen dem frisch vermählten Paar Spalier. Beim anschliessenden
                  Apero konnten wir unseren musikalischen Beitrag geben und Mätthu
                  konnte seine Fähigkeiten als Dirigent beweisen. Petra konnte sich
                  gänzlich auf ihren Ehemann verlassen und spielte auf der Posaune
                  munter mit. Die Musikantin und Musikanten hatten sichtlich Spass
                  den schönsten Tag des jungen Paares noch etwas schöner zu machen.
                  Obwohl noch angefügt werden muss, dass die Ronspatzen GV am Abend
                  zuvor schon noch bei einigen ihre Folgen zeigte.

            \item  28. Mai Kantonal Musiktag Schüpfheim\\
                  Einmal mehr konnte sich die MGH an einem alljährlichen Wettspiel
                  ihr Können zeigen. Nach einer intensiven und erfolgreichen
                  Vorbereitungszeit konnten wir reichlich ernten. So gelang auf der
                  Parademusikstrecke am Samstagnachmittag um 15.30 Uhr den am
                  höchsten bewertete Auftritt in der Geschichte der Musikgesellschaft
                  Hildisrieden. Mit sensationellen 53.1 Punkten wurde der Sieg mit
                  lediglich 0.2 Punkten verfehlt. Auch beim Konzertvortrag mit Lord
                  of all, hatte der Juror Thomas Rüedi neben einigen
                  Verbesserungsmöglichkeiten viel Lob zu verteilen. Insgesamt
                  gesehen waren die Auftritte im Luzerner Hinterland beste Werbung
                  für unseren Verein. Dies konnte dann auch am Abend trotz eines
                  muffigen Geruches gebührend gefeiert werden. Schon eine Woche
                  zuvor konnte unser Nachwuchs, die Brassinis überzeugen. Sie wurden
                  mit dem Prädikat sehr gut ausgezeichnet. Ebenfalls an diesem
                  Wochenende konnten Stephan Wolf, Hans Stöckli und Roli Klaus als
                  Kantonale Veteranen geehrt werden. Unser höchster Luzerner
                  Blasmusikant Christoph Troxler konnte abschliessend Fazit ziehen:
                  Gleich zu Beginn seiner Tätigkeiten als LKBV-Präsident könne er
                  nicht sogleich ganz alles zu unseren Wünschen richten. Bei der
                  FIFA gehe es schliesslich auch nicht ganz so einfach.

            \item 3. Juni Firmung\\
                  Am Pfingstsamstag den 23. Mai, konnten sich die Firmlinge und ihre
                  Angehörigen mit einem Glas Most oder Wein vor der Kirche ein
                  Ständchen der Musikgesellschaft anhören. Die Musikanten gesellten
                  sich danach ebenfalls zu ihnen und genossen den Apero.

            \item 15. Juni Fronleichnam\\
                  Wie gewohnt begleiteten wir die Erstkommunikanten mit dem
                  Prozessionsmarsch durchs Dorf bis vor die Kirche. Nach der Messe
                  zog die ganze Gesellschaft mit dem Marsch Viva Arogno ins Inpuls,
                  wo die MGH den Apero musikalisch umrahmte. Danach wurde von der
                  Kirchgemeinde allen anwesenden Risotto serviert.

            \item 17. und 18. Juni, Reise ins Allgäu\\
                  Da die letzte Musikreise schon etwas länger zurückliegt, gönnten
                  sich die Musikantin und Musikanten der Musikgesellschaft
                  Hildisrieden mit Begleitung am Wochenende vom 17. und 18. Juni
                  2017 einen Ausflug in das schöne Allgäu in Süddeutschland. Am
                  Samstag-Morgen kurz vor 8.00 Uhr versammelte sich die
                  Reisegesellschaft bei bestem Wetter vor dem Inpuls, um mit dem Car
                  der Firma Estermann Reisen AG pünktlich loszufahren.

                  Beim ersten Zwischenhalt in Walenstadt wartete bereits eine
                  Weindegustation mit einem „kalten Plättli“ auf die Gruppe. Zu
                  Ehren der kürzlich ernannten Veteranen nahmen die Ronspatzen die
                  Instrumente mit und spielten zum Ständli auf.

                  Nach dieser gemütlichen und willkommenen Pause ging es weiter nach
                  Sulzberg in Österreich. Im dortigen Restaurant Ochsen gab es ein
                  feines Mittagessen und es blieb anschliessend genügend Zeit für
                  eine Rundwanderung im schönen Gebiet mit Sicht auf die
                  Vorarlberger, wie auch auf die Schweizer Berge.

                  Eine weitere Fahrstunde entfernt lag unser Ziel in Kempten. In der
                  Brauereigaststätte zum Stift in der Innenstadt nahm die
                  Reisegruppe das Nachtessen in Form von feinen, typisch
                  allgäuischen Menus ein. Aber auch für die Ohren gab es nochmals
                  einen besonderen Schmaus. Die Ronspatzen hatten nämlich die
                  Instrumente nicht nur wegen der Veteranen dabei, sondern vor allem
                  aufgrund des kürzlich gewählten Präsidenten des Luzerner
                  Kantonalen Blasmusikverbandes, Christoph Troxler, welcher sich
                  sogar noch zu einer Soloeinlage hinreissen liess. Die gutgelaunte
                  Gruppe kostete danach voller Tatendrang das spätnächtliche Angebot
                  von Kempten aus.

                  Um die Stadt auch bei Tageslicht zu Gesicht zu bekommen, gab es am
                  Sonntagvormittag noch freie Zeit zur Begutachtung der
                  Sehenswürdigkeiten. Am Nachmittag war der Besuch des
                  Skywalk-Naturerlebnisparks in Scheidegg auf dem Reiseprogramm. Der
                  Ausblick vom 40 Meter hohen Baumwipfelpfad bot einiges und war
                  eindrücklich. Natürlich ist das rustikale Brotzeitbuffet noch zu
                  erwähnen, welches zum Abschluss aufgetischt wurde.

                  Dies war dann auch der Schlusspunkt der Musikreise 2017, bevor es
                  wieder zurück in die Schweiz nach Hildisrieden ging. An dieser
                  Stelle einen ganz grossen Dank an die beiden Organisatoren Hans
                  Stöckli (auf der Reise leider abwesend) und Toni Bachmann, welche
                  die Reise tadellos geplant und durchgeführt hatten.


            \item 6. Juli, NJBB Konzert


            \item 24. September Neuuniformierung Hochdorf\\
                  Anlässlich der Neuuniformierung der Feldmusik Hochdorf, wurden wir
                  zur Seetaler Blasmusikparade eingeladen. Bei bestem Wetter wurde
                  die Strasse durch Hochdorf gesperrt und die Musikgesellschaften
                  aus den benachbarten Gemeinden und dem Seetal konnten auf der
                  Marschmusikstrecke ihr Können zeigen, Auffällig war, dass sogleich
                  drei Formationen den selben Marsch auserwählt hatten. Das heisst,
                  vor und hinter uns wurde Viva Arogno gespielt und wird machten es
                  diesen gleich. Erstaunlicherweise wurde dies nur von wenigen
                  Zuhörern bemerkt. Unser Erfolgsmarsch vom Musiktag hat merklich an
                  Bekanntheit gewonnen. Der zweite Teil unserer Präsentation konnten
                  wir im Brauisaal vortragen. Leider wurden die Zuschauerränge erst
                  gegen Ende unserer Aufführung besetzt. Die fehlenden oder erst
                  später einterffenden Zuhörer werden sich gerauen sein. Der
                  restliche Teil des sonntäglichen Anlasses wurde mit einem Bier und
                  einem Wurst gemütlich ausgeklungen.

            \item 20. / 21. Oktober Lotto\\
                  Wie die letzten Jahre war auch im Herbst 2017 unser Lotto ein
                  Erfolg. Der Leue war wiederum sehr gut gefüllt und den
                  Lottospielern schienen die Preise zu gefallen. Scheinbar wurde
                  bereits von einer Spielerin versucht das gewonnene Gold auf dem
                  fruchtbaren Hildisriederboden zu säen um es im nächsten Herbst
                  wieder zu ernten.

            \item 3. Dezember Jubilarenständli\\
                  Auch im Spätherbst 2017 konnten die Jubilaren aus Hildisrieden
                  nach dem sonntäglichen Kirchgang im Foyer Inpuls ein gelungenes
                  Ständli der MGH geniessen. Wie gewohnt wurde das kleine Konzert
                  anschliessend mit einem Apero abgerundet.

      \end{itemize}

\end{history}
