\subsection*{2015}
\begin{history}


      \begin{itemize}

            \item 2. Januar Zunftbot\\
                  Wie immer am 2 Januar organisierte die Gätschizunft im Leue,
                  den Zunftbot zur Wahl des neuen Zunftmeisters. Erwin Wolf war
                  schliesslich der Auserwählte. Als Gratulation an den
                  Zunftmeister spielte die MGH auf der altehrwürdigen Saalbühne
                  ein kleines Konzert, bei dem sich auch Erwin als Dirigent
                  beweisen konnte.

            \item 10. /11. Januar Jahreskonzert\\
                  Die Jahreskonzerte unter dem Motto Visions boten den
                  Konzertbesuchern ein abwechslungsreiches und unterhaltsames
                  Programm. Die eindrückliche Klänge und Solos beeindruckten die
                  Gäste. Wie jedes Jahr wurde auch dieses Konzert feierlich in
                  der Foyerbar ergänzt und ausgetrunken.

            \item 15./17. Februar Fasnachtsumzüge Hildisrieden und Hochdorf\\
                  Unter der Leitung von Beni wurde in den Wochen vor den Umzügen
                  eifrig und mit vollem Einsatz ein erfreuter und politisch
                  engagierter Fasnachtswagen zusammengebaut. Unser Schiff,
                  angeführt vom Hürlimann und getrieben von der Spiess, war für
                  die Umzugsbesucher ein gelungener Scherz zur Zuger Politik
                  Affäre.

            \item 28.Februar Skitag in Andermatt/Sedrun\\
                  Mit einem Kleinbus von Estermann Reisen mit Präsi Chregu am
                  Steuer fuhren wir nach Andermatt, wo wir den Zug Richtung
                  Skigebiet Oberalppass-Sedrun nahmen. Das Wetter wollte nicht
                  so recht; erst am Nachmittag zeigte sich die Sonne vermehrt.
                  Wir liessen aber nicht beirren und gönnten uns dafür das eine
                  oder andere Kaffee (mit Güx☺) mehr. Einige versuchten sich
                  auch noch im Snowpark in der Disziplin Skiakrobatik. Trotz
                  Parkbusse von 10 Fr. (man beachte: die ordentliche Parkgebühr
                  für einen Tag kostet 5 Fr.) – oder gerade deswegen – liessen
                  wir uns es nicht nehmen, im Resort von Samih Sawiris noch
                  einen kleinen Besuch abzustatten und in der gemütlichen Lounge
                  ein Apéröli zu genehmigen, bevor wir dann für das Nachtessen
                  in Flüelen UR einkehrten. Nach wie vor ist uns ein Rätsel,
                  wieso dieses Restaurant nicht schon lange die Tore schliessen
                  musste. Rechnen und Kostentransparenz ist jedenfalls nicht die
                  Stärke dieses Gastro-Betriebes. Für den Schlummertrunk
                  wechselte man dann in die Restaurant-eigene Bar, bevor wir
                  gegen Mitternacht wieder gesund in Hildisrieden ankamen.
                  Organisiert wurde der Skitag von Stefan Barmet – vielen Dank
                  an dieser Stelle!

            \item 15. März Jubilarenständli\\
                  Am Jubilaren Ständli des letzten Frühlings konnten wir den
                  Senioren auf musikalische Weise zu ihrem runden Geburtstag
                  gratulieren. Die Jubilaren waren sichtlich erfreut über den
                  Auftritt und genossen anschliessend an das Ständli den Apero
                  mit uns Musikanten.

            \item 20. März GV\\
                  Nachdem in der Wirtschaft zur Schlacht das Nachtessen gegessen
                  und die ersten Biere und Weine getrunken waren begrüsste
                  Chregu die Vereinsmitglieder zur 141. Generalversammlung.
                  Neben dem kantonalen Veteran Chraso konnte auch Migu als
                  Eidgenössischen Ehrenveteran für die langjährige treue zur MGH
                  gedankt werden zudem konnten drei Musikanten in die MGH
                  aufgenommen werden. Jedoch waren auch 5 Austritte zu
                  vermelden. In einem weiteren Traktandum wurden die Neurungen
                  und Investitionen in den kommenden Jahren diskutiert. So soll
                  sich die MGH auch in Zukunft auf einem soliden Fundament
                  weiter entwickeln können.

            \item 12. April Weisser Sonntag\\
                  Wie jedes Jahr begleiteten wir die Erstkommunikanten mit einem
                  „rassigen“ Marsch durchs Dorf zur Kirche. Da sich jedoch wie
                  gewohnt unsere beiden Euphoniumkollegen gegenseitig fordern
                  blieb der eine zuhause im Bett um die vornächtlichen Strapazen
                  auszukurieren und schickte so den anderen ohne die Noten für
                  den Marsch auf die Dorfstrasse. Oben angekommen konnten sich
                  die Musikanten während der Kirche auf dem Gartensitzplatz des
                  roten Löwens bei sonnigem Frühlingswetter ein erfrischendes
                  Getränk geniessen und dabei die sonntägliche türkische Kultur
                  ein bisschen näher kennen lernen. Anschliessend wurde zur
                  Feier der Erstkommunikanten ein unterhaltsames Ständli
                  gespielt.

            \item  23. Mai Firmung\\
                  Am Pfingstsamstag den 23. Mai, konnten sich die Firmlinge mit
                  ihren Gotti`s und Götti`s mit einem Glas Most oder Wein vor
                  der Kirche ein Ständchen der Musikgesellschaft anhören. Die
                  Musikanten gesellten sich danach ebenfalls zu den Firmlingen
                  und deren Angehörigen und genossen den Apero.

            \item 4. Juni Fronleichnam und Probetag\\
                  Wie immer an Fronleichnam begleiteten wir die den Einzug der
                  Erstkommunikanten mit dem Laudamus. Nach einem kurzen Ständli
                  nach der Messe zog die ganze Gesellschaft mit dem Marsch
                  Gebirgs Füslier 48 ins Inpuls. Nach dem Apero wurde von der
                  Kirchgemeinde allen anwesenden Risotto serviert. Am Nachmittag
                  desselben Tages wurde wieder eifrig geprobt und die letzten
                  falschen Töne wurden für das kantonale Musikfest in Sempach
                  aus den Reihen gestrichen.

            \item 6. Juni Kantonales Musikfest Sempach\\
                  Am zweiten Wochenende des kantonalen Musikfestes konnte
                  endlich auch die MGH in die musikalischen Geschehnisse in
                  Sempach eingreifen. Am Samstagnachmittag viertel vor drei
                  standen die Musikantin und Musikanten unter der brütenden
                  Sonne auf der Marschmusikstrecke bereit für die Parademusik
                  mit dem Marsch \enquote{Geb. Füs. Bataillon 48}. Mit der
                  kräftigen Unterstützung der mitgereisten Hildisrieder Fans
                  gelang nach mehreren Jahren die Zielmarke von 50 Punkten.
                  Diese wurde sogar um 0,6 Punkte übertroffen, weshalb die MGH
                  schliesslich mit dem 5. Rang belohnt wurde.  Bei den Vorträgen
                  vom Selbstwahlstück \enquote{A Malvern Suite} und dem
                  Aufgabenstück \enquote{Strawabar} konnte ebenfalls auf unsere
                  Fans und Musikfreunde gezählt werden, weshalb die Ränge gut
                  gefüllt und die Stimmung fantastisch war.  Diese übertrug sich
                  auf die Musikantin und Musikanten, welche eine gute Darbietung
                  boten und somit auf dem 10. Platz klassiert wurden. Einmal in
                  Fahrt gekommen, konnten sich die Hildisrieder auch am
                  anschliessenden Fest und an der Bar nicht aus der Stimmung
                  bringen lassen. Daher wurde bis tief in die Morgenstunden
                  eifrig gefestet.

            \item 24. Juni Beendung der Zusammenarbeit mit Philipp\\
                  Was während den zwei vorausgegangenen Wochen innerhalb des
                  Vorstandes intensiv diskutiert und argumentiert wurde, musste
                  auch zusammen mit Philipp besprochen werden. Chregu und ich
                  machten sich als Vertreter des Vorstandes also auf den Weg zum
                  abgemachten Treffpunkt im Restaurant Postillion in Beckenried
                  um dort mit Phillipp unsere musikalische Zukunft zu
                  besprechen. Dabei stellte sich aber bald heraus, dass beiden
                  Parteien überraschenderweise die gleiche Meinung über die
                  Direktion der MGH hatten. So hatte sowohl Philipp, wie auch
                  wir dieselbe Meinung, dass sich unsere Wege trennen werden. So
                  könne die MGH mit einer anderen Führung neue Chancen packen
                  und Philipp kann sich zukünftig wieder vermehrt seiner Arbeit
                  als Musikschulleiter und Musiklehrer widmen. Diese Sitzung,
                  welche unangenehm begonnen hatte, fand ihren Schluss
                  schliesslich bei einem Bier auf der Gartenterrasse und man
                  konnte sich so freundschaftlich verabschieden. Angemerkt wurde
                  auch, dass auch in Zukunft das Jahreskonzert im Isenthal
                  besucht werden darf.

            \item 26. Juni Abschlusshöck\\
                  Der diesjährige Abschlusshöck wurde von den hinteren Cornets
                  organisiert. Bei bestem Wetter versammelten sich die MGHler am
                  Freitagabend vor dem Impuls und marschierten sogleich los
                  Richtung Wiederkehr. Dort nahmen uns bereits die
                  Organisatoren, unter anderem auch Toni, mit einem Apero in
                  Empfang. Toni brachte uns während dem Zwischenhalt auf den
                  neusten Stand der Geschichte. So erzählte er von früher wie im
                  Wiederkehr die Entlebucher Suppe assen, im Gimmermee die
                  Nidwalner hausten und im Schlüssel die Eidgenossen am frühen
                  Morgen während eines Gefechtes umkehrten. Oder wie war das
                  Toni? Nach der unterhaltsamen Pause ging es weiter durch den
                  Wald bis an den Steinibüehlweier, wo bei bestem Sommerwetter
                  bereits die Tische gedeckt und der Braten sowie der
                  Kartoffelgratin noch getrennt voneinander in der Wärmebox
                  waren. Das köstliche Essen mit Dessert, Kaffe, Schnaps und
                  Biersorgte bei allen für gute Stimmung. Diese konnte sich bei
                  einigen bis in die Morgenstunden halten. So gönnten sich die
                  letzten das erste Fleischplättli des Tages frisch aus der
                  Metzgerei.

            \item 18. September Gadenkonzert\\
                  An diesem kühlen Septemberabend gab es gleich mehrere
                  Neuigkeiten von der MGH zu bestaunen. Das Gadenkonzert war der
                  erste Auftritt mit dem neuen und trotzdem altbekannten
                  Projektdirigenten Peter Stadelmann. Nach einer kurzen,
                  intensiven, aber trotzdem sehr erfolgreichen Vorbereitungszeit
                  für das Konzert im Stall des Weierhofes konnte die
                  Musikgesellschaft ein sehr abwechslungsreiches und
                  unterhaltsames Programm präsentieren. Auch im Konzertprogramm
                  integriert waren einige Alphorneinlagen. Dieses urchige
                  Ambiente gefiel den Besuchern und man konnte viele tolle
                  Komplimente entgegen nehmen. Der erste Auftritt hatten jedoch
                  auch vier unserer jüngsten Musikanten. Der eigentliche Grund
                  dieses Konzertes war jedoch die Einweihung der neuen Cornets
                  und des Flügelhorns. Auch diese hatten den ersten Auftritt mit
                  der MGH. Ihre Vorgänger wurden wortwörtlich an den Nagel
                  gehängt und machen sich zukünftig als Deko nützlich. Jedoch
                  hatten diese den längsten Auftritt welcher morgens um 4.00 Uhr
                  mit dem Lichterlöschen beendet wurde.

            \item 16. /17. Oktober Lotto\\
                  Das alljährliche Lotto war auch im 2015 ein Erfolg. Besonders
                  am Samstag war der Leue bis auf den letzten Platz überfüllt.
                  Der Andrang war so gross, dass man sich unter den
                  Lottospielern beinahe gegenseitig die Stühle aus den Händen
                  riss. Am ersten Lottoabend hatten wir Besuch von der
                  Kantonalen Gewerbe Polizei. Diese wollten mit unseren
                  Finanzchefs über die Schultern schauen und mit diesen
                  Fachsimpeln. Jedoch soll gemunkelt worden sein, dass die
                  Sympathien nur einseitig waren.

            \item 6. Dezember Jubilarenkonzert\\
                  Und nochmals durften wir unseren Jubilaren musikalisch zu
                  ihrem Geburtstag gratulieren. Der überzeugende Auftritt hatte
                  viele tolle Komplimente zur Folge und wurde von unseren Gästen
                  sehr geschätzt.

      \end{itemize}

\end{history}
