\subsection*{2012}
\begin{history}


    \begin{itemize}

        \item 2. Januar, Zunftbot\\
              Wie jedes Jahr eröffneten wir unser Vereinsjahr mit dem traditionellen
              Zunftbot. Alois Galliker heisst der Regent der Fasnacht 2012. Mit einem
              Ständli gratulierten wir dem neuen Zunftmeisterpaar. Unter dem Motto
              \enquote{Egal was eifach vollgas} wurde anschliessend bis tief in die
              Nacht gefeiert und schlussendlich auch geschlafen. Natürlich nahm die
              MGH mit einem kreativen Wagen zum Thema "Radiosender Beromünster" an den
              Umzügen Hildisrieden und Hochdorf teil. Leider wurde aufgrund des
              feuchten Wetters aus der geplanten Riesensprengung nur ein herziges
              "Geklöpfe" mit einem feinen Hauch Rauch. Hauptsache der Radiosender
              kippte immer um.

        \item 14./15. Januar, Jahreskonzerte\\
              An den Jahreskonzerten 2012 durften wir uns über einen Spezialgast
              freuen. Johanna Bucher sang in der zweiten Hälfte bei einigen Stücken
              mit. Ihre wunderschöne Stimme sorgte da und dort für Gänsehaut. Auch das
              Euphonium-Solo von Stefan Barmet liess nichts zu meckern übrig.
              Anschliessend wurde, wie schon seit Jahren, bis tief in die Nacht in der
              MGH Bar gefeiert.

        \item 4. März, Jubilaren-Apérokonzert\\
              Bereits stand das Frühlings-Jubilarenständchen im Inpuls wieder vor der
              Tür. Da am Vorabend ein Anlass der Lüütertüter stattfand, hatten die
              einen Bläser etwas Mühe mit dem Aufstehen. Unser lieber,
              übermotivierter, angefressener Musiker Stefan Barmet übertrieb am
              Samstag masslos und verschlief deswegen sogar das Einspielen. Tja, die
              Jubilaren hatten auf alle Fälle trotzdem grosse Freude, unsere
              Glückwünsche auf musikalichem Wege zu erhalten und spürten nichts etwas
              von den alkoholischen Überresten der Musikanten.

        \item 10. März, Skitag\\
              Am Samstag besammelten sich einige Schneebegeisterte und reisten
              Richtung Hasliberg. Es wurde fleissig "ge-skit" und "ge-apreskit. Am
              Abend genossen alle ein feines Znacht in Sarnen. Während die einen
              anschliessend gemütlich in eine Bar pilgerten. widmeten sich andere dem
              bekannten "Linde-Tee" in der Linde. Nur schon ein solches Tee sorgte für
              eine sehr amüsante Heimfahrt.

        \item 23. März, Generalversammlung\\
              Die 138. Generalversammlung fand im Restaurant zur Schlacht in Sempach
              statt. Gemeinsam wurde auf das Jahr 2011 zurückgeblickt. Der OK-Chef des
              Musiktages 2013 Jakob Estermann war ebenfalls anwesend und berichtete
              über den aktuellen Stand der Vorbereitungen. Leider verteilte Mumi nicht
              mehr so viele Stangengläser für den guten Probebesuch. Die Aktuarin
              schnappte dummerweise einen Kommentar von unserem lieben Chraso auf. Er
              sagte zu seinem Tischnachbarn; \enquote{Chamer de Musig ächt sone
                  Schachtie Gläser abchaufe? De wörds be mer dehei ufem Regal mole chli
                  besser usgse.} Soll's geben aber Hauptsache du bleibst dabei lieber
              Chroso.


        \item 14. April, LJBB Konzert\\
              Das zwanzigste LJBB-Konzert durfte die MGH in Hildisrieden als Gastgeber
              durchführen. Nachdem sich die LJBB Musikanten/Innen mit Reis Casimir
              gestärkt hatten, begeisterten sie die Zuschauer mit einem musikalisch
              hochstehenden Konzert. Anschliessend sorgten die Ronspatzen für
              Unterhaltung während die restlichen MGH-Musikanten im Service tätig
              waren. Um 2.00 Uhr wurden die letzten Musikanten aus der Bar verjagt, da
              das "Weisser-Sonntags- Ständchen" immer näher kam und die Schlafstunden
              zu schrumpfen begannen.

        \item 15. April, Weisser Sonntag\\
              In der Hoffnung, dass die Prozession aufgrund des schlechten Weiters
              nicht stattfinden wird, schliefen alle Musikanten um 8.30 Uhr noch tief
              und fest. Als jedoch genau zu diesem Zeitpunkt das Rund-SMS mit der
              Aufforderung "9.00 Uhr Besammlung” eintraf, war die Begeisterung überall
              sehr begrenzt vorhanden. So trafen alle relativ knapp zum Abmarsch ein.
              Da es kaum noch fürs Frühstück reichte, genossen wir Musikanten noch ein
              Gipfeli im Löwen bevor wir anschliessend den Apéro im Foyer Inpuls
              musikalisch umrahmten.

        \item 28./29. April, Chilbi\\
              Dieses Jahr fand die Chilbi in einem kleineren Rahmen statt. Der
              Freitagabend viel komplett aus. Am Samstag gab es im Foyer Inpuls eine
              kleine Bar und auf dem Hartplatz ein Kaffeezelt. Die Musik am
              Samstagabend brachte im Gegensatz zu den legendären Ronspatzen am
              Sonntag kaum Stimmung auf. Der Chilbi-Sonntag war dank dem schönen
              Wetter wieder einmal die Rettung, dass die Vereine nicht noch selber in
              die Taschen greifen mussten. Mit unserem Chilbistand wurde hauptsächlich
              Werbung für den Musiktag gemacht.

        \item 13. Mai, Muttertag\\
              Für unsere lieben Mamis besammelten wir uns am 13. Mai vor der Kirche
              für ein Ständli. Unser hitziger Präsi dachte, als er kurz die Sonne sah,
              heute spielen wir mit unserem schönen Polo-Shirt. Leider hat er die
              Rechnung ohne den eisigen, starken Wind gemacht. Abgesehen von den teils
              gefrorenen Finger und den zu Berge stehenden Haaren hatten wir noch
              fleissig mit den ständig umfallenden Notenständern zu kämpfen. Diese
              Bewegung sorgte immerhin ein bisschen für Wärme.

        \item 25. Mai, Vorbereitungskonzert Musiktag in Aesch\\
              Gemeinsam mit der Musikgesellschaft Harmonie Sempach und der
              Musikgesellschaft Schwarzenbach führte die MGH ein Vorbereitungskonzert
              für den Musiktag in Aesch LU durch Patrick Ottiger übernahm die Rolle
              der Jury. Die Kritik an die MGH war wie immer: ruhige Stellen - zu laut,
              laute Stellen - zu laut, Posaunen - zu laut. Dies "Maxxxxx" einfach
              nicht leiden. Die Ronspatzen sorgten anschliessend wiederum für
              Unterhaltung, während die Gäste köstliche Grilladen geniessen durften.

        \item 3. Juni, Firmung\\
              Den frisch Gefirmten gratulierten wir mit einem abwechslungsreichen
              Ständli.

        \item 7. Juni, Fronleichnam\\
              Aufgrund des schlechten Wetters verzichteten die Organisatoren auf die
              Prozession. Der Gottesdienst wurde von einer Kleinformation der MGH
              musikalisch umrahmt. Im Anschluss gab die MGH ein Ständli im Inpuls
              bevor wir das von der Kirchengemeinde offerierte Mittagessen genossen.
              Danach widmeten wir uns der Hauptprobe für Assch.

        \item 9. Juni, Musiktag Aesch\\
              Am frühen Morgen besammelten sich die Musikanten/innen und Ehrendamen
              für die Fahrt nach Aesch. Nach einem Zmorgen mit Zopf und Kaffee ging es
              weiter in das Instrumentendepot, wo wir unseren Bagage deponierten.
              Anschliessend gab es einen kurzen Marsch von 10min ins Einspiellokal.
              Der Auftritt war aus unserer Sicht zufriedenstellend. Bei einer Stelle
              war jedoch aus irgendeinem Grund kein Cornet mehr zu hören. Unser Jurist
              meinte jedoch, wir haben "bon travail" geleistet und sollten noch ein
              bisschen "plus piano" spielen. Von unserem Patzer erwähnte er nichts.
              Danach stand bereits die Marschmusik auf dem Programm. Wunderschön
              gestartet, guter Sound, konzentriertes Marschieren bis -- ja bis wir am
              Schluss in die Höudi-Fan-Kurve gelangten, in welcher fleissig
              applaudiert wurde. Dadurch verstand Philipps "Spiel-Halt" nur noch knapp
              die erste Reihe. Je weiter nach hinten, umso holpriger und verschobener
              war der Halt. Trotzdem reichte es mit 49.4 Punkten auf den 9. Platz von
              16 und es wurde wie immer bis tief in die Nacht gefeiert, was man
              gewissen Uniformen noch immer entnehmen kann. Irgendein Hobby-Clown
              sprühte zu später Zeit weissen Bauschaum umher, von welchem unsere
              ich-geh-erst-nach-Hause-wenns-hell-ist Musikanten leider noch getroffen
              wurden.

        \item 22. Juni, Abschlusshöck\\
              Den diesjährigen Abschlusshöck wurde von unserem 1. Cornet-Register
              inklusive Flügelhorn und Repiano organisiert. Der Festakt fand in Mumi's
              Party-Meile statt. Köstliche Grilladen mit feinsten Salaten wurden uns
              serviert. Natürlich durfte anschliessend auch das berühmte Mumiramisu
              nicht fehlen. Bis tief in die Nacht wurde gefeiert, getrunken und
              gelacht. Alle überstanden den Abend ohne bleibende Schäden, nur der
              Gastgeber, welcher, so nebenbei erwähnt, den Ort am besten kennen
              sollte, marschierte übermotiviert schnell frontal in die Fensterscheibe
              des Wintergartens als er im Haus Nachschub holen wollte. Tja, eins zwei
              Bierchen zu viel und Mumi's Sehkraft schwindet.

        \item 14/15. September, MGH Ausflug\\
              Nach einer kurzen Carfahrt marschierten die Musikanten/innen und ihr
              Anhang mit Sack und Pack dem Vierwaldstädtersee entlang und erreichten
              nach zwei ausgiebigen Apéros und einer steilen Schlussphase schliesslich
              ihr Zielort: die Festung Vitznau. Zwei militärisch gekleidete Herren
              zeigten ihnen, was die Militaristen im Jahre 1943 in diesen Berg gebaut
              hatten. Beeindruckt begutachteten die Musikfreunde die Waffen und
              Kanonen, die düsteren Gänge, einen wunderschönen Aussichtspunkt sowie
              die Unterkunft, ebenfalls im Berg eingebaut. Im "Esssaal' warteten
              bereits das Abendessen in echtem Miltärgeschirr sowie eine traditionelle
              Cremeschnitte auf die hungrigen Hidisrieder. Bis tief in die Nacht wurde
              gefeiert - gestört hat dies bestimmt niemanden. Als endlich alle im Bett
              waren und jeder auf Ruhe hoffte, fing unser Koli Raus an, zu schnarchen
              und schnarchte und schnarchte und schnarchte...! Jeder staunte, wie so
              ein kleiner Musikant lauter schnarcht als ein Riesentraktor beim Mähen!
              Leider nützten alle Kissenwürfe ins Gesicht nichts und es ging die ganze
              Nacht so weiter, bis die "Herren des Bunkers" wieder zum Frühstück
              riefen. Mit gepackten Rucksäcken und vollem Magen begab sich die
              MGH-Truppe anschliessend bei schönstem Wetter mit dem Schiff Richtung
              Luzern. Nach einer kurzen Busfahrt trafen alle wieder froh und munter in
              Hildisrieden ein.

        \item 19./20. Oktober, Lotto\\
              Dieses Jahr war das Lotto wieder einiges erfolgreicher. Wir hatten am
              Freitag und Samstag viele Gäste. Einzig ein vierbeiniger Besucher machte
              uns etwas zu schaffen. Das arme Tierchen musste von 19 Uhr bis am
              Schluss unter dem Tisch sitzen. Als das Lotto dann fertig war und alle
              gleichzeitig aufschossen, war das wohl doch etwas zu viel für den
              kleinen Hunde-Magen. Mann munkelt dass der Lotto-Chef blöderweise Hund
              und vor allem Häufchen übersehen hat. Er konnte sich jedoch nach einer
              kurzen Rutschphase wieder problemlos auffangen. Ebenfalls amüsant war,
              dass am Schluss unser Löwenwirt noch kurz eine Runde Karten für seine
              Mitarbeiter aufwarf und die Köchin mal kurz den Hauptpreis abräumte.

        \item 2. Dezember, Jubilaren-Apérokonzert\\
              Der viele Schnee erschwerte den einen oder anderen am Sonntagmorgen die
              Reise in die Inpulshalle. Trotzdem starteten wir pünktlich mit dem
              Ständli für unsere Jubilaren und gratulieren Ihnen auf die musikalische
              Art zu ihrem Geburtstag. Anschliessen genossen alle noch einen Apero.

    \end{itemize}

\end{history}
