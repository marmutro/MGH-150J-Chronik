\begin{history}

    Über zwei Jahrzehnte hinweg spielte die Musikgesellschaft Hildisrieden eine
    zentrale Rolle im musikalischen und gesellschaftlichen Leben des Dorfes. Von
    den traditionellen Anfängen mit dem Zunftbot am Jahresbeginn bis zu den
    festlichen Jahreskonzerten, prägte die Gesellschaft die lokale Kultur
    tiefgehend. Ihre Konzerte, jedes Jahr unter einem neuen Motto, waren nicht
    nur musikalische, sondern auch gesellschaftliche Highlights, die Jung und
    Alt zusammenbrachten.

    Die Teilnahme an Musikfesten bot der Musikgesellschaft Gelegenheiten, ihre
    musikalischen Fähigkeiten unter Beweis zu stellen und Erfolge zu feiern,
    während gemeinschaftliche Konzerte und Vorbereitungen auf solche Ereignisse
    den Austausch und die Verbundenheit mit anderen Musikvereinen förderten.

    Ein besonderes Augenmerk legte die Gesellschaft auf die Pflege der
    Gemeinschaft und Kameradschaft, was sich in den gemeinsamen Ausflügen und
    Reisen zeigte. Hier ragt besonders die Reise ins Südtirol im Jahr 2002
    heraus, wo Musik, Natur und Geselligkeit im Vordergrund standen. Die
    dreitägige Tour ins Welschland 2007 bot den Musikanten nicht nur kulturelle
    Einblicke, sondern auch unvergessliche Erlebnisse.

    Ein Höhepunkt der jüngeren Vereinsgeschichte war die Ausrichtung des
    Musiktages 2013 in Hildisrieden, ein Ereignis, das die Musikgesellschaft als
    festen Bestandteil der regionalen Blasmusikszene etablierte. Die
    erfolgreiche Organisation dieses Grossereignisses war ein Beweis für das
    Engagement und die Kompetenz der Mitglieder.

    Ein weiterer herausragender Moment war die Jubiläumsreise nach London im
    Jahr 2014, anlässlich des 140-jährigen Bestehens der Musikgesellschaft. Der
    Besuch des Britischen Brass Band-Wettbewerbs in der Royal Albert Hall bot
    den Musikanten nicht nur musikalische Inspiration, sondern auch die
    Gelegenheit, die kulturellen Sehenswürdigkeiten Londons zu erkunden. Diese
    Reise stärkte nicht nur den Zusammenhalt innerhalb der Gruppe, sondern
    hinterliess auch bleibende Eindrücke bei jedem Teilnehmenden.

    Neben diesen besonderen Anlässen war die Musikgesellschaft auch im täglichen
    Dorfleben aktiv, gestaltete kirchliche und gesellschaftliche Feste mit und
    setzte sich für die musikalische Jugendförderung ein. Durch diese
    vielfältigen Aktivitäten bewies die Musikgesellschaft Hildisrieden über
    Jahre hinweg ihre musikalische Vielseitigkeit, ihr Engagement für die
    Gemeinschaft und ihre Leidenschaft für die Musik.

    \subsubsection*{Eidg. Musikfest Fribourg 2001}

    Anfang Juni führte die MGH gemeinsam mit der Feldmusik Hellbühl und
    Feldmusik Neuenkirch ein Vorbereitungskonzert im Inpuls durch, das bei
    vielen Besuchern für Begeisterung sorgte.

    Wenige Tage später, am 16. und 17. Juni, begaben sich die Musikanten trotz
    regnerischem Wetter in bester Stimmung nach Fribourg. Dort absolvierten sie
    im Platzregen stehend ihren Marschmusikvortrag und erreichten mit 103
    Punkten eine Platzierung in der oberen Ranglistenhälfte.

    Die Vorbereitung auf die Konzertstücke \enquote{Alpine Variations} von
    Bertrand Moren und \enquote{Prelude and Celebration} von James Curnow zahlte
    sich aus, denn beide Darbietungen überzeugten die Jury und bescherten der
    Musikgesellschaft mit je 150 Punkten den 19. Rang unter 46 teilnehmenden
    Vereinen.

    Der Ausflug endete mit einer Erkundung der Altstadt von Fribourg und am
    Sonntag mit einem herzlichen Empfang durch die Hildisrieder Bevölkerung.

    \subsubsection*{Reise ins Südtirol 2002}
    Während ihrer dreitägigen Reise ins Südtirol erlebte die Musikgesellschaft
    Hildisrieden neben musikalischen Darbietungen und kulturellen Ausflügen auch
    heitere Momente.

    Gleich nach ihrer Ankunft und dem Beziehen der Zimmer fanden einige der
    jüngeren Mitglieder Gelegenheit für eine aussergewöhnliche Aktion am Pool.
    Getrieben von jugendlicher Ausgelassenheit, wagten sie es, nicht einfach nur
    in den Pool zu springen, sondern vollführten den Sprung gemeinsam mit einem
    Gartenstuhl ins erfrischende Wasser.

    Am nächsten Tag stand der Besuch eines Weinguts in Kaltern auf dem Programm,
    gefolgt von einer Wanderung, Baden und Zeit zur freien Entspannung. Der
    Abend wurde mit einem Kurkonzert in Eppan abgerundet, trotz einiger
    Herausforderungen durch den Wind.

    \begin{MulticolFigure}
        \centering
        \includegraphics[width=0.93\linewidth]{./chap/2001-2024/2002/MGH-Südtirol-2002-3.jpg}
        \captionof{figure}{Konzert mit Windstörungen in Eppan}
    \end{MulticolFigure}

    Den Abschluss der Reise bildete ein Frühschoppenkonzert in St. Maria, bevor
    es zurück nach Hildisrieden ging.

    \begin{MulticolFigure}
        \centering
        \includegraphics[width=0.93\linewidth]{./chap/2001-2024/2002/MGH-Südtirol-2002-1.jpg}
        \captionof{figure}{Die jüngeren der Reisegruppe}
    \end{MulticolFigure}


\end{history}


\groupphoto{0.96}{0.85}{./chap/2001-2024/2010/MGH-2010.jpg}
{\emph{Willisau 2010}\\
    1. Reihe:\\
    Maja Achermann, Beat Koller, Markus Banz, Mathias Niggli, Martin Estermann,
    Josef Wolf, Michael Rösch, Christoph Erni, Franz Dörig, Martin Aregger,
    Silvia Schurtenberger\\
    2. Reihe:\\
    Christoph Schneider, Fabian Hüsler, Stephan Wolf, Benedikt Troxler, Urs
    Niederberger, Beat Disler, Markus Käppeli, Martin Troxler\\
    3. Reihe:\\
    Toni Bachmann, Peter Fleischli, Marina Büchler, Daniel Fleischli, Severin
    Pfister, Stefan Barmet, Dario Fleischli, Beat Bachmann, Christoph Troxler,
    Roland Klaus\\
    4. Reihe:\\
    Stephan Schneider, Peter Estermann, Armin Schmid, Franz Erni, Alexander
    Troxler, Roger Wermelinger, Mathias Rub, Manuel Andermatt } {fig:mgh-2010}


\groupphoto{1.0}{1.0}{./chap/2001-2024/2021/MGH-2021.jpg}
{\emph{Neuuniformierung 2021}\\
    1. Reihe:\\
    Alissa ???, Stephan Wolf, Hans Stöckli, Noe Stadelmann, Franz Dörig,
    Vivienne Bucher, Peter Stadelmann, Lian Wolf, Mattia Klaus, Fabian Hüsler,
    Elia Troxler, André Schmid, Barbara Lingg-Banz\\
    2. Reihe:\\
    Christoph Troxler, Marina Büchler, Robin Estermann, Benedikt Troxler, Filip
    Estermann, Kilian Rüttimann, Beat Disler, Christoph Schneider, Christoph
    Erni, Oliver Schneider, Beat Bachmann, Alexander Troxler\\
    3. Reihe:\\
    Toni Bachmann, Manuel Stirnimann, Yanis Wolf, Jonas Furrer, Stefan Barmet,
    Silvan Wolf, Martin Troxler, Roland Klaus, Peter Estermann, Josef Wolf,
    Stephan Schneider, Mathias Rub } {fig:mgh-2021}

