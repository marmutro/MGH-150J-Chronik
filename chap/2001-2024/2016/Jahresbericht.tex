\subsection*{2016}
\begin{history}

      \begin{itemize}

            \item 2. Januar Zunftbot\\
                  Wie jedes Jahr eröffneten wir unser Vereinsjahr mit dem
                  traditionellen Zunftbot. Sehr zur Freude der MGH stammte der
                  diesjährige Zunfmeister aus den eigenen Reihen. Chraso durfte
                  sich als neuer Zunftmeister krönen lassen. Mit einem Ständli
                  gratulierten wir dem neuen Zunftmeisterpaar. Anschliessend
                  ging es in das Eiholz, wo bis früh in die Morgenstunden auf
                  dir kommende Fasnacht angestossen wurde.

            \item 9./10 Januar Jahreskonzerte\\
                  Eine Woche früher als gewohnt präsentierte die MGH ihr
                  traditionelles Jahreskonzert. Unter dem Motto „I love Bläch“
                  wurde ein sehr forderndes Programm einstudiert. Von sämtlichen
                  Registern wurde ihr bestes Musikkönnen verlangt.  Die
                  Zusammenarbeit mit Peter war ein voller Erfolg, welcher sich
                  in den Konzerten wiederspiegelte. Die Konzertbesucher waren
                  begeistert von den Auftritten und zeigten dies mit viel Lob
                  und Gratulationen. Neben den gesamthaft überzeugenden
                  Konzerten stachen das Cornet Solo von Mattia, wie auch das
                  Solo der Posaunen aus dem restlichen Programm. Zum weiteren
                  konnte Chose zum kantonalen Ehrenveteran geehrt werden.

            \item 6./7./9. Februar Fasnacht\\
                  Einmal mehr konnte unser Fasnachts-Ass Toni sein
                  Fasnächtlicher Trieb entfalten lassen. Die verfügbaren
                  Musikanten sowie einige Verstärkungen wurden von ihm
                  kurzerhand in Chrasos Rüttlerband umgeformt. Max, Tonis rechte
                  Hand, musste aus dem Haufen von Blech und Muskikanten eine
                  wohlklingende Fasnachtsmusik bilden. Zusammen mit der
                  beeindruckenden Fasnachtsuniform ergab dies eine tolle Gruppe,
                  welche an den Fasnachtsumzügen im Rain, in Neudorf und in
                  Hochdorf von A wie Aufritt bis Z wie Zlegscht hei, brillieren
                  konnte.

            \item 13. Februar 100 Jahre Hans Troxler\\
                  Unsere ehemaliges Mitglied Hans Troxler aus der Usserbueche
                  konnte am zweiten Samstag im Februar seinen 100. Geburtstag
                  feiern. Auch die MGH konnte sich zu den Gästen zählen und
                  spielte zu Hans`s Runden  mit einem kleinen Konzert auf.
                  Sichtlich gerührt vom Auftritt nahm usserbueche Hans die
                  Gratulationen und den Geschenkkorb von Chregu und Max
                  entgegen. Als ehemaliger Musikant und 55 Jahre mitgliedschaft
                  bei der MGH wusste der Jubilar dies sehr zu schätzen.

            \item 13. März Jubilaren Konzert\\
                  Das mittlerweile halbjährliche gewohnt gut besuchte
                  Jubilarenkonzert, bot auch im 2016 ein abwechslungsreiches und
                  Unterhaltsames Programm. Die Jubilaren waren sichtlich erfreut
                  und verdankten dies mit einem Applaus und diversen
                  Gratifikationen. Der erste Auftritt mit Pascal Maillard konnte
                  als überaus gelungen abgebucht werden, derweil einer der ganz
                  Jungen im Baritonregister mit dem Wecker nicht auf eine
                  Einigung kam, jedoch trotzdem ganz spontan noch zum Konzert in
                  der Formation erschien.

                  \12. April Weisser Sonntag\\
                  Wie jedes Jahr begleiteten wir die Erstkommunikanten durchs
                  Dorf zur Kirche. Zum Erstaunen der Zuhörer musste dabei der
                  altbewährte Laudamus dem neuen Prozessionsmarsch Credo
                  weichen. Merkwürdige Ereignisse waren jedoch zuvor bei der
                  Besammlung zu beobachten. Den Bildern zur Folge hat unser
                  Kassier bedenken um unsere finanzielle Zukunft. Nichts desto
                  trotz gönnten sich die Musikanten nach dem Einzug einen Kaffe
                  in der frischgeschtrichenen Gartenbeiz des Leuens. Die Spuren
                  auf den Hosen beweisen jedenfalls die Qualität des Lacks.
                  Anschliessend wurde zur Feier der Erstkommunikanten vor der
                  Kirche ein unterhaltsames Ständli gespielt.

            \item 28/29. April Probeweekend Hasliberg\\
                  Nachdem am Samstagmorgen sämtliche Utensilien des
                  Schlagregisters im Ruagtransporter verladen waren, konnte der
                  MGH Karavan pünktlich und ohne Zwischenfall nach Hasliberg
                  verschieben. Im cvjm-Zentrum angekommen wurde die Herberge zu
                  einem Probelokal umfunktioniert. Da auch das Wetter die
                  Musikantinnen Musikanten nicht nach draussen zog, waren die
                  Bedingungen perfekt für ein erfolgreiches Probeweekend. Nicht
                  zuletzt dank der perfekten Organisation von unserem
                  Präsidenten ging alles reibungslos über die Bühne oder eben
                  durch die Kirche. Insgesamt wurde 15 Stunden geprobt. Die
                  spätnächtlichen Gesangsübungen nicht eingerechnet. Die
                  Verpflegung wurde von der engagierten und motivierten Crew der
                  Christlichen Vereinigung junger Männer, jeweils pünktlich zu
                  den Essenszeiten, breitgestellt. So reibungslos wie der Aufbau
                  des Probeweekends war, waren auch der Rückbau und die
                  Heimfahrt wieder Problemlos. Dank der erfolgreichen zwei Tage,
                  werden wir sowohl Musikalisch als auch geistlich, bestens auf
                  das Eidgenössische Musikfest in Montreux vorbereitet sein.

            \item 14. Mai Firmung\\
                  Da der Pfingstsamstag ins Wasser fiel, wurde beim Apero nicht
                  wie gewohnt vor der Kirche, sondern im Foyer des Inpuls mit
                  den Firmlingen angestossen. Die MGH umrahmte dies musikalisch
                  mit einem kleinen und unterhaltsamen Konzert.

            \item 20. Mai Vorbereitungskonzert\\
                  Zur Vorbereitung auf das eidgenössische Musikfest,
                  organisierte die Musikgesellschaft Triengen einen
                  Konzertabend. Nebst den Trienger und Hildisrieder, zeigten
                  auch die Rainer und Schwarzenbacher, was sie in den letzten
                  Monaten einstudiert und geübt hatten oder eben nicht. So
                  wurden noch einige Defizite und unsichere Stellen aufgedeckt.
                  Nichts desto trotz wurde dieser Abend an der Bar bei einem
                  kühlen Bier abgeschlossen.

            \item 26. Mai Fronleichnam und Probetag\\
                  Wie immer an Fronleichnam begleiteten wir die den Einzug der
                  Erstkommunikanten. Dieses Jahr jedoch mit dem neuen
                  Prozessionsmarsch Credo, welchen Max vom Isental importiert
                  hatte. Nach nach der Messe zog die ganze Gesellschaft mit dem
                  Marsch Gebirgs Füslier 48 ins Inpuls, wo die MGH den Apero
                  musikalisch umrahmte. Danach wurde von der Kirchgemeinde allen
                  anwesenden Risotto serviert. Am Nachmittag desselben Tages
                  wurde wieder eifrig geprobt und die letzten falschen Töne
                  wurden für das eidgenössiche aus den Reihen gestrichen.

            \item 27. Mai Veteranenehrung am Jugendmusikfest Gunzwil\\
                  Schon zwei Wochen vor dem Eidg. Musikfest, wurde an der
                  Veteranenehrung anlässlich des Kantonal Jugendmusikfestes in
                  Gunzwil unser Eb-Tubist Chosé Wolf für seine langjährigen
                  Dienste für die Musik offiziell als eidgenössischer Veteran
                  geehrt. Ganze 35 Jahre ist er bereits in der MGH aktiv.

            \item 11. Juni Eidgenössisches Musikfest Montreux\\
                  Am Samstag 11. Juni 2016 konnte die Musikgesellschaft
                  Hildisrieden (MGH) ihr Können am Eidgenössischen Musikfest in
                  Montreux auf nationaler Ebene in der Kategorie „2. Klasse
                  Brass Band“ unter Beweis stellen. Nach der dreistündigen
                  Anreise inklusive Kaffeepause im Greyerzerland mussten sich
                  die Musikanten auf dem gefüllten Festareal unglücklicherweise
                  alleine zurechtfinden, da bei einigen Musikgesellschaften auf
                  eine Begleitperson verzichtet wurde. Der strömende Regen
                  drückte zusätzlich etwas auf die Stimmung bei allen
                  Musikantinnen, Musikanten und Festbesuchern. Folglich mussten
                  jegliche Auftritte auf der Parademusikstrecke zwischenzeitlich
                  bis um 15.00 Uhr abgesagt werden. Da aber die Hildisrieder
                  ihren Parademusik-Auftritt erst um 16.03 Uhr hatten, konnten
                  sie an den Start gehen und das Geübte präsentieren. Dies
                  sorgte nach zehn Minuten Konzentration, militärischer
                  Disziplin und musikalischem Vortragen des Marsches „Geb. Füs.
                  Bat. 48“ von Hans Flury für den ersten Höhepunkt. Die
                  Vorstellung der MGH wurde von den Experten mit traumhaften 90
                  Punkten belohnt. Dies war gleichbedeutend mit dem 2. Rang in
                  der Ranglisten-Gruppe. Diese umfasst alle Vereine, welche auf
                  der gleichen Parademusikstrecke vom gleichen Jury-Team
                  bewertet wurden. Der Konzertvortrag, welcher ein Aufgabestück
                  sowie ein Selbstwahlstück beinhaltet, wurde am Abend um 20.00
                  Uhr der Jury vorgetragen. Dort konnte die MGH die während
                  viereinhalb Monaten eingeübten musikalischen Finessen der
                  beiden Stücke präsentieren. Mit dem 10. Rang innerhalb der
                  Ranglisten-Gruppe wurde das Ziel erreicht und auch dort konnte
                  man sich in der vorderen Tabellenhälfte etablieren. Die
                  grossartigen Erfolge, welche die MGH mit dem
                  Projekt-Dirigenten Pascal Maillard aus dem Elsass mit Fleiss,
                  Disziplin und grossem Teamgeist hart erarbeiteten, wurden
                  anschliessend an der Riviera von Montreux mit Bier, Stumpen
                  und musikalischen Höchstleistungen der Stimmbänder gefeiert.
                  Somit war der Car der Hildisrieder einer der letzten, welcher
                  den Festplatz am frühen Sonntagmorgen verliess. 17. Juni
                  Abschlusshöck Um das erfolgreiche Musigjahr würdig zu beenden,
                  trafen sich die Mitglieder der MGH nach einem kurzen
                  Fussmarsch in der St.Anna auf dem landwirtschaftlichen Betrieb
                  der Familie Fleischli. Dort Dort gabe es reichlich Grilladen
                  und Salate für die hungrigen und noch mehr Bier und weitere
                  Getränke für die durstigen. Das herrliche Sommerwetter genügte
                  jedoch nicht für den ganzen Abend. Deshalb wurde das
                  Dessertbuffet in der gemütlichen Stube aufgetischt. Die
                  fröhliche und gesellige Stimmung zog sich bis tief in die
                  Nacht. An dieser Stelle noch einmal besten Dank an die
                  Organisatoren aus dem Schlagregister.

            \item 17. September Gadekonzärt\\
                  Im bereits gewohnten Umfeld in der Halde konnten wir die
                  dritte Ausgabe des Gadenkonzertes durchführen. Der
                  Werbeaufwand im vorfed hatte sich bemerkbar gelohnt. Die
                  Tischgarnituren unter der Einfahrt zum Heuboden und in der
                  Tenne füllten sich wunschgemäss. So konnte das Jodelduett
                  Karin Schmid und Armin Steffen mit Begleitung durch Martin
                  Flury auf dem Akkordeon das Konzert eröffnen. Auch die MGH
                  konnte mit ihrem unterhaltsamen und abwechslungsreichen
                  Konzertprogramm die Konzertbesucherinnen und Konzertbesucher
                  überzeugen. Die Lehren aus den letzten Jahren wurden sichtlich
                  gezogen, was sich in der Festwirtschaft wiederspiegelte. So
                  konnte das Gadekonärt 2016 gesellig und bis in die Nacht in
                  einer gemütlichen Umgebung ausgekostet werden.

            \item 18. September Bettag\\
                  Nach zwei Jahren Unterbruch waren nun die Hildisrieder wieder
                  an der Reihe um die Messe wie auch den Apero musikalisch zu
                  umrahmen. Bei einigen war das vornächtliche Gadekonzärt nicht
                  nur im Kopf oder im Bauch spürbar, auch der bekannte
                  gädeli-Geruch des Stalles war bis in die Kirchenränge zu
                  riechen. Nichts desto trotz erfreuten sich die Messebesucher
                  an dem sonntäglichen Auftritt der MGH.

            \item 14. /15. Oktober Lotto\\
                  Das alljährliche Lotto war auch im 2016 ein Erfolg. An beiden
                  Lottoabenden war der Leue sehr gut gefüllt. Unter der
                  professionellen Anleitung von Lottoprofi Hanspeter Koch
                  konnten einige Verbesserungen rund um die Organisation und
                  Werbung vorgenommen werden. Dies schien Früchte zu tragen.

            \item 4. Dezember Jubilarenkonzert\\
                  Einmal mehr konnten die Jubilaren aus Hildisrieden nach dem
                  sonntäglichen Kirchgang im Foyer Inpuls, ein gelungenes
                  Ständli der MGH geniessen. Wie gewohnt, wurde das kleine
                  Konzert anschliessend mit einem Apero abgerundet. 2. Januar
                  Zunftbot Wie es in Hildisrieden Tradition ist, wird am 2.
                  Jahrestag der neue Zunftmeister erkoren.  Und auch diesmal
                  begleitete die MGH zuerst die Zunftdelegation, nach einem
                  kleinen Apero mit vielen Neujahrswünschen, mit dem Bundesrat
                  Gnägi Marsch durchs Dorf bis zum Leuen. Anschliessend wurde
                  der Zunftbot im grossen Saal abgehalten. Der Zunftmeister fürs
                  2017 erschien schlussendlich in qualmenden Rauch auf der Bühne
                  und liess sich feiern. Jedoch war die Stimmung im Saal weniger
                  ausgelassen als auch schon, da das Gesicht des neuen
                  Zunftmeisters in der Dorfbevölkerung eher unbekannt war. Erwin
                  Bieri wurde schliesslich vorgestellt und so konnten ihm alle
                  zur Wahl gratulieren. Auch die Musikgesellschaft tat dies mit
                  einem kleinen Ständli. Anschliessend verschiebten sich die
                  Feierlichkeiten in den Malorain.


      \end{itemize}

\end{history}
