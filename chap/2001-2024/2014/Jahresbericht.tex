\subsection{2014}
\begin{history}


    \begin{itemize}

        \item 2. Januar, Zunftbot\\
              Wie jedes Jahr eröffneten wir unser Vereinsjahr mit dem traditionellen
              Zunftbot. Ivo Berchtold heisst der Regent der Fasnacht 2013. Mit einem
              Ständli gratulierten wir dem neuen Zunftmeisterpaar. Das Motto für die
              Fasnacht 2014 war hiess \enquote{Es farbigs Fäscht mit sportliche
                  Gäscht} und wurde unter \enquote{schärfsten} Bedingungen ghörig
              eingeweiht.

        \item 19. und 20. Januar Jahreskonzerte\\
              Unter dem Motto Feuer im Blut wurde unseren Konzertbesuchern mächtig
              eingeheizt. Das unterhaltsame Programm beanspruchte einige Musikanten so
              sehr, dass diese ihr Feuer in der legendären Musigbar im Foyer mit ein
              zwei Schnäpsen löschen mussten. Vereinzelt wollte man sogar ganz sicher
              gehen und auch die letzten Gluten ausmachen, was schliesslich auch
              gelang. Trotzdem konnte auch am Sonntag den Besuchern ein überzeugender
              Auftritt mit einem mächtigen Sound gezeigt werden.

        \item 28. März GV\\
              Die GV des 28 März 2014 wurde aus verschiedenen Betrachtungsweisen eine
              legendäre. Zum einen war es die letzte GV im Restaurant Kreuz, das am
              darauf folgenden Tag die Türen für immer schloss und zum anderen konnte
              das vergangene Jahr an der 140 GV mit den überaus positiven Zahlen des
              Musiktages abgeschlossen werden. Nach der GV gab es noch ein Ständli für
              die Kreuz-Crew und für uns ein feines Kreuz Cordonbleu.  Die letzten
              Stunden im Kreuz wurde von den Musikanten  genossen und mit viel
              Kafischnaps abgeschlossen.

        \item 6. April Jubilarenständli\\
              Am Jubilaren Ständli des letzten Frühlings konnten wir den Senioren auf
              musikalische Weise zu ihrem runden Geburtstag gratulieren. Die Jubilaren
              waren sichtlich erfreut über den Auftritt und genossen anschliessend an
              das Ständli den Apero mit uns Musikanten.

        \item 26. und 27. April Kilbi\\
              Unter regnerischen Bedingungen wurde auch im letzten Jahr die Kilbi von
              uns Musikanten zusammen mit der Zunft und dem HSV auf die Beine
              gestellt. Die schlechten Wetterbedingungen konnten jedoch nicht allzu
              viele Besucher anlocken.  Der kräftige Regen bescherte den Kilbigästen
              am Samstagabend jedoch ein strömender Fluss quer durch das Zelt bis an
              die Bar  wo dann das Wasser in einem Schacht seinen Meister fand. Auch
              am Sonntag war die Kilbi vom Regen gezeichnet. Man suchte eine warme
              bleibe im Zelt und genoss dort die Unterhaltung der Ronspatzen. Diese
              fanden sich nach dem Auftritt ebenfalls hinter einem Mass Bier  und
              führten die Feierlichkeiten bis zum Schluss. Zwei dieser Musikanten
              gönnten sich zum Schluss noch eine gegenseitige Bierdusche womit die
              Kilbi 2014 abgeschlossen werden konnte.

        \item 4. Mai 1. Kommunion\\
              Wie alle Jahre begleiteten die Musikanten die Erstkommunikanten mit dem
              traditionellen Laudamus beim Einzug durch das Dorf. Nach einer kurzen
              Pause im roten Löwen spielte die MGH nach der Messe ein Ständchen.

        \item 10. Mai 100 Jahre Kaspar Troxler\\
              Am 10. Mai wurde unser Ehrenmitglied Kaspar Troxler 100 Jahre alt. Ihm
              zu Ehren spielten wir an der Geburtstagsfeier im Altersheim in
              Beromünster ein kleines Konzert. Da der Jubilar selber über 60 Jahre
              musiziert hatte und in unserer Musikgesellschaft tätig war, konnten wir
              ihm eine grosse Freude bereiten. Das Programm bestehend aus Märschen und
              Polkas war ganz auf seinen Musikgeschmack angepasst. Wir Musikanten
              hatten jedoch zu kämpfen. Die einen mit dem starken Wind und die anderen
              mit dem Hut.

        \item 11. Mai Muttertag\\
              Der Wind war auch am folgenden Tag ein grosses Thema. Das alljährliche
              Muttertagsständchen wurde aus Gründen von feinen Wetter-Kenntnissen vom
              Vorstand allen Erwartungen entgegen am Sonntagmorgen durchgeführt. Wie
              erwartet blieb der Regen fern jedoch nicht der Wind. Die Auslastung der
              Klämmerli war an diesen Tagen gross. Sogar die in den Föhntälern
              wohnhaften Musikanten waren solche Verhältnisse nicht gewachsen und
              mussten sich gegenseitig behilflich sein. Der Dirigent schickte dem
              Posaunen-Solisten seine Frau zur Hilfe da diesem die Noten um die Ohren
              flogen.

        \item 31. Mai Musiktag Wauwil\\
              Die Organisatoren des Musiktages in Wauwil konnten anders als ihre
              letztjährigen  Vorgänger aus Hildisrieden von ihrem Wetterglück
              profitieren. Sie stellten ein gelungenes Fest auf die Beine wo alle ihre
              Erwartungen erfüllen konnten. Durch die Kompaktheit des Festgeländes
              konnten die Verschiebungszeiten kurz gehalten werden. Die Musikanten und
              Musikantinnen der Musikgesellschaft Hildisrieden konnten schon am
              Samstagmorgen um 10.15 Uhr ihr Vortragsstück dem Publikum sowie dem
              Juror vortragen. Kurz darauf konnten die schöneren Musikanten weiter
              brillieren, wobei die anderen bei der Fotoaufnahme in die hinteren
              Reihen stehen mussten. Schon bald war dann auch Mittag und allesamt
              gesellten sich am Bankett an den Tisch. Auch dort konnten sich mit einem
              zügigen Ablauf die Organisatoren und Helfer auszeichnen. Um 14.15 Uhr
              spielte die MGH zur Marschmusik auf. Die zufriedenstellende Darbietung
              konnte jedoch nicht alle überzeugen, da doch plötzlich von einer
              gewissen Jurorin nicht genannte Körperteile zum \enquote{Verhängnis}
              wurden. Trotzdem konnte niemand den Durst der Musikanten stillen und der
              Bierstand wurde rege benutzt. Dies zog sich bis in die frühen
              Morgenstunden an den vielen Bars weiter.  Um 3 Uhr morgens wurden dann
              alle mit dem Car wieder sicher nach Hildisrieden gefahren. Jedoch lud
              der Sonnenrain 15 noch zum Morgenbuffet ein welches mit grosser
              Begeisterung rege benutzt wurde. So sah man noch bei Tageslicht
              Musikanten durchs Dorf ziehen.

        \item 7. Juni Firmung\\
              Am Pfingstsamstag konnten sich die Firmlinge mit ihren Gotti's und
              Götti's mit einem Glas Most oder Wein vor der Kirche ein Ständchen der
              Musikgesellschaft anhören. Jedoch konnten die Musikanten erst im dritten
              Stück mit der kompletten Besetzung auffahren, da die Spitze des
              Euphonium-Registers den freitäglichen Ausgang nicht verträgt - oder sind
              dies bereits Anzeichen von Starallüren?

        \item 19. Juni Fronleichnam\\
              Wie immer an Fronleichnam begleiteten wir die den Einzug der
              Erstkommunikanten mit dem Laudamus. Nach einem kurzen Ständli nach der
              Messe zog die ganze Gesellschaft mit dem Bundesrat Gnägi Marsch ins
              Inpuls. Nach dem Apero wurde von der Kirchgemeinde allen Anwesenden
              Risotto serviert.

        \item 20. Juni Abschlusshöck\\
              Dieses Jahr war das Bassmadli verantwortlich für den Abschlusshöck. Sie
              organisierten den geselligen Abend in der Schützenstube. Bei herrlichem
              Wetter konnten wir uns mit köstlichen Grilladen verpflegen lassen und
              anschliessend den WM-Match zwischen der Schweiz und Frankreich
              mitverfolgen. Leider ging dieser nur knapp verloren. Nichtsdestotrotz
              gaben einige unsere härtesten Krieger auch in den Morgenstunden nicht
              auf und verlegten die Tätigkeiten in die Lütertüterbar wo sie dann den
              Sonnenaufgang geniessen konnten.

        \item 29. Juni Tag der offenen Tür auf dem Weierhof\\
              Am letzten Sonntag des Junis öffnete die Familie Hüsler alle Türen und
              Tore des neu erbauten Stalles um diesen der Bevölkerung zugänglich zu
              machen. Da mit einem grossen Besucheraufmarsch gerechnet wurde, übernahm
              die MGH die Festwirtschaft und konnte so die Familie Hüsler entlasten.
              Trotz des durchzogenen Wetters war der Besucheraufmarsch riesig und von
              allen Regionen der Schweiz. Die über 2000 Besucher und Besucherinnen
              brachten die Küchenmannschaft kurz ins Rudern da auch unser erfahrene
              Festwirt Migu nicht mit so vielen hungrigen Mäulern gerechnet hatte. Das
              Ganze war jedoch ein voller Erfolg und gab einen schönen Bazen in die
              Kasse.

        \item 10./11./12./13. Oktober Jubiläumsreise London\\
              Da wir im 2014 unser 140-jähriges Jubiläum feiern durften, konnte sich
              Max ein etwas grosszügigeres und weiter entferntes Reiseziel aussuchen.
              Er organisierte die gesamte Reise mit Unterstützung von Dani. Ziel der
              Reise war der Besuch des Britischen Brass Band-Wettbewerbs in der Royal
              Albert Hall. Auch das Wembleystadion, der Big Ben, die Tower Brig und
              viele weitere Sehenswürdigkeiten wurden in diesen Tagen von den
              Reiseteilnehmern besucht und bestaunt. Ein ebenfalls gut besuchter Event
              war das Londoner Nachtleben sowie die Hotelbar, welche überaus
              grosszügig eingestellt war.  Die perfekt organisierte Reise war ein
              voller Erfolg und an dieser Stelle noch einmal besten Dank an Max für
              seine grossartige Reiseplanung.

        \item 17. /18. Oktober Lotto\\
              Das Lotto war ein voller Erfolg. Einerseits waren die Tische gut
              besetzt, andererseits sind die finanziellen Ergebnisse sehr erfreulich.

        \item 15. /16. November Galakonzert Isenthal\\
              Da die MG Isenthal ihr 75-Jähriges bestehen feiern konnte wurden wir
              angefragt, um dieses Jubiläum musikalisch zu umranden. Diese ehrenvolle
              Anfrage konnten wir natürlich nicht ablehnen und somit wurde daraus für
              uns ein perfektes Vorbereitungskonzert für das Jahreskonzert im Januar.
              Unser kurzes Programm gefiel den Gästen und war für die Isenthaler
              Mehrzweckhalle ein denkwürdiger Event. Als Gratulation überbrachten wir
              der MG Isenthal eine grosse Flasche Wein mit einer Jubiläumsetikette.
              Ein ebenfalls gut besuchter Ort war die Bar. Diese wurde von einigen
              Hildisriedern auf Herz und Niere getestet. In den frühen Morgenstunden
              mussten sich jedoch auch die härtesten Krieger unter uns geschlagen
              geben. Glücklicherweise fanden diese eine bleibe bei unseren
              Isenthalerkollegen. So konnten sie sich am nächsten Tag ebenfalls auf
              die Heimreise begeben.


    \end{itemize}

\end{history}
