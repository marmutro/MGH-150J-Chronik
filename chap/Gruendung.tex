\begin{multicols}{2}

    \subsection{Die Gründung im Jahre 1874}

    Am 25. Jänner des Jahres 1874 ist auf die Initiative
    von Josef Troxler, Moos, die Musikgesellschaft Hildisrieden
    gegründet worden.

    Ihr Ziel und Zweck waren
    die Pflege der Blechmusik, Mitwirkung bei kirchlichen
    Anlässen und nicht zuletzt die gesellige Unterhaltung.
    Acht Männer mit zäher Ausdauer legten den
    Grundstein zur heutigen Musikgesellschaft. Oft brachten finanzielle und musikalische
    Schwierigkeiten das Vereinsschiff ins Wanken.
    Nur eine uneigennützige Hingabe zur Musik und das
    dem jungen Gebilde vonseiten der Bevölkerung geschenkte
    Zutrauen stärkten die Gründer in ihrem
    Durchhaltewillen.

    Die Musikanten erklärten und verpflichteten sich,
    in ihrer Vereinigung sich nicht mit Angelegenheiten
    beschäftigen zu wollen, welche der Musik fremd sind.
    (Politik). Die ersten Statuten von 1874 sind leider
    nicht mehr auffindbar. Ein glücklicher Zufall wollte
    es, dass die ersten Aufzeichnungen im Jahre 1876 revidiert
    und vollumfänglich von Hand geschrieben, zur
    Verfügung stehen.

    Die Musikfreunde, die den Verein gegründet haben, sind:\\

    \noindent
    Josef Disler, Landwirt, Dorf\\
    Jakob Troxler, Metzger, Moos\\
    Alois Troxler, Landwirt, Moos\\
    Josef Troxler, Landwirt, Dorf\\
    Niklaus Troxler, Landwirt, Schopfen\\
    Peter Troxler, Landwirt, Unterschlüssel\\
    Josef Wolf, Schreiner, Breite\\
    Adam Wolf, Tischmacher, Dorf\\


    Dirigent und Kapellmeister, wie er damals genannt
    wurde, war Militärtrompeter Jakob Troxler, Moos.
    Er überwachte den Probenbetrieb, führte die Kasse
    und sorgte für den nötigen Schliff.

    Ihre Einnahmen kamen von den Kilbianlässen. Das
    "Vermögen" wurde auf Jahresende brüderlich verteilt.

\end{multicols}
