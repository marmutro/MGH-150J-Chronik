\begin{multicols}{2}

    \subsection{1925-1950}

    In diesem Vierteljahrhundert machte die Musikgesell-
    schaft auf musikalischem Gebiet grosse Fortschritte.
    In diese Zeitspanne fallen die Besuche der Musikanlässe
    ‘von Hochdorf, Rickenbach, Wolhusen, Willisau,
    'Ebikon, Rain und an das Eidg. Musikfest in St. Gallen.
    Die Teilnahme an diesen Festen, welche an die
    Direktion und an die Bläser hohe Anforderungen
    stellte, liess erkennen, dass der Verein gewillt war,
    eine höhere Stufe zu erreichen und nach Möglichkeit
    diese zu halten. Die vermehrte und auf Blasmusik
    zugeschnittene Literatur verlangte intensivere Proben-
    arbeit, Tüchtiges und zielbewusstes Schaffen brachte
    der Musikgesellschaft Hildisrieden Lorbeeren ein, auf
    die sie sicher stolz sein konnte,

    Ausser dem Besuch von Musikfesten fand der Verein
    noch Zeit, an verschiedenen Veranstaltungen mitzu-
    wirken. So treffen wir ihn 1929 am Festzug des
    Schweiz. Katholikentages in Luzern, als Festmusik an
    der Pferde-Springkonkurrenz und 1943 an der
    Vereranentagung des Kantonal-Schützenverbandes in
    Hildisrieden.

    An der Generalversammlung von 1932 wurde der
    Vorstand für eine weitere Amtsdauer (von 2 Jahren)
    bestätigt:

    Präsident: Josef Disler, Dorf
    Vize-Präsident: Alois Estermann, Traselingen
    Aktuar: Karl Estermann, Traselingen
    Kassier: Heinrich Estermann, Oele
    Beisitzer: Leo Erni, Wirt zum Kreuz.
    Jakob Estermann, Dorf
    Dirigent: Alois Estermann, Traselingen


    Eine besondere Anerkennung sei hier kurz der
    Vereinsführung gezollt. Im Vorstand treffen wir bis
    1945 keinen Wechsel von grosser Bedeutung. Man
    fand stets wieder die geeignete Person, um entstandene
    Lücken zu füllen. Josef Disler, welcher der Musik-
    gesellschaft volle 17 Jahre vorstand, wurde zum Dank
    für die geleisteten treuen Dienste zum Ehren-
    präsidenten ernannt.

    Dass in einer Uniform der Musikant erst so recht zur
    Geltung kommt, ist altbekannt, Die Finanzierung der
    Neu-Uniformierung konnte die Kasse nicht verkraften.
    Durch eine Sammelaktion wurde der benötigte Betrag
    aufgebracht.

    Das Jahr 1939 brachte der Musikgesellschaft etwas
    Ungewolltes. Die in diesem Jahre ausgebrochene Maul-
    und Klauenseuche verhinderte einen geordneten.
    Probenbetrieb. Aber der im September entfachte Zweite
    Weltkrieg, der die Grosszahl der Mitglieder an die
    Grenze rief, wirkte nicht weniger hemmend. Die Musik-
    gesellschaft hatte die Ehre, damals schon eine
    stattliche Zahl Militärtrompeter zu stellen, die während
    der langen Aktivdienstzeit musikalische Weiterbildung
    genossen.

    Solange die Musikgesellschaft bestand, fehlte sie nie
    an einer kirchlichen Feier. Am Fronleichnamstag 1940
    aber waren so viele Musikanten an der Grenze, dass
    der Verein ausserstande war, die Prozession würdig

    zu gestalten. Das Spiel des Inf Reg 86, das in Hildis-
    tieden einquartiert war, gab dem Allerheiligsten das
    Geleit und bekundete so die Verbundenheit von Volk
    und Armee.

    Im Frühjahr 1945 nahm der mörderische Krieg, der
    so viele Tränen fliessen liess, endlich ein Ende. Der
    denkwürdige 8. Mai wurde zum Waffenstillstandstag
    erklärt, Mit dem Wegfall des Aktivdienstes begann
    wieder ein regelmässiges Vereinsleben mit Probenbetrieb
    in alter Form. Der neu gewählte Präsident, Josef
    Rüttimann, wurde allen Aktiven ein Vorbild.

    1949 war für die Musikgesellschaft ein Jubeljahr. Der
    fünfundsiebzigste Geburtstag wurde gebührend gefeiert.


\end{multicols}
