\subsection{1995}

\begin{history}


    \begin{itemize}

        \item 2. Jan.\\
              Bei heftigem Schneetreiben eröffnen wir das Zunftbot mit einem Marsch
              durchs Dorf. Mit einem kurzen Ständchen gratulieren wir dem neuen
              Zunftmeiser Dieter Kern.

        \item 8., 12. und 14. Jan.\\
              Jahreskonzerte im roten Löwen.

        \item 21. Jan.\\
              Wir halten das Konzert im Shopping Center Emmen.

        \item 7. und 8. März\\
              Zusammen mit der Götschizunft meistern wir die Festwirtschaft im
              Festzelt. In diesen 2 Tagen wird ein Rekordumsatz pro Quadratmeter
              erzielt.

        \item 8. April\\
              Am Vorabendgottesdienst zum Palmsonntag halten wir wie gewohnt unser
              Kirchenkonzert.

        \item 23. April\\
              Nach dem Festgottesdienst halten wir unser Weisssontag Ständchen.

        \item 6. und 7. Mai\\
              Auch dieses Jahr betreiben wir unseren obligaten Chilbipfeilbogenstand.

        \item 21. Mai\\
              In der Mehrzweckhalle in Rain tasten wir uns an die Festform heran. Im
              Hinblick auf das Luzerner Kantonale Musikfest in Reiden bestreiten wir
              einen intensiven Probesonntag.

        \item 27. Mai\\
              Firmungständchen

        \item 7. Juni\\
              Gemeinschaftskonzert in Neuenkirch. Zusammen mit der Brass Band Harmonie
              Neuenkirch und der Feldmusik Nottwil halten wir ein Vorbereitungskonzert
              für Musikfest in Reiden. Die Organisation sowie die Zuhörerzahl lässt zu
              wünschen übrig.

        \item 8. Juni\\
              Gemeinschaftskonzert in Rain. Zusammen mit der Harmoniemusik
              Hitzkirchertal und der Harmonie Rain halten wir vor einer ansehnlichen
              Zuhörerzahl das 2. Vorbereitungskonzert.

        \item 17. Juni\\
              Kant. Musikfest Reiden.

              Gut vorbereitet nehmen wir an diesem musikalischen Wettstreit in der 2.
              Klasse Brass Band teil. Wir stellen uns mit dem Selbstwahlstück
              \enquote{Triptych for Brass Band} von Philip Sparke und dem
              Aufgabenstück \enquote{Un Soupcon de Paganini} der Jury. Mit 85.2 im
              Aufgabe- und 89.0 Punkten im Selbstwahlstück (Total 174.2 Punkte)
              belegen wir den 7. Rang von 14 teilnehmenden Vereinen. Mit diesem
              Mittelfeldplatz sind wir zufrieden, kalkulierten jedoch bei unserem
              anspruchsvollen Selbstwahlstück mit etwas mehr Punkten. Bei gutem
              Festwetter bestreiten wir anschliessend den Marschmusikwettbewerb.
              Dieser Auftritt lässt bezüglich Marschdisziplin sowie der musikalischen
              Leistung zu wünschen übrig, was uns auch die Jury bestätigt. (49. Rang,
              85.4 Punkte)

        \item 18. Juni\\
              Von sehr vielen Vereinen und Freunden werden wir am Sonntag herzlich
              empfangen. Sie begleiten uns zum Löwenplatz, wo wir auf das vergangene
              Musikfest anstossen. Anschliessend geniessen wir aus der Löwenküche ein
              feines Nachtessen.

        \item 9. Juli\\
              Bei prächtigem Sommerwetter führen wir unser Waldfest im Meierholzwald
              durch. Dieses Jahr beschränken wir uns wieder auf 1 Festtag. Dieses Jahr
              haben wir 2 neue Attraktionen: Neben der Steinstossdemonstration von
              Ruedi Muri und Ruedi Schuler errichten wir erstmals ein Waldpub.

        \item 1. Aug.\\
              Wie gewohnt nehmen wir an der Bundesfeier auf dem Schulhausplatz teil.

        \item 16. Sept.\\
              Bei heftigen Regenschauer empfangen wir die Schützengesellschaft vom
              eidg. Schützenfest in Thun.

        \item 24. Dez.\\
              Mit einigen Weihnachtsliedern stimmen wir die Dorfbevölkerung auf das
              bevorstehende Fest ein.


    \end{itemize}

\end{history}
