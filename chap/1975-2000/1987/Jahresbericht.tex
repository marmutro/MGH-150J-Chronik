\subsection*{1987}

\begin{history}


    \begin{itemize}

        \item 2. Jan.\\
              Am Zunftbot Josef Fleischli, Holzmatt, zum Zunftmeister erkoren.

        \item 11. Jan.\\
              Konzerthauptprobe im Löwensaal.

        \item 14. und 17. Jan.\\
              Jahreskonzerte mit Theater im Löwensaal.

        \item 24. Jan.\\
              Konzert im Shopping Center Emmen.

        \item 20. Feb.\\
              Wir umrahmen den Schulbesuch vom Zunftmeister.

        \item 22. April\\
              Wir spielen an der der CVP-Wahlversammlung einige Märsche.

        \item 7. Mai\\
              Heidi Koch-Amberg wird in den grossen Rat gewählt. Die Dorfbevölkerung
              ehrt sie mit einem Festzug durchs Dorf.

        \item 8. Mai\\
              Expertisenprobe mit Franz Renggli für das Unterwaldner Musikfest in
              Sarnen.

        \item 13. und 14. Juni\\
              Unterwaldner Musikfest Sarnen.

              Wir pielen zuerst den Marsch
              \enquote{Diavolezza}. Am Nachmittag spielen wir in der Turnhalle die
              Ouverture \enquote{A Suite for Switzerland} von Roy Newsome. Da die
              Resultate erst am Sonntag bei der Rangverkündigung bekannt gegeben
              werden, beginnt nun das grosse Rätselraten um gewonnene oder evtl.
              verlorene Punkte. Nach dem Bankett stieg dann in der Linden in Sarnen
              ein Fest, so dass die Wirtschaft den Musikanten und die Musikanten dem
              Wirt sicher noch einige Zeit in Erinnerung bleiben wird.

              Auch am Sonntag traf sich eine stattliche Zahl Hildisrieder Musikanten
              zur Rangverkündigung in Sarnen ein. Der 20. Platz von 28 Teilnehmern war
              eher eine magere Ausbeute. Besser ging es in der Marschmusik: Mit
              90$\sfrac{1}{2}$ Punkten belegen wir den 5. Rang.

        \item 20. Juni\\
              Wir spielen nach der Firmung auf dem Schulhausplatz.

        \item 12. Juli\\
              Waldfest im Meierholzwald. Das unsichere Wetter am Sonntagvormittag
              wirkt sich nicht positiv auf. Der Besucheransturm blieb im Rahmen.

        \item 14. Juli\\
              Wir gratulieren den neuen Gemeinderäten Walter Gemperli, Artur Maron und
              Zita Müller mit einem Ständchen.

        \item 18. Juli\\
              Wir helfen eine Unteroffizier-Brevetierung in der Schlacht musikalisch
              gestalten.

        \item 29. Aug.\\
              Musikhock im Wagenschopf in der Waldegg. Bei einem feinen Chäsbuffet,
              Wein und Kaffe pflegen wir die Gemütlichkeit.

        \item 25. Sept. - 6. Okt.\\
              Die MG führt das grosse Festzelt für den Kirchenbazar.

    \end{itemize}

\end{history}
