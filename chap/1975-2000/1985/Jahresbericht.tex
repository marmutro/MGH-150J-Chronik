\begin{history}

    % \subsection{Jahresbericht}

    \begin{itemize}

        \item[]2. Jan.\\
        Nach einem Glas Wein im Pavillon beim Schulhaus gehts mit Marschmusik
        zum Löwen, wo am Zunftbot der neue Zunftmeister Robert
        Estermann-Wangeler Baumeister inthronisiert wird.

        \item[]5. und 12. Jan.\\
        Am ersten Jahreskonzertabend mochte sich der Löwensaal nicht so recht
        füllen, was sicher der eisigen Kälte zuzuschreiben war.

        \item[]19. Jan.\\
        Konzert im Shopping Center Emmen.

        \item[]17. Feb.\\
        Unser Fasnachtswagenmotto sind die Wiener Philharmoniker. Die Musik und
        die riesengrosse Bassgeige -- gebastelt und gespielt von Toni Bachmann
        -- fand beim Publikum grossen Anklang.

        \item[]21. April\\
        Eine Gruppe aus jüngeren Musikanten hilft den Samstagabend- und
        Sontagsgottesdienst gestalten.

        \item[]23. April\\
        Wir beschliessen den Kauf von Trommeln für die Tambourengruppe.

        \item[]28. April\\
        52 Kinder erhalten die Firmung. Wir geben ein Ständchen.

        \item[]7. Mai\\
        Wir ersetzen 5 Trompeten mit neuen Trompeten der Marke Getzen. Auch die
        Anschaffung eines Bach-Flügelhorns wird beschlossen.

        \item[]2. Juni\\
        Das neue Schulhaus wird eingeweiht, wir spielen beim Festakt.

        \item[]21. Juni\\
        Das Jahr 1985 wird zum Jahr der Musik ernannt. Aus diesem Anlass
        bestimmt der eidg. Musikverband den 21. zum Tag der Blasmusik. Im ganzen
        Land geben viele Vereine ein Ständchen.

        Wir ziehen mit Marschmusik durchs Dorf und in die Quartiere. Leider hat
        es nur wenige Zuschauer.

        \item[]23. Juni\\
        Wir geben in der Gartenwirtschaft vom Rest. Schlacht ein Ständchen.

        \item[]7. Juli\\
        Nach vierjährigem Unterbruch organisieren wir wieder ein Waldfest,
        diesmal beim neuen Festplatz im Meierholz-Wald. Der Besucheransturm war
        aber nicht so gross.

        \item[]9. Juli\\
        Brötlete im Traselinger Wald nach der Probe.

        \item[]12. Juli\\
        Das Bassregister lädt uns in den neu erstellten Wagenschopf in die
        Waldegg ein.

        \item[]21. Juli\\
        Am Vormittag spielen wir beim Kurplatz in Luzern ein unterhaltsames
        Programm.

        \item[]1. Aug.\\
        Wir helfen wieder die 1. August Feier zu gestalten.

        \item[]7. Sept.\\
        Musikhock im Schopf von Walter Rüttimann.

        \item[]16. Nov.\\
        Im Radio-DRS-Spiel \enquote{Spielplatz} kämpft die Gemeinde Hildisrieden
        um Punkte. Zur Eröffnung wird unsere Aufnahme vom Marsch
        \enquote{Diavolezza} gesendet.

        \item[]6. Dez.\\
        Der Samichlaus besucht uns während der Probe. Nebst Lobworten musste er
        auch einige Ermahnungen anbringen.

    \end{itemize}

\end{history}
