\subsection{1977}

\begin{history}


    \begin{itemize}

        \item 7. Jan.\\
              Nach einem Glas Wein im Rest. Kreuz geht es mit Marschmusik zum Löwen.
              Wir spielen dem neu gewählten Zunftmeister Martin Estermann Traselingen
              2 Märsche. Erst in den späten Morgenstunden begibt man sich auf den
              Heimweg.

        \item 15. und 22. Jan.\\
              Angesichts der grossen Schneefälle dürfen wir nicht allzu grossen
              Andrang am Winterkonzert erwarten.

        \item 17. April\\
              Nach 2 Jahren wetterbedingtem Unterbruch konnten wir wieder die
              Kommunionskinder bei strahlendem Frühlingswetter zur Kirche begleiten.
              Nach dem Ständchen offerierte Kreuzwirt Hugo Fleischli einen Aperitif.

        \item 22. Mai\\
              Mit Privatautos begeben wir uns an den Musiktag in Hergiswil b.W.
              Experte Alois Gschwind Dornach begutachtete unseren Vortrag von
              \enquote{Simfonietta Pastorale}. Am Nachmittag ziehen wir mit dem Marsch
              Kameraden am aufmerksamen Publikum vorbei.

        \item 24. Juli\\
              Nachdem wir das Waldfest am 17. Juli verschieben mussten, können wir uns
              heute an einem herrlichen Sommertag erfreuen. Wir beginnen das Fest um
              11 Uhr mit Braten vom Grill, 60 kg können wir verkaufen. Dieses Jahr
              wird die Reisekasse wieder einen rechten Zustupf bekommen.

        \item 1. Aug.\\
              Diese Jahr findet die Feier des Nationalfeiertag auf dem Schulhausplatz
              statt.

        \item 28. Sept.\\
              Das Dorforiginal und Hofcoiffeur von Hildisrieden Fritz Gradwohl stirbt
              mit 81 Jahren. Mit 2 Chorälen geben wir ihm das letzte Geleit.

        \item 17. Dez.\\
              Am Frühschoppenkonzert im Shopping Center Emmen spielen wir Stücke, die
              wir am Winterkonzert 78 vortragen werden.

    \end{itemize}

\end{history}
