\begin{history}

    % \subsection{Jahresbericht} 1989

    \begin{itemize}

        \item[]2. Jan.\\
        Nach dem Einzug in den roten Löwen wird am Zunftbot Walter Schmid zum
        Zunftmeister gewählt.

        \item[]8., 11. und 14. Jan.\\
        Wir laden zum Jahreskonzert ein.

        \item[]21. Jan.\\
        Einen Teil unseres Konzertprogramms spielen wir heute Vormittag im
        Shopping Center Emmen.

        \item[]2. April\\
        Die MGH spielt beim Einzug am Weissen Sonntag einen Parademarsch. Nach
        dem Gottesdienst geben wir noch ein Ständchen.

        \item[]13. Mai\\
        Die Musikgesellschaft hilft den Muttertagsgottesdienst gestalten. Danach
        spielen wir einige Märsche.

        \item[]28. Mai\\
        Bei schönstem Sonntagsmorgenwetter halten wir das Kurplatzkonzert in
        Luzern. Am Nachmittag nehmen wir ander Fahnenweihe der Feldmusik Rain
        teil. Weil wir nur eine kurze Mittagspause haben, wird das Mittagessen
        gemeinsam im roten Löwen eingenommen.

        \item[]10. und 11. Juni\\
        Musiktag in Menznau. Wir treffen am Samstagabend um 19 Uhr ein. Die
        Vorprobe haben wir im Saal des Gasthof Krone. In der vollen Rickenhalle
        spielen wir um 20.47 Uhr das Stück \enquote{March Prelude} von Edward
        Gregson. Im anschliessenden Gespräch mit dem Experten fand dieser dann
        durchwegs lobende Worte. Zur Marschmusik mussten wir am Sonntag um 14.30
        antreten. Wir spielen den Marsch \enquote{King Size} von Fred L. Frank.
        Um 16 Uhr ist im Festzelt die Veteranenehrung. Drei unserer Kameraden
        befinden sich auch im Kreis der geehrten. Zum ersten mal in der
        Geschichte der MGH kan ein Vereinsmitglied für 60 Jahre aktives
        musizieren geehrt werden. Kaspar Troxler bekommt zu diesem Anlass die
        Medaille des Internationalen Musikbundes CISM. Am Abend feiern wir die
        Veteranen im Löwen.

        \item[]20. Juni\\
        Das Es-Horn Register lädt zum Musikhock in den Wagenschopf von Kaspar
        Dörig ein.

        \item[]9. Juli\\
        Im 2. Anlauf können wir das Waldfest doch noch durchführen. Wir starten
        erstmals mit einem Feldgottesdienst.

        \item[]27. Aug.\\
        Anstelle des Musikhocks führen wir diese Jahr erstmals ein
        Familienpicknick durch. Dank des grosszügigen Wagenschopfs von Walter
        Rüttimann legt uns Petrus mit seinen Regenwolken keinen Stein in den
        Weg.

        \item[]24. Sept.\\
        Die Schützengesellschaft weiht seine neue Fahne. Wir nehmen am Festumzug
        und am Festakt im Zelt teil.

    \end{itemize}

\end{history}
