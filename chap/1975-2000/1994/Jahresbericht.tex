\subsection{1994}

\begin{history}


    \begin{itemize}

        \item 2. Jan.\\
              Die Musikgesellschaft nimmt am Zunftbot teil. Wir können mit einem
              Ständchen dem neuen Zunftmeister Alois Estermann Traselingen
              gratulieren.

        \item 9., 13. und 15. Jan.\\
              Im Löwensaal halten wir unsere Jahreskonzerte.

        \item 22. Jan.\\
              Wir geben das jährliche Frühschoppenkonzert im Shopping Center Emmen.

        \item 10. April\\
              Wegen der misslichen Wetterbedinguen wird der festliche Einzug zum
              Weissen Sonntag abgesagt. Wir geben nach dem Gottesdienst ein Ständchen.

        \item 30. April und 1. Mai\\
              Auch an der Chilbi 94 betreiben wir unseren obligaten Pfeilbogenstand.

        \item 1. Mai\\
              Unter dem Motto \enquote{3 x 30} findet das 90. Kirchenweihfest statt.
              Wir geben nach dem Festgottesdienst der Bevölkerung ein Ständchen.

        \item 8. Mai\\
              Mit unseren rassigen Klängen verleihen wir auch dieses Jahr dem
              Muttertag eine besondere Note.

        \item 2. Juni\\
              Eine Kleinformation begleitet den Fronleichnahmsgottesdienst auf dem
              Schulhausplatz. Anschliessend führen wir die feierliche Prozession zur
              Kirche. Dieser Auftritt endet dan mit einem kurzen Ständchen auf dem
              Schulhausplatz.

        \item 4. Juni\\
              Kant. Musiktag in Ermensee.

              Mit dem Marsch \enquote{Gret little Army}
              bestreiten wir die Marschmusikkonkurrenz. Die Marschdisziplin wird als
              sehr gut, die musikalische Darbietung eher als mässig beurteilt. Am
              Abend stellen wir uns dann in der voll besetzten Mehrzweckhalle mit der
              \enquote{Romantischen Ouverture} von Stephan Jäggi der Jury. Mit voller
              Konzentration und grossem Einsatz geben wir diese Komposition zum
              Besten. Dies wurde mit grossem Applaus und einem guten Jurybericht
              belohnt. Natürlich widmen wir uns jetzt auch mit gleichem Einsatz dem
              gemütlichen Teil dieses Festes, was bei einigen bis in die Morgenstunden
              anhielt.

        \item 16. Juni\\
              An diesem schönen Sommerabend geben wir auf dem Kurplatz in Engelberg
              ein Konzert. Vom internationalen Publikum können wir viel Applaus
              entgegennehmen. Anschliessend geniessen wir den Abend bei einem feinen
              Nachtessen in der Bäknlialp.

        \item 18. Juni\\
              Nach dem Gottesdienst halten wir für die Neugefirmten ein Ständchen.

        \item 21. Juni\\
              Heute unterhalten wir die Bevölkerung von der Waldmatt mit unseren
              Klängen.

        \item 2. und 3. Juli\\
              Be super schönem Sommerwetter halten wir unser Waldfest. Am Sonntag
              servieren wir Braten, am Samstag Spagetti. Für die Unterhaltung sind die
              Rigi- und die Ronspatzen verantwortlich. Der Besucheraufmarsch wie auch
              der Reingewinn waren andere Jahre auch schon besser.

        \item 7. Juli\\
              Das 2. Konzert in Engelberg bestreiten wir im Kursaal.

        \item 10. Juli\\
              Der HSV organisiert ein Fussballspiel GC gegen Aalborg aus Dänemark. Wir
              eröffnen dieses Spiel mit einem Ständchen und Marschmusik auf dem Rasen.

        \item 27. und 28. Aug.\\
              Im gewohnten 2-Jahrestournus besammeln wir uns zur Musikreise.
              Reiseführer Armin Schmid führt uns mit dem Reisecar nach Elm zur
              Erbsenalp. Nach einer kurzen Pause geniessen wir dann bei schönem Wetter
              und herrlicher Fernsicht die Wanderung zu unserem Nachtlager Skihaus
              Schabell in Empächli. Dort erleben wir einen gemütlichen Abend im Kreise
              der Kollegen. Nach einer kurzen Nacht gestaltet jeder den Sonntag
              individuell. Nach einer kürzerern oder längeren Wanderung, einem
              erfrischenden Bad im Chuebodensee oder einem Jass treffen wir uns dann
              um 15 Uhr 47zur Abfahrt in Elm.

        \item 27. Sept.\\
              Einstimmig beschliessen wir die Teilnahme am kant. Musikfest 1995 in
              Reiden und den Kauf von einem B-Bass, einem Es-Horn und einem
              Occasion-Euphonium.

        \item 29. Okt.\\
              Erstes Galakonzert im Bahnhof-Shopping Luzern. Unser rassiges und
              abwechslungsreiches Repertoir lockt sofort viele Zuhörer an.

        \item 6. Nov.\\
              Der Schwingclub Oberseetal feiert sein 75-jähriges Bestehen mit einer
              Fahnenweihe. Das Festbankett im Löwen umrahmen wir musikalisch.

    \end{itemize}

\end{history}
