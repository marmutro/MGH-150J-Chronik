\subsection{Jahresbericht 1990}

\begin{history}


    \begin{itemize}

        \item[]2. Jan.\\
        Die Musikgesellschaft nimmt am Zunftbot teil. Alfred Bachmann wird am
        Zunftbot zum Zunftmeister gewählt.

        \item[]14., 17. und 20. Jan.\\
        Jahreskonzert, zum ersten mal dirigiert von unserem neuen Direktor
        Marcel Sennhauser.

        \item[]27. Jan.\\
        Wir spielen im Shopping Center Emmen das traditionelle
        Vormittagskonzert.

        \item[]7. April\\
        In der Kirche halten wir nach dem Gottesdienst ein Kirchenkonzert.

        \item[]22. April\\
        Weisser Sonntag. Wir begleiten mit einem Parademarsch die
        Erstkommunikanten in die Kirche.

        \item[]27. Mai\\
        Für eine ganztägige Musikprobe dislozieren wir ins Lehrerseminar nach
        Hitzkirch. Unter besten Voraussetzungen arbeiten wir an diesem Sonntag
        sehr intensiv an den beiden Wettkampfstücken für das kant. Musikfest in
        Schüpfheim.

        \item[]29. Mai\\
        Nach der Probe entscheiden wir über eine Teilnahme am Eidg. Musikfest in
        Lugano im Jahr 1991, was vom Verein klar abgelehnt wird.

        \item[]5. Juni\\
        Gemeinschaftskonzert in der Festhalle Sempach. Gemeinsam mit den
        Musikgesellschaften Harmonie Sempach und Harmonie Rain tragen wir die
        Wettstücke des kant. Musikfestes Schüpfheim vor.

        \item[]16. und 17. Juni\\
        Kantonales Musikfest Schüpfheim.

        Bereits um 7.40 müssen wir
        in Schüpfheim sein un dmit der Vorprobe beginnen. Bei einem lockeren
        Einblasen versuchen wir die Nervosität in den Zügeln zu halten. Um 8.45
        gilt es zum erstenmal ernst. Im Saal des Hotels Adler tragen wir das
        Aufgabenstück \enquote{Dies acterna} von Pascal Faver vor. Die Jury
        bewertet unsere Leistung mit 159.5 Punkten. Diese Punktzahl erweckte
        zuerst ein bisschen Missmut, war aber in der Endabrechnung gar nicht so
        schlecht. Um 9.30 stellten wir uns der Jury in der Pfarrkirche mit dem
        Selbstwahlstück \enquote{A Holiday Suite} von Eric Ball. Uns gelang die
        Vorstellung super, die Experten bedankten sich mit 168 Punkten. Dies gab
        zusammen 327.5 Punkte, was den ausgezeichneten 5. Rang bedeutete.

        Nächster Auftritt war die Marschmusikkonkurrenz am Nachmittag. Der Start
        zum Marsch \enquote{Hessen} gelang nicht ganz nach Wunsch, mit 41
        Punkten landeten wir im Mittelfeld. Danach kam der gemütliche Teil des
        Festes. Schüpfheim zeigte sich natürlich auch da von der besten Seite.

        Am Sonntagabend werden wir, wieder heimgekehrt, von den Dorfvereinen
        empfangen.

        \item[]15. Juli\\
        Bei bestem Sommerwetter findet das Waldfest statt.

        \item[]25. und 26. Aug.\\
        Musikreise. Um 12 Uhr besammeln wir uns beim Schulhaus. Die Firma
        Estermann Beromünster führt uns via Brünig bis zum Thunersee wo wir in
        Faulensee einen Kaffeehalt machen. Weiter gehts durchs Simmental bis
        nach Schönried. Die Gondelbahn bringt uns auf den Rellerligrat. Leider
        treffen wir nicht sonderlich schönes Wetter an, so dass die
        Vorabendwanderung ins Restaurant verlegt wird. Bei einem feinen
        Nachtessen und anschliessendem Tanz ging es ganz gemütlich zu. Erst ein
        paar ernste Worte des Wirtes mochten die letzten fürs Schlafengehen zu
        überzeugen. Der Fussmarsch ging am Sonntag wieder nach Schönried. Der
        Car brachte uns dann nach Greyerz, wo wir das Städtchen besichtigten.
        Auf der Weiterfahrt über Friboug-Bern nach Ursenbach holte mancher den
        versäumten Schlaf der letzten Nacht nach. Bei Georg Duss in Ursenbach
        machten wir den Zobighalt, ehe es wieder Richtung engere Heimat ging. Um
        20 Uhr kehrten wir nach Hildisrieden zurück.


    \end{itemize}

\end{history}
