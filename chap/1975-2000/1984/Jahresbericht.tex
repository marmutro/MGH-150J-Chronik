\begin{history}

    % \subsection{Jahresbericht} 1984

    \begin{itemize}

        \item[]2. Jan.\\
        Nach einem Glas Wein im Pavillon beim Schulhaus gehts mit Marschmusik
        zum Löwen, wo am Zunftbot unser Musikkamerad Alois Estermann zum
        Zunftmeister gewählt wird.

        \item[]7. und 14. Jan.\\
        Jahreskonzert. Die drei Tänzerinnen Hildegard Dörig, Margrith Muri und
        Marie-Louise Elmiger begleitet von den Ronspatzen konnten mit ihrer
        Aerobiceinlage das Publikum voll begeistern.

        \item[]29. Feb.\\
        Zunftmeister Alois Estermann besucht die Schule. Wie immer ist auch die
        Musik dabei mit einer kleineren Besetzung.

        \item[]29. April\\
        Bei schönem aber kühlem Frühlingswetter begleiten wir die
        Kommunionkinder mit einem Prozessionsmarsch in die Kirche.

        \item[]30. April\\
        Experte Franz Renggli begutachtet unser Können während einer Probe im
        Löwen. Danach erläutert er uns die Schwachstellen von \enquote{London
            River} und \enquote{Capriccio für Blasmusik}.

        \item[]5. und 6. Mai\\
        Die MGH darf dieses Jahr die Chilbi durchführen und wir machen einen
        Reingewinn von 10700 Fr.

        \item[]13. Mai\\
        Muttertagsständchen bei fast Spätherbstwetter.

        \item[]26. und 27. Mai\\
        Der Musiktag in Neuenkirch ist Hauptprobe für das Solothurner
        Kantonalmusikfest in Balsthal. Am Samstagabend spielen wir in der Kirche
        das Selbstwahlstück \enquote{London River}. Als Experte amtet Fritz
        Vögelin, der unseren Vortrag sehr gelungen fand. Am Sonntagnachmittag
        treten wir an zur Marschmusik mit dem Experten Josef Keist.

        \item[]5. Juni\\
        Franz Renggli hält mit uns eine sehr anspruchsvolle und lehrreiche
        Expertisenprobe.

        \item[]14. Juni\\
        Vorbereitungskonzert in Buttisholz mit der Feldmusik Ettiswil und der
        Feldmusik Buttisholz.

        \item[]15. Juni\\
        Vorbereitungskonzert in der Kirche Hildisrieden mit der
        Musikgesellschaft Egolzwil und der Feldmusik Buttisholz.

        \item[]24. Juni\\
        Nachdem mit etwas Verspätung auch die Gundolinger auf dem Schulhausplatz
        eingetroffen sind, geht es mit dem Car von Edi Scherz, Lenzburg, nach
        Balsthal.

        Zuerst spielen wir das Aufgabstück 3. Klasse \enquote{Capriccio für
            Blasmusik} in der Kirche. Danach das Selbstwahlstück \enquote{London
            River} in der Turnhalle. Wir erreichen Total 338 Punkte. Auf dem
        Computer sind wir auf den 1. Platz vorgerrückt. Am Nachmittag
        erreichen wir bei der Marschmusik 88 Punkte, d.h. 16. von 56
        Vereinen.

        Wegen eines heftigen Gewitters findet der Festakt im Festzelt statt, wo
        wir wegen Platzmangel in der Mitte zusammenstehen. Als unser 1. Rang
        verkündet wurde flogen die Hüte unters Zeltdach. Auf dem Hildisrieder
        Schulhausplatz erwartete uns viel Volk und die Dorfvereine, um uns zu
        gratulieren.

        \item[]1. Juli\\
        Wir eröffnen den Festzug bei der Neuuniformierung der Harmonie Rain.
        Auch im Festzelt sollen wir als erste unser Ständli spielen, leider hat
        unser Dirigent seine Noten zu Hause gelassen.

        \item[]1. Aug.\\
        Nationalfeier auf dem Schulhausplatz.

        \item[]1. und 2. Sept.\\
        Bei herrlichem Sonnenschein fahren wir mit dem Car von Estermann
        Beromünster via Entlebuch nach Interlaken. Dort besteigen wir das Schiff
        bis Iseltwald. Von da gehts zu Fuss dem Ufer entlang zu einem
        Picknickplatz und weiter zu den Giessbachfällen. Nach dem Zöbig fährt
        uns der Car hoch zur Axalp. Nach dem feinen Nachtessen gehts mit Musik
        und Gesang bis in die Nacht.

        Am Sonntagnachmittag haben wir Zeit für kleine Wanderungen. Das
        Mittagessen nehmen wir in Brienz, danach haben wir Zeit zum Pedalofahren
        oder Minigolfspielen.

        \item[]21. - 25. Sept.\\
        Den Auftakt der Fahnenweihe-Feiern bildet am Freitagabend der Höck der
        Musikfreunde und der einheimischen Bevölkerung mit dem Patenpaar Lisbeth
        Lang-Ruckli und Fritz Amrein. Trotz heftigen Sturmböen sind wir mit der
        Besucherzahl zufrieden.

        Am Samstagabend sorgt das Tanzorchester Dreams trotz Regen für gute
        Stimmung. Auch die Musikantenbar ist komplett überfüllt.

        Am Sonntag muss die Fahnenweihe wegen des schlechten Wetters in die
        Kirche verlegt werden. Das Ständchen und den Apero nach dem Gottesdienst
        müssen wir fluchtartig in das Festzelt verschieben. Beim anschliessenden
        Bankett sind einige prominente Persönlichkeiten eingeladen. Beim
        anschliessenden Festzug marschieren 17 Gruppen mit. Nach dem Festzug
        besammelt sich Alt und Jung im Festzelt um den Vorträgen der
        benachbarten Musikvereine zuzuhören. OK Präsident Walter Schmid hielt
        Rückblick auf die 110-jährige Vereinsgeschichte.

        Den Dienstagabend bestreiten unsere Dorfvereine und die Swiss Rolling
        Brass Band unter der Leitung von Franz Renggli. Als Conferencier führt
        Klaus Estermann durch den Abend.

        \item[]1. Dez.\\
        Als Dank und Anerkennung laden wir über 100 Helferinnen und Helfer zu
        einem Schlusshock die Schulaula ein.

    \end{itemize}

\end{history}
