\begin{history}

    % \subsection{Jahresbericht} 1992

    \begin{itemize}

        \item[]2. Jan.\\
        Die Musikgesellschaft nimmt am Zunftbot der Götschizunft teil. Mit Josef
        Rüttimann wird ein Ehrenmitglied der Musikgesellschaft zum Zunftmeister
        gewählt.

        \item[]5., 9. und 11. Jan.\\
        Wir laden die Bevölkerung zu unserem Jahreskonzert ein.

        \item[]25. Jan.\\
        Im Shopping Center Emmen geben wir das alljährliche Konzert.

        \item[]11. April\\
        In der Kirche spielen wir ein besinnliches Konzert. Erstmals spielen wir
        bereits während dem Gottesdienst.

        \item[]26. April\\
        Wir begleiten die Erstkommunikanten an ihren Freudentag.

        \item[]2. und 3. Mai\\
        Dieses Jahr ist die Musikgesellschaft für die Durchführung der Chilbi
        verantwortlich. Trotz kühlem Wetter wurde auch diese Chilbi wieder zu
        einem Grosserfolg.

        \item[]10. Mai\\
        Nach dem Gottesdienst ehren wir die Frauen und Mütter mit dem
        traditionellen Muttertagsständchen.

        \item[]21. Mai\\
        Aus der Uniformenkollektion der Firma Kleider Frei stellen wir heute
        unsere neue Uniform zusammen. Wir wägen den ganzen Abend aus der
        Vielfalt der Möglichkeiten ab. Am Schluss des Abends sind wir soweit,
        dass wir zwei Modelle in der engeren Auswahl haben. Das eine ist ein
        grauer Kittel mit anthrazitfarbenen Hosen, und das andere ist ein
        weinroter Kittel mit grauen Hosen. In der nächsten Runde sehen wir
        Modelle dieser beiden Uniformen.

        \item[]23. Mai\\
        Anlässlich der Firmspende geben wir ein Ständchen.

        \item[]31. Mai\\
        Zur Vorbereitung des Unterwaldner Kantonalen Musikfestes bereiten wir
        uns an einem Probesonntag in Hitzkirch vor.

        \item[]13. Juni\\
        Wir besuchen den Unterwaldner Musikfest in Hergiswil. Gut vorbereitet
        stellen wir uns um 14.25 in der Aula des Schulhauses mit dem
        Selbstwahlstück \enquote{Oregon} von Jacob de Haan der Jury. Diese
        bewertet unseren Vortrag mit 152 Punkten. Um 15.00 spielen wir in der
        Turnhalle das Aufgabenstück \enquote{Der Torero und die Zigeunerin} vom
        Luzerner Komponisten Otto Haas. Mit unserem Vortrag mochten wir der Jury
        sehr gut zu gefallen. Sie bedankte sich mit 169 Punkten. Mit einem Total
        von 321 Punkten reicht dies zum 4. Platz von 12 Konkurrenten in der 2.
        Stärkeklasse. Um 17.15 marschierten wir mit dem Marsch \enquote{Front
            and Center} von Pat Lee an den Kampfrichtern vorbei. Wir mochten nicht
        ganz zu überzeugen, die erreichten 83 Punkte reichten nur zum 17. Rang.
        Nach einem feinen Nachtessen geniessen wir  noch manche frohe Stunde im
        Lopperdorf.

        \item[]20. und 21. Juni\\
        In Rothenburg findet der Jubiläums-Musiktag statt. Um 18.45 spielen wir
        in der Chärnshalle das Stück \enquote{Oregon}. Wir mochten unseren
        Experten Ives Illi mit dem Vortrag zu überzeugen, was er mit einem guten
        Expertenbericht honorierte. Am Sonntag Mittag eröffnen wir die
        Marschmusikvorträge mit dem Marsch \enquote{Bärner Musikante} von Walter
        Joseph.

        \item[]5. und 12. Juli\\
        Das Waldfest fällt dieses Jahr wegen schlechtem Wetter ins Wasser.

        \item[]29. und 30. Aug.\\
        Der diesjährige Ausflug führt ins Bündnerland. Unter der Reiseleitung
        von Toni Bachmann reisen wir mit Estermann Reisen via Walensee nach
        Maienfeld. Noch bei Regen gibt es hier den Kaffehalt. Weiter geht es ins
        Prättigau nach Davos. Nach einer Irrfahrt von Sepp Estermann finden wir
        doch noch die Talstation der Jakobshornbahn. Die Gondel bringt uns bis
        zur Mittelstation. Im Massenlager beziehen wir unser Zimmer. Leider ist
        das Wetter schlecht, so dass wir die Zeit mit Jassen verbringen. Nach
        einem feinen Nachtessen geniessen wir einen gemütlichen Abend im
        Bergrestaurant, der für einige bis fast zum Morgen dauert. Der Sonntag
        zeigt sich von der allerschönsten Seite. Die Sonnenstrahlen animieren
        jeden zu einer Wanderung. Dir Rückreise führt uns über Tiefencastel,
        Thusis nach Reichenau, von dort durch das Rheintal nach Sedrun. Hier
        machen wir Zobighalt. Über den Oberalppass erreichen wir dann wieder die
        Innerschweiz. Müde aber mit einer superschönen Reiseerinnerung erreichen
        wir am Sonntagabend wieder Hildisrieden.

        \item[]2. Sept.\\
        Die Firma Kleider Frey stellt uns die zwei Modelle vor. Die Mehrheit
        entscheidet sich für das weinrote Modell mit Schattenstreifen und grauer
        Hose.

        \item[]3. Okt.\\
        Die Musikgesellschaft beteiligt sich am Bazar zugunsten des Alters- und
        Pflegeheims Sonnmatt in Hochdorf. Wir führen das \enquote{Jägerstübli}.


    \end{itemize}

\end{history}
