Jahresbericht 1974:

Das Jahr 1974 begann mit einer Diskussion über finanzielle Angelegenheiten, bei
der beschlossen wurde, beim Winterkonzert kein Nachtessen mehr zu bezahlen,
sondern stattdessen den Vereinsmitgliedern ein Bier anzubieten. Ein
Jubiläumskonzert im Januar verzeichnete eine beachtliche Zuschauerzahl. Das
Winterkonzert in Emmen konnte hingegen nur eine begrenzte Anzahl Zuhörer
verzeichnen. Ein Zunftbot, die 100. Generalversammlung, Uniformanpassungen und
die Begrüßung der Neuzuzüger prägten das erste Quartal. Die Firma Schuler aus
Rothenturm lieferte die neuen Uniformen, und das Jahr gipfelte in der Teilnahme
an der 100-Jahr-Feier der Musikgesellschaft Hildisrieden. Das Fest umfasste
Konzerte, Festumzüge und verschiedene musikalische Darbietungen. Nach dem Fest
erfolgten interne Entscheidungen bezüglich der Finanzierung der Uniformen und
der Teilnahme an kommenden Veranstaltungen.

Jahresbericht 1975:

Der Jahresbeginn 1975 war geprägt von der Wahl des neuen Zunftmeisters und der
Entscheidung, auf den traditionellen Musikhock am schmutzigen Donnerstag zu
verzichten. Das Winterkonzert verzeichnete erneut hohe Besucherzahlen, während
das Konzert in Emmen auf geringes Interesse stieß. Im Februar erfolgte die
Anpassung der Uniformen, gefolgt von der Adressenverteilung für die
Bettelaktion. Die Begrüßung der Neuzuzüger und die Lieferung der neuen Uniformen
markierten den März und April. Die Teilnahme an der 100-Jahr-Feier der
Musikgesellschaft Hildisrieden im Mai war ein herausragendes Ereignis des
Jahres. Das Vereinsintern beschlossene Selbstfinanzierungsmodell für die Hemden
und die Teilnahme am Kantonalmusiktag in Reiden im Juni waren weitere
Highlights. Die Jahresmitte wurde mit einem Schlusshock und der Teilnahme an der
Schlachtfeier in Sempach gefeiert. Ein verschobenes Waldfest im Juli und die
Teilnahme an der 1. August-Feier rundeten das Sommerprogramm ab. Der
Musikausflug auf die große Scheidegg und die Besprechung der
Hundertjahrfeierabrechnung im September schlossen das Vereinsjahr ab.

Jahresbericht 1976:

Das Jahr 1976 begann mit der Wahl des neuen Zunftmeisters Franz Estermann. Die
Winterkonzerte im Januar zogen erneut viele Besucher an. Im Februar wurden
Leintücher für den Fastnachtsumzug vorbereitet, bei dem das Vereinssujet
präsentiert wurde. Die Frühjahrsmonate waren von Prozessionen und
Lohnanpassungen für den Direktor geprägt. Der Mai brachte einen musikalischen
Beitrag zur Orientierungsversammlung über die Uni Luzern. Der Musikausflug im
August führte die Mitglieder nach Adelboden. Der Jahresabschluss umfasste das
Frühschoppenkonzert im Shopping Center Emmen mit Blick auf das Winterkonzert im
nächsten Jahr.

Jahresbericht 1977:

Die Berichte von 1977 beginnen mit der Wahl des neuen Zunftmeisters Martin
Estermann. Die Winterkonzerte im Januar verliefen trotz großer Schneefälle
erfolgreich. Im April begleitete der Verein die Kommunionskinder, und im Mai
trat er beim Musiktag in Hergiswil auf. Das Waldfest im Juli und die
musikalische Umrahmung des Nationalfeiertags im August waren Höhepunkte des
Sommers. Im September wurde dem Pfarrer zum 40-jährigen Priesterjubiläum ein
Ständchen dargeboten. Das Jahr endete mit einem Frühschoppenkonzert im Shopping
Center Emmen.

Jahresbericht 1978:

Das Jahr 1978 begann mit der Wahl des neuen Zunftmeisters Josef Zwinggi-Amstutz.
Die Jahreskonzerte im Januar verliefen erfolgreich, gefolgt von einem
traditionellen Nachtessen im Löwen. Der Musikausflug führte die Mitglieder im
September nach Flumserberge Tannenboden. Das Jahr endete mit einem Konzert im
Shopping Center Emmen, bei dem die Stücke für das Winterkonzert im nächsten Jahr
präsentiert wurden. Im November nahm der Verein Abschied von seinem
Musikkameraden Marcel Stierli.