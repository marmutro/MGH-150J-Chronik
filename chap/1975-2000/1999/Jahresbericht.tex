\subsection*{1999}

\begin{history}


    \begin{itemize}

        \item 2. Jan.\\
              Am Zunftbot wird unser Präsident Franz Dörig zum 62. Zunftmeister der
              Götschizunft ernannt.

        \item 10., 14. und 16. Jan.\\
              Jahreskonzerte zum 125 Jahr Jubiläum.

        \item 23. Jan.\\
              Konzert im Shopping Center Emmen.

        \item 12. Febr.\\
              Am Dorfball rufen wir in einem Sketch verschiedene Anektoten von Franz
              in Erinnerung.

        \item 14. und 16. Febr.\\
              Mit dem Wagen mit dem Motto \enquote{Hobbies von Franz 7.} nehmen wir an
              den Umzügen in Beromünster und Hochdorf teil.

        \item 27. März\\
              Messebegleitung an Palmsamstags-Gottesdienst.

        \item 11. April\\
              Mit einem Prozessionsmarsch und einem anschliessenden Ständchen umrahmen
              wir den Weisssonntag. Spezielles Vorkommnis: Anstatt mit unserer eigenen
              Fahne marschieren wir mit der Fahne der Feldschützen in die Kirche.

        \item 24. April\\
              Wir begleiten die Einweihung des neu erstellten Feuerwehrgebäudes.

        \item 30. Mai\\
              Am 1. Sommertag wird in allen Gemeinden der Schweiz die neue Institution
              \enquote{Jugend und Musik} gefeiert. Alle musikalischen Vereine tragen
              zum guten Gelinge dieses Anlasses bei.

        \item 3. Juni\\
              Eine Kleinformation begleitet den Feldgottesdienst in der Arena.
              Anschliessend führen wir die Prozession durchs Dorf an.

        \item 4., 22., 25. und 29. Juni\\
              In den Quartieren Birkeweg, Schlüssel, Grossacher, Malorain und Waldmatt
              tragen wir unser Sommerprogramm der anwesenden Quartierbevölkerung vor.

        \item 11. Juni\\
              Wir geben im Pavillon in Luzern ein kurzes Konzert.

        \item 19. Juni\\
              Am Vormittag geben wir nach der Firmung ein Ständchen. Am Abend halten
              wir vor dem Löwen unter der Linde ein Sommerkonzert. Anlässlich unseres
              125. Geburtstags haben wir diese einmalige Konzert im Freien
              organisiert.

        \item 28. und 29. Aug.\\
              Vereinsausflug Südschwarzwald. Am Samstagmittag fahren wir mit dem Car
              los Richtung Basel, Badische Weinstrasse nach Wettelbrunn. Dort
              versuchen wir die Produkte des Weingutes von Andreas Neymeyer. Bei der
              Weiterfahrt nach Belchen-Multen machen wir einen Zwischenhalt im
              Städtchen Staufen. Im Jägerstüble erleben wir bei Speis, Trank, Musik
              und Gesang einen illustren Abend. Früher oder später fanden alle die
              Nachtruhe im Haus Bergfried. Am Sonntag fahren und wandern wir zum
              Aussichtspunkt Belchen bevor wir in St. Blasien die wunderschöne
              Klosterkirche besichtigen. Über Häusern, Waldhut, Brugg fahren wir
              zurück nach Hildisrieden.6

        \item 24., 26. Sept. und 3. Okt.\\
              Bazar Inpuls. Wir umrahmen die Bazareröffnung musikalisch. An den beiden
              Festsonntagen organisieren wir das Bankettessen, sowie den Brunch.

        \item 18. Okt.\\
              Erste Probe in unserem neuen Probelokal auf der Bühne im Inpuls. Der
              erste Eindruck ist sehr gut.

        \item 29. und 30. Okt.\\
              Lotto im Löwen. Der Ertrag von letztem Jahr wird übertroffen.

        \item 7. Dez.\\
              Da eine Mehrheit nicht mehr hinter ihre arbeit steht, lösen wir die
              Zusammenarbeit mit Dirigentin Beatrice Renkewitz per sofort auf.

        \item 16. Dez.\\
              Interimistisch bis Sommer 2000 übernimmt Kobi Banz die Direktion.

    \end{itemize}

\end{history}
