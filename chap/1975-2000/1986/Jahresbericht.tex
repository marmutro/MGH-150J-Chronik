\subsection{1986}

\begin{history}


    \begin{itemize}

        \item 2. Jan.\\
              Wir besammeln uns im Schulhaus, um den abtretenden Zunftmeister mit
              Marschmusik bis zum Löwen zu begleiten. Beim Zunftbot wird Toni
              Schumacher als neuer Zunftmeister gewählt. Er zahlt an diesem Abend den
              Ehrenmitgliederbeitrag, was mir mit einem Extramarsch bestens verdanken.

        \item 4. Jan.\\
              Weil an unserem Jahreskonzert wieder ein Theater gespielt wird, machen
              wir eine Kinderaufführung. 100 Kinder besuchen uns am Nachmittag.

        \item 8. und 11. Jan.\\
              An beiden Jahreskonzertabenden können wir einen vollen Saal begrüssen.

        \item 25. Jan.\\
              Konzert im Shopping Center Emmen. Wegen grossem Schneefall blieb die
              Pauke im Tiefschnee stecken, sodass Hans' fehlende Paukenschläge noch
              den ganzen Vormittag für Diskussion sorgen.

        \item 31. Jan.\\
              Beim Schulbesuch des Zunftmeisters erfreut die MGH die Schuljugend und
              die Zünftigen mit einem Ständchen auf dem Schulhausplatz.

        \item 5. April\\
              Wir begleiten die Kommunionkinder mit einem Parademarsch zur Kirche.
              Nach der Messe geben wir ein Ständchen.

        \item 11. Mai\\
              Muttertagständchen auf dem Löwenplatz.

        \item 23. - 25. Mai \textbf{Ausflug nach Lauterbach/Deutschland}\\
              Erstmals in der Geschichte der Musikgesellschaft Hildisrieden wurde auf
              Vermittlung unseres Direktors ein 3-tägiger Ausflug in unser Nachbarland
              Deutschland unternommen. Ziel der Reise war das etwa 10'000 Einwohner
              zählende Städtchen Lauterbach, welches rund 25 Km von der Kreisstadt
              Fulda und ca. 45 Km von der \enquote{Zonengrenze} zur DDR entfernt
              liegt.

              Bei herrlichem Sommerwetter bestieg die muntere Reiseschar den modernen,
              doppelstöckigen Reisecar der Auto AG Rothenburg und nahm den langen Weg
              über Zürich, Schaffhausen, Donaueschingen, Stuttgart, Würzburg nach
              Lauterbach unter die Räder. Mit 2 Rastaufenthalten und der üblichen
              Verkehrsstockung im Raume Stuttgart sind wir frohgelaunt an unserem
              Reiseziel angelangt. Nach einem kühlendem Trunk im Pfarreizentrum wurde
              die vom örtlichen Mittelsmann vorbereitete und von uns mit Spannung
              erwartete Einquartierung vorgenommen, waren doch fast alle Teilnehmer
              bei Gastfamilien untergebracht.

              Nach einem ersten gegenseitigen Kennenlernen ging's mit Privatwagen zu
              unseren Gastfamilien, mit welchen auch der erste Abend ohne offiziellen
              Teil verbracht wurde. Doch es: kam so wie es kommen musste, man traf
              sich irgendwann, irgendwo wieder, ob beim Nachbar im Gartenhaus oder in
              einer der vielen Gaststätten, auf jedenfalls gab es bereits tolle Feste
              und eine erste kurze Nacht.

              Der Samstag war dann eher etwas hektischer, denn bereits um 09.30 Uhr
              war Empfang durch den Bürgermeister, welcher im Zusammenhang mit zwei
              Patengemeinden von Lauterbach stattfand. Ob der vielen Reden wurde um
              10.30 Uhr erst um 11.15 zur Eröffnung des 216. Prämienmarktes
              musikalisch auf den Marktplatz einmarschieren. Zu dieser Zeit wollte der
              Wettergott nicht so recht mitspielen und liess es während dem Festakt
              immer leicht regnen. Im Anschluss an diese offizielle Eröffnung wurde
              zum Eintrag ins \enquote{Goldene Buch der Stadt Lauterbach} und Apero
              ins Rathaus eingeladen.

              Doch die Zeit drängte schon bald wieder und es hiess einsteigen in
              unseren Reisebus. Der örtliche Organisator Herr Hahner, Vice-Polizeichef
              von Lauterbach, gab uns während der Fahrt eine überaus interessante
              Lektion in deutscher Geographie und Geschichte. Das Ziel unserer
              Informationstour war die Bundes-Grenzschutz-Station Hünfeld, wo uns ein
              Dokumentarfilm über die Entstehung und bis zur Unmenschlichkeit:
              ausgebaute Zonengrenze der DDR vorgeführt wurde. Beim anschliessendem
              Mittagessen in der Kantine des BGS konnten die ersten filmischen
              Eindrücke verarbeitet werden. Aber so richtig eindrücklich, wie es in
              der Wirklichkeit auch aussieht, war erst die nach dem Mittagessen unter
              der Führung von zwei Beamten des: Bundes-Grenzschutzes vorgenommene
              Grenzbesichtigung bei Rasdorf. Die beiden Beamten mussten eine Flut von
              Fragen beantworten und es war erstaunlich, was für neue Erkenntnis wir
              mit auf den Heimweg nehmen konnten.

              Die anschliessende Fahrt führte uns über Fulda wieder nach Lauterbach
              zurück, doch die Besichtigung des Doms von Fulda kam aus zeitlichen
              Gründen nicht: mehr infrage, da wir die am Morgen einkassierte
              Verspätung nicht mehr aufholen konnten.

              Nach dem Nachtessen bei unseren Gastfamilien warteten bereits die
              nächsten offiziellen Auftritte auf uns, so wirkte eine Bläsergruppe bei
              der Gestaltung des Abendgottesdienstes mit und die Ronspatzen erfreuten
              die Kirchgänger mit einem kleinen Ständchen. Ab 21.00 Uhr begann unser
              Unterhaltungskonzert in der Festhütte, welche leider unmittelbar am
              Rummelplatz stand und wir somit fast sämtliche Kräfte freimachen
              mussten, um mit unseren musikalischen Darbietungen über die Bühne zu
              kommen. Für unsere Solisten standen Mikrofone zur Verfügung, so mussten
              sie wenigstens nicht fast die Lunge aus dem Leib blasen. Trotzdem es war
              ein Erfolg und ein Erlebnis für uns.

              Im Anschluss an dieses Konzert wurde kräftig gefeiert und gefestet, so
              kam es auch, dass die Einen und die Andern die Amseln pfeifen hörten als
              sie zu Bett gingen. Allzu viel Schlaf gab es wohl für die meisten
              Teilnehmer nicht, denn das Sonntagsprogramm rief.

              Während etwa $\sfrac{2}{3}$ der Reiseteilnehmer um 9 Uhr einen
              Gutsbetrieb besichtigten, verbrachte der Rest die Zeit mit Besichtigung
              des Städtchens Lauterbach oder schlief doch ganz einfach ein bisschen
              aus.

              Doch um 11.00 Uhr hiess es bereitstehen für ein Platzkonzert zu Ehren
              des Generalvikars von Mains, der in Lauterbach die Hl. Firmung erteilte.

              Nach dem fliegenden Tenüwechsel (von der Uniform zur Reisekleidung) in
              der Eingangshalle des Pfarreizentrums wurde uns zum Abschied im
              Pfarreisaal ein schmackhaftes Mittagessen serviert, dessen Spender uns
              bis heute unbekannt blieb, hingegen erfuhren wir, dass das süffige Bier
              vom Braumeister aus Lauterbach gespendet wurde. Herzlichen Dank.

              Nach gegenseitigen Abschiedsworten durch unseren Präsidenten und dem
              örtlichen Organisator Herr Hahner, welchem wir aus Dank eine Kuhglocke
              mit Widmung überreichten, mussten wir auch von unseren Gastgebern
              Abschied nehmen. Wie gut wir es getroffen hatten, konnte man erst da so
              richtig sehen -- es gab alles -- Umarmungen, Freundschaftsküsschen,
              langes Händeschütteln und zuletzt ein Winken mit dem Taschentuch und
              dann Adieu mein schönes Lauterbach.

              Und so wurde die Heimfahrt über die Autobahn ab Alsfeld, Frankfurt,
              Freiburg, Basel angetreten. Eine ungehinderte Fahrt ohne irgendwelche
              Stockungen vom Wochenendausflug brachte uns rasch der heimatlichen
              Scholle zu. Trotz Müdigkeit in den Knochen wurde im Reisebus gescherzt
              und gelacht und natürlich auch ein wenig geprahlt über die Erlebnisse
              der vergangenen Stunden. Der untere Stock des Cars gehörte den Jassern
              und zum Teil auch den Nachschläfern.

              Nach einer Fahrt von 6 $\sfrac{1}{4}$ Stunden mit einem
              dazwischenliegenden Halt von einer halben Stunde sind wir früher als
              geplant wieder zu Hause eingetroffen. Es war für alle Teilnehmer eine
              schöne und eindrückliche Reise und bestimmt wird dieser Ausflug den
              Meisten ein unvergessliches: Erlebnis bleiben.

        \item 8. Juni\ Mit einem Unterhaltungskonzert erfreuen wir ein
              internationales Publikum am Gestade des Vierwaldstättersees.

        \item 8. Juli\\
              Die 1. Stimmen laden zum Registerhöck ein im Wagenschopf in der
              Holzmatt.

        \item 13. Juli\\
              Überraschend wurde das Wetter besser und wir können das Waldfest doch
              durchführen. Der Besucheransturm hielt sich jedoch in Grenzen, was
              einige Musikanten bewog, selber für einen steigenden Umsatz zu sorgen.

        \item 1. Aug.\\
              Teilnahme an der Feier auf dem Schulhausplatz.

        \item 7. Sept.\\
              Wir umrahmen den Empfang der Goldkranz-Schützen musikalisch. Auch die
              Turnerinnen vom SVKT kommen mit einem Glanzresultat vom Turnfest in
              Brugg heim.

        \item 14. Sept.\\
              Der neu erstellte Fussballplatz Bogenhüsli wird eingeweiht.

        \item 7. Okt.\\
              Wir entscheiden, anstatt Posaunen der Marke Bach welche von Yamaha zu
              kaufen, da deren Qualität ebenbürtig sei. Wir kaufen 2 Posaunen in
              normaler und 2 mit Quartventilen.



    \end{itemize}

\end{history}
