\subsection{Jahresbericht 2000}

\begin{history}


    \begin{itemize}

        \item[]31. Dez.- 1. Jan.\\
        Eine Bläsergruppe nimmt an der grossen Einweihungsfeier des Zentrum
        \enquote{Inpuls} in der Silvesternacht 1999/2000 teil.

        \item[]2. Jan.\\
        Am Zunftbot wird Ruedi Kneubühler zum Zunftmeister gewählt.

        \item[]15. und 16. Jan.\\
        Zum ersten Mal dürfen wir in der neuen Mehrzweckhalle Inpuls
        konzertieren. Dieser Neubau erweist sich als praktisch, geräumig und
        akustisch als sehr gut. In der kurzen Vorbereitungsphase hat uns der
        neue Dirigent Kobi Banz auf ein beachtliches Niveau gebracht.

        \item[]5. März\\
        Bei strahlendem Sonnenschein und einem grossen Publikumsaufmarsch nehmen
        wir mit dem Motto \enquote{Marla Glenn} oder \enquote{Ohne Münz kei
            Brünz} teil.

        \item[]4. April\\
        Wir wählen Kobi Banz einstimmig zu unserem neuen Dirigenten.

        \item[]30. April\\
        Bei prächtigem Frühlingswetter begleiten wir die Erstkommunikanten in
        einer Prozession zur Kirche.

        \item[]11. Mai\\
        Als Vorbereitung auf das Luzerner Kantonale Musikfest halten wir im
        Zentrum Inpuls ein Vorbereitungskonzert. Es nehmen die MG Flühli und die
        Feldmusik Rothenburg teil.

        \item[]13. Mai\\
        Muttertagsständli vor der Kirche.

        \item[]19. Mai\\
        Gemeinschaftskonzert in Neuenkirch. Mit der Brass Band Harmonie
        Neuenkirch, der MG Romoos und wir testen ein 2. mal die Form für das
        Musikfest.

        \item[]27. Mai\\
        Kant. Musikfest Kriens.

        Bei strömendem Regen fahren wir am Morgen mit der Auto AG Rothenburg
        nach Kriens. Nach dem Einspielen tragen wir in der Kirche das
        Aufgabestück \enquote{Focus} von Leon Vargas und in der Mehrzweckhalle
        das Selbstwahlstück \enquote{North-West Passage} von Roy Newsome vor.

        In der Rangverkündigung wurden wir mit einem schlechten Ergebnis
        konfrontiert. Mit 155 Punkten belegen wir den 13. Rang von 14 Vereinen
        in derselben Klasse. In der Marschmusik konnten wir einen
        Mittelfeldplatz erreichen.

        \item[]13. Juni\\
        Ständchen für die neue gewählten Gemeinderäte beim Partyraum Stross.

        \item[]22. Juni\\
        Fronleichnam.

        \item[]24. Juni\\
        Firmung

        \item[]27. Juni\\
        An einem schönen Sommerabend unterhalten wir die Gäste von Weggis mit
        einem Konzert im Freien.

        \item[]1. Aug.\\
        Wir umrahmen die Bundesfeier in der Schür musikalisch.

        \item[]9. Sept.\\
        Am Plausch-Wettkampf \enquote{Spiel ohne Grenzen} nimmt eine Delegation
        der MGH an der Einweihung der Aussensportanlage teil.

        \item[]27. und 28. Okt.\\
        Bei durchschnittlicher Beteiligung führen wir im Löwen unser Lotto
        durch. Das Lotto bringt uns einen beträchtlichen Betrag in die
        Vereinskasse.

    \end{itemize}

\end{history}
