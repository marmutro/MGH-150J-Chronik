\subsection{Jahresbericht 1991}

\begin{history}


    \begin{itemize}

        \item[]2. Jan.\\
        Unter Mitwirkung der MGH findet der Zunftbot statt. Neuer Zunftmeister
        wird Walter Gemperli.

        \item[]6., 10. und 12. Jan.\\
        Wir halten unser Jahreskonzert.

        \item[]19. Jan.\\
        Wir geben im Shopping Center Emmen unser Konzert.

        \item[]23. März\\
        Nach dem Abendgottesdienst halten wir das Kirchenkonzert.

        \item[]12. Mai\\
        Zum Muttertagsständchen laden wir nach dem Sonntagsgottesdienst ein.

        \item[]2. Juni\\
        Am Vormittag spielen wir an der Seepromenade in Luzern ein
        unterhaltsames Morgenkonzert.

        Anschliessend begeben wir uns mit unseren Familien in den Traslingerwald
        wo wir zum ersten mal ein Familienpicknick organisiert haben.

        \item[]6. und 20. Juni\\
        An diesen beiden Abenden sind wir in Engelber zu Gast. Im Casino
        konzertieren wir für die Feriengäste.

        \item[]15., 16. und 22., 23. Juni\\
        In Lugano findet das eidgenössische Musikfest ohne uns statt.

        \item[]9. Juli\\
        Wir besuchen den Schlüsselrain und spielen an drei Plätzen. Diese
        Ständchen wird von den Quartierbewohnern sehr geschätzt und endete mit
        einem nächtlichen Quartierfest auf offener Strasse.

        \item[]14. Juli\\
        Das Waldfest halten wir bei schönstem Wetter. Die grosse Hitze lockt
        viele Besucher in den Meierholzwald. Am Abend ziehen jedoch dunkle
        Wolken auf und die offizielle Wirtschaft musste vorzeitig geschlossen
        werden.

        \item[]1. Aug.\\
        Die Musikgesellschaft nimmt an der Jubiläumsbundesfeier 700 Jahre
        Eidgenossenschaft auf dem Schulhausplatz teil.

        \item[]18. Aug.\\
        Auf der Rigi findet ein Begegnungstag statt. Die Musikgesellschaft
        Hildisrieden spielt als Gast am Vormittag auf Rigi-Kulm und am
        Nachmittag auf Rigi-Kloster ein Ständchen.

        \item[]29. Sept.\\
        Die Schützen kehren goldbekränzt vom Luzerner Kantonalen Schützenfest
        heim. Wir beteiligen uns am Empfang.

    \end{itemize}

\end{history}
