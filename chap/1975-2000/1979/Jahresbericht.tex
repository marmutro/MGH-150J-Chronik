\subsection{1979}

\begin{history}


    \begin{itemize}

        \item 2. Jan.\\
              Bei bissiger Kälte und Schneesturm besammeln wir uns für das Zunftbot im
              Kreuz bei einem Glas Wein. Mit Marschmusik begleiten wir den abtretenden
              Zunftmeister zum Löwen. Oskar Estermann wird mit tosendem Applaus zum
              Zunftmeister gewählt.

        \item 6. und 13. Jan.\\
              Winterkonzert im Löwen.

        \item 19. Jan.\\
              Wir verschönern den Abend der Neuzuzüger musikalisch.

        \item 25. Feb.\\
              Bei kaltem Wetter findet der Fastnachtsumzug statt. Unser Verein
              verkörpert am Umzug die Bundessicherheitspolizei (Busipro). Unser Tenü
              besteht aus halbhohen Stiefeln, blauem Kombiüberkleid, einem roten Hut
              und an der Seite einen grossen Knüppel. Im Marschtempo marschieren wir
              durchs Dorf an über 5000 Zuschauern vorbei.

        \item 12. April\\
              Im Alter von erst 21 Jahren müssen wir heute von unserem Musikkamerad
              Erwin Luterbach abschied nehmen. Er hatte einen Arbeitsunfall. Auf dem
              Friedhof geben wir ihm die letzte Ehre mit dem Ordonanzmarsch und dem
              Choral \enquote{der gute Kamerad}.

        \item 22. April\\
              Bei herrlich schönem Frühlingswetter dürfen wir am weissen Sonntag die
              Kommunionskinder durchs Dorf in die Kirche begleiten. Nach der Messe
              geben wir ein Platzkonzert, dann marschieren wir zum Kreuz, wo wir zu
              einem Glas Wein eingeladen sind.

        \item 24. April\\
              Ständchen für den neuen Gemeindepräsidenten Oskar Estermann.

        \item 28. April\\
              Firmung von 52 Kindern durch Bischof Anton Hänggi. Nach der Feier
              spielen wir zur musikalischen Umrahmung.

        \item 13. Mai\\
              Zu Ehren der Mütter geben wir nach der Maiandacht auf dem Dorfplatz ein
              Ständchen.

        \item 23. Juni\\
              Ständchen in der Schlacht bei warmem Sommerwetter.

        \item 1. Juli\\
              50 Jahr Feier der Harmonie Neuenkirch. Am Festzug spielen wir den Marsch
              \enquote{Schwyzer Soldaten} und im Festzelt ein Solo für 3 Trompeten.

        \item 3. Juli\\
              Wir gratulieren unserem Präsidenten Fritz Disler zur Wiederwahl als
              Gemeinderat und Sozialvorsteher.

        \item 10. Juli\\
              Mit einem Ständchen gratulieren wir dem neugewählten Gemeindepräsidenten
              Franz Widmer.

        \item 22. Juli\\
              Obwohl das Wetter eher unstabil ist führen wir das Waldfest durch.
              Trotzdem machen wir einen schönen Reinerlös.

        \item 1. Aug.\\
              Wir machen mit bei der Feier auf dem Schulhausplatz.

        \item 1. Sept.\\
              Wir empfangen unsere Schützen, darunter auch 4 Musikanten, vom eidg.
              Schützenfest in Luzern. In einem kleinen Festzug geht es durchs Dorf auf
              den Schulhausplatz.

        \item 15. Sept.\\
              Heute halten wir das erste Mal einen Musikantenhock mit den Frauen, im
              extra für solche Anlässe ausgebaute Schopf bei Walter Rüttimann. Zur
              Unterhaltung spielt Josef Amrein und sein Sohn aus Pfeffikon. Auch
              Pirmin Troxler, der als Feldwebel und Trompeter auftritt weiss viel zur
              Unterhaltung beizutragen.

        \item 1. Dez.\\
              Konzert im Shopping Center Emmen

    \end{itemize}

\end{history}
