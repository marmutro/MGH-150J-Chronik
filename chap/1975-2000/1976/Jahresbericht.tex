\begin{history}

    % \subsection{Jahresbericht}
    % 1976

    \begin{itemize}

        \item[]5. Jan.\\
        Wir begleiten den abtretenden Zunftmeister Werner Troxler mit
        Marschmusik zum Rest. Kreuz. Als Zunftmeister 1976 wird Franz Estermann,
        Handlung, gewählt. Wir danken beiden auf musikalische Weise.

        \item[]10. und 17. Jan.\\
        Unsere Winterkonzerte locken rekordverdächtige 722 Besucher in den
        Löwensaal.

        \item[]15. Jan.\\
        Im Emmmer Pfarreisaal spielen wir vor 60 Personen.

        \item[]17. Feb.\\
        Zur Probe bringt jeder Musikant ein Leintuch mit. Es wird von fleissigen
        Musikantenfrauen zu einer Pellerine zurecht geschneidert und dann
        eingefärbt. Sie werden für den Fastnachtumzug gebraucht.

        \item[]29. Feb.\\
        Bei herrlich warmem Frühlingswetter findet der Fastnachtumzug statt.
        Unser Sujet bedeutet die Verschandelung des Pilatus durch die Montierung
        von gelben Kugeln an den Drähten der Pilatusbahn, die zu Sicherung gegen
        Flugzeuge dienen. Marsch und Walzer spielend marschieren wir vor 5000
        Zuschauern. Es nehmen 40 Gruppen teil.

        \item[]25. April\\
        Wegen schlechter Witterung muss die Weiss-Sonntag Prozession abgesagt
        werden.

        \item[]9. Mai\\
        Bei herrlichem Wetter spielen wir am Abend nach der Maiandacht das
        Muttertagständchen.

        \item[]30. Mai\\
        Neuuniformierung und 75 Jahrfeier der Feldmusik Rain. Nach dem
        Ehrentrunk in der Rüti gehts mit Marschmusik nach Sandblatten in die
        Festhalle, wo wir 2 Stücke spielen.

        \item[]12. Juli\\
        Das Dorforiginal und Hofcoiffeur von Hildisrieden Fritz Gradwohl feiert
        den 80. Geburtstag. Wir bringen im ein Ständchen bei der Feier im Kreuz.

        \item[]18. Juli\\
        Bei herrlichem Wetter werden am Waldfest 48 kg Braten verkauft.
        Festwirtschaft und Kaffeebude laufen auf Hochtouren. Am Nachmittag
        müssen sogar Tische und Bänke im Gormund geholt werden.

        \item[]6. Aug.\\
        Am Kurplatzkonzert in Luzern spielen wir den Marsch de Medici, eine
        Konzertpolka und weitere Märsche. Die Stücke werden mit grossem Applaus
        verdankt. Im Rest. Emmenbaum zahlt der Kassier noch ein Bier.

        \item[]21. und 22. Aug.\\
        Bei strahlendem Sommerwetter fahren wir mit Bus und Bahn nach Adelboden.
        Unsere Ronspatzen spielen im Dorfzentrum einige gefällige Weisen. Nach
        diversen Erfrischungen wandern wir nach Gilbach, wo wir im alpinen
        Ferienlager übernachten. Das Nachtessen gibts im Berggastaush \enquote{Des
            Alpes}. Danach spielen die Ronspatzen zum Tanz. Am Sonntagmorgen gehts
        weiter nach Interlaken, wo uns ein Schiff nach Brienz bringt.


    \end{itemize}

\end{history}
