\subsection{1993}

\begin{history}


    \begin{itemize}

        \item 2. Jan.\\
              Die MG nimmt am Zunftbot der Götschizunft zeil. Als Zunftmeister 1993
              wird Hugo Fleischli vom Rest. Kreuz gewählt.

        \item 10., 14. und 16. Jan.\\
              An unserem Jahreskonzert dürfen wir an allen drei Aufführungen viele
              Besucher begrüssen.

        \item 5., 7., 8. und 9. Mai\\
              Neuuniformierung und Teilinstrumentierung. Am Mittwoch beginnt unser
              Fest mit einem Galakonzert der Brassband Bürgermusik Luzern. Bei recht
              frischem Wetter können wir eine beachtliche Schar Blasmusikfreunde im
              Festzelt begrüssen. Dieses Konzert unter der Leitung von Ludwig Wicki
              wird zum vollen Erfolg. Ebenfalls das Tessinerstübli im alten
              Genossenschaftsgebäude, das Café Symphonie im Suppigerhaus, die Bar und
              die Bierschwemme fanden beim Publikum guten Anklang.

              Am Freitagabend laden wir unsere Gönner zu einem Gönnerabend ein. Nach
              dem Apéro wurde im Festzelt ein Nachtessen serviert. Höhepunkt von
              verschiedenen Darbietungen war die Vorstellung der neuen Uniform.

              Am Ende dieses Abends war allen Eingeladenen klar, \enquote{Gönner zu sein
                  bei der MG Hildisrieden lohnt sich}.

              Ein Grossaufmarsch der jüngeren Generation durften wir zum Tanzabend am
              Samstagabend zur Kenntnis nehmen. Auch unsere Beizli waren praktisch den
              ganzen Abend bis auf den letzten Platz belegt.

              Am Festsonntag zeigten wir uns zum ersten Mal in der neuen Uniform.

              Im feierlichen Festgottesdienst weihte Pfarrer Paolo Brenni unsere neue Uniform.

              Mit Marschmusik ging es anschliessen zum Festzelt auf dem
              Schulhausplatz. Zum dort stattfindenden Zmorgebrunch hat sich fast die
              ganze Gemeinde eingefunden, doch die Wirtecrew meisterte diesen
              Grossaufmarsch bestens.

              Am Festumzug am Nachmittag nimmt die Harmonie Rain, viele Dorfvereine
              und Fahnendelegationen von Nachbarvereinen teil. Bei strahlendem
              Sonnenschein zog ein farbenprächtiger Festzug durchs Dorf.

              Der anschliessende Festakt wurde geprägt von kurzen Ansprachen
              von OK-Mitgliedern und Gästen. Anschliessend fand die Uraufführung des
              Marsches \enquote{Free of Fog} des einheimischen Komponisten Otto
              Troxler statt. Als Überraschung wurde die MG Hildisrieden mit einem
              Saxophonregister verstärkt. Nach dem Festakt spielen im Festzelt die
              Lake City Stompers alten Jazz. In der Aula findet die Versteigerung der
              Collagen statt. Mithilfe einer Presse hat der Künstler Stefan Bucher aus
              unseren alten Instrumenten Wandbilder hergestellt. Diese werden den
              Meistbietenden verkauft. In den Beizli findet irgendwann am späten Abend
              das Ende einer gelungen Neuuniformierung statt.

        \item 23. Juni\\
              Zum Chrampfer-Abend der Neuuniformierung ladet die MG in den Wagenschopf
              in der Waldegg ein.

        \item 3. und 4. Juli\\
              Erstmals beginnen wir unser Waldfest bereits am Samtsagabend. Zur
              Unterhaltung haben wir die Rigispatzen aus Küssnacht engagiert. Sie
              begeistern das Publikum mit volkstümlicher Blasmusik.

        \item 6. Juli\\
              Zu einem Ständchen im Schlüsselrain haben uns unsere Veteranen Franz und
              Toni eingeladen. Der Schopf von Toni erweist sich als hervorragend
              geeignet zur Beherbergung unseres Vereins.

        \item 25. Juli\\
              Am Seenachtsfest in Küssnacht spielen wir ein Unterhaltungsprogramm im
              Anschluss an den Frühstücksbrunch.

        \item 1. Aug.\\
              Wir nehmen am der traditionellen Bundesfeier auf dem Schulhausplatz
              teil.

        \item 28. und 29. Aug.\\
              Die MG Eich weiht an diesem Wochenende ihre neue Uniform. Wir spielen am
              Samstagabend im Gemeindesaal.



    \end{itemize}

\end{history}
