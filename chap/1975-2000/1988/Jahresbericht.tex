\subsection*{1988}

\begin{history}


    \begin{itemize}

        \item 2. Jan.\\
              Zunftbot im Rest. Kreuz, zum Zunftmeister gewählt wird Dorfkäser Walter
              Kaufmann.

        \item 9. und 10. Jan.\\
              Jahreskonzerte in der Turnhalle Matte, weil der Löwen wegen einer
              Explosion geschlossen ist.

        \item 23. Jan.\\
              Konzert im Shopping Center Emmen.

        \item 14. März\\
              Am Fasnachtsumzug beteiligt sich die Musikgesellschaft mit einem Wagen.
              Das Sujet ist das Gipfeltreffen von Reagan und Gorbatschow.

        \item 10. April\\
              Wegen schlechtem Wetter müssen die Erstkommunikanten auf unser Ständchen
              verzichten.

        \item 30. April und 1. Mai\\
              Bei besten Bedingungen führen wir diesen Jahr wieder die Chilbi durch.
              Der Besucheransturm dürfte Rekorddimensionen erreicht haben.

        \item 13. Mai\\
              Die 2. Stimmen laden zum Registerhöch nach Omelingen ein.

        \item 3. Juli\\
              Wir sind zur Fahnenweihe der Musikgesellschaft Eich eingeladen. Nach dem
              Festzug spielen wir im Festzelt 2 Stücke.

        \item 10. Juli\\
              Bei schönem Wetter aber wenig Besuchern führen wir das Waldfest durch.

              Eine Fahnendelegation nimmmt am Festzug der Fahnenweihe der
              Musikgesellschaft Harmonie Rain teil.

        \item 16. Juli\\
              Da das Kurplatzkonzert wegen Regens ins Wasser fiel spielen wir das
              Konzertprogramm nach dem Abendgottesdienst auf dem Schulhausplatz.

        \item 1. Aug.\\
              1. August-Feier auf dem Schulhausplatz.

        \item 27. und 28. Aug.\\
              Musikreise ins Montafon. Mit einem Car der Firma Estermann Beromünster
              und einem zugemieteten Kleinbus unter der Regie von Toni Kleemann ging
              es um 12 Uhr via Hirzel-Walensee bis nach Vaduz wo wir einen Kaffeehalt
              einschalten. Über die Grenze geht es dann nach Feldkirch-Bludenz-Schruns
              bis nach Partennen am Silvrettapass. In drei verschiedenen
              Frühstückspensionen beziehen wir unser Unterkunft. Bei einem Vorabend
              bummel lenrt man das Dorf Partennen und die ersten kulinarischen
              Getränke kennen. Zum Nachtessen trifft man sich dann im
              Unterhaltungskeller des Hotels Hubertusklause. Es gibt ein Nachtessen,
              das qualitativ und quantitativ alle Grenzen sprengt. Der Wirt sorgte
              dann selber für musikalische Unterhaltung. Unter dem Motto \enquote{Auf
                  die Bäume ihr Affen, der Wald wird geputzt} vergingen die Stunden im
              Fluge. Der Sonttagmorgen kam in schnellen Schritten auf uns zu. Nur dank
              des schönen Wetters konnte sich die ganze Gesellschaft entschliessen am
              frühen Vormittag schon wieder einen Ausflug mit dem Car auf die
              Bielerhöhe oder einen Fussmarsch auf die Alp Tafmunt zu unternehmen. Um
              halb drei mussten wir dann schon wieder Abschied nehmen von der
              österreichischen Gastfreundschaft. Auf der Heimfahrt ist ein
              Überaschungszobig eingeschaltet. Wir fahre über Wildhaus durchs
              Toggenburg und über den Ricken. In St. Gallenkappel hält der Car bei der
              Käserei von unserem ehemaligen Musikkamerad Markus Dörig. Er und seine
              Frau haben für uns ein Käsebuffet vorbereitet. Erst gegen acht Uhr
              verabschieden wir uns von der tollen Stimmung auf dem Käsereivorplatz
              und fahren über den Seedamm von Rapperswil nach Hause.

        \item Im Okt.\\
              Um einen Nachfolger für unseren Direktor zu finden inserieren wir in der
              Musikzeitung. Auf diese Inserat haben sich 4 Bewerber gemeldet. Es sind
              dies Herr Huber aus Kriens, Herr Birrer aus Horw, Herr Sennhauser aus
              Adligenswil und Herr Tuor aus Reussbüel. Alle 4 Bewerber wurden zu einer
              Stellprobe eingeladen. In einem Gespräch nach der Probe mit dem Vorstand
              und der Musikkommission lernte amn sich dann gegenseitig noch besser
              kennen.

        \item 6. Dez.\\
              Zur Wahl des neuen Dirigenten finden wir uns im Roten Löwen ein. Schon
              in den Diskussionen vor der Versammlung merkte man, dass Herr Marcel
              Sennhauser al Kronfavorit ins Rennen gehen wird. Nach kurzer Diskussion
              wirrd er von 35 Anwesenden einstimmig zu unserem neuen Direktor gewählt.

        \item 8. Dez.\\
              Die neu renovierte Kirche wird eingeweiht. Die Musikgesellschaft hilft
              den Festgottesdienst gestalten.


    \end{itemize}

\end{history}
