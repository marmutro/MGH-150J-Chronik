\subsection{Jahresbericht 1996}

\begin{history}


    \begin{itemize}

        \item[]2. Jan.\\
        Auf der Löwenbühne gratulieren wir mit einigen Märschen dem
        frischgebackenen Zunftmeister Claude Cornaz.

        \item[]7., 11. und 13. Jan.\\
        Jahreskonzert auf der Löwenbühne.

        \item[]20. Jan.\\
        Konzert im Shopping Center Emmen.

        \item[]14. April\\
        Prozession am Weissen Sonntag.

        \item[]4. und. 5. Mai\\
        Mit dem Motto \enquote{Chilbi Total} haben wir für die Chilbi 96 ein
        neues Konzept erarbeitet. Bei prächtigem Frühlingswetter ist der
        Publikumsaufmarsch sehr gross. Dank dieser Neuorganisation, dem
        Gesamteintritt, und dem grossen Engagement jedes Einzelnen wir diese
        Chilbi zum Grosserfolg.

        \item[]12. Mai\\
        Bei regnerischem Wetter halten wir unser Muttertagsständchen.

        \item[]19. Mai\\
        In Hitzkirch bereiten wir uns in einer intensiven Probe auf das eidg.
        Musikfest in Interlaken vor.

        \item[]24. Mai\\
        Kurplatzkonzert Luzern. Auch dieses Konzert wird als Vorbereitung für
        das bevorstehende Musikfest benützt.

        \item[]5. Juni\\
        Gemeinschaftskonzert in Römerswil. Mit den beiden Nachbarvereinen Brass
        Band Römerswil und Harmonie Rain halten wir dieses Vorbereitungskonzert.
        Die einzelnen Vorträge stossen auf grosses Interesse und werden mit
        grossem Applaus quittiert.

        \item[]6. Juni\\
        Gemeinschaftskonzert in Eich. Zusammen mit der Musikgesellschaft Eich
        und der Harmonie Sempach halten wir unser 2. Vorbereitungskonzert.

        \item[]15. und 16. Juni\\
        Eidgenössisches Musikfest Interlaken.

        Im Löwen spielen wir uns auf den musikalischen Wettstreit ein. Vor der
        Abfahrt geben wir unseren Firmlingen noch ein Ständchen. Anschliessen
        fahren wir via Lungern (Mittagessen) nach Interlaken. Das sommerliche
        Wetter verspricht ein Superfest. Angespannt tragen wir im vollgestopften
        Aareparksaal das Aufgabestück \enquote{Offside} von Christian Henting
        der Jury vor. Mit 138 erreichten Punkte sind wir nur mässig zufrieden.
        Nun verschieben wir in die Kirche von Unterseen. Gut vorbereitet spielen
        wir da unser Selbstwahlstück \enquote{A Saddleworth Festival Ouverture}
        von Goff Richards. Nach diesem Vortrag haben wir ein gutes Gefühl.
        Hoffentlich bestätigen dies auch die Experten. Und siehe da, mit 149
        Punkten im Selbstwahlstück, Total 287 sind wir ganz vorne in der
        Rangliste. Im 9. Rang von 41 klassierten Vereinen sind wir mehr als
        zufrieden. Nun wollen wir auch in der Marschmusik einen guten Eindruck
        hinterlassen. Diesmal hat uns das Gefühl getäuscht. Mit 107 Punkten und
        einem 12. Rang von 41 Vereinen werden unsere Erwartungen weit
        übertroffen. Nun werden diese erzielten Punkte bis in die Morgenstunden
        gefeiert. Gutgelaunt kehren wir am Sonntag via Brünig wieder nach
        Hildisrieden zurück. Mit einem Grossaufmarsch werden wir von der ganzen
        Bevölkerung empfangen. Den Abschluss feiern wir im Löwen mit einem
        Nachtessen.

        \item[]1. Aug.\\
        Bei herrlichem Sommerwetter nehmen wir an der Bundesfeier auf dem
        Schulhausplatz teil.

        \item[]15. Aug.\\
        Pfarrer Brenni feiert seinen 70. Geburtstag und sogleich seinen Abschied
        von Hildisrieden. Wir spielen den Marsch \enquote{Alte Kameraden}, den
        er selber dirigiert.

        \item[]8. Sept.\\
        Mit Marschmusik und einem Platzständchen erweisen wir der Feldmusik Rain
        an deren Neuuniformierung die Ehre.

        \item[]20. Okt.\\
        Pfarrereinsetzung Pfarrer Josef Hauser. Mit einem Marsch durchs Dorf
        begeben wir uns in die Aula. Dort unterhalten wir die Gäste beim Apéro.

    \end{itemize}

\end{history}
