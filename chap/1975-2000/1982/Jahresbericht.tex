\subsection{Jahresbericht 1982}

\begin{history}


    \begin{itemize}

        \item[]2. Jan.\\
        Nach einem Glas Wein im Pavillon beim Schulhaus gehts mit Marschmusik
        zum Löwen, wo am Zunftbot der neue Zunftmeister Anton Estermann
        inthronisiert wird. Wir spielen ihm 2 extra Märsche.

        \item[]9. und 16. Jan.\\
        Erstes Jahreskonzert unter der neuen Leitung von Thomas Zemp.

        \item[]23. Jan.\\
        Konzert im Shopping Center Emmen.

        \item[]20. März\\
        Das Wetter an der Hochzeit von Direktor Thomas Zemp ist so schlecht,
        sodass wir nicht beim Schloss Heidegg sondern im Don Bosco in
        Beromünster den Apperitiv mit einem Ständchen verschönern.

        \item[]27. März\\
        Zum Passionsonntag helfen wir auch dieses Jahr den
        Samstagabendgottesdienst musikalisch zu bereichern.

        \item[]18. April\\
        Bei strahlendem Frühlingswetter dürfen wir die Kommunionkinder in die
        Kirche begleiten.

        \item[]9. Mai\\
        Muttertagständchen nach der Maiandacht

        \item[]10. Juni\\
        Begleitung des Fronleichnamgottesdienstes mit Chorälen und
        anschliessender Prozession.

        \item[]18. Juni\\
        Quartierständchen im Schlüsselrain.

        \item[]29. Juni\\
        Quartierständchen im Sonnbühl.

        \item[]1. Aug.\\
        Nationalfeier auf dem Schulhausplatz.

        \item[]28. und 29. Aug.\\
        Bei strömendem Regen gehts mit Car und Kleinbus von Estermann
        Beromünster ins Wallis. Nach Kaffeehalt in Kandersteg verladen wir und
        fahren wir mit der Bahn durch den Lötschberg nach Salgesch. Wir besuchen
        den Weinkeller der Gebrüder Matthier. Dann geht weiter nach Leukerbad
        und durch dichten Nebel auf die Torrentalp. Oberhalb der Nebelgrenze
        beziehen wir unsere Unterkunft im Hotel Briand. Nach dem Nachtessen
        spielen die Ronspatzen, die letzten gehen erst gegen Morgen ins Bett. Am
        Sonntagmorgen wandern die einen nach Leukerbad während die anderen den
        Kleinbus nehmen, und in den Naturheilquellen ein wohltuendes Bad nehmen.
        Dann gehts mit dem Car weiter nach Gampel zum Mittagessen. Die Heimreise
        führt über Nufenen und Gotthardtunnel nach Altdorf, wo wir die heilige
        Messe besuchen.

        \item[]25. und 26. Sept.\\
        Anstatt ein Waldfest führen wir ein Herbstfest auf dem Schulhausplatz
        durch. Am Samstag ist im Festzelt ein Tanzabend und am Sonntag gibts
        Braten vom Spiess. Wir erwirtschaften einen Reingewinn von 4000 Fr.

    \end{itemize}

\end{history}
