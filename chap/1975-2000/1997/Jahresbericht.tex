\subsection*{1997}

\begin{history}


    \begin{itemize}

        \item 2. Jan.\\
              Wir gratulieren dem neugewählten Zunftmeister Franz Stocker zu seiner
              Wahl.

        \item 5., 9. und 11. Jan.\\
              Jahreskonzerte im Löwen.

        \item 18. Jan.\\
              Konzert im Shopping Center Emmen.

        \item 20. Jan.\\
              Kurzfristige Mitgliederversammlung im Löwen, weil Dirigent Marcel
              Sennhauser den sofortigen Rücktritt bekannt gegeben hat.

        \item 9. Feb.\\
              Bei prächtigem Frühlingswetter präsentieren wir unseren Wagen mit dem
              Motto \enquote{Undichte Pilatusbahnen} den Umzugsbesuchern.

        \item 22. März\\
              Das Kirchenkonzert gestalten wir dieses Jahr mit 4 Kleinformationen.

        \item 1. April\\
              Auflösung des Vertrages mit Dirigent Marcel Sennhauser.

        \item 6. April\\
              Wir begleiten die Erstkommunikanten in die Kirche. Das anschliessende
              Ständchen wird wegen heftigen Regenfällen abgesagt.

        \item 27. April\\
              Ständchen für die Bewohner vom Altersheim Rosenhügel Hochdorf.

        \item 11. Mai\\
              Bei strahlendem Sonnenschein geben wir vor der Kirche ein Ständchen zum
              Muttertag unter der Leitung von Vizedirigent Christoph Troxler.

        \item 29. Mai\\
              Begleitung des Fronleichnam-Gottesdienstes und anschliessende Prozession
              durchs Dorf.

        \item 6. Juni\\
              Kurplatzkonzert in Luzern unter der Leitung von Alexander Troxler.

        \item 7. Juni\\
              Ständchen zur Firmung.

        \item 10. Juni\\
              Aus 4 Bewerbern werden Beatrice Renkewitz und Markus Balli zur Wahl
              vorgeschlagen. Beatrice Renkewitz wird einstimmig zur Dirigentin der MGH
              gewählt.

        \item 3. Juli\\
              Das Konzert in Engelberg muss wegen unsicherer Witterung in den Kursaal
              verlegt werden. Abwechslungsweise mit dem Jodelclub und dem Alphorntrio
              Engelberg gestalten wir dieses Konzert.

        \item 11. - 13. Juli\\
              Ausflug nach Seefeld Österreich.

              Wir fahren mit dem Car von Estermann
              Beromünster via St. Margrethen, Bregenz, Oberstaufen, Immenstatt los und
              treffen dann um 18 Uhr in Seefeld ein. Nach dem Zimmerbezug und dem
              gemeinsamen Nachtessen geh es dann bereits das 1. Mal in den Ausgang.
              Während die Einen zuerst das schmucke Seefeld besichtigen, lassen sich
              die Anderen sofort im Bräukeller nieder. Dort steigt dann bereits das 1.
              Fest bei Musik und Bier. Am Samstag fahren wir mit der Standseilbahn zum
              Rest. Rosshütte. Dort geben wir ein Frühschoppenkonzert. Am Abend
              eröffnen wir mit einem Marsch durch die Fussgängerzone das Konzert im
              Kurpark. Dieses unterhaltsame Konzert wird von unserem Gastdirigenten
              Josef Brun geleitet. Als Abwechslung gibt unser Alphontrio ein Stück zum
              Besten. Anschliessend treffen wir uns sofort wieder in unserem
              Bräukeller. Bei ausgelassener Stimmung spielen noch spontan unsere 6er
              Musik. Für die meisten endete dann diese Festnacht im Western Saloon.
              Nach dem Sonntagsbrunch fahren wir dann über Landeck, Arlbergpass wieder
              in die Schweiz zurück.

        \item 10. Aug.\\
              Bei herrlichem Sommerwetter halten wir unser Waldfest. Am Abend spielen
              die Ronspatzen zur Unterhaltung auf. Weil trotz der guten Bedingungen
              die Besucherzahl mässig ist, soll dies unser letztes Waldfest sein.

        \item 26. Aug.\\
              1. Probe mit Dirigentin Beatrice Renkewitz.


    \end{itemize}

\end{history}
