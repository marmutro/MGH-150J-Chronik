\subsection{Jahresbericht 1998}
\begin{history}


    \begin{itemize}

        \item[]2. Jan.\\
        Mit einem Ständchen gratulieren wir Willy Wey zum Zunftmeister der
        Götschizunft.

        \item[]8., 10. und 11. Jan.\\
        Jahreskonzerte unter der neuen Leitung von Beatrice Renkewitz.

        \item[]24. Jan.\\
        Konzert im Shopping Center Emmen.

        \item[]4. April\\
        Wir umrahmen die Abendmesse zum Palmsonntag musikalisch.

        \item[]19. April\\
        Bei nasskaltem Wetter begleiten wir die Erstkommunikanten in einer
        feierlichen Prozession zur Kirche.

        \item[]8. Mai\\
        An diesem 1. schönen Sommerabend geben wir auf dem Kurplatz in Luzern
        ein Konzert.

        \item[]9. Mai\\
        Im Schulhaus organisieren wir ein Instrumentenparcours. Viele
        interessierte Schülerinnen und Schüler besuchen die Vorstellung der
        einzelnen Instrumente und geniessen eine Kostprobe der Ronspatzen.

        \item[]10. Mai\\
        Wir geben vor der Kirche ein Ständchen zum Muttertag.

        \item[]13. Juni\\
        Luzerner Kant. Musiktag in Hergiswil bei Willisau.

        Um 17.12 beginnen wir
        mit dem Marsch \enquote{Fidelity} von Henk Hogestein. Nach einem kurzen
        Imbiss tragen wir in der vollen Mehrzweckhalle unser Konzertstück
        \enquote{Four Little Maids} von John Carr vor. Von Thomas Rüedi bekommen
        wir einen sehr guten Expertenbericht. Einige Verbesserungsmöglichkeiten
        sieht er in der Intonation und der Dynamik.

        \item[]18. Juni\\
        Mit einem Marsch durch Engelberg eröffnen wir das Konzert im Park.

        \item[]20. Juni\\
        Ständchen zur Firmung.

        \item[]20. Sept.\\
        Bettagsgottesdienst in Gundolingen. Wir begleiten den Feldgottesdienst
        und geben anschliessen ein Ständchen.

        \item[]18. Ok.\\
        10 Jahre Neueröffnung Roter Löwen. Auf dem Löwenplatz geben wir einige
        Märsche zum Besten.

        \item[]24. und 25. Okt.\\
        Als Ersatz für das Waldfest führen wir das erste Mal ein Lotto durch.


    \end{itemize}

\end{history}
