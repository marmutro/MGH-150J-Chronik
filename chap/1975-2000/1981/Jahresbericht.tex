\begin{history}

    % \subsection{Jahresbericht} 1981

    \begin{itemize}

        \item[]2. Jan.\\
        Wir besammeln uns bei der Schmiede fürs Zunftbot. Mit Marschmusik geht
        es durchs Dorf zum Rest. Kreuz. Wir spielen dem neuen Zunftmeister Jakob
        Estermann Omelingen einen extra Marsch. Die Zunftbotnacht in Omelingen
        bleibt uns noch lange in Erinnerung.

        \item[]17. Jan.\\
        Da der Löwen über längere Zeit geschlossen ist sind wir gezwungen unser
        Winterkonzert in die Turnhalle zu verlegen. Diese Besonderheit fällt
        auch mit dem Jubiläum unseres Direktors Franz Limacher zusammen, der mit
        uns das 20. Winterkonzert durchführt.

        \item[]7. März\\
        An der GV informiert uns Direktor Franz Limacher, dass er nach 20 Jahren
        auf Ende Jahr demissionieren will

        \item[]26. April\\
        Bei strahlendem Frühlingswetter begleiten wir die Kommunionkinder mit
        einem Prozessionsmarsch durchs Dorf in die Kirche. Wie üblich geben wir
        nach dem Gottesdienst auf dem Dorfplatz ein Ständchen.

        \item[]1. Mai\\
        Die Götschizunft weit ihr neues Banner ein. Musikalisch umrahmen wir das
        Fest mit einem Ständchen. Beim Festumzug am 2. Mai sind wir auch dabei
        und der grosse Zunftabend am 5. Mai wird auch von uns eröffnet.

        \item[]10. Mai\\
        Muttertagständchen auf dem Dorfplatz.

        \item[]11. Mai\\
        Experte André Winkler begutachtet eine Probe im Löwen und erläutert uns
        danach mit träfen Worten, welche Teile von \enquote{Synfonietta Pastorale} und
        \enquote{London Scherzo} wir noch verbessern sollten.

        \item[]16. Mai\\
        Firmung von über 50 Kindern durch den Abt von Engelberg. Wir spielen
        danach ein Ständchen.

        \item[]2. Juni\\
        Gemeinsames Vorbereitungskonzert fürs Eidgenössische mit der Feldmusik
        Gunzwil in der Kantonschule Beromünster.

        \item[]11. Juni\\
        Vorbereitungskonzert in der Pfarrkirche Rothenburg zusammen mit der
        Feldmusik Rothenburg. Mit einem Einzug in den Flecken üben wir auch die
        Marschmusik.

        \item[]14. Juni\\
        Wir danken Hermann Wolf mit einem Ständchen für die Benutzung seines
        Obstmagazins als Probelokal.

        \item[]17. Juni\\
        Gönnerkonzert im Löwen. Trotz Gratiseintritt kann der Kassier 3000 Fr.
        verbuchen.

        \item[]20. und 21. Juni\\
        Nach sehr harter Probearbeit -- in 100 Tagen war der Verein 50-mal
        beisammen -- fahren wir mit einem Car der Auto AG Rothenburg nach
        Lausanne.

        Leider bemerkt der Chauffeur, dass an einem Rad die Luft fehlt, sodass
        wir nach Rothenburg fahren in der Garage pumpen müssen. Der Car wurde
        mit uns mit dem Lift unters Dach gehoben und wir warteten eine halbe
        Stunde, aber das Pumpen wollte nicht klappen. Wir mussten nun unser
        Gepäck und die Instrumente in einen Ersatzcar umladen und fuhren deshalb
        mit einer Stunde Verspätung ab. Gerade noch rechtzeitig kamen wir in
        Lausanne zur Vorprobe an.

        Wir spielen im Palais de Beaulieu Saal vor leider nur etwa 30 Personen.
        Zuerst spielen wir unser Selbstwahlstück \enquote{Synfonietta Pastorale} und
        danach das Aufgabenstück \enquote{London Scherzo}. Nach einem Gesamtfoto zur
        Erinnerung stellen wir uns am Nachmittag auf zum \enquote{Marsch zur Feier des
            Tages}. Am Sonntag nehmen wir Teil am farbenprächtigen Einzug von 150
        Musikvereinen ins Fussballstadion. Leider war die erreichte Punktzahl
        von Total 103.5 Punkten nicht das was wir erwartet hatten, dafür
        erhielten wir für die Marschmuskik 46 Punkte. Nach einem Nachtessen in
        der Sonne Reiden werden wir von der Hildisrieder Vereinen und der
        Dorfbevölkerung empfangen.

        \item[]23. Juni\\
        Im Bauernschopf vom Rest. Schlacht nehmen den Rest der Stücke auf
        Tonband auf.

        \item[]7. - 21. Juli\\
        Probedirigate von Hanspeter Häcki, Toni Kunz, Guido Ruckstuhl und Thomas
        Zemp.

        \item[]1. Aug.\\
        Teilnahme an der Feier auf dem Schulhausplatz.

        \item[]2. Aug.\\
        Nach 2 verregneten Sonntagen führen wir doch noch das Waldfest durch.
        Leider ist wohl der 2. Aug. kein ideales Datum weshalb der
        Besucherandrang recht gering ist.

        \item[]18. Aug.\\
        An einer ausserordentlichen GV wählen wir Thomas Zemp zum neuen
        Dirigenten.

        \item[]8. Sept.\\
        Erste Probe mit dem neuen Direktor Thomas Zemp.

        \item[]26. Sept.\\
        Musikhock im Schopf von Walter Rüttimann. Wir verabschieden unseren
        langjährigen Direktor Franz Limacher.

        \item[]15. Nov.\\
        Wir spielen zur Einweihung vom neuen Bank- und Postgebäude und während
        des Banketts ein kleines Konzert.

        \item[]11. Dez.\\
        Wir eröffnen den Neuzuzügerabend mit einem Ständchen.


    \end{itemize}

\end{history}
