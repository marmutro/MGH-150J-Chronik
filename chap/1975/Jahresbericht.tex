\begin{multicols}{2}

    % \subsection{Jahresbericht}

    \begin{itemize}

        \item[]3. Jan.\\
        Um 20 Uhr holen wir den Zunftmeister von 1974 Alois Gassmann
        bei seinem Hause ab. Dann geht es mit Fackeln und Marschmusik durchs Dorf in den Löwen.
        Nach den üblichen Traktanden wird mit grossem Applaus Gemeindepräsident
        Werner Troxler-Käppeli, Moos, zum neuen Zunftmeister erkoren.

        \item[]7. Jan.\\
        Wir beschliessen, in diesem Jahr auf den Musikhock im Kreuz am schmutzigen Donnerstag zu verzichten,
        weil die Fastnacht so kurz ist.

        \item[]11. und 18. Jan.\\
        An diesen Abenden bringen wir das Winterkonzert zu Gehör. Der Löwensaal ist jeweils sehr gut besetzt.

        \item[]16. Jan.\\
        Wir begeben uns nach Emmen, um im Pfarreisaal unser Konzert der Emmer Bevölkerung aufzuführen.
        Leider sind nur sehr wenig Zuhörer anwesend.

        \item[]4. Feb.\\
        Wir begleiten den Zunftmeister zum Schulbesuch. Mit Marschmusik geht es durchs Dorf auf den alten Schulhausplatz.
        Während der Orangenschlacht spielen wir einige Märsche.

        \item[]6. April\\
        Wegen starkem Schneefall kann der Einzug am Weissen Sonntag nicht abgehalten werden.

        \item[]15. April\\
        Wir beschliessen, den Lohn von Direktor Franz Limacher von 3000 auf 3400 Fr. zu erhöhen.

        \item[]11. Mai\\
        Muttertagständchen. Löwenwirt J. Schnarwiler zahlt danach noch ein Bier.

        \item[]24. Mai\\
        Zur Expertise für das kommende kantonale Musikfest in Sempach laden wir André Winkler zur
        Probe in die Turnhalle ein. Mit träfen Worten erläutert er uns die schwachen Stellen der
        "`nostalgischen Ouvertüre"' und der "`Ouverture Symphonique"'.

        \item[]13. und 14. Juni\\
        Im Löwen und in der Kirche bringen die Musikgesellschaft Römerswil und die Musikgesellschaft Hildisrieden
        abwechslungsweise ihre Wettstücke für das kant. Musikfest in Sempach zur Aufführung.

        \item[]22. Juni\\
        Kant. Musikfest in Sempach.

        Wir spielen in der Festhalle das Aufgabestück "`Nostalgische Ouverture"',
        wir bekommen von der Jury schöne 153 von 180 Punkten. In der Kirche spielen wir das Selbstwahlstück
        "'Ouverture Symphonique"'. Hier erhalten wir 160 Punkte.
        Mit der Totalpunktzahl von 313 erreichen wir den 5. Rang in der 2. Stärkeklasse.

        Von der Marschmusikjury erhalten wir am Nachmittag leider nur 29 Punkte.
        Am Abend wird die MGH von der Götschizunft und der Dorfbevölkerung empfangen.
        Im Löwen werden wir von der Trachtengruppe und dem Kirchenchor unterhalten.

        \item[]20. Juli\\
        Wir können das Waldfest bei schönem Wetter durchführen. Die Tanzmusik "`The King Drivers"'
        bringt die Gäste zum Tanzen.



    \end{itemize}

\end{multicols}
