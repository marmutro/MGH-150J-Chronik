\begin{multicols}{2}

    \section{Jahresbericht}

    \begin{itemize}
        \item[-]3. Jan.\\
        Bei der Probe macht der Präsident den Vorschlag
        des Vorstandes, beim Winterkonzert dem Verein aus finanziellen
        Gründen kein Nachtessen mehr zu bezahlen, dafür beim
        Konzert in Emmen ein Bier zu offerieren. Ohne
        Diskussion wird der Vorschlag angenommen.

        \item[-]5. und 12. Jan.\\
        An diesen Abenden bringen wir das Jubiläumskonzert
        zu Gehör. Wir können uns an einem vollbesetzten
        Saal erfreuen. Eintritte 600 Personen. Seit mehreren Jahren
        konnten wir nicht mehr so viele Zuhörer begrüssen.

        \item[-]10. Jan.\\
        Auch dieses Jahr pilgern wir nach Emmen um das
        Winterkonzert auch der Emmer Bevölkerung zu Gehör zu
        bringen. Der Reinerlös soll dieses Jahr an das seraphische
        Liebeswerk weitergeleitet werden. Aber es wird sich bloss um ein
        Trinkgeld handeln, denn die Zuhörerzahl beschränkt sich auf
        ca. 60 Stück. Nachher inkludieren wir im Restaurant
        Sternen noch ein Bier.

        \item[-]18. Jan.\\
        Zunftbot im Rest. Kreuz. Um 1/4 vor 8 Uhr besammeln
        wir uns beim letztjährigen Zunftmeister Josef Estermann
        Bauernhof um noch unter seiner Macht den letzten Tropfen
        zu geniessen. Aber es soll noch nicht der letzte sein an diesem
        Abend, denn es regnet während des Einzuges ins Kreuz in Strömen.
        Das schreckt uns aber nicht zurück
        vor dem Rätsel wer für das Jahr 1974 zum Zunftmeister
        gewählt wird.

        Der zum letztenmal amtierende Zunftpräsident
        Albin Estermann gibt die Nomination von Alois
        Gassmann Sandgütsch als neuer Zunftmeister bekannt.
        Mit grossem Applaus wird er in sein Amt eingesetzt.
        Es ist schon sehr lange her, dass ein Aktivehrenmitglied
        der Musikgesellschaft zu diesem hohen Amt gewählt
        wird. Wir wünschen ihm viel Glück und viel fastnächtliche
        Stunde in diesem Jahr.

        \item[-]1. Feb.\\
        100. Generalversammlung im Rest. Kreuz

        \item[-]2. Feb.\\
        Die Firma Schuler aus Rothenturm hat unsere Uniform im
        Rohbau soweit fertig, dass sie heute jedem Musikanten
        angepasst werden kann. Wie es scheint, sind nur geringe
        Abänderungen worzunehmen.

        \item[-]15. Feb.\\
        Der Zunftmeister Alois Gassmann besucht die
        Schüler. Um 15 Uhr besammeln wir uns beim neuen
        Schulhaus. Dann gehts mit Marschmusik durchs Dorf
        zum alten Schulhauszplatz. Dort bringen wir einge Stücke zu
        Gehör und der Zunftmeister lädt uns zu einme Imbiss
        im Kreuz ein. Besten Dank.

        \item[-]21. Feb.\\
        Da es schmutziger Donnerstag ist führen wir am
        Abend den traditionellen Musikhock im Kreuz durch.
        Es geht bald so richtig fastnächtlich zu. Es erscheinen gegen
        20 Maskierte, was für den kleinen Raum doch schon viel ist.
        Bei einer zügigen Tanzmusik, die Hugo Fleischlin organisiert
        hat kommt Jung und Alt in den Genuss des Beinschwingens.

        \item[-]5. März\\
        Herr Schuler jun. aus Rothenturm orientiert uns über sein
        Kravattensortiment. Mit grossem Mehr wird beschlossen,
        eine gemusterte Kravatte zur neuen Uniform zu kaufen.

        \item[-]11. März\\
        OK Präsident Julius Bieri lädt den Verein ein zur Verteilung
        der Adressen für die Bettelaktion. Jedem
        anwesenden Musikant werden einige Posten zugeteilt, sodass
        dem Kassier von heute an die Säcke bald gefüllt werden sollten.



    \end{itemize}

\end{multicols}
