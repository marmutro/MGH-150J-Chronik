\begin{multicols}{2}

    \subsection{1900-1925}

    Wenn auch diese Zeitspanne Mitgliederwechsel brachte,
    so konnte doch ein langsames Wachsen und Gedeihen
    der Musikgesellschaft festgestellt werden. Ihre Mitgliederzahl erhöhte sich von zehn auf neunzehn Mann.
    Die Tätigkeit bestand in den ersten Jahren noch vorwiegend aus dem Musizieren bei Ständchen, Ausmärschen
    und Tanzanlässen.

    Dass die Musikgesellschaft Hildisrieden bemüht war,
    gute Beziehungen zu Kirchenchor und Schützengesellschaft zu pflegen,
    beweisen die gemeinsamen Konzerte und Ausflüge auf die Rigi, nach Engelberg und
    Andermatt. Die Musik betrachtete ihre Aufgabe aber
    nicht bloss im Tanzmusikspielen, Sie war jederzeit bereit aufzutreten, wenn sich irgendetwas tat
    in der Gemeinde. Das zeigte die Einweihungsfeier unserer neuen
    Kirche im Jahre 1904. Der Verein war bestrebt, diesem Anlass ein festliches Gepräge zu geben und erfreute
    mit seinen Beiträgen den Bischof Leonard Haas und all die vielen Gäste.

    Im Verlaufe des Herbstes gleichen Jahres wurde die
    Gesellschaft zur Unterhaltung zu einem Grossanlass
    nach Münster aufgeboten. Es ging um das umstrittene Wynentalbahn-Projekt, das damals viel Gesprächsstoff
    lieferte, hätte doch die Weiterführung der Linie über
    Hildisrieden bis Emmenbrücke unser Dorf in der Entwicklung beeinflusst. (Geplanter Bahnhof bei der
    heutigen Bäckerei Arnold).

    Die Musikgesellschaft wurde 1905 erstmals photografiert.

    Dass der Probenbesuch nicht immer den Wünschen des Kapellmeisters entsprach,
    zeigt das Probenverzeichnis von 1907. Von den 15 abgehaltenen Proben trug
    nur eine den Vermerk "`Alle anwesend"'. Stets war die
    Musik bereit, ihren Möglichkeiten. entsprechend mitzuwirken, so 1909 am Pfarraufritte von Pfarrer Alois
    Hodel.

    In den Jahren 1910 und 1911 sind nebst den kirchlichen
    und den weltlichen Anlässen keine weiteren Taten zu verzeichnen,
    ebenso schweigt das Probenverzeichnis.

    An der Generalversammlung vom 31. März 1912 wurden die Statuten
    wieder revidiert und von folgenden Mitgliedern unterzeichnet:

    Karl Estermann, Traselingen
    Peter Muff, Lehrer
    Alois Estermann, Traselingen
    Kaspar Troxler, Moos
    Kaspar Suter, Dorf
    Alois Troxler, Schopfen
    Franz Troxler, Schopfen
    Josef Troxler, Schopfen
    Alois Wolf, Sandgütsch
    Josef Gassmann, Gigen
    Balz Gassmann, Gigen
    Heinrich Estermann, Dorf

    Der Rechnungsbericht, abgelegt von Kassier Kaspar
    Troxler, verzeigte an Einnahmen Fr. 298.90 und an
    Ausgaben Fr. 79.22. Die Mehreinnahmen wurden verteilt, und jedes Mitglied konnte 15 Franken als
    "`Dividende"' entgegennehmen.

    Dass die Musikgesellschaft 1915 neue Mitglieder aufnehmen konnte, bezeugt, dass der Verein bestrebt war,
    sich weiterzuentwickeln und an sich zu arbeiten.
    Neu wurden in den Verein aufgenommen: Josef Disler,
    Dorf; Jakob Estermann, Bethlehem; Heinrich Ester-
    mann, Ohmenlingen; Heinrich Estermann, Dorf; Josef
    Jutz, Löwen, und Leo Erni, Schmiede.

    Diese Mitglieder haben das Vereinsschifflein in gesellschaftlicher Hinsicht wieder
    belebt und flotte Probenarbeit geleistet. Die Musikgesellschaft hatte sich nun
    schon derart gefestigt, dass der entfachte Weltkrieg das
    Vereinsleben nicht mehr zum Stillstand zu bringen vermochte.
    Bei jeder, wenn auch nur selten sich bietenden
    Gelegenheit, wurden Proben abgehalten.

    Im Jahre 1915 hatte der Vorstand folgendes
    Aussehen:

    Präsident: Kaspar Suter, Dorf
    Dirigent Alois Estermann, Traselingen
    Akt: Karl Estermann, Traselingen
    Kassier: Kaspar Troxler, Moos
    Mitglied: Alois Troxler, Lenzenhof

    «Friede ist es zwar geworden, das Elend und der
    Hass unter den Völkern ist geblieben», so beginnt der
    Tätigkeitsbericht von 1919, Arbeitslosigkeit und
    Unzufriedenheit herrschten in der Heimat. Als in den
    ersten Augusttagen das tapfere Inf Reg 19 nach
    Zürich gerufen wurde, um Ruhe und Ordnung zu
    schaffen, betraf das auch Hildisrieder Musikanten,

    Am Jubiläumsschiessen der Schützengesellschaft Hildisrieden
    gab die Musik im Löwen ein Konzert zum besten.

    Das Symbol des Vereins, die Fahne, die 40 Jahre Wind
    und Wetter getrotzt hatte, sollte ersetzt werden. Eine
    von Frauen und Töchtern Hildisriedens im Frühjahr
    1921 durchgeführte Sammlung, ermöglichte dieses
    Vorhaben. Besondere Aufmerksamkeit schenkte man
    der Musikgesellschaft bei der Teilnahme am
    Kantonalen Musiktag in Sempach unter der Leitung
    von Lehrer Alois Troxler.

    An der Gemeindeversammlung vom 25. März 1924
    beantragte Josef Disler sen, dass der jährliche Beitrag
    der Gemeinde von Fr. 40.— auf Fr. 100,— erhöht
    werden sollte, was auch einstimmig genehmigt wurde.
    Von nun an wurde intensiver und freudiger musiziert


\end{multicols}
