
\begin{figure}[ht]
    \centering
    \subfloat{%
        \includegraphics[width=0.45\textwidth]{./chap/1874-1900/MGH-Statuten-1874-S1.jpg}%

    }\hfil
    \subfloat{%
        \includegraphics[width=0.45\textwidth]{./chap/1874-1900/MGH-Statuten-1874-S2.jpg}%
    }
    \caption{Statuten 25. März 1876}
    \label{fig:Statuten-1874}
\end{figure}

\begin{history}

    \subsubsection{Statuten vom 25. März 1876}

    Es hat sich in der Gemeinde Hildisrieden unter obigem Datum eine neue
    Musikgesellschaft gebildet, welche den Mitgliedern zum besseren Gedeihen der
    Gesellschaft folgende Bedingungen festgesetzt:

    \S1 Jedes Mitglied soll, wenn möglich, lange bei der Gesellschaft verbleiben
    und dieselbe fördern.

    \S2 Die Gesellschaft bildet sich nicht nur zum Schein, sie will auch etwas
    erlernen, um sich auch produzieren zu können, daher verpflichtet sich jedes
    Mitglied, an den vorhergesagten Proben zu erscheinen.

    \S3 Zum Bestimmen der Proben, sowie zum Dirigieren wählt die Gesellschaft
    einen Kapellmeister.

    \S4 Wer an der Probe 20 Minuten zu spät erscheint, der wird um 20 Cts., wer
    eine halbe Stunde zu spät kommt um 30, und wer gar nicht erscheint, um 50
    Cts. bestraft.

    \S5 Wenn es gewisse Umstände verlangen, so ist der Kapellmeister berechtigt,
    festzustellen, was von den Mitgliedern bis zur nächsten Probe gelernt werden
    soll.

    \S6 Wird etwas nicht gelernt und man sieht, dass es flissentlich nicht
    geschehen ist, so ist der Betreffende in eine Strafe von 30 Cts. verfallen,
    die jedoch auch erhöht werden kann bis auf einen Franken.

    \S7 Kann sich ein Mitglied mit der Gesellschaft nicht mehr vertragen, so
    steht es dieser frei, das Mitglied auszuschliessen,

    \S8 Tritt ein Mitglied nach eigener Willkür aus der Gesellschaft, so hat es
    eine Entschädigung von 20 Fr. zu entrichten.

    \S9 Wer nicht mehr an den Proben erscheint, der schliesst sich von selbst
    aus der Gesellschaft und die Entschädigung folgt. Die Gelder, Strafgelder
    sowie andere, sind vom Präsidenten einzuziehen und er hat die Pflicht, zu
    besorgen und am Ende des Jahres oder wenn es die Gesellschaft verlangt,
    genaue Rechnung abzugeben.

    \S10 | Was Instrumente anbetrifft, so hat jedes Mitglied das seinige selbst
    anzuschaffen und zu besorgen. Ein schlechtes Instrument wird nicht geduldet.

    \S12 Wenn 2/3 der Mitglieder einen Ausmarsch verlangen, und ein Mitglied
    will oder kann nicht kommen, so hat es dieser sofort dem Kapellmeister
    anzuzeigen, sonst ist es in eine Strafe von drei Fr. verfallen. Die
    Gesellschaft zieht dann eine andere Person zu und das betreffende Mitglied
    hat selbe zu entschädigen und ihr sein Instrument zu leihen.

    \S13 Jedes Mitglied soll sich dem Befehl des Kapellmeisters und Präsidenten
    fügen soweit ihnen die Gesellschaft das Recht in die Hand gelegt hat.

    \S14 Kommt etwas vor, das in dem Rechte dieser zwei Personen liegt, so wird
    die Gesellschaft angefragt, und es wird abgestimmt.

    \S15 Tritt ein neues Mitglied in die Gesellschaft, so hat es eine
    Eintrittszahlung von zehn Fr. zu entrichten und sein Instrument
    anzuschaffen. Dann tritt es in die gleichen Rechte wie die andern
    Mitglieder.

    \S16 Auch diese, so wie alle andern Gelder werden vom Präsidenten bezogen
    und er hat am Ende des Jahres genaue Rechnung abzugeben, wonach die Gelder
    unter die Mitglieder gleichmässig verteilt werden. Am Ende Jahres hat jedoch
    ein Saldo von 20 Fr. in der Kasse zu bleiben, das zum Anschaffen neuer
    Musikstücke aufs folgende Jahr verwendet wird.

    \S17 Die Musikstücke werden vom Präsidenten oder Kapellmeister bestellt, sie
    haben jedoch die Gesellschaft zuerst anzufragen.

    \S18 Die bestellten Stücke werden vom Präsidenten bezahlt und er hat darüber
    Rechnung zu geben.

    \S19 Würde es der Fall sein, dass ein Mitglied stürbe, so würde die
    Gesellschaft für dasselbe einen angemessenen Gottesdienst halten lassen,

    \S20 Wer bei der Musik sein will, muss auch in die Gesellschaft treten.

    \S21 Die Gesellschaft wählt einen Vorstand von 3 Mitgliedern: Kapellmeister,
    Präsident und Schreiber auf eine Amtsdauer von zwei Jahren.

    \S22 Der Schreiber schreibt die Stücke und alles, was die Gesellschaft
    beschlägt, wofür er angemessen entschädigt wird.

    \S23 Als Entschuldigung wird in allen Fällen nur Krankheit oder
    Militärdienst angenommen.

    \S24 Jedes Mitglied hat sich eigenhändig zu unterzeichnen.\\


\end{history}
