\begin{history}

    \subsection{1875-1900}

    Die Wiegenjahre brachten der Musikgesellschaft Hildisrieden
    abwechslungsreiche Begebenheiten.

    Die Hauptbetätigung blieb weiterhin das Konzertieren in Form von Ständchen
    und Ausmärschen, in und ausserhalb der Gemeinde. So treffen wir die
    Musikgesellschaft 1882 im Emmenbaum anlässlich einer Theateraufführung und
    1887 beritten am Auffahrtsumritt in Sempach. Die Entschädigung, welche die
    Musikgesellschaft von der Kirchenverwaltung Sempach erhielt, betrug immerhin
    45 Franken. Ein Gesuch, den Auffahrtsumritt abwechslungsweise mit der
    Musikgesellschaft Sempach alle drei Jahre bestreiten zu dürfen, wurde von
    der Kirchenverwaltung abgelehnt. Was unseren Musikanten in den 1890er Jahren
    vorgeworfen werden konnte, war der Mangel an Ordnung    und Disziplin. Wohl
    hatte man 1884 neue Statuten aufgestellt und neue Strafbestimmungen
    festgesetzt; diese sind aber nicht von allen befolgt worden. Diese unruhigen
    Vereinsjahre, fünf an der Zahl, hatten zur Folge, dass man mit dem Gedanken
    spielte, die Musikgesellschaft aufzulösen oder sich den gegebenen Statuten
    zu fügen.

    Man fand sich hin und wieder zusammen und kam ebenso schnell wieder
    auseinander, bis schliesslich die Vernunft siegte. Ein weiterer Grund, der
    den Zusammenhang des Vereins trübte, war der kleine finanzielle Beitrag der
    Gemeinde. Jährlich 20 Franken waren wirklich kein grosses Honorar, wenn man
    bedenkt, dass jeder Musikant sein Instrument selbst zu beschaffen und die
    Musikgesellschaft an sämtlichen Prozessionen teilzunehmen hatte.

    Im Jahre 1899 wurde das Restaurant Kreuz eröffnet und der Saal im Löwen
    eingeweiht. Die Einladung zu diesen Festen fügte das schwache Gebilde wieder
    zusammen.

    Die Statuten von 1874 und 1884 erneuert, haben 10 Mitglieder unterzeichnet.
    Es sind: Silvester Schnieper, Josef Wolf, Leonz Geisshüsler, Silvester
    Disler, Josef Disler, Peter Troxler, Peter Muff, Rudolf Bühlmann, Josef Wolf
    und Niklaus Süess.

    Weitere Einnahmen wurden mit dem \enquote{Umblasen}  während der Weihnachtszeit in
    der Gemeinde erzielt. Vor den Häusern wurde jeweils musiziert, wofür die
    Musik-freundlichen Bewohner eine Geldgabe spendeten. Im Jahre 1886 brachte
    der Verein auf diese Weise Fr. 259.60 zusammen. Bei kaltem Wetter war das
    Musizieren kein Vergnügen, denn öfters froren die Ventile ein und mussten
    durch Einhauchen wieder beweglich gemacht werden. Gerne wurde dann die
    Einladung angenommen, ins Haus zu kommen.

    In der warmen Stube liess es sich bequemer musizieren, besonders da, wo auch
    \enquote{Mittel} gegen trockene Kehlen bereitstanden. Allzu lange aber durfte die
    Gastfreundschaft nicht beansprucht werden, warteten doch noch viele
    Einwohner auf den Besuch der Musikanten. War aber dann das gesteckte
    Tagesziel erreicht, wurde es mit dem Aufbruch nicht so genau genommen, gar
    wenn es oft etwas Feines zum Beissen gab.

    Leider ist nun auch dieser alte Brauch verschwunden. Dass es hin und wieder
    gemütlich wurde, berichtet die Chronik. Als beim Ausflug auf den Napf ein
    Leiterwagen mit Blumen und Girlanden geschmückt dem Sempachersee entlang
    rasselte, löste sich vom Wagen ein Rad, sodass ein Musikant nach dem andern
    vom Wagen kippte. Dabei gab es einige defekte Instrumente. Diese konnten in
    Willisau beim Instrumentenmacher Badmann wieder hergestellt werden. Die
    Fahrt führte ins Luthernbad und von da an zu Fuss auf den Napf. Dem
    Bassisten Troxler war das Bergsteigen nicht besonders angenehm, denn er soll
    sich damals geäussert haben: \enquote{Dass mer au uf ne so ne Morgelandssiech cha
        go!}


\end{history}
