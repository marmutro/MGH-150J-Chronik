
\cleardoublepage

\begin{multicols*}{2}

    % erwinge Text nur in rechter Spalte
    \vphantom{1em}
    \vfill
    \columnbreak

    \section*{Die Entstehung der Chronik}

    Während der ersten Sitzung zur Planung unserer Jubiläumsfeierlichkeiten kam
    das Thema einer Chronik auf den Tisch. Bereits zum 100-jährigen Jubiläum
    wurde eine solche erstellt. Angesichts dessen, und nach einigem Zögern,
    übernahm ich die Aufgabe, eine Chronik für das 150-jährige Bestehen der
    Musikgesellschaft zu verfassen. Damals war mir noch nicht bewusst, wieviel
    Arbeit das bedeutete.

    Wichtige Ereignisse wurden aus den Jahresberichten gefiltert und in eine
    stimmige historische Reihenfolge gebracht. Die älteren, von Hand
    geschriebenen Berichte in den Büchern, boten trotz ihrer teilweise
    anspruchsvollen Lesbarkeit einen wertvollen und vollständigen Einblick. Die
    jüngeren Berichte hingegen, digital gespeichert auf diversen Datenträgern,
    stellten eine andere Herausforderung dar, da einige Informationen im Laufe
    der Zeit verloren gegangen sind.

    \bigskip

    Um die Herausforderungen und Eigenheiten von Microsoft Word zu umgehen,
    entschied ich mich für \LaTeX, ein Satzsystem, das vor allem im akademischen
    Bereich für die Erstellung wissenschaftlicher Arbeiten genutzt wird. Die
    Erstellung der Quelltexte erfolgte in einem Texteditor, gefolgt von der
    Übersetzung in ein PDF-Dokument. Um den Fortschritt sicherzustellen,
    speicherte ich regelmässig Versionen im einem git Repository und sicherte
    dieses auf GitHub.

    \bigskip

    Diese Chronik ist daher nicht nur ein Zeugnis der reichen Geschichte unserer
    Musikgesellschaft, sondern auch ein Produkt der modernen Technik und einer
    Portion Widerwillen, die schlussendlich in Hingabe umschlug.

    Ich hoffe, dass dieses Werk die Erinnerungen an unsere vielfältige
    Vergangenheit lebendig hält und kommende Generationen inspiriert.

    \bigskip

    \raggedleft Der Chronist\\
    \raggedleft Martin \enquote{Mumi} Troxler\\
    \raggedleft Sonnerain, Hildisrieden\\

\end{multicols*}
