% \usepackage[utf8]{inputenc}
\usepackage{polyglossia}
\usepackage[a4paper, landscape, margin=2cm, twoside, bindingoffset=1cm]{geometry}
\usepackage{multicol}
\usepackage[labelformat=simple]{subfig}
\usepackage{graphicx}
\usepackage{wrapfig}
% \usepackage{showframe}
\usepackage{lipsum}
\usepackage{verbatim}
\usepackage{eso-pic}
\usepackage{fancyhdr}
\usepackage{tikz}
\usepackage[newparttoc]{titlesec}
\usepackage{titletoc}
\usepackage{xfrac}
\usepackage{pdfpages}
\usepackage{forloop}
\usepackage[export]{adjustbox}
\usepackage{caption}
\usepackage{csquotes}
\usepackage{threeparttable}
\usepackage{longtable}
\usepackage{floatpag}
\usepackage{enumitem}
\usepackage{needspace}

\setdefaultlanguage[variant=swiss]{german}

\makeatletter
% change title size
\patchcmd{\@maketitle}{\normalsize}{\HUGE}{}{}
% change author size
\patchcmd{\@setauthors}{\MakeUppercase {\authors }}{\MakeUppercase{\Large\authors}}{}{}
% change section title size
\patchcmd{\section}{\normalfont}{\normalfont\Large}{}{}
\makeatother

% TOC formatting
\setcounter{tocdepth}{0} % Show only \part and \chapter in tableofcontents
\makeatletter
\let\latexl@part\l@part
\def\l@part#1#2{\begingroup\let\numberline\@gobble\latexl@part{#1}{#2}\endgroup}
\makeatother

\makeatletter
\let\latexl@chapter\l@chapter
\def\l@chapter#1#2{\begingroup\let\numberline\@gobble\latexl@chapter{#1}{#2}\endgroup}
\makeatother

\titleformat{\part}{\normalfont\Huge\bfseries}{}{0em}{}
\titleformat{\chapter}{\normalfont\huge\bfseries}{}{0em}{}
\titleformat{\section}{\normalfont\Large\bfseries}{}{0em}{}
\titleformat{\subsection}{\normalfont\large\bfseries}{}{0em}{}
\titleformat{\subsubsection}{\normalfont\normalsize\bfseries}{}{0em}{}
\titleformat{\paragraph}[runin]{\normalfont\normalsize\bfseries}{}{0em}{}
\titleformat{\subparagraph}[runin]{\normalfont\normalsize\bfseries}{0}{0em}{}

% Definieren, wie viel Platz mindestens vorhanden sein muss
\newcommand{\subsectionbreak}{\needspace{8\baselineskip}}

\titleformat{\subsection}[hang]
{\normalfont\normalsize\bfseries}{\thesubsection}{1em}{\subsectionbreak}
[\titlerule\vspace{1ex}]  % Fügt eine horizontale Linie nach dem Titel ein


% Anpassen des Formats für Unterabschnittstitel
\titlespacing{\part}{0em}{0em}{0em}
\titlespacing{\chapter}{0em}{0em}{0em}


\setlist[itemize]{leftmargin=0cm}

% redefine plain style (used by first pages of chapters)
\fancypagestyle{plain}{
    \lhead{}
    \fancyhead{}
    \fancyfoot{}
    \renewcommand{\headrulewidth}{0pt}
}

\fancyhf{}
\fancyhead[RO]{\leftmark{}}
\fancyhead[LE]{\leftmark{}}
\fancyfoot[RO]{\thepage}
\fancyfoot[LE]{\thepage}
\pagestyle{fancy}
\floatpagestyle{fancy} % required for float-only pages

\renewcommand{\headrulewidth}{0pt}
\renewcommand{\chaptermark}[1]{\markboth{#1}{}}
\renewcommand{\sectionmark}[1]{\markboth{#1}{}}
\renewcommand{\subsectionmark}[1]{} % ignore subsections in header
\AddToHook{cmd/section/before}{\clearpage} % new page on each section

\renewcommand{\thesubfigure}{\relax}  % Do nothing for the counter »subfigure«


\setlength\columnsep{1cm} %distance between multicolumn columns
\captionsetup{font=small, labelformat=empty} % Bildtitel ohne Kapitelangabe
\captionsetup[subfigure]{justification=justified,singlelinecheck=false}

\newcommand{\groupphoto}[5]{
    {
            \captionsetup{singlelinecheck=false} % left aligned
            \begin{figure}[!htbp]
                \thisfloatpagestyle{empty}
                \centering
                \begin{measuredfigure}
                    \includegraphics[width=#1\textwidth,height=#2\textheight, keepaspectratio]{#3}
                \end{measuredfigure}
                \caption{#4}
                \label{#5}
            \end{figure}
        }
}

\newcommand{\portrait}[3][0.8]{
    {
            \captionsetup{singlelinecheck=false} % left aligned
            \begin{figure}[!htbp]
                \def\theight{#1}
                \begin{measuredfigure}
                    \includegraphics[width=0.93\textwidth, height=\theight\textheight, keepaspectratio]{#2}
                    \caption{#3}
                \end{measuredfigure}
            \end{figure}
        }
}
\newcommand{\ConcertProg}[2][0.8]{{
            \begin{figure}[!htbp]
                \centering
                \includegraphics[height=#1\textheight,keepaspectratio]{#2}%
            \end{figure}
        }}

\newcommand{\ConcertProgsTwoVertical}[3][0.485]{{
            \begin{figure}[!htbp]
                \centering
                \subfloat{%
                    \includegraphics[height=#1\textheight,keepaspectratio]{#2}%
                }\\
                \subfloat{%
                    \includegraphics[height=#1\textheight,keepaspectratio]{#3}%
                }
            \end{figure}
        }}

\newcommand{\ConcertProgsTwoHorizontal}[3][0.85]{{
            \begin{figure}[!htbp]
                \centering
                \subfloat{%
                    \includegraphics[height=#1\textheight,keepaspectratio]{#2}%
                }\hfil
                \subfloat{%
                    \includegraphics[height=#1\textheight,keepaspectratio]{#3}%
                }
            \end{figure}
        }}

\newcommand{\ConcertProgsFourOnPage}[5][0.47]{{
            \begin{figure}[!htbp]
                \centering
                \def\theight{#1}
                \subfloat{%
                    \includegraphics[height=\theight\textheight,keepaspectratio]{#2}%
                }\hfil
                \subfloat{%
                    \includegraphics[height=\theight\textheight,keepaspectratio]{#3}%
                }\\
                \subfloat{%
                    \includegraphics[height=\theight\textheight,keepaspectratio]{#4}%
                }\hfil
                \subfloat{%
                    \includegraphics[height=\theight\textheight,keepaspectratio]{#5}%
                }
            \end{figure}
        }}

\newenvironment{history}{\begin{multicols}{2}}{\end{multicols}} % multicolumn for history text


\newenvironment{MulticolFigure}
{\par\medskip\noindent\minipage{\linewidth}}
{\endminipage\par\medskip}

\newcommand{\multicolphoto}[2][0.93]{{\includegraphics[width=#1\columnwidth]{#2}}}
% \newcommand{\multicolphoto}[2][0.93]{}

